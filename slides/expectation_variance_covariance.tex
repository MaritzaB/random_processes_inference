\documentclass{beamer}

\mode<presentation>{
  \usetheme{Warsaw}
}


\usepackage[spanish,es-tabla,es-nodecimaldot]{babel}
\usepackage{tikz}
\usepackage{pgf}  %Para realizar figures
\usepackage{xcolor} % Para los colores
\usepackage{enumerate}
\usepackage{graphicx}
\usepackage{array}
\usepackage{cancel}
\usepackage{amssymb}
\usepackage{hyperref}
\usepackage{tcolorbox}  %Cuadros de teoremas


\title[ \hspace{21mm} \insertframenumber \ de \inserttotalframenumber ]
{Momentos de una variable aleatoria. \\Esperanza, Varianza y Covarianza}

\subtitle
{Probabilidad, procesos aleatorios e inferencia}

\author[]
{Ana Maritza Bello Yáñez}


\institute[Instituto Polit\'ecnico Nacional]
{
  \inst{1}
  Centro de Investigaci\'on en Computaci\'on
  }

\date[Short Occasion]
{\today}

\keywords{}

\begin{document}

\begin{frame}
  \titlepage
\end{frame}

\begin{frame}{Momentos de una v.a.}
Los momentos de una variable aleatoria son cantidades que reflejan ciertas
características de un conjunto de datos numérico y se definen de la
siguiente manera:

\begin{itemize}
\item $E(X)$ también se le denomina como primer momento. También se puede
denotar como $\bar{X}$ o $\mu{X}$, que implican una \textit{operación de
promedio} sobre la variable $X$.
  \item $E(X^2)$ es el segundo momento.
  \item $E(X^3)$ es el tercer momento.
  ...
  \item $E(X^n)$ es el $n$-ésimo momento.
\end{itemize}

\end{frame}


\begin{frame}{Esperanza}
La esperanza es un número que representa el valor promedio que toma una variable
aleatoria.\\

Sea $X$ una variable aleatoria discreta con función de probabilidad $f(x)$, que
puede tomar los valores $x_i=(i=1,2,...)$.\\

Entonces, se define la esperanza de $X$, como la suma sobre todos los posibles
valores que la variable aleatoria puede tomar de $X$ por la probabilidad $P(X=x_i)=f_{X}(x_i)$.

\begin{equation}
  \bar{X} = E(X) = \sum_{i=1}^{n} x_i \cdot P(X=x_i)
\end{equation}

Siempre y cuando esta suma sea absolutamente convergente.

\begin{equation}
  \sum_{x} |x| f(x) < \infty
\end{equation}

\end{frame}

\begin{frame}{Esperanza}
  La esperanza también es llamada el primer momento de una variable aleatoria.
\end{frame}

\begin{frame}{}
\end{frame}

\begin{frame}{}
\end{frame}

\begin{frame}{}
\end{frame}

\end{document}