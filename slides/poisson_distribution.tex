\documentclass{beamer}

\mode<presentation>{
  \usetheme{Warsaw}
}


\usepackage[spanish,es-tabla,es-nodecimaldot]{babel}
\usepackage{tikz}
\usepackage{pgf}  %Para realizar figures
\usepackage{xcolor} % Para los colores
\usepackage{enumerate}
\usepackage{graphicx}
\usepackage{array}
\usepackage{cancel}
\usepackage{amssymb}
\usepackage{hyperref}
\usepackage{tcolorbox}  %Cuadros de teoremas


\title[ \hspace{21mm} \insertframenumber \ de \inserttotalframenumber ]
{Distribución de Poisson}

\subtitle
{Probabilidad, procesos aleatorios e inferencia}

\author[]
{Ana Maritza Bello Yáñez}


\institute[Instituto Polit\'ecnico Nacional]
{
  \inst{1}
  Centro de Investigaci\'on en Computaci\'on
  }

\date[Short Occasion]
{\today}

\keywords{}

\begin{document}



\begin{frame}
  \titlepage
\end{frame}

\begin{frame}{Distribución de Poisson}
  \begin{block}{}
    Esta distribución es útil cuando se quiere estudiar la ocurrencia de eventos por
    unidad de tiempo:\\
    \vfill
    errores/mes, quejas/semana, defectos/día. \\
    \vfill
    Para su aplicación, la probabilidad de ocurrencia del evento debe ser constante
    en tiempo o espacio y debe haber independencia de ocurrencia de eventos.
  \end{block}
\end{frame}

\begin{frame}{Distribución de Poisson}
  \begin{block}{Definición}
    También se puede usar como una aproximación de la distribución binomial, esto es
    cuando:

    $n \to \infty $ y $p \to 0 $

    De manera que el promedio $\lambda = np$ se hace constante.
  \end{block}

  \begin{block}{}
    La expresión binomial de la función de densidad de probabilidad para tales
    sucesos tiende a la siguiente forma simplificada:

    \begin{equation}
      P(x)=\frac{\lambda ^ x}{x!}e^{-\lambda}
    \end{equation}
  \end{block}
\end{frame}


\begin{frame}{Distribución binomial}
  \begin{block}{Retomando la distribución binomial...}
La variable aleatoria $X$ representa el número de éxitos con probabilidad $p$,
obtenidos en $n$ intentos.

La función de densidad de probabilidad está dada por:

    \begin{equation}
      P(X=x) = f(x) = \binom{n}{x} p^x q^{n-x}
    \end{equation}

    Donde $\binom{n}{x}$ es la combinatoria de $n$ en $x$.

    La función de distribución está dada por:

    \begin{equation}
      P(X \leq x) = \sum_{i}^{} f(x_i)
    \end{equation}

  \end{block}
\end{frame}


\begin{frame}{Distribución Binomial}
  Para una v.a. discreta binomial podemos definir tres parámetros importantes:

  \begin{enumerate}
    \item Valor esperado:
    \begin{equation}
      \mu = E(x) = \sum_{x}^{} x f(x) = \sum_{x}^{} x \binom{n}{x} p^x (1-p)^(n-x) = np
    \end{equation}

    \item Varianza:
    \begin{equation}
      \sigma^2 = \sum_{x} x^2 \binom{n}{x} p^x (1-p)^{n-x} - (\sum_{x} \binom{n}{x} p^x (1-p)^{n-x})^2 = np(1-p)
    \end{equation}

    \item Desviación estándar:
    \begin{equation}
      \sigma = \sqrt{\sigma^2}
    \end{equation}

  \end{enumerate}

\end{frame}


\begin{frame}{Aproximación de la dist. Binomial a Poisson}
  La distribución binomial depende de 3 parámetros:

  \begin{equation}
    f(x,n,p) = \binom{n}{x} p^x (1-p)^{n-x}
  \end{equation}

\end{frame}


\begin{frame}{}
\end{frame}



\end{document}

