
\section{Binomio de Newton}

El binomio de Newton consiste en una fórmula que permite obtener los
coeficientes de un término enésimo de un binomio elevado a un exponente
determinado.

La fórmula matemática del binomio de Newton es la siguiente:

\begin{equation}
    (a+b)^n = \sum_{k=0}^{n} \binom{n}{k} a^{n-k} b^k
\end{equation}

\subsection{Desarrollo del binomio de Newton de la potencia 1 a la 10}

\begin{landscape}
	\begin{equation}
		\begin{split}
			(a+b)^0 &   = 1 \\
			(a+b)^1 &   = a+b \\
			(a+b)^2 &   = a^2+2ab+b^2 \\
			(a+b)^3 &   = a^3+3a^2b+3ab^2+b^3 \\
			(a+b)^4 &   = a^4+4a^3b+6a^2b^2+4ab^3+b^4 \\
			(a+b)^5 &   = a^5+5a^4b+10a^3b^2+10a^2b^3+5ab^4+b^5\\
			(a+b)^6 &   = a^6+6a^5b+15a^4b^2+20a^3b^3+15a^2b^4+6ab^5+b^6\\
			(a+b)^7 &   = a^7+7a^6b+21a^5b^2+35a^4b^3+35a^3b^4+21ba^2b^5+21ab^6+b^7 \\
			(a+b)^8 &   = a^8+8a^7b+28a^6b^2+56a^5b^3+70a^4b^4+56a^3b^5+28a^2b^6+8ab^7+b^8 \\
			(a+b)^9 &   = a^9+9a^8b+36a^7b^2+84a^6b^3+126a^5b^4+126a^4b^5+84a^3b^6+36a^2b^7+9ab^8+b \\
			(a+b)^{10} &    = a^{10}+10a^9b+45a^8b^2+120a^7b^3+210a^6b^4+252a^5b^5+210a^4b^6+120a^3b^7+45a^2b^8+10ab^9+b^{10} \\
		\end{split}
	\end{equation}
\end{landscape}
