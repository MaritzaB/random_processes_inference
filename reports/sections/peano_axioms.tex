\section{Axiomas de Peano}\label{sect: Axiomas de Peano}
Los axiomas de Peano son un sistema de postulados para la aritmética ideados por el matem\'atico Giuseppe Peano en el siglo XIX para definir los n\'umeros naturales.

Estos axiomas fueron publicados en 1889 en un art\'iculo denominado \textit{Aritmetices principia, nova methodo exposita}.

Los cinco axiomas son los siguientes:
\begin{enumerate}
	\item $N(1)$: El \textbf{1} es un n\'umero natural.
	\item $\forall x(N(x)\to N(x'))$:Todo n\'umero natural, tiene un sucesor \textbf{$n^*$}
	\item $\neg \exists x(N(x)\land 1=x')$: El \textbf{1} no es el sucesor de ning\'un n\'umero natural.
	\item $\forall x \forall y ((N(x)\land N(y)\land x'=y')\to x=y)$: Si hay dos n\'umeros naturales n y m con el mismo sucesor, entonces n y m son el mismo n\'umero natural.
	\item $\left(\phi(1)\land \forall x(\phi(x)\to \phi(x'))\right) \to \forall x \phi(x)$: Si el 1 pertenece a un conjunto de n\'umeros naturales, y dado un elemento cualquiera, el sucesor tambi\'en pertenece al conjunto. Entonces, todos los n\'umeros naturales pertencen a ese conjunto.
\end{enumerate}

Cabe mencionar que el axioma 5 es el principio de la inducci\'on matem\'atica.

Existe un debate para determinar si el $0$ pertenece o no al conjunto de los n\'umeros naturales. Sin embargo, se tomar\'a en cuenta dependiendo de la aplicaci\'on. Es por esa raz\'on que los 5 axiomas existen en la versi\'on donde el $0$ es el n\'umero natural. As\'i, los axiomas 1 al 5, donde se indique el n\'umero 1, se cambia por el 0.

Adem\'as de sus 5 axiomas, Peano define la suma y la multiplicaci\'on:
\begin{itemize}
	\item suma
	\begin{eqnarray*}
		\forall n(n+1=n')\\
		\forall n\forall m(n+m'=(n+m)')
	\end{eqnarray*}
	\item Multiplicaci\'on
	\begin{eqnarray*}
		\forall n(nx1=n)\\
		\forall n \forall m(nxm'=(nxm)+n)
	\end{eqnarray*}
\end{itemize}
De igual manera, cuando interviene el 0, \'este se colocar\'a en lugar del 1.