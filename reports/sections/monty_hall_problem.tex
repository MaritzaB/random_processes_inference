\subsection{El problema de Monty Hall}

El problema de \textit{Monty Hall} o paradoja del presentador es un problema de
probailidad condicional basado en el concurso de televisi\'on estadounidense
\textit{Let's Make a Deal}. El problema fue bautizado as\'i por el nombre del
presentador de dicho programa.

En 1975, el matem\'atico Steve Selvin resuelve este problema y es publicado en
la revista \textit{American Statistician}. Posteriormente fue popularizado por
Marilyn vos Savant en la revista \textit{Parade Magazine}, en 1990. El problema
dice as\'i:

\textbf{Planteamiento del problema:}

Hay tres puertas. Detr\'as de una de ellas, hay un premio. Detr\'as de cada una
de las dos puertas restantes hay una cabra.

Cada puerta tiene la misma probabilidad de contener el premio.

Se te da a seleccionar una de las tres puertas como una oportunidad de ganar el
premio. Sin embargo, una vez que se ha seleccionado la puerta, el presentador
destapa una de ellas y te muestra que detrás hab\'ia una cabra.Una vez dada esa
informaci\'on, el presentador te pregunta si deseas cambiar de puerta o
conservar tu elecci\'on.

La pregunta crucial es:

Dado que el presentador ha destapado una puerta y mostrado que no estaba el
premio, ¿es m\'as conveniente cambiar mi opci\'on o conservar mi elecci\'on de
la puerta inicial?

%\begin{figure}[h]
%	\centering
%	\includegraphics[scale=0.5]{Images/Monty Hall}
%	\caption{Imagen ilustratuiva del problema de Monty Hall}
%\end{figure}

\subsection{Soluci\'on:}
Este problema es un ejercicios relativamente simple de probabilidad condicional.
Sin embargo, se debe tener cuidado, pues es un problema particularmente confuso.
Se abrodará desde dos enfoques.

Dos observaciones son cruciales para entender el problema:

\begin{itemize}
\item El presentador conoce la puerta donde se encuentra el premio y obviamente
no revelará esa puerta. 
\item Monty revelar\'a cualquiera de las otras dos puertas de forma aleatoria y
con con igual probabilidad de que esto ocurra.
\end{itemize}
Supongase que selecciona la puerta 1. Entonces pueden ocurrir tres casos:
\begin{enumerate}
\item El premio está en la puerta 1. Monty abre la puerta 2 o la 3. Si se cambia
la elecci\'on, se pierde. Si se conserva la elecci\'on, ganas.
\item El premio está en la puerta 2. Monty abre la puerta 3. Si se cambia de
elecci\'on, se gana. Si se conserva la elecci\'on, pierdes.
\item El premio está en la pierta 3. Monty abre la puerta 2. Si se cambia de
elecci\'on, se gana. Si se conserva la elecci\'on, pierdes.
\end{enumerate}
Se debe notar que al cambiar de elecci\'on se gana en 2 de los 3 casos; mientras
que al conservar la elecci\'on, se gana s\'olo en 1 de los 3 casos. Por lo
tanto, la probabilidad de ganar si se cambia de puerta, es de $\frac{2}{3}$. Por
otro lado, la probabilidad de ganar con la elecci\'on inicial es de
$\frac{1}{3}$.

Conclusi\'on: ¡Cambia de puerta!
