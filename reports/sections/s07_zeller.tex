\subsection{Congruencia de Zeller}

La congruencia de Zeller es un algoritmo que permite obtener, a partir de una
fecha, el día de la semana que le corresponde.

Se atribuye su creación a Julius Christian Johannes Zeller, un sacerdote
protestante alemán que vivió en el siglo XIX.

Zeller observó que existía una dependencia entre las fechas del calendario
gregoriano y el día de la semana que les correspondía. A raíz de esa
observación, obtuvo (se dice que por tanteo), esta fórmula, en apariencia
mágica, que lleva su nombre.

La fórmula en sí es muy sencilla, y se basa en algunas operaciones de aritmética
modular (el resto, también llamado módulo, de las divisiones)

Es necesario tener en cuenta que la fórmula presentada a continuación es válida
sólo para el calendario gregoriano, promulgado por el papa Gregorio XIII en
1582, pero adoptado en distintas fechas en cada país.

Para el caledario gregoriano la congruencia de Zeller es:

\begin{equation}
    h = (q + [\frac{13(m+1)}{5}] + K [\frac{K}{4}] + [\frac{J}{4}] - 2J ) \text{ mod } 7
\end{equation}

Para el calendario juliano es:

\begin{equation}
    h = (q + [\frac{13(m+1)}{5}] + K [\frac{K}{4}] + 5 - J ) \text{ mod } 7
\end{equation}

Donde:

\begin{itemize}
    \item $h$ es el día de la semana.
    \item $q$ es el día del mes.
    \item $m$ es el mes.
    \item $K$ el año del siglo (año mod 100)
    \item $J$ es el siglo de base cero.
\end{itemize}

Veáse código en Sec. (\ref{sec:zeller})
