
\section*{Experimentos deterministas vs aleatorios}


\begin{tcolorbox}[colback=blue!5!white,colframe=blue!60!black,title=Definición: Observación]
    Cualquier registro de informaci\'on, ya sea num\'erico o categ\'orico.
\end{tcolorbox}

Ejemplos: 

Los n\'umeros 2, 0, 1 y 2, que representan el n\'umero de accidentes que
ocurrieron cada mes, de enero a abril, durante el a\~no pasado, constituyen un
conjunto de observaciones. 

Lo mismo ocurre con los datos categ\'oricos N, D, N, N y D, que representan los
art\'iculos defectuosos o no defectuosos cuando se inspeccionan cinco
art\'iculos y se registran como observaciones \cite{walpole2012probabilidad}.

\begin{tcolorbox}[colback=blue!5!white,colframe=blue!60!black,title=Definición: Experimento]
    Cualquier proceso que genere un conjunto de datos.    
\end{tcolorbox}

Un ejemplo simple de experimento estad\'istico es el lanzamiento de una moneda
al aire. En tal experimento s\'olo hay dos resultados posibles: águila o sol.
\cite{walpole2012probabilidad}.

\begin{tcolorbox}[colback=blue!5!white,colframe=blue!60!black,title=Definición: Tipos de experimentos]

    \textbf{Experimento determinista.}
    Un experimento determinista es aquel que produce el mismo resultado cuando se le
    repite bajo las mismas condiciones.

    \tcblower

    \textbf{Experimento aleatorio.}
    Un experimento aleatorio es aquel que, cuando se le repite bajo las mismas
    condiciones, el resultado que se observa no siempre es el mismo y tampoco es
    predecible.
\end{tcolorbox}

