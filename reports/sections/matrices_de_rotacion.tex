
\section*{Matrices de rotaci\'on}
Una matriz de rotaci\'on es una matriz que representa una rotaci\'on en el
espacio euclidiano. Las matrices de rotaci\'on pueden ser en dos o tres
dimensiones.
	
Por ejemplo, si la matriz de rotaci\'on es de dos dimensiones y se busca rotar
en sentido horario, la matriz tendrá la siguiente forma:

\begin{equation}
	R(\theta)=
	\begin{bmatrix}
		\cos \theta & -\sin \theta\\
		\sin \theta & \cos \theta
	\end{bmatrix}
\end{equation}

De esta manera, las nuevas coordenadas, una vez aplicando la rotación son de la
siguiente manera:

\begin{eqnarray}
	\left(
	\begin{matrix}
		x'\\
		y'
	\end{matrix}
	\right)
	=
	\begin{bmatrix}
		\cos \theta & -\sin \theta\\
		\sin \theta & \cos \theta
	\end{bmatrix}
	\left(
	\begin{matrix}
		x\\
		y
	\end{matrix}
	\right)\\
	x'=x\cos \theta -y\sin \theta\\
	y'=x\sin \theta + y \cos \theta
\end{eqnarray}

En caso de querer que la rotación sea antihoraria, se utiliza la siguiente
matriz:

\begin{equation}
	R(-\theta)=
	\begin{bmatrix}
		\cos \theta & \sin \theta\\
		-\sin \theta & \cos \theta
	\end{bmatrix}
\end{equation}

Si lo que se busca es realizar rotaciones en el espacio tridimensional, se deben
modificar las matrices de la siguiente manera:

\begin{equation}
	R_x(\theta)=
	\begin{bmatrix}
		1 & 0 & 0\\
		0 & \cos \theta & \sin \theta\\
		0 & -\sin \theta & \cos \theta
	\end{bmatrix}
\end{equation}

\begin{equation}
	R_y(\theta)=
	\begin{bmatrix}
		\cos \theta & 0 & \sin \theta\\
		0 & 1 & 0\\
		-\sin \theta & 0 & \cos \theta
	\end{bmatrix}
\end{equation}

\begin{equation}
	R_z(\theta)=
	\begin{bmatrix}
		\cos \theta & -\sin \theta & 0\\
		\sin \theta & \cos \theta & 0\\
		0 & 0 & 1
	\end{bmatrix}
\end{equation}
