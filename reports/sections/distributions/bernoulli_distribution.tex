\section{Distribución Bernoulli}

Un ensayo Bernoulli se define como aquel experimento aleatorio con únicamente
dos posibles resultados, llamados genéricamente: éxito y fracaso. Supondremos
que las probabilidades de estos resultados son $p$ y $1 - p$, respectivamente.
Si se define la variable aleatoria $X$ como aquella función que lleva el
resultado éxito al número 1 y el resultado fracaso al número 0, enton- ces
decimos que $X$ tiene una distribución Bernoulli con parámetro $p \in (0, 1)$ y
escribimos $X \sim Ber(p)$. La función de probabilidad se puede escribir de la
siguiente forma.

\begin{equation}
    f(x) = \left\lbrace
        \begin{array}{ll} 
            1-p & \text{ si } x = 0, \\
            p   & \text{ si } x=1, \\
            0   & \text{ en otro caso.}
        \end{array}\right.
\end{equation}

O bien de manera compacta,

\begin{equation}
    f(x) = \left\lbrace
        \begin{array}{ll} 
            p^x (1-p)^{1-x} & \text{ si } x = 0,1, \\
            0   & \text{ en otro caso.}
        \end{array}\right.
\end{equation}

En la realización de todo experimento aleatorio siempre es posible pregun-
tarse por la ocurrencia o no ocurrencia de un evento cualquiera. Este es el
esquema general donde surge esta distribución de probabilidad. La distri-
bución Bernoulli es sencilla, pero de muy amplia aplicación.

\subsection{Ejemplo de evento tipo Bernoulli}

Un jugador de baloncesto tiene un 80\% de acierto en tiros libres. Si tira tres
lanzamientos seguidos, ¿Cuál es la probabilidad de que acierte los tres?

Para un tiro:

Probabilidad de éxito $p=0.8$

Probabilidad de fracaso: $q = 1 - 0.8 = 0.2$

Para tres tiros:

Probabilidad de éxito $P(A \cap A \cap A) = p \cdot p \cdot p = 0.8 \cdot 0.8
\cdot 0.8 = 0.512$

Probabilidad de fracaso $P(\bar{A} \cap \bar{A} \cap \bar{A}) = q \cdot q \cdot q = 0.2 \cdot 0.2
\cdot 0.2 = 0.008$
