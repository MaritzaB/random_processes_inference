\section{Distribuci\'on exponencial}

Decimos que una variable aleatoria continua $X$ tiene distribución exponencial
con parámetro $\lambda > 0$, y escribimos $X \sim \text{exp}(\lambda)$, cuando
su función de densidad es:

\begin{equation}
    f(x) = \left\lbrace \begin{array}{ll}
        \lambda e^{-\lambda x} & \text{ si } x>0, \\
        0 & \text{ en otro caso.}
    \end{array}\right.
\end{equation}

La gráfica de esta función, cuando el parámetro $\lambda$ toma el valor
particular 3. La correspondiente función de distribución aparece a su derecha.
Es muy sencillo verificar que la función $f(x)$ arriba definida, es
efectivamente una función de densidad para cualquier valor del parámetro
$\lambda > 0$. Se trata pues de una variable aleatoria continua con valores en
el intervalo $0, \infty$. Esta distribución se usa para modelar tiempos de
espera para la ocurrencia de un cierto evento.

