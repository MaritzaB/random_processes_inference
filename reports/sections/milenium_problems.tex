
\section{Problemas del milenio}

\subsection{Problemas P vs NP}
Se trata del primero de los problemas del milenio de las matemáticas aplicadas,
y alude más concretamente al campo de la complejidad computacional, dentro del
ámbito de la informática. Su planteamiento se remonta a los años 70, cuando
además de por Alan Turing, fue planteado paralelamente por los programadores
Stephen Cook y Leonid Levin.

A grandes rasgos, el problema P frente a NP busca clasificar los problemas en
dos clases: los que pueden ser resueltos con una cantidad determinada de
recursos, y aquellos que no. Los recursos a los que nos referiríamos serían el
tiempo empleado para realizar los cálculos, y la memoria requerida (no olvidemos
que nos encontramos en el campo de la informática computacional) para procesar
los datos del problema.

Por ejemplo, los problemas P serían de fácil resolución para las computadoras, es
decir sus soluciones serían fáciles de encontrar en una cantidad razonable de
tiempo. En los problemas NP, por el contrario, la solución podría ser muy
difícil de encontrar, o quizá requeriría una gran cantidad de recursos (miles de
años), para ser hallada, aunque una vez encontrada la solución sería fácil de
comprobar. Un ejemplo muy ilustrativo de este tipo de problemas podría ser la
resolución de un puzle, donde encontrar el orden de las piezas, podría requerir
gran cantidad de recursos, pero una vez terminado el puzzle, la solución
correcta saltaría a la vista, y sería fácil de comprobar.

El problema P versus NP plantea si todos los problemas NP son también un
problema P. Si P es igual a NP, todos los problemas NP contendrían un atajo
oculto, que permitiría que los ordenadores encontrasen rápidamente soluciones
perfectas. Pero si P no es igual a NP, entonces no existen dichos atajos, lo que
demostraría que la potencia de resolución de problemas de los ordenadores es
limitada.

\subsection{Hipótesis de Riemann}

La hipótesis de Riemann fue formulada por primera vez por Bernhard Riemann en
1859, y por su relación con la distribución de los números primos en el conjunto
de los naturales, es uno de los problemas abiertos más importantes en la
matemática contemporánea. Riemann sugirió que la distribución de estos números
está estrechamente relacionada con el comportamiento de la llamada "función zeta
de Riemann", la cual tiene dos tipos de ceros: los ceros "triviales", que son
todos los números enteros pares y negativos; y los ceros "no triviales", cuya
parte real está siempre entre 0 y 1.

La hipótesis de Riemann afirma que todos los ceros no triviales de la función
zeta se encuentran en la recta $x = 1/2$. A día de hoy, más de diez billones de
ceros han sido calculados para la función $z$, todos alineados sobre la recta
crítica, los cuales corroboran la sospecha de Riemann. Sin embargo todavía nadie
aún ha podido demostrar en la actualidad que la función zeta no tenga ceros no
triviales fuera de dicha recta.

\subsection{La conjetura de Poincaré}

La conjetura de Poincaré es un problema topológico, establecido en 1904 por el
matemático francés Henri Poincaré. Se trataba de uno de los problemas de más
difícil resolución de los 7 problemas del milenio. Decimos "se trataba" por que
fue resuelto en el año 2006, convirtiéndose en el Teorema de Poincaré como fruto
del trabajo del matemático ruso Grigori Perelman, quien renunció a la cuantía
económica del premio.

El teorema sostiene que la esfera cuatridimensional, también llamada 3-esfera o
hiperesfera, es la única variedad compacta cuatridimensional en la que todo lazo
o círculo cerrado (1-esfera) se puede deformar (transformar) en un punto. Este
último enunciado es equivalente a decir que solo hay una variedad cerrada y
simplemente conexa de dimensión: la esfera cuatridimensional.

