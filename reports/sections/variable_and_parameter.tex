\pagebreak
\section{Definiciones}

\subsection{Parámetro}

Variable y parámetro son dos términos muy utilizados en matemáticas y física.
Una variable es una entidad que cambia con respecto a otra entidad. Un parámetro
es una entidad que se utiliza para conectar variables. Los conceptos de variable
y parámetro son muy importantes en campos como las matemáticas, la física, la
estadística, el análisis y cualquier otro campo que tenga usos de las
matemáticas.

Un parámetro es una cantidad que influye en la salida o el comportamiento de un
objeto matemático, pero se considera que se mantiene constante. Los parámetros
están estrechamente relacionados con las variables y, a veces, la diferencia es
solo una cuestión de perspectiva. Se considera que las variables cambian,
mientras que los parámetros generalmente no cambian o cambian más lentamente. En
algunos contextos, uno puede imaginarse realizando múltiples experimentos, donde
las variables cambian a través de cada experimento, pero los parámetros se
mantienen fijos durante cada experimento y solo cambian entre experimentos.

Un lugar donde aparecen los parámetros es dentro de las funciones. Por ejemplo,
una función podría ser una función cuadrática genérica como $f(x)=ax2+bx+c$.
Aquí, la variable $x$ se considera como la entrada de la función. Los símbolos
$a$, $b$ y $c$ son parámetros que determinan el comportamiento de la función $f$.
Para cada valor de los parámetros, obtenemos una función diferente

\subsection{Población y muestra}

\begin{tabular}{l|p{6cm}|p{6cm}}
                & \textbf{Población}    &   \textbf{Muestra}    \\
\hline

Definición  & 	Universo de elementos que se van a estudiar.    &   Selección de
una parte de la población que se va a ser sujeto de estudio.   \\

\hline

Características &   

\begin{itemize} 
    \item Se puede clasificar según la cantidad de individuos que la conforman. 
    \item Posee variables estadísticas. 
\end{itemize}
& \begin{itemize}
    \item Forma parte de la población: debería comprender entre 5\% y 10\% para ser más
efectiva. 
    \item Los elementos deben ser aleatorios. 
    \item Debe ser representativa de la población.
\end{itemize} \\
\hline

    Objetivos & Analizar los datos recabados referentes a las características comunes que
comparten los elementos con diversos propósitos. & Estudiar el comportamiento,
características, gustos o propiedades de una parte representativa de la
población. \\

\hline

Ejemplos & \begin{itemize} \item Las personas que habitan un país.
\item La cantidad de carros en una ciudad.
\item Los estudiantes de un país.
\end{itemize} & Para el estudio del desempeño de los estudiantes de cinco
universidades de una ciudad en una materia específica, se toma como muestra a
500 estudiantes aleatoriamente (100 de cada institución) que estén cursando el
mismo nivel para que la muestra sea representativa.

\end{tabular}


\subsection{Tipos de poblaciones}

\begin{itemize} 
    \item Población finita: es aquella que se puede contar y se pueden
estudiar con mayor facilidad a sus integrantes. Por ejemplo, la cantidad de
personas inscritas en un gimnasio. 
    \item Población infinita: son inmensas poblaciones
donde se hace muy difícil contabilizar a sus integrantes, por lo que suele
tomarse en cuenta solo una porción de ella a la hora de realizar un estudio,
seleccionando así una muestra. Por ejemplo, la cantidad de granos de arena en
una playa. 
    \item Población real: son grupos de integrantes tangibles. Por ejemplo, la
cantidad de animales en un zoológico.
\end{itemize}

\subsection{Tipos de muestras}

\textbf{Muestreo aleatorio}. Es una técnica que ofrece la misma posibilidad a los
elementos de ser seleccionados, por ser tomados al azar. Los tipos de muestreo
aleatorio son:

\begin{itemize}

    \item Muestreo aleatorio simple: los elementos se eligen de una lista al azar.
Funciona más eficazmente cuando el universo es reducido y homogéneo. 

\item Muestreo sistemático: el primer elemento se elige al azar y luego se
escogen a intervalos constantes los elementos restantes. 

\item Muestreo estratificado: se realiza dividiendo a la población en partes o
estratos que respondan a características establecidas y luego se eligen
aleatoriamente los individuos que se van a estudiar. 

\item Muestreo por conglomerado: la población se divide en grupos heterogéneos y
éstos a su vez se subdividen en grupos homogéneos con características comunes
para ser estudiados de acuerdo a lo requerido por el investigador.
\end{itemize}

\textbf{Muestreo no aleatorio o por selección intencionada} Se elige con
base en el manejo de información de los elementos a estudiar, por lo que la
representatividad de la muestra puede ser subjetiva. En este caso, se corre el
riesgo de que los resultados sean sesgados.

\section{Variable aleatoria}

Una variable es una entidad que cambia en un sistema dado. Considere un ejemplo
simple de una partícula en movimiento a través del espacio. En tal caso,
entidades como el tiempo, la distancia recorrida por la partícula, la dirección
de viaje se denominan variables.

Hay dos tipos principales de variables en un experimento dado. Estas se conocen
como variables independientes y variables dependientes. Las variables
independientes son las variables que se cambian o que son naturalmente
inmutables. En un ejemplo simple, si se mide la tensión de una banda elástica
mientras se cambia la tensión de la banda, la tensión es la variable dependiente
y la tensión es la variable independiente. La dependencia se aplica cuando la
variable dependiente es dependiente de la variable independiente.

Las variables también se pueden categorizar como variables discretas y variables
continuas. Esta clasificación se utiliza principalmente en matemáticas y
estadística. Los problemas se pueden categorizar dependiendo del número de
variables. El número de variables es muy importante en campos como las
ecuaciones diferenciales y la optimización.

Una variable aleatoria $X$ es una función definida sobre el espacio muestral
$\omega$ (conjunto de los resultados de un experimento aleatorio) que toma
valores en el cuerpo de los números reales $\mathbb{R}$, es decir:


$ X : \omega \rightarrow \mathbb{R} $


Una variable aleatoria puede ser discreta o continua según sea el rango de esta
aplicación.

%\subsection{Variable aleatoria discreta}

Una \textbf{variable aleatoria es discreta} si toma un número de valores finito
o infinito numerable. Estas variables corresponden a experimentos en los que se
cuenta el número de veces que ha ocurrido un suceso.

%\subsection{Variable aleatoria continua}

Una \textbf{variable aleatoria es continua} cuando puede tomar cualquier valor
de un intervalo real de la forma $(a, b)$,$(a,\infty)$,$(-\infty, b)$,$(-\infty,
+\infty)$ o uniones de ellos. Por ejemplo, el peso de una persona, el tiempo de
duración de un suceso, etc.

