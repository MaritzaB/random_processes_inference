\pagebreak

\section{Función de distribución}

Toda variable aleatoria tiene asociada una función llamada función de
distribución.

\begin{tcolorbox}[colback=gray!5!white,colframe=gray!60!black,title=Definición: Función de Distribución]

    La función de distribución de una variable aleatoria $X$ es la función 
    $F(x): \mathbb{R} \rightarrow [0,1]$, definida de la siguiente manera:

\begin{equation}
    F(x) = P(X \leq x)
\end{equation}

\end{tcolorbox}

Cuando sea necesario especificar la variable aleatoria en cuestión se escribe
$F_X (x)$, pero en general se omite el subíndice $X$ cuando no haya posibilidad
de confusión. El argumento de la función es la letra minúscula $x$ que puede
tomar cualquier valor real. Por razones obvias a esta función se le conoce
también con el nombre de función de acumulación de probabilidad, o función de
probabilidad acumulada. Observe que la función de distribución de una variable
aleatoria está definida sobre la totalidad del conjunto de números reales, y
siendo una probabilidad, toma valores en el intervalo $[0, 1]$.


La función de distribución es importante pues, contiene toda la información de
la variable aleatoria y la correspondiente medida de probabilidad.

\begin{tcolorbox}[colback=gray!5!white,colframe=gray!60!black,title=Propiedades: Función de Distribución]
    Sea $F(x)$ la función de distribución de una variable aleatoria. Entonces:

    \begin{enumerate}
        \item \begin{equation}
            \lim_{x \rightarrow +\infty} F(x) =1
        \end{equation}

        \item \begin{equation}
            \lim_{x \rightarrow -\infty} F(x) =1
        \end{equation}

        \item Si $x_1 \leq x_2$
        \begin{equation}
            F(x_1) \leq F(x_2)
        \end{equation}

        \item $F(x)$ es continua por la derecha, es decir,
        \begin{equation}
            F(x+) = F(x)
        \end{equation}
    \end{enumerate}

Donde $F(x+)$, es el límite por la derecha de la función $F$ en el punto $x$, y
$F(x=)$, es el límite por la izquierda de la función $F$ en el punto $x$.

\end{tcolorbox}

\subsection{Demostración de las propiedades de la función de distribución}

Una función $F(x) : \mathbb{R} \rightarrow [0,1]$ es llamada función de
distribución si cumple con las siguientes propiedades:

\begin{enumerate}
\item Sea $x_1, x_2, ...$ una sucesión cualquiera de números reales creciente a
infinito, y sean los eventos $A_n = (X \leq x_n)$. Entonces ${A_n : n \in
\mathbb{N}}$ es una sucesión de eventos creciente cuyo límite es $\omega$. Por
la propiedad de continuidad:

\begin{equation}
    \lim_{n \rightarrow \infty} F(x_n) = \lim_{n \rightarrow \infty} P(A_n) = P(\omega) = 1
\end{equation}

Dado que $\mathbb{R}$ es un espacio métrico, lo anterior implica que $F(x)$
converge a 1 cuando $x$ tiende a infinito.

\item Sea ahora $\{x_n : n \in \mathbb{N} \}$ una sucesión cualquiera de números
reales decreciente a menos infinito, y sean los eventos $A_n = (X \leq x_n )$.
Entonces $\{A_n : n \in \mathbb{N}\}$ es una sucesión de eventos decreciente al
conjunto vacío. Nuevamente por la propiedad de continuidad

\begin{equation}
    \lim_{n \rightarrow \infty} F(x_n) = \lim_{n \rightarrow \infty} P(A_n) = P(0) = 0
\end{equation}

Por lo tanto, $F(x)$ converge a cero cuando $x$ tiende a menos infinito.

\item Para $x_1 \leq x_2$:

\begin{equation}
    \begin{array}{ll}
        F(x)    & \leq F(x_1) + P(x_1 < X \leq x_2) \\
                & = P[(X \leq x_1) \cup (x_1 < X \leq x_2)] \\
                & = P(X \leq x_2) \\
                & = P(x_2) \\
    \end{array}
\end{equation}

\item Sea $x_1 , x_2 ,...$ una sucesión cualquiera de números reales no negativos
y decreciente a cero. Entonces:

\begin{equation}
    F (x + x_n ) = F (x) + P (x < X \leq x + x_n )
\end{equation}

en donde $A_n = (x < X \leq x + x_n )$ es una sucesión de eventos decreciente al
conjunto vacío. Por lo tanto $\lim_{n \rightarrow \infty} F(x + x_n ) = F (x)$.
Es decir $F (x+) = F (x)$.

\end{enumerate}

Por lo tanto basta dar una función de distribución específica para saber que
existe un cierto espacio de probabilidad y una variable aleatoria definida sobre
él y cuya función de distribución es la especificada. Este es el punto de vista
que a menudo se adopta en el estudio de las variables aleatorias, quedando un
espacio de probabilidad no especificado en el fondo como elemento base en todas
las consideraciones.


\subsection{Variable aleatoria discreta}

\begin{tcolorbox}[colback=gray!5!white,colframe=gray!60!black,title=Definición: Variable aleatoria discreta]
La variable aleatoria $X$ se llama discreta si su correspondiente función de
distribución $F(x)$ es una función constante por pedazos, Fig. (\ref{fig: discrete_function}). Sean $x_1, x_2 , ...$
los puntos de discontinuidad de $F(x)$. En cada uno de estos puntos el tamaño de
la discontinuidad es $P(X = x_i ) = F (x_i ) - F (x_i -) > 0$. A la función
$f(x)$ que indica estos incrementos se le llama función de probabilidad de $X$,
y se define como sigue:

\begin{equation}
    f(x) =
    \left\lbrace
    \begin{array}{ll} 
        P(X=x)  & \text{si } x = x_1, x_2, ... \\
        0       & \text{otro caso.}
    \end{array}\right.
\end{equation}

La función de distribución se reconstruye de la forma siguiente:

\begin{equation}
    F(x) = \sum_{u \leq x} f(u)
\end{equation}

\end{tcolorbox}

En este caso se dice también que la función de distribución es discreta, además
la función de probabilidad $f(x)$ siempre existe, y se le llama también función
de masa de probabilidad. También se acostumbra usar el término función de
densidad, como una analogía con el caso de variables aleatorias continuas
definidas más adelante. Observe que la función de probabilidad $f(x)$ es una
función no negativa que suma uno en el sentido $\sum_{i} f(x_i) = 1$.
Recíprocamente, toda función que cumpla las dos propiedades anteriores se le
llama función de probabilidad, sin que haya necesariamente una variable
aleatoria de por medio.

\begin{figure}[h!]
    \centering
\begin{tikzpicture}
    \begin{axis}[
        clip=false,
        jump mark left,
        ymin=0,ymax=1,
        xmin=0, xmax=5,
        discontinuous,
        table/create on use/cumulative distribution/.style={
            create col/expr={\pgfmathaccuma + \thisrow{f(x)}}
            }
        ]
        \addplot [red] table [y=cumulative distribution]{
            x f(x)
            0 1/15
            1 2/15
            2 1/5
            3 4/15
            4 1/3
            5 0
        };
    \end{axis}
\end{tikzpicture}
\label{fig: discrete_function}
\caption{Función de distribución discreta}
\end{figure}


\subsection{Variable aleatoria continua}

\begin{tcolorbox}[colback=gray!5!white,colframe=gray!60!black,title=Definición: Variable aleatoria continua]
La variable aleatoria continua $X$ con función de distribución $F(x)$ se llama
absolutamente continua, si existe una función no negativa e integrable $f$ tal que
para cualquier valor de $x$ se cumple que:

\begin{equation}
    F(x) = \int_{- \infty}^{x} f(x) dx.
    \label{eq:continous_distribution}
\end{equation}

En tal caso $f(x)$ se le llama función de densidad de $X$.

\end{tcolorbox}

Aún cuando exista una función no negativa e integrable $f$ que cumpla con lo
anterior, ésta puede no ser única, pues basta modificarla en un punto para que
sea ligeramente distinta pero aún así lo seguirá cumpliendo. A pesar de ello,
nos referiremos a la función de densidad como si ésta fuera única, y ello se
justifica por el hecho de que las probabilidades son las mismas, ya sea usando
una función de densidad o modificaciones de ella que cumplan. Es claro que la
función de densidad de una variable aleatoria absolutamente continua es no
negativa y su integral sobre toda la recta real es uno. Recíprocamente, toda
función $f(x)$ no negativa que integre uno en $\mathbb{R}$ se llama función de
densidad. Si $X$ es absolutamente continua con función de distribución $F(x)$ y
función de densidad continua $f(x)$, entonces el teorema fundamental del cálculo
establece que, a partir de la Ec. \eqref{eq:continous_distribution}, $F'(x) = f
(x)$. Además, la probabilidad de que $X$ tome un valor en el intervalo $(a, b)$
es el área bajo la función de densidad sobre dicho intervalo, esto se ilustra en
la Fig. \ref{fig:continous_distribution}.

% GAUSSIANs: basic properties

\begin{figure}[h!]
    \centering
    \begin{tikzpicture}
    \message{Cumulative probability^^J}
    
    \def\B{11};
    \def\Bs{3.0};
    \def\xmax{\B+3.2*\Bs};
    \def\ymin{{-0.1*gauss(\B,\B,\Bs)}};
    \def\h{0.07*gauss(\B,\B,\Bs)};
    \def\a{\B-0.8*\Bs};
    
    \begin{axis}[every axis plot post/.append style={
                 mark=none,domain={-0.05*(\xmax)}:{1.08*\xmax},samples=\N,smooth},
                 xmin={-0.1*(\xmax)}, xmax=\xmax,
                 ymin=\ymin, ymax={1.1*gauss(\B,\B,\Bs)},
                 axis lines=middle,
                 axis line style=thick,
                 enlargelimits=upper, % extend the axes a bit to the right and top
                 ticks=none,
                 xlabel=$x$,
                 every axis x label/.style={at={(current axis.right of origin)},anchor=north},
                 width=0.7*\textwidth, height=0.55*\textwidth,
                 y=700pt,
                 clip=false
                ]
      
      % PLOTS
      \addplot[red,thick,name path=B] {gauss(x,\B,\Bs)};
      
      % FILL
      \path[name path=xaxis]
        (0,0) -- (\pgfkeysvalueof{/pgfplots/xmax},0);
      \addplot[red!25] fill between[of=xaxis and B, soft clip={domain=-1:{\a}}];
      
      % LINES
      \addplot[mydarkred,dashed,thick]
        coordinates {({\a},{1.2*gauss(\a,\B,\Bs)}) ({\a},{-\h})}
        node[mydarkred,below=-2pt] {$a$};
      \node[mydarkred,above right] at ({\B+\Bs},{1.2*gauss(\B+\Bs,\B,\Bs)}) {$f(x)$};
      \node[red!60!black,above left] at ({0.85*(\a)},{1.0*gauss(0.85*(\a),\B,\Bs)}) {};
      
    \end{axis}
  \end{tikzpicture}
  \label{fig:continous_distribution}
  \caption{Función de distribución continua}
\end{figure}