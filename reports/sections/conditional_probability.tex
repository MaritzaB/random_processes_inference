
\section{Probabilidad Condicional}

La probabilidad condicional nos provee una forma de razonar acerca de la salida
o resultado de un experimento, basado en información parcial. Algunos ejemplos
de estos experimentos podrían ser los siguientes:

\begin{itemize}
\item En un experimento en el cual tiramos dos dados sucesivamente, te dicen que
la suma de los dados es 9. ¿Qué tan probable es que el primer dado haya caído 6?

\item En un juego de adivinanzas de palabras, la primera letra de la palabra es
\textit{t}. ¿Cuál es la probabilidad de que la siguiente palabra sea \textit{h}?

\item ¿Qué  tan probable es que una persona tenga cierta enfermedad dado que su
examen médico dió negativo?
\end{itemize}

En términos más precisos, dado un experimento, un espacio de muestreo y una ley
de probabilidad, supongamos que sabemos que la salida está dentro de un evento
dado $B$. Deseamos cuantificar la probabilidad de que la salida pertenece a
algún otro evento $A$. Así, podemos construir una nueva probabilidad que tome en
cuenta el conocimiento disponiblw: una ley de probabilidad para cuaquier evento
$A$. Especificamente, la probabilidad condicional de $A$ dado $B$, denotado por
$P(A|B)$.
