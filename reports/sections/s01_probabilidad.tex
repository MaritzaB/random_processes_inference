\section{Probabilidad y conteo}

\subsection{Definici\'on cl\'asica}

La definici\'on cl\'asica de la probabilidad \cite{faber2012statistics} de un
evento $A$, puede ser formulada de la siguiente manera:

\begin{theorem}{Probabilidad}{probabilidad}
    \begin{equation}
        P(A)= \frac{n_A}{n_{total}}
    \end{equation}
    \label{eq:probabilidad}
\end{theorem}

Donde $n_A$ es el n\'umero de formas igualmente probables por las cuales un
experimento puede conducir a A; $n_{total}$ n\'umero total de formas igualmente
probables en el experimento.

- En la definici\'on cl\'asica el experimento no necesariamente se lleva a cabo
ya que la respuesta es conocida de antemano.

- La teor\'ia cl\'asica no da soluci\'on a menos de que todas las formas
igualmente posibles puedan ser derivadas anal\'iticamente.

La definición clásica desprende una serie de propiedades:

\begin{tcolorbox}[colback=gray!5!white,colframe=gray!60!black,title=Axiomas de la probabilidad]

\begin{itemize}
\item $P(A) \geq 0$. La probabilida se he definido como el cociente del número
de casos favorables al suceso $A$ y el número de casos posibles, por lo que el
cociente no puede ser negativo, y su límite inferior 0 se alcanza cuando el
número de casos favorables sea nulo.

\item $P(A) \leq 1$. El número de casos favorables nunca puede ser mayor que el
número total de casos, a lo sumo igual.


Estas dos propiedade conducen  que la probabilidad de un suceso esté acotada, 
\begin{equation}
    0 \leq P(S) \leq 1
    \label{eq:acotadaProb}
\end{equation}

\item Si $A$ y $B$ son conjuntos disjuntos, entonces:

\begin{equation}
    P(A \cup B) = P(A) + P(B)
    \label{eq:probaAunionB}
\end{equation}

\item $P(0)=\emptyset$

\item \textbf{Regla del complemento}

Sea $S$ el espacio muestral de un experimento $E$. Si $A$ es un evento, es decir
$A \subseteq S$, entonces:

\begin{equation}
    P(\bar{A}) = 1 - P(A)
    \label{eq:complementRule}
\end{equation}

\end{itemize}
\end{tcolorbox}
