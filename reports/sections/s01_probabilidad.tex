\section*{Probabilidad y conteo}

\subsection*{Definici\'on cl\'asica}

La definici\'on cl\'asica de la probabilidad \cite{faber2012statistics} de un
evento $A$, puede ser formulada de la siguiente manera:

\begin{theorem}{Probabilidad}{probabilidad}
    \begin{equation}
        P(A)= \frac{n_A}{n_(total)}
    \end{equation}
\end{theorem}

Donde $n_A$ es el n\'umero de formas igualmente probables por las cuales un
experimento puede conducir a A; $n_(total)$ n\'umero total de formas igualmente
probables en el experimento.

- En la definici\'on cl\'asica el experimento no necesariamente se lleva a cabo
ya que la respuesta es conocida de antemano.

- La teor\'ia cl\'asica no da soluci\'on a menos de que todas las formas
igualmente posibles puedan ser derivadas anal\'iticamente.
