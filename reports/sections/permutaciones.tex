
\subsection{Permutaciones}

\begin{tcolorbox}[colback=gray!5!white,colframe=gray!60!black,title=Definición: Permutación]
	Una \textbf{permutaci\'on} de un conjunto es un arreglo u ordenamiento de sus
	elementos sin repeticiones ni omisiones.
	
	Si el conjunto está ordenado, el proceso de reordenar sus elementos, es una permutación.
	\label{Permutaciones_definition}
\end{tcolorbox}

Para saber el n\'umero de permutaciones que existen de un conjunto se debe
conocer su cardinalidad. Tomemos como ejemplo, el conjunto de n\'umeros enteros
$A=\{1, 2, 3\}$. Para cualquier permutaci\'on, en la primera posici\'on pueden
colocarse cualquiera de los tres elementos; en la segunda posición se pueden
colocar dos posibles elementos; mientras que al final, se puede colocar una
posibilidad.

En general, para cualquier conjunto de elementos $X$ de cardinalidad $|n|$, en
la primera posici\'on se pueden colocar $n$ elementos, para la siguiente
posici\'on $n-1$. Así sucesivamente hasta las \'ultimas posiciones, donde se
pueden colocar $3, 2$ y $1$ elementos. De esta manera, se dice que el n\'umero
de permutaciones es:

\begin{theorem}{Permutación}{permutacion}
	\begin{equation}
		P_n=n*(n-1)*(n-2)*...*3*2*1=n!
		\label{permutaciones_totales}
	\end{equation}
	
	Remarcando el hecho de que $0! = 1$.

\end{theorem}

\subsubsection{Permutaciones de $k$-elementos}

Sea un conjunto de elementos $X$ de cardinalidad $|n|$, como en el caso
anterior. Supongamos que deseamos tomar $k$ elementos del conjunto y contar las
diferentes formas en las que podríamos agarrarlos.

De primera instancia, podemos tomar cualquiera de los $n$ objetos como el primer
elemento. Posteriormente, solo hay $n-1$ elementos para la segunda opción y asi
sucesivamente hasta que llegamos al $k$-ésimo elemento. Lo que nos deja
$n-(k-1)$ opciones para la última opción.

Podemos partir de la Ec. (\ref{permutaciones_totales}) para hacer la permutación
de los $n$ elementos, sin embargo, tendríamos que ajustar la fórmula para que
escojamos solo $k$ de los $n$ elementos ($n-k$).
\begin{equation}
n(n-1)(n-2)...(n-(k-1))\overbrace{(n-k)(n-(k+1))...1}^{(n-k)!}
\end{equation}

Para quitar los elementos repetidos, dividimos la ecuación anterior entre el
número de elementos repetidos, de la siguiente manera:

\begin{equation}
 = \frac{ n (n-1) (n-2) ... (n-k+1) \overbrace{(n-k)(n-(k+1))...1}^{(n-k)!}}{\underbrace{(n-k)(n-k-1) ... 1}_{(n-k)!}}
\end{equation}

Lo que nos lleva a la Ec. (\ref{k-permutaciones}).

\begin{theorem}{Permutación}{permutacion}
El total de permutaciones de $n$ objetos distintos tomados de $k$ formas a la
vez, está dado por:

	\begin{equation}
		P(n,r) = \frac{n!}{(n-k)!}
		\label{k-permutaciones}
	\end{equation}
\end{theorem}


\subsubsection{Paridad de una permutación}

En una permutación, ocurre una \textbf{inversi\'on} cuando un elemento mayor
precede a un elemento menor. Para conocer el n\'umero de inversiones se siguen
los siguientes pasos:

\begin{itemize}
\item Tomar el primer elemento de la permutaci\'on
\item Contar los enteros menores a la derecha del elemento en cuesti\'on
\item Realizar los dos pasos anteriores para cada elemento de la permutaci\'on
\item Sumar el total de inversiones contadas para cada elemento
\end{itemize}

\textbf{Ejemplo}:

Se toma la permutación \textbf{$A=\{6,1,3,4,5,2\}$}\\

Primer elemento: $6$; menores a la derecha: $1,3,4,5,2$; n\'umero de
inversiones: \textbf{5}
Primer elemento: $1$; menores a la derecha: $\emptyset$; n\'umero de
inversiones: \textbf{0}
Primer elemento: $3$; menores a la derecha: $2$; n\'umero de inversiones:
\textbf{1}\\
Primer elemento: $4$; menores a la derecha: $2$; n\'umero de inversiones:
\textbf{1}\\
Primer elemento: $5$; menores a la derecha: $2$; n\'umero de inversiones:
\textbf{1}\\

Total de inversiones: \textbf{8}
\begin{tcolorbox}[colback=gray!5!white,colframe=gray!60!black,title=Definición: Tipos de permutaciones]
	\textbf{Permutaci\'on par:} aquella en la que el total de inversiones es un
	entero par.
	
	\tcblower

	\textbf{Permutaci\'on impar}: aquella en la que el total de inversiones es un
	entero impar.
	
\end{tcolorbox}
