\section{Criterios de divisibilidad}

Los criterios de divisibilidad son herramientas matem\'aticas que nos permiten
identificar por simple inspecci\'on si un n\'umero es divisible\footnote{Se dice
que $a$ es divisible por $b$ si al realizar la operaci\'on, el resultado es un
n\'umero entero y el residuo es 0} por otro.

Cuando se desea conocer la divisibilidad de un n\'umero por otro en forma
aritm\'etica (``calculando a mano''), se tienen distintas pruebas, enunciadas en
el siguiente teorema:


	\begin{itemize}
\item \textbf{Divisibilidad por 2:} Un n\'umero es divisible por 2 cuando
termina en cero o cifra par.
\item \textbf{Divisibilidad por 5:} Un n\'umero es divisible entre 5 cuando
termina en cero o cinco.
\item \textbf{Divisibilidad por 4:} Un n\'umero es divisible entre 4 cuando sus
dos \'ultimas cifras de la derecha son ceros o forman un m\'ultiplo de 4.
\item  \textbf{Divisibilidad por 3:} Un n\'umero es divisible entre 3, si la
suma de los valores absolutos de sus cifras es un m\'ultiplo de 3.
\item \textbf{Divisibilidad por 11:} Un n\'umero es divisible entre 11 cuando la
diferencia entre la suma de los valores absolutos de sus cifras de lugar impar y
la suma de los valores absolutos de sus cifras con lugar par, de derecha a
izquierda, es 0 o m\'ultiplo de 11.
\item \textbf{Divisibilidad por 7:} Un n\'umero es divisible entre 7 cuando
separando la primera cifra de la derecha, multiplic\'andola por 2, restando el
producto del resto de cifras que quedaron a la izquierda y as\'i sucesivamente,
da cero o un m\'ultiplo de 7.		
\item \textbf{Divisibilidad por 13:} Un n\'umero es divisible por 13 si, al
tomar la primera cifra de la derecha, multiplicarla por 9 y rest\'andola del
resto de cifras de la izquierda y as\'i sucesivamente, resulta 0 o un m\'ultiplo
de 13.
\item \textbf{Divisibilidad por 17:} Un n\'umero es divisible por 17 cuando,
tomando la primera cifra de la derecha, multiplic\'andola por 5 y rest\'andola
del resto de sus cifras de la izquierda y as\'i sucesivamente, resulta 0 o un
m\'ultiplo de 17.
	\end{itemize}

\pagebreak

Un n\'umero en sistema decimal se representa en sistema binario mediante una
cadena de 1's y 0's, a los que podemos denotar como alfabeto. Las palabras
\textbf{cadena} y \textbf{alfabeto} nos hacen pensar que es posible analizar un
n\'umero mediante un \textbf{aut\'omata finito}. Adem\'as, que los criterios de
divisibilidad pueden ser llevados acabo por un aut\'omata finito y, como \'estos
tienen una representaci\'on en grafos, entonces existe un grafo para cada
criterio de divisibilidad.

Consideremos el caso de la divisibilidad por 2. Es claro que cualquier n\'umero
dividido por 2 tiene dos posibles residuos: 1 si es impar y 0 si es par. Si los
estados del grafo son los posibles residuos de dividir cualquier n\'umero por
dos, entocnes hay dos estados: 0 y 1.

Los n\'umeros a evaluar, son cadenas de 1's y 0's. Por lo tanto, las
transiciones a cada estado dependen de qu\'e caracter se va ``leyendo''.
\begin{itemize}
\item Grafo de divisibilidad por 2:
	\begin{figure}[h]
		\centering
		\begin{tikzpicture}[node distance=2cm]
			% \draw [help lines] (-5,-5) grid (5,5);
\node (q0)[state, initial, accepting, initial text={Inicio}] at (0,0) {$0$};
\node (q1)[state, right=of q0] {$1$};
			\path[-stealth,thick]
(q0) edge [loop above] node {0} () (q0) edge [bend right] node [below] {1} (q1)
(q1) edge [bend right] node [above] {0} (q0) (q1) edge [loop below] node {1} ();
		\end{tikzpicture}
	\end{figure}
\item Grafo de divisibilidad por 3:
	\begin{figure}[h]
		\centering
		\begin{tikzpicture}[node distance=2cm]
			% \draw [help lines] (-5,-5) grid (5,5);
\node (q0)[state, initial, accepting, initial text={Inicio}] at (0,0) {$0$};
\node (q1)[state, right=of q0] {$1$}; \node (q2)[state, right=of q1] {$2$};
			
			\path[-stealth,thick]
(q0) edge [loop above] node {0} () (q0) edge [bend right] node [below] {1} (q1)
(q1) edge [bend right] node [above] {1} (q0) (q1) edge [bend right] node [below]
{0} (q2) (q2) edge [bend right] node [above] {0} (q1) (q2) edge [loop below]
node {1} ();
		\end{tikzpicture}
	\end{figure}
\pagebreak
\item Grafo de divisibilidad por 5:
	\begin{figure}[h]
		\centering
		\begin{tikzpicture}[node distance=2cm]
			% \draw [help lines] (-5,-5) grid (5,5);
\node (q0)[state, initial, accepting, initial text={Inicio}] at (0,0) {$0$};
\node (q1)[state, right=of q0] {$1$}; \node (q2)[state, right=of q1] {$2$};
\node (q3)[state, below=of q1] {$3$}; \node (q4)[state, below=of q2] {$4$};
			
			\path[-stealth,thick]
(q0) edge [loop above] node {0} () (q0) edge node [below] {1} (q1) (q1) edge
node [above] {0} (q2) (q1) edge node [right] {1} (q3) (q2) edge [bend right]
node [above] {1} (q0) (q2) edge node [right] {0} (q4) (q3) edge [bend left] node
[left] {0} (q1) (q3) edge node [right] {1} (q2) (q4) edge node [below] {0} (q3)
(q4) edge [loop below] node {1} ();
		\end{tikzpicture}
	\end{figure}

\item Grafo de divisibilidad por 7:
	\begin{figure}[h]
		\centering
		\begin{tikzpicture}[node distance=2cm]
			% \draw [help lines] (-5,-5) grid (5,5);
\node (q0)[state, initial, accepting, initial text={Inicio}] at (0,0) {$0$};
\node (q1)[state, right=of q0] {$1$}; \node (q2)[state, below=of q1] {$2$};
\node (q3)[state, right=of q1] {$3$}; \node (q4)[state, below=of q3] {$4$};
\node (q6)[state, right=of q3] {$6$}; \node (q5)[state, below=of q6] {$5$};
			
			\path[-stealth,thick]
(q0) edge [loop above] node {0} () (q0) edge node [above] {1} (q1) (q1) edge
node [above] {1} (q3) (q1) edge node [left] {0} (q2) (q2) edge [bend right] node
[below] {1} (q5) (q2) edge node [below] {0} (q4) (q3) edge node [above] {0} (q6)
(q3) edge [bend right] node [right=1cm] {1} (q0) (q4) edge node [above] {0} (q1)
(q4) edge [bend right] node [above] {1} (q2) (q5) edge node [above] {0} (q3)
(q5) edge node [above] {1} (q4) (q6) edge [loop above] node [above] {1} () (q6)
edge node [right] {0} (q5);
		\end{tikzpicture}
	\end{figure}
\end{itemize}