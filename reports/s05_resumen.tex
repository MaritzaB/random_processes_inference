\section{Resumen: La belleza y utilidad de las matem\'aticas}

Las matem\'aticas han impregnado todos los campos de la actividad cient\'ifica y
desempe\~nan un papel inestimable en la biolog\'ia, la f\'isica, la qu\'imica,
la econom\'ia, la sociolog\'ia y la ingenier\'ia.


El descubrimiento simult\'aneo ha ocurrido a menudo en la historia de las
matem\'aticas.

Por ejemplo el del c\'alculo del erudito ingl\'es Isaac Newton (1643-1727) y el
matem\'atico alem\'an Gottfried Wilhelm Leibniz (1646-1716). Esto nos hacen
preguntarnos por qu\'e se hicieron estos descubrimientos cient\'ificos en al
mismo tiempo por personas que trabajan de forma independiente. Para dar otro
ejemplo, los naturalistas brit\'anicos Charles Darwin (1809–1882) y Alfred
Wallace (1823–1913) desarrollaron la teor\'ia de la evoluci\'on de manera
independiente y simult\'anea. De manera similar, el matem\'atico h\'ungaro
J\'anos Bolyai (1802–1860) y el matem\'atico ruso Nikolai Lobachevsky
(1793–1856) parec\'ian haber desarrollado la geometr\'ia hiperb\'olica de forma
independiente y al mismo tiempo.

Lo m\'as probable es que tales descubrimientos simult\'aneos hayan ocurrido
porque el momento era propicio para tales descubrimientos, dado el conocimiento
acumulado por la humanidad en el momento en que se realizaron los
descubrimientos. A veces, dos cient\'ificos se sienten estimulados al leer la
misma investigaci\'on preliminar de uno de sus contempor\'aneos. Por otro lado,
los m\'isticos han sugerido que existe un significado m\'as profundo para tales
coincidencias.

En ocasiones, se han utilizado teor\'ias matem\'aticas para predecir fen\'omenos
que no se confirmaron hasta a\~nos despu\'es. Por ejemplo, las ecuaciones de
Maxwell, llamadas as\'i por el f\'isico James Clerk Maxwell, predijeron las
ondas de radio. Las ecuaciones de campo de Einstein sugirieron que la gravedad
doblar\'ia la luz y que el universo se est\'a expandiendo. El f\'isico Paul
Dirac se\~nal\'o una vez que las matem\'aticas abstractas que estudiamos ahora
nos dan una idea de la f\'isica en el futuro. De hecho, sus ecuaciones
predijeron la existencia de antimateria, que posteriormente fue descubierta. De
manera similar, el matem\'atico Nikolai Lobachevsky dijo que “no hay rama de las
matem\'aticas, por abstracta que sea, que alg\'un d\'ia no pueda aplicarse a los
fen\'omenos del mundo real”.
