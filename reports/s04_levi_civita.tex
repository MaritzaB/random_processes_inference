
\section{Convenio de suma de Einstein}

Se denomina notación de Einstein o notación indexada a la convención utilizada
para abreviar la escritura de las sumatorias, donde se suprime el término de la
sumatoria $(\sum)$. Este convenio fue introducido por Albert Einstein en 1916.

Dada una expresión lineal en $\mathbb{R^n}$ en la que se escriban todos sus
términos en forma explícita:

\begin{equation}
    u = u_1 + u_2 + ... = u_n
\end{equation}

Se puede escribir de la forma:

\begin{equation}
    u = \sum_{i=1}^{n} u_ix_i
\end{equation}

La notación de Einstein obtiene una expresión aún más condensada eliminando el
signo de la sumatoria y entendiendo que la expresión resultante en un índice
indica la suma sobre todos los posibles valores del mismo.

\begin{equation}
    u = u_ix_i
\end{equation}

\subsection{Tensor Levi-Civita}
Está definido por:

\begin{equation}
    \epsilon_{ijk}=
    \left\lbrace\begin{array}{clllr} 
        +1 & ijk, & kji, & jki & \text{Permutación cíclica o par}\\ 
        -1 & ikj, & kji, & jik & \text{Permutación anticíclica o impar}\\
        0  & iii, & jjj, & kkk & \text{Si se repiten}
    \end{array}\right.
\end{equation}

\subsection{Producto escalar de dos vectores}

También conocido como producto punto, es una operación algebraíca que toma dos
vectores y retorna un escalar. esta definido de la siguiente manera:

\begin{equation}
    \bar{A} \cdot \bar{B}= \bar{\abs{A}} \bar{\abs{B}} \cos \theta
    = \sum_{i=1}^{n} a_i b_i
\end{equation}

Que de acuerdo al convenio de suma de Einstein podemos escribir como:

\begin{equation}
    \bar{A} \cdot \bar{B} = a_i b_i
\end{equation}

\subsection{Producto vectorial de dos vectores}

EL producto vectorial o producto cruz de dos vectores es una operación binaria
entre dos vectores en un espacio tridimensional. El resultado es un vector
perpendicular a los vectores que se multiplican, y por lo tanto normal al plano
que los contiene.

Está definido de la siguiente manera:

\begin{equation}
    \bar{A} \times \bar{B}= \bar{\abs{A}} \bar{\abs{B}} \sin \theta \hat{r}
\end{equation}

Donde $\hat{r}$ es el vector unitario y ortogonal a los vectores $\bar{A}$ y
$\bar{B}$, y $\theta$ el ángulo entre $\bar{A}$ y $\bar{B}$.

Que mediante determinantes tenemos:

\begin{equation}
    \bar{A} \times \bar{B}=
    \begin{bmatrix}
        \hat{i} & \hat{j} & \hat{k}\\ 
         a_1 & a_1 & a_1\\ 
         b_1 & b_1 & b_1
    \end{bmatrix}
\end{equation}

Desarrollamos:

\begin{equation}
    \bar{A} \times \bar{B}= \hat{i}(a_2 b_3 - a_3 b_2) - \hat{j}(a_1 b_3 - a_3 b_1) + \hat{k}(a_1 b_2 - a_2 b_1)
\end{equation}

Desde un punto de vista tensorial el producto generalizado de $n$ vectores vendrá
dado por:

\begin{equation}
    (\bar{A} \times \bar{B})_{i} = \epsilon_{ijk} a_{j} b_{k}
\end{equation}

Desarrollando:

\begin{equation}
    \epsilon_{ijk} a_{jbk} =
    \begin{array}{ccccc}
        \cancel{\epsilon_{111} a_1 b_1} & + & \cancel{\epsilon_{112} a_1 b_2}   & + & \cancel{\epsilon_{112} a_1 b_3}   \\
        \cancel{\epsilon_{121} a_2 b_1} & + & \cancel{\epsilon_{122} a_2 b_2}   & + & \epsilon_{123} a_2 b_3            \\
        \cancel{\epsilon_{131} a_3 b_1} & + & \epsilon_{132} a_3 b_2            & + & \cancel{\epsilon_{133} a_3 b_3}
    \end{array}
    = a_2 b_3 - a_3 b_2
\end{equation}
