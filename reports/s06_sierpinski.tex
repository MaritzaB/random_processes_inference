
\section{Tri\'angulo de Sierpinski}

El matem\'atico polaco Waclav Sierpinski (1882-1969), construy\'o este fractal
en 1919 del modo siguiente: tom\'o un tri\'angulo equil\'atero, uni\'o los
puntos medios de los lados y form\'o cuatro tri\'angulos interiores: tres
tri\'angulos equil\'ateros sombreados y un hueco que es otro tri\'angulo
equil\'atero. Repiti\'o el proceso en cada uno de los tri\'angulos sombreados, y
sigui\'o hasta el infinito el proceso en los tres tri\'angulos restantes como el
primero.


\begin{figure}[h]
    \centering
    \usetikzlibrary{lindenmayersystems}
\def\trianglewidth{3cm} \pgfdeclarelindenmayersystem{Sierpinski triangle}{
        \symbol{X}{\pgflsystemdrawforward}
        \symbol{Y}{\pgflsystemdrawforward}
        \rule{X -> X-Y+X+Y-X}
        \rule{Y -> YY}
} \foreach \level in {1,3,5}{ \tikzset{
l-system={step=\trianglewidth/(2^\level), order=\level, angle=-120} }

        \begin{tikzpicture}
\fill [black] (0,0) -- ++(0:\trianglewidth) -- ++(120:\trianglewidth) -- cycle;
\draw [draw=none] (0,0) l-system [l-system={Sierpinski triangle,
axiom=X},fill=white];
        \end{tikzpicture}
}
    \caption{Tri\'angulo de Sierpinski para los niveles 1,3 y 5.}
    \label{fig:sierpinski}
\end{figure}

Como se muestra en la Fig.(\ref{fig:sierpinski})

