\documentclass[12pt]{article}
\usepackage[spanish,es-tabla,es-nodecimaldot]{babel}
\usepackage{tikz}
\usepackage{amsmath}    %Caracteres especiales de matematicas
\usepackage[hidelinks]{hyperref}    % Vinculos [que no estén subrayados]
\usepackage{listings}   %Introducir codigo
\usepackage{setspace}
\usepackage{color}
\usepackage{amssymb}
\usepackage{pdflscape}  %Voltear la página
\usepackage{graphicx}
\usepackage{caption}


\begin{document}
\title{Ejercicios de Técnicas de Conteo}
\author{Ana Maritza Bello Ya\~nez}
\maketitle
\setlength{\parindent}{0pt}
\setlength{\parskip}{1em}

\section*{Ejercicios de Probabilidad (Schaum)}
% Problemas 1.7, 1.8, 1.29, 1.39, 1.40, 1.77, 1.78, 1.87
% 5 ejemplos de muestras ordenadas

\subsection*{Problema 1.7}

\subsection*{Problema 1.8}

\subsection*{Problema 1.29}

\subsection*{Problema 1.39}

\subsection*{Problema 1.40}

\subsection*{Problema 1.77}

\subsection*{Problema 1.78}

\subsection*{Problema 1.87}

\section*{Ejercicios de Combinatoria (Schaum)}
% Problemas 2.29, 2.30, 2.68 al 2.74

\subsection*{Problema 2.29}
Sea $A$ un subconjunto $r$ de un $n$ conjunto $X$. ¿Cuántas permutaciones de $X$ inducen
alteraciones de $A$?

Sea $A_i (i = 1, 2,..., r)$ el subconjunto de permutaciones de $X$ que dejan fijo el
$i$-ésimo elemento de $A$. Entonces 

$n(X)=n!$

$s_i = C(r,j)(n-j)!$

$(j=1,2,...,r)$

y el Teorema 2.5 da la respuesta como $n(X) -s_1 + ... + (-1)^r s_r .$


\subsection*{Problema 2.30}

Muestre que $D_n = (n - 1)(D_{n_ 1} + D_{n_2})$, donde $n \geq 3$. (Por supuesto,
$D_2 = 1$ y $D_1 = 0$.)

\textbf{Solución}

Considere esos desarreglos de $X = {1, 2,..., n}$ en los que $r$ ocupa la primera
posición. Entonces 1 no ocupa la $r$-ésima posición o la ocupa. Hay trastornos
$D_{n-1}$ del primer tipo (con respecto a $X - \{r\}$, la posición $r$ funciona como la
posición 1) y $D_{n-2}$ del último tipo. El elemento $r$ se puede elegir de $n-1$.
maneras.

\subsection*{Problema 2.68}

\subsection*{Problema 2.69}
De los 100 estudiantes del problema 2.20, ¿cuántos toman 

\begin{itemize}
    \item exactamente 1 curso? 
    \item exactamente 2 cursos? 
    \item exactamente 3 cursos? 
    \item al menos 1 curso?
    \item al menos 2 cursos? 
    \item al menos 3 cursos?
\end{itemize}

Problema 20:

En un dormitorio, hay 12 estudiantes que toman un curso de arte (A), 20 que
toman un curso de biología (B), 20 que toman un curso de química (C) y 8 que
toman un curso de teatro (D). Hay 5 estudiantes que toman A y B, 7 estudiantes
que toman A y C, 4 estudiantes que toman A y D, 16 estudiantes que toman B y C,
4 estudiantes que toman B y D, y 3 estudiantes que toman tanto C como D. Hay 3
que toman A, B y C; 2 que toman A, B y D; 2 que toman B, C y D; 3 que toman A, C
y D.

Finalmente, hay 2 en los cuatro cursos. También se sabe que hay 71 estudiantes
en la residencia que no se han apuntado a ninguno de estos cursos. Encuentra el
número total de estudiantes en el dormitorio.

\subsection*{Problema 2.70}

Una función $w$ de un conjunto $X$ al conjunto de números reales se denomina
\textbf{función de peso} en $X$. Si $X$ es finito y $A$ es un subconjunto de
$X$, \textbf{el peso de $A$}, denotado $w(A)$, es la suma de todos los $w(x)$
para $x \in A$. Si $\amalg$ es un conjunto de $m$ propiedades (cf. Problema
2.65), sea $A_j (j = 0, 1, 2,..., m)$ el conjunto de todos los elementos en $X$
que tienen \textit{exactamente} $j$ propiedades y $j$ sea $B_j (j = 1, 2,...,
m)$ el conjunto de todos los elementos en $X$ que tienen al menos $j$
propiedades. Para cada $j$, escribe $E_j \equiv w(A_j)$ y $F_j w(B_j)$. Si $Q$
es un subconjunto de $\amalg$, $w(Q)$ se define como la suma de los pesos de
todos los elementos en $X$ que tienen todas las propiedades de $Q$. Finalmente,
en analogía con el $s_k$ del problema 2.65, defina

\begin{equation}
    \label{eq:aqui-le-mostramos-como-hacerle-la-llave-grande}
    S_k = \left\{
          \begin{array}{ll}
        0      & \mathrm{si\ } k=0 \\
        x - 30 & \mathrm{si\ } k=1,2,...,m
          \end{array}
        \right.
  \end{equation}

\subsection*{Problema 2.71}

\subsection*{Problema 2.72}

\subsection*{Problema 2.73}

\subsection*{Problema 2.74}

\end{document}