\documentclass[12pt]{article}
\usepackage[spanish,es-tabla,es-nodecimaldot]{babel}
\usepackage{tikz}
\usepackage{amsmath, amssymb, amsbsy}    %Caracteres especiales de matematicas
\usepackage[hidelinks]{hyperref}    % Vinculos [que no estén subrayados]
\usepackage{listings}   %Introducir codigo
\usepackage{setspace}
\usepackage{color}
\usepackage{pdflscape}  %Voltear la página
\usepackage{graphicx}
\usepackage{caption}


\begin{document}
\title{Ejercicios de Técnicas de Conteo}
\author{Ana Maritza Bello Ya\~nez}
\maketitle
\setlength{\parindent}{0pt}
\setlength{\parskip}{1em}

\section*{Ejercicios de Combinatoria (Schaum)}
% Problemas 1.7, 1.8, 1.29, 1.39, 1.40, 1.77, 1.78, 1.87
% 5 ejemplos de muestras ordenadas

\subsection*{Problema 1.7}
Demostrar que un número palindrómico (decimal) de longitud par es divisible por
11.

La prueba inductiva explota el hecho de que cuando se quitan el primer y el
último carácter de un palíndromo, queda un palíndromo. Así, sea $N$ un número
palindrómico de longitud $2k$. Si $k = 1$, el teorema obviamente se cumple. Si
$k \geq 2$, tenemos

$ N = a_{2k-1} 10^{2k-1} + a_{2k-2} 10^{2k-2} + ... + a_{k} 10^{k} + ... +
a_{2k-2} 10^{1} + a_{2k-1} 10^{0} $

$ = a_{2k-1}(10^{2k}+10^0) + (a_{2k-2}10^{2k-2}+...+ a_{2k-2}10^{1}) $

$ \equiv a_{2k-1} P + Q$

Donde:

$P = \underbrace{100...001}_{\text{longitud } 2k} =
\underbrace{9090...9091}_{\text{longitud } 2k-2}$

y ya sea $Q=0$ (divisible por 11) o, para algún $1 \leq r \leq k-1$,

$Q = 10^r$ \{palindrome de longitud $ 2(k-r)\} = 10^r\{11R\} $

donde el último paso se sigue de la hipótesis de inducción. Por lo tanto, $N$ es
divisible por 11 y la demostración está completa.

\subsection*{Problema 1.8}

En un palíndromo binario, el primer dígito es 1 y cada dígito subsiguiente puede
ser 0 o 1. Cuente los palíndromos binarios de longitud $n$.

Tenemos $[(n + 1)/21- 1 ]= [(n - 1)/2]$ posiciones libres, por lo tanto el número
deseado es:

$2^{[(n-1)/2]}$


\subsection*{Problema 1.29}
Encuentre la probabilidad $p_n$ de que un grupo de $n$ personas reunidas al azar
incluya al menos 2 personas con el mismo cumpleaños (día del año).

\textbf{Solución}

Aquí no tratamos con una muestra de personas, sino con una muestra de
cumpleaños, es decir, números enteros del 1 al 365 inclusivo. Nuestra noción de
probabilidad es:

$\text{Probabilidad} = \frac{\text{número de muestras favorables}}{\text{Número total de muestras}}$

En este problema es simple considerar el evento complementario: todos los $n$
cumpleaños son distintos. Este evento es realizado en $P(365,n)$ muestras; y el
total de muestras es $365^n$. Por lo tanto, $1-p_n = P(365,n)/365^n$ o:

$p_n = 1 - \frac{P(365,n)}{365^n} = 1 -
\frac{(365)(365-1)(365-2)...[365-(n-1)]}{365^n} $

$= 1-(1 - \frac{1}{365})(1 - \frac{2}{365})...(1 - \frac{n-1}{365})$

Puede verificarse que $p_n \geq 1/2 $ cuando $n>25$.

\subsection*{Problema 1.39}

\subsection*{Problema 1.40}

\subsection*{Problema 1.77}

\subsection*{Problema 1.78}

\subsection*{Problema 1.87}

\section*{Ejercicios de Probabilidad (Schaum)}
% Problemas 2.29, 2.30, 2.68 al 2.74

\subsection*{Problema 2.29}


\subsection*{Problema 2.30}


\subsection*{Problema 2.68}

\subsection*{Problema 2.69}


\subsection*{Problema 2.70}


\subsection*{Problema 2.71}

\subsection*{Problema 2.72}

\subsection*{Problema 2.73}

\subsection*{Problema 2.74}

\end{document}