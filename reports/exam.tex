\section{Examen}

\textbf{Problema}
Una empresa de manufactura que emplea tres planos analíticos para el diseño y
desarrollo de un productos específico. Por razones de costos los tres se
utilizan en momentos diferentes. De hecho, los planos 1, 2 y 3 se utilizan para
30\%, 20\% y 50\% de los productos respectivamente. La tasa de defectos difiere
en los 3 procedimientos de la siguiente manera,\

\begin{center}
$P(D|P_1)=0.01 P(D|P_2)=0.03 P(D|P_3)=0.02 $ \
\end{center}

En donde $P(D|P_j)$ es la probabilidad de que un producto esté defectuoso, dado
el plano j.\

Si se observa un producto al azar y se descubre que está defectuoso, ¿cuál de
los planos tiene más probabilidades de haberse utilizado y, por lo tanto, de ser
el responsable?\\

\textbf{Solución} \

A partir del planteamiento del problema \

\begin{center}
$P(P_1)=0.30, P(P_2)=0.20 y P(P_3)=0.50$ debemos calcular $P(P_j|D)$ para $j =
1, 2, 3.$
\end{center}

La regla de Bayes muestra que\

\begin{center}
$P(P_1|D)=\frac{(P(P_1)P(D|P1))}{P(P_1)P(P_1|D)+P(P_2)P(D|P_2)+P(P_3)P(D|P_3)} = $
\end{center}

\begin{center}
$\frac{(0.30)(0.01)}{(0.3)(0.01)+(0.20)(0.03)+(0.50)(0.02)}=\frac{0.003}{0.019}=0.158$\ 
\end{center}

De igual manera,\

\begin{center}
$P(P_2|D) = \frac{(0.03)(0.20)}{0.019} = 0.316$ y
\end{center}

\begin{center}
$P(P_3|D)=\frac{(0.02)(0.50)}{0.019}=0.526.$
\end{center}

La probabilidad condicional de un defecto, dado el plano 3, es la mayor de las
tres; por consiguiente, un defecto en un producto elegido al azar tiene más
probabilidad de ser el resultado de haber usado el plano 3.

\textbf{Problema}
Un bit es 0 ó 1, un byte una secuencia de 8 bits. Encuentra (a) El número de
bytes (b) El número de bytes que comienzan con 11 y no terminan con 11 (c) El
número de bytes que comienzan con 11 o terminan con 11. (d) El número de bytes
que comienzan con 11 o terminan con 11

\textbf{Solución}\

(a) $2^8 = 256$.

(b) Si un byte tiene 8 bits, entonces las 4 posiciones de en medio pueden
llenarse de $2^4 = 16$ maneras. 

(c) $2^6 = 64$ bytes comienzan con 11; por lo tanto, menos los $2^4 = 16$
espacios que no terminan en 11, tenemos $64 - 16 = 48$.

(d) Si usamos (b), 64 bytes comienzan con 11; y de igual manera 64 bytes
terminan con 11. Sumándolos, tenemos 64+64 = 128, y cada byte que empieza y
termina con 11 se cuenta doble. Así, la respuesta es 128 - 16 = 112 bytes. 

\textbf{Problema}
Entre un grupo de programadores, 49 estudiaron Pascal. 37 estudiaron COBOL y 21
estudiaron FORTRAN. Si 9 de esos programadores estudiaron Pascal y COBOL, 5
estudiaron Pascal y FORTRAN, 4 estudiaron COBOL y FORTRAN, 3 estudiaron Pascal,
cobol y FORTRAN, ¿cuanto programadores hay en este grupo?

\textbf{Solución}

Pascal = C, Cobol = C, Fortran = F, el número de programadores en el grupo está
dado por: $|P \cup C \cup F|$.

El total de estudiantes está dado por $n_1 = |P|+|C|+|F| = 49+37+21=107$ $n_2
=|P \cap C| + |P \cap F| + |C \cap F| = 9+5+4 = 18$ $n_3 = |P \cap C \cap F| =
3$

Por lo tanto por el principio de inclusión-exclusión $|P \cup C \cup F| = n_1 - n_2 +
n_3 = 107-18+3 = 92$

\textbf{Problema}
¿Cuál es la probabilidad de que un número entero positivo seleccionado de un
conjunto de enteros positivos, que no excedan 100, sea divisible ya sea entre 2
o 5. 

\textbf{Solución}

Sea el $E_1$ el evento en el que el entero seleccionado es divisible entre 2, y
sea $E_2$ el evento en el que es divisible entre 5. Entonces $E_1 \cup E_2$ es
el evento en el que el número seleccionado es divisible entre 2 y entre .

Por otra parte $E_1 \cap E_2$ es el evento en el que el número seleccionado es
divisible entre 2 y entre 6, o equivalentemente, que sea divisible entre 10.

Tenemos que:
$|E_1|=50, |E_2|=20,|E_1 \cap E_2|=10,$

entonces

\begin{center}
$p(E_1 \cup E_2)=p(E_1)+p(E_2)-p(E_1 \cap E_2) = \frac{50}{100} + \frac{20}{100}
- \frac{10}{100} = \frac{3}{5}$
\end{center}


\textbf{Problema}

Los alces moran en cierto bosque. Hay $N$ alces, de los cuales se captura y
etiqueta una muestra aleatoria simple de tamaño $n$ ("muestra aleatoria simple"
significa que todos los $\binom{n}{k}$ conjuntos de $n$ alces son igualmente
probables). Los alces capturados se devuelven a la población y luego se extrae
una nueva muestra, esta vez de tamaño $m$. Este es un método importante que se
usa ampliamente en ecología, conocido como \textit{captura-recaptura}. ¿Cuál es
la probabilidad de que exactamente $k$ de los $m$ alces en la nueva muestra
hayan sido marcados previamente? Suponga que un alce que fue capturado antes no
tiene más o menos probabilidades de ser capturado nuevamente.

\textbf{Solución}

Asumimos que todas las muestras de tamaño $m$ son igualmente probables. Para
tener exactamente k sea elk etiquetado, necesitamos elegir $k$ de los $n$ alces
etiquetados, y luego $m-k$ de los $N-n$ elk no etiquetados. Entonces la
probabilidad es

\begin{center}
$\frac{\binom{n}{k}*\binom{N-n}{m-k}}{\binom{N}{m}}$
\end{center}

para $k$ tal que $0 \leq k \leq n$ y $0 \leq m-k \leq N-n$, y la probabilidad es
0 para todos los demas valores de $k$ (por ejemplo, si $k > n$ la
probabilidad es 0 ya que entonces ni siquiera hay $k$ alces etiquetados en toda
la población).

\textbf{Problema}

Hay 100 pasajeros en fila para abordar un avión con 100 asientos (con cada
asiento asignado a uno de los pasajeros). El primer pasajero en la fila decide
locamente sentarse en un asiento elegido al azar (con todos los asientos
igualmente probables). Cada pasajero subsiguiente toma su asiento asignado si
está disponible y, de lo contrario, se sienta en un asiento disponible al azar.

¿Cuál es la probabilidad de que el último pasajero de la fila se siente en su
asiento asignado?

\textbf{Solución}

Observemos la situación cuando entra el $k$-ésimo pasajero. Ninguno de los
pasajeros anteriores mostró preferencia por el $k$-ésimo asiento frente al
asiento del primer pasajero. Esto en particular es cierto cuando $k=n$. Pero el
$n$-ésimo pasajero solo puede ocupar su asiento o el del primer pasajero. Por lo
tanto la probabilidad es $\frac{1}{2}$.

\textbf{El problema de los dos mentirosos}

Los sujetos $A$ y $B$ dicen la verdad con la probabilidad de $1/3$ y mienten con
una probabilidad de $(2/3)$.

$A$ dice una declaración, y $B$ confirma que la declaración hecha por $A$ es
cierta.

¿Cuál es la probabilidad que $A$ estuviera diciendo la verdad?

\textbf{Solución}

Sea $E$ el evento en el que $A$ hace una declaración y $B$ la confirma. Esto
puede pasar de dos maneras:

\begin{itemize}
    \item $T$: $A$ y $B$, ambos dicen la verdad.
    \item $L$: $A$ y $B$, ambos están mintiendo.
\end{itemize}

\begin{center}
$P(E)=P(T)+P(L)=\frac{1}{3}*\frac{1}{3}+\frac{2}{3}*\frac{2}{3}-\frac{5}{9}$
\end{center}

Dado que observamos $E$, estamos interesados en encontrar la probabilidad de que
el proceso subyacente sea, de hecho, $T$, es decir, $P(T|E)$. De la fórmula de las
probabilidades condicionales, tenemos que:

\begin{center}
$P(T|E)=\frac{P(T \cap E)}{P(E)}$  y $P(T \cap E) = P(T)$
\end{center}

ya que $T \subset E$.

En conclusión:

\begin{center}
$P(T|E) = \frac{P(T)}{P(E)} = \frac{1/9}{5/9} = \frac{1}{5}$
\end{center}

\textbf{La ameba sobreviviente}

Una población comienza con una sola ameba. Para esta y para las generaciones
posteriores, existe una probabilidad de 3/4 de que una ameba individual se
divida para crear dos amebas, y una probabilidad de 1/4 de que se extinga sin
producir descendencia. ¿Cuál es la probabilidad de que el árbol genealógico de
la ameba original continúe para siempre?

\textbf{Solución}

Esta es una caminata aleatoria sesgada disfrazada con un estado de absorción en
0 que comienza en 1. Si esta caminata aleatoria está en $N$, considerar $N$
movimientos corresponde a considerar el destino de $N$ amebas simultáneamente.
Esto produce $1-\frac{1-p}{p}=1-\frac{1/4}{3/4}=\frac{2}{3}$ por los argumentos
habituales.

%\subsection{Duelo justo}
%
%Pepe y Juan se han retado a un duelo. Tomarán turnos disparando uno al otro
%hasta que uno sea herido. Pepe, quien puede atinarle a Juan solamente el 40\% de
%las veces, es el disparador más débil, por lo que le será permitido disparar
%primero. Ellos han determinado que en el duelo no se favorezca a nadie.
%
%¿Cuál es la probabilidad de herir a Pepe?



