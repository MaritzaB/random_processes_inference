\documentclass{article}
\usepackage[spanish,es-tabla,es-nodecimaldot]{babel}
\usepackage{tikz}


\begin{document}
\title{Probabilidad, procesos aleatorios e inferencia}
\author{Ana Maritza Bello Yañez}
\maketitle
\setlength{\parindent}{0pt}
\setlength{\parskip}{1em}

\tableofcontents

\section{Probabilidad y conteo}

\subsection{Definición clásica}

La definición clásica de la probabilidad \cite{faber2012statistics} de un evento $A$, puede ser formulada
de la siguiente manera:

\begin{equation}
    P(A)= \frac{n_A}{n_(total)}
\end{equation}

Donde $n_A$ es el número de formas igualmente probables por las cuales un
experimento puede conducir a A; $n_(total)$ número total de formas igualmente
probables en el experimento.

- En la definición clásica el experimento no necesariamente se lleva a cabo ya que
la respuesta es conocida de antemano.

- La teoría clásica no da solución a menos de que todas las formas igualmente
posibles puedan ser derivadas analíticamente.

\section{Experimentos deterministas vs aleatorios}

\subsection{Definición: Observación}

Cualquier registro de información, ya sea numérico o categórico. Por
consiguiente, los números 2, 0, 1 y 2, que representan el número de accidentes
que ocurrieron cada mes, de enero a abril, durante el año pasado en la
intersección de Driftwood Lane y Royal Oak Drive, constituyen un conjunto de
observaciones. Lo mismo ocurre con los datos categóricos N, D, N, N y D, que
representan los artículos defectuosos o no defectuosos cuando se inspeccionan
cinco artículos y se registran como observaciones \cite{walpole2012probabilidad}.

\subsection{Definición: Experimento}

Cualquier proceso que genere un conjunto de datos. Un ejemplo simple de
experimento estadístico es el lanzamiento de una moneda al aire. En tal
experimento sólo hay dos resultados posibles: cara o cruz
\cite{walpole2012probabilidad}.

\subsection{Experimento determinista}
Un experimento determinista es aquel que produce el mismo resultado cuando se le
repite bajo las mismas condiciones.

\subsection{Experimento aleatorio}
Un experimento aleatorio es aquel que, cuando se le repite bajo las mismas
condiciones, el resultado que se observa no siempre es el mismo y tampoco es
predecible.

%Fig.(\reference{gatito})
%Tab.()
%Ec.()
\section{Triángulo de Sierpinski}

El matemático polaco Waclav Sierpinski (1882-1969), construyó este fractal en
1919 del modo siguiente: tomó un triángulo equilátero, unió los puntos medios de
los lados y formó cuatro triángulos interiores: tres triángulos equiláteros
sombreados y un hueco que es otro triángulo equilátero. Repitió el proceso en
cada uno de los triángulos sombreados, y siguió hasta el infinito el proceso en
los tres triángulos restantes como el primero.
\begin{figure}[htp]
    \centering
    \usetikzlibrary{lindenmayersystems}
    \def\trianglewidth{3cm}
    \pgfdeclarelindenmayersystem{Sierpinski triangle}{
        \symbol{X}{\pgflsystemdrawforward}
        \symbol{Y}{\pgflsystemdrawforward}
        \rule{X -> X-Y+X+Y-X}
        \rule{Y -> YY}
    }
    \foreach \level in {1,3,5}{
        \tikzset{
            l-system={step=\trianglewidth/(2^\level), order=\level, angle=-120}
        }

        \begin{tikzpicture}
            \fill [black] (0,0) -- ++(0:\trianglewidth) -- ++(120:\trianglewidth) -- cycle;
            \draw [draw=none] (0,0) l-system
            [l-system={Sierpinski triangle, axiom=X},fill=white];
        \end{tikzpicture}
    }
    \caption{Tirángulo de Sierpinski para los niveles 1,3 y 5.}
    \label{fig:sierpinski}
\end{figure}

\section{La belleza y utilidad de las matemáticas}

Las matemáticas han impregnado todos los campos de la actividad científica y
desempeñan un papel inestimable en la biología, la física, la química, la
economía, la sociología y la ingeniería. Las matemáticas se pueden utilizar para
ayudar a explicar los colores de una puesta de sol o la arquitectura de nuestro
cerebro. Las matemáticas nos ayudan a construir aviones supersónicos y montañas
rusas, simular el flujo de los recursos naturales de la Tierra, explorar
realidades cuánticas subatómicas y obtener imágenes de galaxias lejanas.

Las matemáticas han cambiado la forma en que vemos el cosmos.

La utilidad de las matemáticas nos permite construir naves espaciales e
investigar la geometría de nuestro universo. Los números pueden ser nuestro
primer medio de comunicación con razas alienígenas inteligentes. Algunos físicos
incluso han especulado que una comprensión de las dimensiones superiores y de la
topología, el estudio de las formas y sus interrelaciones, podría algún día
permitirnos escapar de nuestro universo, cuando termine en un gran calor o frío,
y entonces podríamos llamar a todo el espacio. -tiempo nuestro hogar.

El descubrimiento simultáneo ha ocurrido a menudo en la historia de las
matemáticas.

En 1858 el matemático alemán August Möbius (1790–1868) descubrió de forma simultánea e independiente
la cinta de Möbius (un maravilloso objeto retorcido de un solo lado) junto con
un erudito contemporáneo, el matemático alemán Johann Benedict Listado
(1808–1882). Este descubrimiento simultáneo de la banda de Möbius por parte de
Möbius y Listing, al igual que el del cálculo del erudito inglés Isaac Newton
(1643-1727) y el matemático alemán Gottfried Wilhelm Leibniz (1646-1716), nos
hacen preguntarnos por qué se hicieron tantos descubrimientos científicos en al
mismo tiempo por personas que trabajan de forma independiente. Para dar otro
ejemplo, los naturalistas británicos Charles Darwin (1809–1882) y Alfred Wallace
(1823–1913) desarrollaron la teoría de la evolución de manera independiente y
simultánea. De manera similar, el matemático húngaro János Bolyai (1802–1860) y
el matemático ruso Nikolai Lobachevsky (1793–1856) parecían haber desarrollado
la geometría hiperbólica de forma independiente y al mismo tiempo. \\

Lo más probable es que tales descubrimientos simultáneos hayan ocurrido porque
el momento era propicio para tales descubrimientos, dado el conocimiento
acumulado por la humanidad en el momento en que se realizaron los
descubrimientos. A veces, dos científicos se sienten estimulados al leer la
misma investigación preliminar de uno de sus contemporáneos. Por otro lado, los
místicos han sugerido que existe un significado más profundo para tales
coincidencias. El biólogo austriaco Paul Kammerer (1880-1926) escribió:
\textit{“Llegamos así a la imagen de un mosaico mundial o caleidoscopio cósmico que, a
pesar de las constantes reorganizaciones y reorganizaciones, también se ocupa de
unir lo similar”}. Comparó los eventos en nuestro mundo con la parte superior de
las olas del océano que parecen aisladas y sin relación. De acuerdo con su
controvertida teoría, notamos la parte superior de las olas, pero debajo de la
superficie puede existir algún tipo de mecanismo sincrónico que misteriosamente
conecta eventos en nuestro mundo y hace que se agrupen.

Georges Ifrah en The Universal History of Numbers analiza la simultaneidad al
escribir sobre las matemáticas mayas:

\textit{Por lo tanto, vemos una vez más cómo las personas que han estado muy separadas
en el tiempo o en el espacio han... llegado a resultados muy similares, si no
idénticos... En algunos casos, la explicación de esto se puede encontrar en los
contactos e influencias entre diferentes grupos de personas... La verdadera
explicación está en lo que antes hemos llamado la unidad profunda de la cultura:
la inteligencia del Homo sapiens es universal y su potencial es notablemente
uniforme en todas partes del mundo.}

Los pueblos antiguos, como los griegos, tenían una profunda fascinación por los
números. ¿Podría ser que en tiempos difíciles los números fueran lo único
constante en un mundo en constante cambio? Para los pitagóricos, una antigua
secta griega, los números eran tangibles, inmutables, cómodos, eternos, más
confiables que los amigos, menos amenazantes que Apolo y Zeus.

El brillante matemático Paul Erdös (1913–1996) estaba fascinado por la teoría de
los números, el estudio de los números enteros, y no tuvo problemas para
plantear problemas, usando números enteros, que a menudo eran simples de
plantear pero notoriamente difíciles de resolver. Erdös creía que si uno puede
plantear un problema de matemáticas que no ha sido resuelto durante más de un
siglo, entonces es un problema de teoría de números.

Muchos aspectos del universo se pueden expresar con números enteros. Los
patrones numéricos describen la disposición de los pétalos en una margarita, la
reproducción de los conejos, la órbita de los planetas, las armonías de la
música y las relaciones entre los elementos de la tabla periódica. Leopold
Kronecker (1823–1891), un algebrista y teórico de los números alemán, dijo una
vez: \textit{“Los números enteros provienen de Dios y todo lo demás fue creado por el
hombre”}. Su implicación fue que la fuente principal de todas las matemáticas son
los números enteros.

Desde la época de Pitágoras, el papel de las proporciones de números enteros en
las escalas musicales ha sido ampliamente apreciado. Más importante aún, los
números enteros han sido cruciales en la evolución de la comprensión científica
de la humanidad. Por ejemplo, el químico francés Antoine Lavoisier (1743–1794)
descubrió que los compuestos químicos se componen de proporciones fijas de
elementos que corresponden a las proporciones de números enteros pequeños. Esta
fue una evidencia muy fuerte de la existencia de los átomos. En 1925, ciertas
relaciones enteras entre las longitudes de onda de las líneas espectrales
emitidas por los átomos excitados dieron las primeras pistas sobre la estructura
de los átomos. Las proporciones casi enteras de los pesos atómicos fueron
evidencia de que el núcleo atómico está formado por un número entero de
nucleones similares (protones y neutrones).

Las desviaciones de las proporciones de enteros condujeron al descubrimiento de
isótopos elementales (variantes con un comportamiento químico casi idéntico pero
con diferente número de neutrones). Pequeñas divergencias en las masas atómicas
de los isótopos puros de los números enteros exactos confirmaron la famosa
ecuación de Einstein E = mc 2 y también la posibilidad de bombas atómicas. Los
números enteros están en todas partes en la física atómica. Las relaciones
enteras son hebras fundamentales en el entramado matemático, o como dijo el
matemático alemán Carl Friedrich Gauss (1777-1855): "Las matemáticas son la
reina de las ciencias, y la teoría de números es la reina de las matemáticas".

Nuestra descripción matemática del universo crece para siempre, pero nuestros
cerebros y habilidades lingüísticas permanecen arraigados. Se están descubriendo
o creando nuevos tipos de matemáticas todo el tiempo, pero necesitamos nuevas
formas de pensar y comprender. Por ejemplo, en los últimos años, se han ofrecido
pruebas matemáticas para problemas famosos de la historia de las matemáticas,
pero los argumentos han sido demasiado largos y complicados para que los
expertos estén seguros de que son correctos. El matemático Thomas Hales tuvo que
esperar cinco años antes de que los revisores expertos de su artículo de
geometría, enviado a la revista Annals of Mathematics, finalmente decidieran que
no podían encontrar errores y que la revista debería publicar la prueba de
Hales, pero solo con el descargo de responsabilidad que decía que no eran
errores. ¡seguro que tenía razón! Además, matemáticos como Keith Devlin han
admitido en el New York Times que “la historia de las matemáticas ha llegado a
una etapa de tal abstracción que muchos de sus problemas fronterizos ni siquiera
pueden ser entendidos por los expertos”. Si los expertos tienen tantos
problemas, uno puede ver fácilmente el desafío de transmitir este tipo de
información a una audiencia general. Hacemos lo mejor que podemos. Los
matemáticos pueden construir teorías y realizar cálculos, pero es posible que no
sean lo suficientemente capaces de comprender, explicar o comunicar
completamente estas ideas.

Una analogía física es relevante aquí. Cuando Werner Heisenberg se preocupó de
que los seres humanos nunca llegaran a entender verdaderamente los átomos, Niels
Bohr fue un poco más optimista. Él respondió a principios de la década de 1920:
\textit{"Creo que aún podemos hacerlo, pero en el proceso es posible que tengamos que
aprender qué significa realmente la palabra comprensión"}. Hoy, usamos
computadoras para ayudarnos a razonar más allá de las limitaciones de nuestra
propia intuición. De hecho, los experimentos con computadoras están llevando a
los matemáticos a descubrimientos e ideas nunca antes soñados antes de la
ubicuidad de estos dispositivos. Las computadoras y los gráficos por computadora
permiten a los matemáticos descubrir resultados mucho antes de que puedan
probarlos formalmente y abren campos matemáticos completamente nuevos. Incluso
las herramientas informáticas simples, como las hojas de cálculo, brindan a los
matemáticos modernos el poder que Gauss, Leonhard Euler y Newton habrían
deseado. Como solo un ejemplo, a fines de la década de 1990, los programas de
computadora diseñados por David Bailey y Helaman Ferguson ayudaron a producir
nuevas fórmulas que relacionaban pi con log 5 y otras dos constantes. Como
informa Erica Klarreich en Science News, una vez que la computadora produjo la
fórmula, probar que era correcta fue extremadamente fácil. A menudo, el simple
hecho de saber la respuesta es el mayor obstáculo que se debe superar al
formular una prueba.

En ocasiones, se han utilizado teorías matemáticas para predecir fenómenos que
no se confirmaron hasta años después. Por ejemplo, las ecuaciones de Maxwell,
llamadas así por el físico James Clerk Maxwell, predijeron las ondas de radio.
Las ecuaciones de campo de Einstein sugirieron que la gravedad doblaría la luz y
que el universo se está expandiendo. El físico Paul Dirac señaló una vez que las
matemáticas abstractas que estudiamos ahora nos dan una idea de la física en el
futuro. De hecho, sus ecuaciones predijeron la existencia de antimateria, que
posteriormente fue descubierta. De manera similar, el matemático Nikolai
Lobachevsky dijo que “no hay rama de las matemáticas, por abstracta que sea, que
algún día no pueda aplicarse a los fenómenos del mundo real”.

\pagebreak
\bibliography{../references/references.bib} 
\bibliographystyle{apalike}

\end{document}
