\documentclass[12pt]{article}
\usepackage[spanish,es-tabla,es-nodecimaldot]{babel}
\usepackage{tikz}
\usepackage{amsmath}    %Caracteres especiales de matematicas
\usepackage[hidelinks]{hyperref}    % Vinculos [que no estén subrayados]
\usepackage{listings}   %Introducir codigo
\usepackage{setspace}
\usepackage{color}
\usepackage{amssymb}
\usepackage[makeroom]{cancel}
\usepackage{pdflscape}  %Voltear la página
\usepackage{tcolorbox}  %Cuadros de teoremas
\usepackage{fancyhdr, lastpage}
\usepgflibrary{fpu}
\usepackage{ifthen}
\usepackage{graphicx}
\usepackage{caption}
\usepackage{subcaption}
\usepackage{multirow}
\usepackage{tikz-3dplot}
\usepackage{pgfplots}
\usepackage{pgfplotstable}
\usepgfplotslibrary{external}
\tcbuselibrary{listings,theorems}
\usepackage{wrapfig}
\usepackage{rotating}
\usetikzlibrary{automata, arrows.meta, positioning}


\pagestyle{fancy}   %Encabezado en las páginas
\chead{Probabilidad, procesos aleatorios e inferencia}  %Encabezado central
\lhead{}    %Encabezado izquiero vacio
\rhead{}    %Encabezado derecho vacio
%% Cambiar letra a tamaño 12

\usepackage[Glenn]{fncychap}
% Sonny, Lenny, Glenn, Conny, Rejne, Bjarne, Bjornstrup

\providecommand{\abs}[1]{\lvert#1\rvert}

\input{custom/specifications}
\input{custom/tikz-3dplot_documentation_figures}

\begin{document}
\title{Probabilidad, procesos aleatorios e inferencia}
\author{Ana Maritza Bello Ya\~nez}
\maketitle


%%%%%%%%%%%%%%%%%%%%%%%%%%%%%%%%%%%%%%%%%%%%%%%%%%%%%%%%%%%%%%%%%%%%%
%                       Caja del teorema                            %
%                                                                   %
\newtcbtheorem{theorem}{Teorema}             %
{colback=gray!5,colframe=gray!35!black,fonttitle=\bfseries}{th}   %
%%%%%%%%%%%%%%%%%%%%%%%%%%%%%%%%%%%%%%%%%%%%%%%%%%%%%%%%%%%%%%%%%%%%%

\tableofcontents

\setlength{\parindent}{0pt}
\setlength{\parskip}{1em}
\pagebreak
\section{Resumen: La belleza y utilidad de las matem\'aticas \ref{hestenes2012clifford}}

Las matem\'aticas han impregnado todos los campos de la actividad cient\'ifica y
desempe\~nan un papel inestimable en la biolog\'ia, la f\'isica, la qu\'imica,
la econom\'ia, la sociolog\'ia y la ingenier\'ia.


El descubrimiento simult\'aneo ha ocurrido a menudo en la historia de las
matem\'aticas.

Por ejemplo el del c\'alculo del erudito ingl\'es Isaac Newton (1643-1727) y el
matem\'atico alem\'an Gottfried Wilhelm Leibniz (1646-1716). Esto nos hacen
preguntarnos por qu\'e se hicieron estos descubrimientos cient\'ificos en al
mismo tiempo por personas que trabajan de forma independiente. Para dar otro
ejemplo, los naturalistas brit\'anicos Charles Darwin (1809-1882) y Alfred
Wallace (1823-1913) desarrollaron la teor\'ia de la evoluci\'on de manera
independiente y simult\'anea. De manera similar, el matem\'atico h\'ungaro
J\'anos Bolyai (1802-1860) y el matem\'atico ruso Nikolai Lobachevsky
(1793-1856) parec\'ian haber desarrollado la geometr\'ia hiperb\'olica de forma
independiente y al mismo tiempo.

Lo m\'as probable es que tales descubrimientos simult\'aneos hayan ocurrido
porque el momento era propicio para tales descubrimientos, dado el conocimiento
acumulado por la humanidad en el momento en que se realizaron los
descubrimientos. A veces, dos cient\'ificos se sienten estimulados al leer la
misma investigaci\'on preliminar de uno de sus contempor\'aneos. Por otro lado,
los m\'isticos han sugerido que existe un significado m\'as profundo para tales
coincidencias.

En ocasiones, se han utilizado teor\'ias matem\'aticas para predecir fen\'omenos
que no se confirmaron hasta a\~nos despu\'es. Por ejemplo, las ecuaciones de
Maxwell, llamadas as\'i por el f\'isico James Clerk Maxwell, predijeron las
ondas de radio. Las ecuaciones de campo de Einstein sugirieron que la gravedad
doblar\'ia la luz y que el universo se est\'a expandiendo. El f\'isico Paul
Dirac se\~nal\'o una vez que las matem\'aticas abstractas que estudiamos ahora
nos dan una idea de la f\'isica en el futuro. De hecho, sus ecuaciones
predijeron la existencia de antimateria, que posteriormente fue descubierta. De
manera similar, el matem\'atico Nikolai Lobachevsky dijo que “no hay rama de las
matem\'aticas, por abstracta que sea, que alg\'un d\'ia no pueda aplicarse a los
fen\'omenos del mundo real”.



\section{Problemas del milenio}

\subsection{Problemas P vs NP}
Se trata del primero de los problemas del milenio de las matemáticas aplicadas,
y alude más concretamente al campo de la complejidad computacional, dentro del
ámbito de la informática. Su planteamiento se remonta a los años 70, cuando
además de por Alan Turing, fue planteado paralelamente por los programadores
Stephen Cook y Leonid Levin.

A grandes rasgos, el problema P frente a NP busca clasificar los problemas en
dos clases: los que pueden ser resueltos con una cantidad determinada de
recursos, y aquellos que no. Los recursos a los que nos referiríamos serían el
tiempo empleado para realizar los cálculos, y la memoria requerida (no olvidemos
que nos encontramos en el campo de la informática computacional) para procesar
los datos del problema.

Por ejemplo, los problemas P serían de fácil resolución para las computadoras, es
decir sus soluciones serían fáciles de encontrar en una cantidad razonable de
tiempo. En los problemas NP, por el contrario, la solución podría ser muy
difícil de encontrar, o quizá requeriría una gran cantidad de recursos (miles de
años), para ser hallada, aunque una vez encontrada la solución sería fácil de
comprobar. Un ejemplo muy ilustrativo de este tipo de problemas podría ser la
resolución de un puzle, donde encontrar el orden de las piezas, podría requerir
gran cantidad de recursos, pero una vez terminado el puzzle, la solución
correcta saltaría a la vista, y sería fácil de comprobar.

El problema P versus NP plantea si todos los problemas NP son también un
problema P. Si P es igual a NP, todos los problemas NP contendrían un atajo
oculto, que permitiría que los ordenadores encontrasen rápidamente soluciones
perfectas. Pero si P no es igual a NP, entonces no existen dichos atajos, lo que
demostraría que la potencia de resolución de problemas de los ordenadores es
limitada.

\subsection{Hipótesis de Riemann}

La hipótesis de Riemann fue formulada por primera vez por Bernhard Riemann en
1859, y por su relación con la distribución de los números primos en el conjunto
de los naturales, es uno de los problemas abiertos más importantes en la
matemática contemporánea. Riemann sugirió que la distribución de estos números
está estrechamente relacionada con el comportamiento de la llamada "función zeta
de Riemann", la cual tiene dos tipos de ceros: los ceros "triviales", que son
todos los números enteros pares y negativos; y los ceros "no triviales", cuya
parte real está siempre entre 0 y 1.

La hipótesis de Riemann afirma que todos los ceros no triviales de la función
zeta se encuentran en la recta $x = 1/2$. A día de hoy, más de diez billones de
ceros han sido calculados para la función $z$, todos alineados sobre la recta
crítica, los cuales corroboran la sospecha de Riemann. Sin embargo todavía nadie
aún ha podido demostrar en la actualidad que la función zeta no tenga ceros no
triviales fuera de dicha recta.

\subsection{La conjetura de Poincaré}

La conjetura de Poincaré es un problema topológico, establecido en 1904 por el
matemático francés Henri Poincaré. Se trataba de uno de los problemas de más
difícil resolución de los 7 problemas del milenio. Decimos "se trataba" por que
fue resuelto en el año 2006, convirtiéndose en el Teorema de Poincaré como fruto
del trabajo del matemático ruso Grigori Perelman, quien renunció a la cuantía
económica del premio.

El teorema sostiene que la esfera cuatridimensional, también llamada 3-esfera o
hiperesfera, es la única variedad compacta cuatridimensional en la que todo lazo
o círculo cerrado (1-esfera) se puede deformar (transformar) en un punto. Este
último enunciado es equivalente a decir que solo hay una variedad cerrada y
simplemente conexa de dimensión: la esfera cuatridimensional.




\section*{Experimentos deterministas vs aleatorios}


\begin{tcolorbox}[colback=blue!5!white,colframe=blue!60!black,title=Definición: Observación]
    Cualquier registro de informaci\'on, ya sea num\'erico o categ\'orico.
\end{tcolorbox}

Ejemplos: 

Los n\'umeros 2, 0, 1 y 2, que representan el n\'umero de accidentes que
ocurrieron cada mes, de enero a abril, durante el a\~no pasado, constituyen un
conjunto de observaciones. 

Lo mismo ocurre con los datos categ\'oricos N, D, N, N y D, que representan los
art\'iculos defectuosos o no defectuosos cuando se inspeccionan cinco
art\'iculos y se registran como observaciones \cite{walpole2012probabilidad}.

\begin{tcolorbox}[colback=blue!5!white,colframe=blue!60!black,title=Definición: Experimento]
    Cualquier proceso que genere un conjunto de datos.    
\end{tcolorbox}

Un ejemplo simple de experimento estad\'istico es el lanzamiento de una moneda
al aire. En tal experimento s\'olo hay dos resultados posibles: águila o sol.
\cite{walpole2012probabilidad}.

\begin{tcolorbox}[colback=blue!5!white,colframe=blue!60!black,title=Definición: Tipos de experimentos]

    \textbf{Experimento determinista.}
    Un experimento determinista es aquel que produce el mismo resultado cuando se le
    repite bajo las mismas condiciones.

    \tcblower

    \textbf{Experimento aleatorio.}
    Un experimento aleatorio es aquel que, cuando se le repite bajo las mismas
    condiciones, el resultado que se observa no siempre es el mismo y tampoco es
    predecible.
\end{tcolorbox}



\section{Teoría de conjuntos}

\subsection{Introducción}

El concepto de conjunto, es fundamental en matemáticas. Su estudio, se basa en
el hecho de que éstos pueden ser combinados, mediante ciertas operaciones, para
formar otros conjuntos. 

El estudio de las operaciones con los conjuntos, constituye el álgebra de
conjuntos; que tiene semejanzas formales (aunque también presenta diferencias)
con el álgebra de los números. El álgebra de conjuntos, resulta valiosa en la
reducción de los conceptos matemáticos a sus fundamentos lógicos. 

La disciplina matemática que estudia las propiedades generales de los conjuntos
es la teoría de conjuntos. Esta disciplina se comenzó a desarrollar,
rigurosamente, a finales del siglo XIX y principios del XX. El fundador de dicha
teoría es el matemático alemán de origen ruso, George Ferdinand Ludwing
Phillipp Cantor (1845, 1918). Las ideas y conceptos de la teoría de conjuntos
han irrumpido literalmente en todas las ramas de las matemáticas y cambiaron su
faz por completo.

\subsection{Definición básica de conjunto}

Un conjunto es cualquier colección de objetos bien definidos por medio de alguna
o algunas propiedades en común, de dichos objetos. Por objeto entenderemos no
sólo cosas físicas, como discos, computadores, etc., si no también abstractos,
como son números, letras, etc. A los objetos se les llama elementos del
conjunto.

Representamos a los conjuntos por medio de letras mayúsculas, así A, B, C, etc.
nos representan conjuntos.

\subsection{Elemento}

Se llaman elementos o miembros a los objetos que componen un conjunto; y se
denotan con letras minúsculas, como: $a, b, x, y$, etc. Para indicar que un
objeto $x$ pertenece o es miembro de un conjunto $A$, escribimos $x \in A$, que
se lee \textit{“x elemento de A”} y, si no pertenece al conjunto $A$, escribimos
$x \notin A$, que se lee \textit{“x no es elemento de A”}.

\subsection{Conjunto universo}
Llamamos conjunto universo y lo denotamos por U, al conjunto del cual se
seleccionan los elementos para formar conjuntos.

En general identificamos al conjunto universo como un todo, pero no
representamos de una manera única a este “todo”, así habrá ocasiones que un
conjunto sea considerado conjunto universo y en otras no, por ejemplo:
considerando el conjunto P de los número pares, se tiene que $\mathbb Z$
(conjunto enteros) es un conjunto universo para él; pero si consideramos el
mismo conjunto $\mathbb Z$, tenemos que $\mathbb Q$ (racionales) es un conjunto
universo de él. Así Z es un conjunto universo en ocasiones y en otras no. Además
puede haber varios conjuntos universos para un solo conjunto, por ejemplo el
mismo conjunto P de números pares tiene por conjunto universo a $\mathbb Z$
(conjunto de los enteros) o a Q (conjunto se los racionales) o a $\mathbb R$
(conjunto de los reales) o incluso a $\mathbb C$ (conjunto de los complejos), en
general podemos decir que para números el conjunto universo más grande es
$\mathbb C$ (números complejos).

\subsection{Conjunto vacío}

El conjunto vacío o nulo es un conjunto que no tiene elementos, el conjunto
vacío se representa por: $\emptyset$, o bien por {}.

No confundir el conjunto $A = \emptyset $ con el conjunto $ A = {\emptyset} $,
ya que el primer conjunto indica que no tiene ningún elemento y el segundo
conjunto indica que tiene un elemento y ese elemento es el conjunto vacío.

\subsection{Igualdad de conjuntos}

Decimos que dos conjuntos $A$ y $B$ son iguales y denotamos $A=B$, si $A$ y $B$
constan de los mismos elementos, i.e. si cada elemento de $A$ pertenece a $B$ y
si cada elemento de $B$ pertenece a $A$.

\subsection{Subconjunto}

Si todos los elementos de un conjunto A son también elementos de un conjunto
$B$, esto es, si cuando $x \in A$ entonces $x \in B$ (simbólicamente $x \in A
\rightarrow x \in B$), decimos que A es un subconjunto de B o que A está
contenido en B y se escribe:

\begin{equation}
    \begin{array}{l}
        A \subseteq B \\
        B \subseteq A
    \end{array}
\end{equation}

Si $A$ no es subconjunto de $B$ se escribe $A \not \subset B$ .

Si además existe un elemento de $B$ que no este en $A$, decimos que $A$ es un
subconjunto propio de $B$ y se denota $A \subset B$ o $B \subset A$.

\subsection{Propiedades de los conjuntos}

\begin{enumerate}
    \item Si A es cualquier conjunto diferente del vacío, entonces $A \subseteq A$ . Esto es, cualquier conjunto
    diferente del vacío es subconjunto de si mismo.
    
\item El conjunto vacío es subconjunto de cualquier conjunto distinto del vacío,
esto es $\emptyset \subset A$

\item Si $n>0$ es el número de elementos de $A$ entonces el número de elementos de $P(A)$ es $2^n$
\end{enumerate}

\begin{theorem}{Conjuntos}{conjuntos}
Sean $A$ y $B$ dos conjuntos. Decimos que $A=B$ si y solo si $A \subseteq B$ y
$B \subseteq A$
\end{theorem}

\begin{theorem}{Conjuntos}{conjuntos}
    Si $ A \subset B $ y $ B \subset C $, entonces $ A \subset C $
\end{theorem}

\subsection{Cardinalidad}

Sea $A$ un conjunto, la cardinalidad de $A$ es el número de elementos diferentes del conjunto
$A$ y se representa por $|A|$.

\subsection{Conjunto potencia}
Sea $A$ un conjunto finito, llamaremos conjunto potencia al conjunto formado por todos los
subconjuntos de $A$. El conjunto potencia se denota como $P(A)$ o bien $2^A$.

\subsection{Operaciones entre conjuntos}

% Definition of circles
\def\firstcircle{(0,0) circle (1.5cm)}
\def\secondcircle{(0:2cm) circle (1.5cm)}

\colorlet{circle edge}{blue!50}
\colorlet{circle area}{blue!20}

\tikzset{filled/.style={fill=circle area, draw=circle edge, thick},
    outline/.style={draw=circle edge, thick}}

\setlength{\parskip}{5mm}

\subsubsection{Unión}

La unión La unión (o reunión) de dos conjuntos $A$ y $B$, denotada por $ A \cup
B $ (que se lee “A unión B”) es un nuevo conjunto formado por los elementos que
pertenecen a A o a B o a ambos conjuntos.

\begin{equation}
    A \cup B = {x | x \in A \vee x \in B}
\end{equation}

Si $x \in A \cup B $ entonces $x \in A $ ó $x \in B$ o $x$ pertenece a ambos
conjuntos.

% Set A or B
\begin{figure}[h]
    \centering
    \begin{tikzpicture}
        \draw[filled] \firstcircle node {$A$}
                      \secondcircle node {$B$};
        \node[anchor=south] at (current bounding box.north) {$A \cup B$};
    \end{tikzpicture}
\end{figure}

\subsubsection{Intersección}

La intersección de dos conjuntos $A$ y $B$, denotada por $A \cap B$ (que se lee “ A intersección B”),
es un nuevo conjunto formado por los elementos que pertenecen a A y a B al mismo tiempo, es
decir, por los elementos comunes a ambos conjuntos.

\begin{equation}
    A \cap B={x | x \in A \wedge x \in B}
\end{equation}

% Set A and B
\begin{figure}[h]
    \centering
    \begin{tikzpicture}
        \begin{scope}
            \clip \firstcircle;
            \fill[filled] \secondcircle;
        \end{scope}
        \draw[outline] \firstcircle node {$A$};
        \draw[outline] \secondcircle node {$B$};
        \node[anchor=south] at (current bounding box.north) {$A \cap B$};
    \end{tikzpicture}
\end{figure}

\subsubsection{Complemento}

Sea $A \subseteq U$ un conjunto. El complemento de A denotado por $A'$, $A^C$ ó $U-A$, se define
como el conjunto de todos los elementos que están en $U$ pero no están en $A$.
Simbólicamente:

\begin{equation}
    \bar{A}= {x | x \in U \wedge x \notin A}
\end{equation}

Es decir, el complemento de $A$ es el conjunto de los todos los elementos que no
están en el conjunto $A$. Simbólicamente: $ \bar{A}={x | x \notin A}$.

\subsubsection{Diferencia}

% Set A but not B
\begin{figure}[h]
    \centering
    \begin{subfigure}[b]{0.45\textwidth}
        \centering
        \begin{tikzpicture}
            \begin{scope}
                \clip \firstcircle;
                \draw[filled, even odd rule] \firstcircle node {$A$}
                                             \secondcircle;
            \end{scope}
            \draw[outline] \firstcircle
                           \secondcircle node {$B$};
            \node[anchor=south] at (current bounding box.north) {$A - B$};
        \end{tikzpicture}
    \end{subfigure}
\hfill
    \begin{subfigure}[b]{0.45\textwidth}
        \centering
        % Set B but not A
        \begin{tikzpicture}
            \begin{scope}
                \clip \secondcircle;
                \draw[filled, even odd rule] \firstcircle
                                             \secondcircle node {$B$};
            \end{scope}
            \draw[outline] \firstcircle node {$A$}
                           \secondcircle;
            \node[anchor=south] at (current bounding box.north) {$B - A$};
        \end{tikzpicture}
\end{subfigure}
\end{figure}

\subsubsection{Diferencia simétrica}



%Set A or B but not (A and B) also known a A xor B
\begin{figure}[h]
    \centering
    \begin{tikzpicture}
        \draw[filled, even odd rule] \firstcircle node {$A$}
                                     \secondcircle node{$B$};
        \node[anchor=south] at (current bounding box.north) {$\overline{A \cap B}$};
    \end{tikzpicture}
\end{figure}






\section{Binomio de Newton}

El binomio de Newton consiste en una fórmula que permite obtener los
coeficientes de un término enésimo de un binomio elevado a un exponente
determinado.

La fórmula matemática del binomio de Newton es la siguiente:

\begin{equation}
    (a+b)^n = \sum_{k=0}^{n} \binom{n}{k} a^{n-k} b^k
\end{equation}

\subsection{Desarrollo del binomio de Newton de la potencia 1 a la 10}

\begin{landscape}
	\begin{equation}
		\begin{split}
			(a+b)^0 &   = 1 \\
			(a+b)^1 &   = a+b \\
			(a+b)^2 &   = a^2+2ab+b^2 \\
			(a+b)^3 &   = a^3+3a^2b+3ab^2+b^3 \\
			(a+b)^4 &   = a^4+4a^3b+6a^2b^2+4ab^3+b^4 \\
			(a+b)^5 &   = a^5+5a^4b+10a^3b^2+10a^2b^3+5ab^4+b^5\\
			(a+b)^6 &   = a^6+6a^5b+15a^4b^2+20a^3b^3+15a^2b^4+6ab^5+b^6\\
			(a+b)^7 &   = a^7+7a^6b+21a^5b^2+35a^4b^3+35a^3b^4+21ba^2b^5+21ab^6+b^7 \\
			(a+b)^8 &   = a^8+8a^7b+28a^6b^2+56a^5b^3+70a^4b^4+56a^3b^5+28a^2b^6+8ab^7+b^8 \\
			(a+b)^9 &   = a^9+9a^8b+36a^7b^2+84a^6b^3+126a^5b^4+126a^4b^5+84a^3b^6+36a^2b^7+9ab^8+b \\
			(a+b)^{10} &    = a^{10}+10a^9b+45a^8b^2+120a^7b^3+210a^6b^4+252a^5b^5+210a^4b^6+120a^3b^7+45a^2b^8+10ab^9+b^{10} \\
		\end{split}
	\end{equation}
\end{landscape}


\section*{Triángulo de Pascal}

En las matemáticas, el triángulo de Pascal es una representación de los
coeficientes binomiales ordenados en forma de triángulo. Es llamado así en honor
al filósofo y matemático francés Blaise Pascal, quien introdujo esta notación en
1654, en su Tratado del triángulo aritmético.

\newcommand{\pasc}[2]{
	\pgfkeys{/pgf/fpu}
	\pgfmathparse{round(#1!/((#1-#2)!*#2!))}
	\pgfmathfloattoint{\pgfmathresult}
	\pgfmathresult
}

\begin{figure}[h]
	\centering
	\begin{tikzpicture}
	\pgfmathsetmacro{\N}{10};
	\foreach \i in {0,...,\N}{
		\foreach \j in {0,...,\i}{
			%\pgfmathtruncatemacro{\var}{\pasc{\i}{\j}}
			\node at ({-0.5*\i+\j-0.2},-\i){\pasc{\i}{\j}};
			
			%\ifthenelse{\pasc{\i}{\j}=}{then clause}{else clause}
			% \draw ({-0.5*\i+\j-0.5},-\i+0.5) rectangle
			% ({-0.5*\i+\j-0.5},-\i-0.5);
			\draw ({-0.5*\i+\j},-\i) circle (0.5) ;
		}
	}
	\end{tikzpicture}
	\caption{Triángulo de Pascal para n=10}
	\label{fig:pascalTriangle}
\end{figure}

Este triángulo fue ideado para desarrollar las potencias de binomios.

La fórmula del binomio de Newton desarrolla los coeficientes de cada fila en el
triángulo de Pascal. Es por esto que existe una estrecha relación entre el
triángulo de Pascal y el binomios de Newton.

\subsection{Propiedades del triángulo de Pascal}

Algunas de las propiedades más representativas del triángulo de Pascal son las
siguientes:

\begin{itemize}
	\item Cada número es la suma de los dos números encima de él.
	\item Todos los números exteriores son iguales a 1.
	\item El triángulo de Pascal es simétrico.
	\item La primera diagonal muestra los números de conteo.
	\item Las sumas de las filas dan las potencias del 2.
	\item Cada fila da los dígitos de las potencias del 11.
	\item Cada elemento representa a la combinación , en donde, m es la fila del elemento y n es la posición del elemento en la fila.
	\item Cada fila representa a los coeficientes binomiales.
	\item Los números Fibonacci están a lo largo de las diagonales.
\end{itemize}

\subsection{Patrones del triángulo de Pascal}

\subsubsection{Suma de las filas}

Una de las propiedades interesantes del triángulo es que la suma de los números
en una fila es igual a $2^n$, en donde, n corresponde al número de la fila. Por
ejemplo, tenemos:

$ 1 = 1 = 2^0 $ \\
$ 1 + 1 = 1 = 2 = 2^1 $ \\
$ 1 + 2 = 1 = 4 = 2^2 $\\

como se puede observar en la Fig. (\ref{fig:pascalTriangle})

\subsubsection{Números primos en el triángulo}

Otro patrón visible en el triángulo se relaciona a los números primos. Si es que
una fila empieza con un número primo o es una fila con número primo, todos los
números que están en esa fila, sin incluir al 1, son divisibles para ese número
primo.

Si es que miramos a la fila 5 (1, 5, 10, 10, 5, 1) podemos ver que el 5 y el 10
son divisibles por 5. Sin embargo, para una fila compuesta como la fila 8 (1, 8,
28, 56, 70, 56, 28, 8, 1), 28 y 70 no son divisibles por 8.

\subsubsection{Sucesión de Fibonacci en el triángulo}
En el triángulo de Pascal se puede apreciar una relación entre un modo de sumar
las diagonales y la sucesión de Fibonacci. Los primeros términos de esta
sucesión son: $1,1,2,3,5,8,13,21,34,55$ como se puede apreciar en la Fig.
(\ref{fig:pascalTriangle}).

\subsubsection{Expansión binomial con el triángulo de Pascal}

El triángulo de Pascal define a los coeficientes que aparecen en las expresiones
binomiales. Eso significa que la fila n del triángulo de Pascal contiene a los
coeficientes de la expresión expandida del binomio $(x+y)^n$.


\section{T\'ecnicas de conteo}

Muchos problemas de combinatoria implican conteo. Ya que el número de objetos a
contar podría ser muy grande, es importante ser capaz de conta el conjunto de
objetos sin tener que enlistarlos a todos.
% ARREGLAR SALTOS DE PÁGINA
% EIGENVALORES
% VALORES PROPIOS
\subsection{Principio de multiplicaci\'on}

El principio de multiplicación lo utilizamos para contar el número de formas en
las que pueden ocurrir dos eventos simultáneos. El principio de la multiplicación
establece que si un evento puede ocurrir de $n_1$ formas diferentes, y para cada
una de estas ppuede ocurrir un segundo evento simultáneo en $n_2$ formas
diferentes, entonces los dos eventos pueden ocurrir de $n_1 * n_2$ formas
diferentes.

Así, para una serie de $k$ eventos, tenemos:
\begin{theorem}{Principio de multiplicación}{multiplicación}
	\begin{equation}
		n_1*n_2*n_3*...*n_k
		\label{eq:reglaMultiplicacion}
	\end{equation}
\end{theorem}


\subsection{Diagrama de Árbol}

Los diagramas de árbol muestran todos los resultados posibles de un evento. Cada
rama en un diagrama de árbol representa un posible resultado.
Los diagramas de \'arbol pueden usarse para encontrar el n\'umero de resultados
posibles y calcular la probabilidad de los posibles resultados.

%Por ejemplo,

\begin{figure}[h]
	\begin{center}		
	\begin{tikzpicture}[level distance=1cm, level 1/.style={sibling distance=3cm},
		level 2/.style={sibling distance=2cm}, every node/.style={circle, draw,
		align=center} ]
				\centering
				\node[circle,draw]{$P_n$}
		child{node{1} child{node{2} child{node{3}} } child{node{3} child{node{2}}} }
		child{node{2} child{node{1} child{node{3}} } child{node{3} child{node{1}}} }
		child{node{3} child{node{1} child{node{2}} } child{node{2} child{node{1}}} };
			\end{tikzpicture}
		\end{center}
		\caption{Permutaciones para 3 elementos}
		\label{permutaciones_tree}
\end{figure}


\subsection{Principio de adición}

Supongamos que existen $k$ conjuntos de elementos con $n_1$ elementos en el
primer conjunto, $n_2$ elementos en el segundo conjuntos, etc. Si todos los
elementos son distintos, es decir, si si todos los pares dek conjunto $k$ son
disjuntos, entonces el número de elementos de la unión de los conjuntos es $n_1
+ n_2 + ... + n_k $.

Usando la nomenclatura de conjuntos, el principio de adición está definido de la
siguiente manera:

\begin{theorem}{Principio de adición}{adicion}
Sean $A$ y $B$ dos eventos disjuntos, entonces la probabilidad de la unión de
los conjuntos está dada por:

\begin{equation}
	P(A \cup B) = P(A) + P(B)
\end{equation}

De manera general, si el espacio de muestreo tiene un número infinito de
elementos y $A_1, A_2, ... $ son una secuencia de eventos disjuntos, entonces la
probabilidad de la unión es:

\begin{equation}
	P(A_1 \cup A_2 \cup ...) = P(A_1) + P(A_2) + ...
\end{equation}

\end{theorem}


\subsection{Principio del palomar o Principio de Dirichlet}

El principio del palomar, también llamado principio de Dirichlet o principio de
las cajas, establece que si $n$ palomas se distribuyen en $m$ palomares, y si $n
> m$, entonces al menos habrá un palomar con más de una paloma. Otra forma de
decirlo es que $m$ huecos pueden albergar como mucho $m$ objetos si cada uno de
los objetos está en un hueco distinto, así que el hecho de añadir otro objeto
fuerza a volver a utilizar alguno de los huecos, como se observa en la Fig.
(\ref{palomar}). A manera de ejemplo: si se toman trece personas, al menos dos
habrán nacido el mismo mes.

\begin{figure}[h]
	\centering
	\begin{tikzpicture}
		% draw the sets
		\filldraw[fill=gray!20, draw=gray!60] (-1.5,0) circle (2cm);
		\filldraw[fill=green!20, draw=green!60] (3,0) circle (2cm);
	
	
		% the texts
		\node at (-1.5,3) {Palomas};
		\node at (3,3) {Palomares};
	
		% the circles and the arrows
		\node (x1) at (-1.5,0.8) {$p_1$};
		\node (x2) at (-1.5,0.3) {$p_2$};
		\node (x3) at (-1.5,-0.3) {$p_3$};
		\node (x4) at (-1.5,-0.8) {$p_4$};
		
		\node (y1) at (3,0.7) {$1$};
		\node (y2) at (3,0) {$2$};
		\node (y3) at (3,-0.7) {$3$};
	
		% draw the arrows
		\draw[-latex] (x1) -- (y1);
		\draw[-latex] (x2) -- (y2);
		\draw[-latex] (x3) -- (y3);
		\draw[-latex] (x4) -- (y3);
	
	\end{tikzpicture}
\caption{Ilustración del principio del palomar. El conjunto de las palomas es
mayor al del palomar, por lo tanto en al menos un palomar deben de haber dos
palomas.}
\label{palomar}
	\end{figure}




\subsection{Permutaciones}

\begin{tcolorbox}[colback=gray!5!white,colframe=gray!60!black,title=Definición: Permutación]
	Una \textbf{permutaci\'on} de un conjunto es un arreglo u ordenamiento de sus
	elementos sin repeticiones ni omisiones.
	
	Si el conjunto está ordenado, el proceso de reordenar sus elementos, es una permutación.
	\label{Permutaciones_definition}
\end{tcolorbox}

Para saber el n\'umero de permutaciones que existen de un conjunto se debe
conocer su cardinalidad. Tomemos como ejemplo, el conjunto de n\'umeros enteros
$A=\{1, 2, 3\}$. Para cualquier permutaci\'on, en la primera posici\'on pueden
colocarse cualquiera de los tres elementos; en la segunda posición se pueden
colocar dos posibles elementos; mientras que al final, se puede colocar una
posibilidad.

En general, para cualquier conjunto de elementos $X$ de cardinalidad $|n|$, en
la primera posici\'on se pueden colocar $n$ elementos, para la siguiente
posici\'on $n-1$. Así sucesivamente hasta las \'ultimas posiciones, donde se
pueden colocar $3, 2$ y $1$ elementos. De esta manera, se dice que el n\'umero
de permutaciones es:

\begin{theorem}{Permutación}{permutacion}
	\begin{equation}
		P_n=n*(n-1)*(n-2)*...*3*2*1=n!
		\label{permutaciones_totales}
	\end{equation}
	
	Remarcando el hecho de que $0! = 1$.

\end{theorem}

\subsubsection{Permutaciones de $k$-elementos}

Sea un conjunto de elementos $X$ de cardinalidad $|n|$, como en el caso
anterior. Supongamos que deseamos tomar $k$ elementos del conjunto y contar las
diferentes formas en las que podríamos agarrarlos.

De primera instancia, podemos tomar cualquiera de los $n$ objetos como el primer
elemento. Posteriormente, solo hay $n-1$ elementos para la segunda opción y asi
sucesivamente hasta que llegamos al $k$-ésimo elemento. Lo que nos deja
$n-(k-1)$ opciones para la última opción.

Podemos partir de la Ec. (\ref{permutaciones_totales}) para hacer la permutación
de los $n$ elementos, sin embargo, tendríamos que ajustar la fórmula para que
escojamos solo $k$ de los $n$ elementos ($n-k$).
\begin{equation}
n(n-1)(n-2)...(n-(k-1))\overbrace{(n-k)(n-(k+1))...1}^{(n-k)!}
\end{equation}

Para quitar los elementos repetidos, dividimos la ecuación anterior entre el
número de elementos repetidos, de la siguiente manera:

\begin{equation}
 = \frac{ n (n-1) (n-2) ... (n-k+1) \overbrace{(n-k)(n-(k+1))...1}^{(n-k)!}}{\underbrace{(n-k)(n-k-1) ... 1}_{(n-k)!}}
\end{equation}

Lo que nos lleva a la Ec. (\ref{k-permutaciones}).

\begin{theorem}{Permutación}{permutacion}
El total de permutaciones de $n$ objetos distintos tomados de $k$ formas a la
vez, está dado por:

	\begin{equation}
		P(n,r) = \frac{n!}{(n-k)!}
		\label{k-permutaciones}
	\end{equation}
\end{theorem}


\subsubsection{Paridad de una permutación}

En una permutación, ocurre una \textbf{inversi\'on} cuando un elemento mayor
precede a un elemento menor. Para conocer el n\'umero de inversiones se siguen
los siguientes pasos:

\begin{itemize}
\item Tomar el primer elemento de la permutaci\'on
\item Contar los enteros menores a la derecha del elemento en cuesti\'on
\item Realizar los dos pasos anteriores para cada elemento de la permutaci\'on
\item Sumar el total de inversiones contadas para cada elemento
\end{itemize}

\textbf{Ejemplo}:

Se toma la permutación \textbf{$A=\{6,1,3,4,5,2\}$}\\

Primer elemento: $6$; menores a la derecha: $1,3,4,5,2$; n\'umero de
inversiones: \textbf{5}
Primer elemento: $1$; menores a la derecha: $\emptyset$; n\'umero de
inversiones: \textbf{0}
Primer elemento: $3$; menores a la derecha: $2$; n\'umero de inversiones:
\textbf{1}\\
Primer elemento: $4$; menores a la derecha: $2$; n\'umero de inversiones:
\textbf{1}\\
Primer elemento: $5$; menores a la derecha: $2$; n\'umero de inversiones:
\textbf{1}\\

Total de inversiones: \textbf{8}
\begin{tcolorbox}[colback=gray!5!white,colframe=gray!60!black,title=Definición: Tipos de permutaciones]
	\textbf{Permutaci\'on par:} aquella en la que el total de inversiones es un
	entero par.
	
	\tcblower

	\textbf{Permutaci\'on impar}: aquella en la que el total de inversiones es un
	entero impar.
	
\end{tcolorbox}


\subsection{Combinación}

Existe una estrecha relaci\'on entre el n\'umero de combinaciones y el
coeficiente binomial ya que el n\'umero de combinaciones es igual al coeficiente
binomial $\binom{n}{k}$.

Para contar el n\'umero de combinaciones, tenga en cuenta que al seleccionar una
$k$-permutación es lo mismo que seleccionar primero una combinación de $k$ elementos
y luego ordenarlos. Ya que hay $k!$ formas de ordenar los $k$ elementos
seleccionados, vemos que el número de $k$-permutaciones es igual al n\'umero de
combinaciones por $k!$. Por lo tanto, la n\'umero de combinaciones posibles,
viene dado por:

\begin{equation}
	\binom{n}{k} = \frac{n!}{k!(n-k)!}
\end{equation}

Ejemplo: El n\'umero de combinaciones de dos de las cuatro letras A, B, C, y D
se encuentra haciendo que $n = 4$ y $ k = 2$, está dado por:

\begin{center}
	$\binom{4}{2} = \frac{4!}{2!(4-2)!}$
\end{center}


\subsection{El problema de Monty Hall}

El problema de \textit{Monty Hall} o paradoja del presentador es un problema de
probailidad condicional basado en el concurso de televisi\'on estadounidense
\textit{Let's Make a Deal}. El problema fue bautizado as\'i por el nombre del
presentador de dicho programa.

En 1975, el matem\'atico Steve Selvin resuelve este problema y es publicado en
la revista \textit{American Statistician}. Posteriormente fue popularizado por
Marilyn vos Savant en la revista \textit{Parade Magazine}, en 1990. El problema
dice as\'i:

\textbf{Planteamiento del problema:}

Hay tres puertas. Detr\'as de una de ellas, hay un premio. Detr\'as de cada una
de las dos puertas restantes hay una cabra.

Cada puerta tiene la misma probabilidad de contener el premio.

Se te da a seleccionar una de las tres puertas como una oportunidad de ganar el
premio. Sin embargo, una vez que se ha seleccionado la puerta, el presentador
destapa una de ellas y te muestra que detrás hab\'ia una cabra.Una vez dada esa
informaci\'on, el presentador te pregunta si deseas cambiar de puerta o
conservar tu elecci\'on.

La pregunta crucial es:

Dado que el presentador ha destapado una puerta y mostrado que no estaba el
premio, ¿es m\'as conveniente cambiar mi opci\'on o conservar mi elecci\'on de
la puerta inicial?

%\begin{figure}[h]
%	\centering
%	\includegraphics[scale=0.5]{Images/Monty Hall}
%	\caption{Imagen ilustratuiva del problema de Monty Hall}
%\end{figure}

\subsection{Soluci\'on:}
Este problema es un ejercicios relativamente simple de probabilidad condicional.
Sin embargo, se debe tener cuidado, pues es un problema particularmente confuso.
Se abrodará desde dos enfoques.

Dos observaciones son cruciales para entender el problema:

\begin{itemize}
\item El presentador conoce la puerta donde se encuentra el premio y obviamente
no revelará esa puerta. 
\item Monty revelar\'a cualquiera de las otras dos puertas de forma aleatoria y
con con igual probabilidad de que esto ocurra.
\end{itemize}
Supongase que selecciona la puerta 1. Entonces pueden ocurrir tres casos:
\begin{enumerate}
\item El premio está en la puerta 1. Monty abre la puerta 2 o la 3. Si se cambia
la elecci\'on, se pierde. Si se conserva la elecci\'on, ganas.
\item El premio está en la puerta 2. Monty abre la puerta 3. Si se cambia de
elecci\'on, se gana. Si se conserva la elecci\'on, pierdes.
\item El premio está en la pierta 3. Monty abre la puerta 2. Si se cambia de
elecci\'on, se gana. Si se conserva la elecci\'on, pierdes.
\end{enumerate}
Se debe notar que al cambiar de elecci\'on se gana en 2 de los 3 casos; mientras
que al conservar la elecci\'on, se gana s\'olo en 1 de los 3 casos. Por lo
tanto, la probabilidad de ganar si se cambia de puerta, es de $\frac{2}{3}$. Por
otro lado, la probabilidad de ganar con la elecci\'on inicial es de
$\frac{1}{3}$.

Conclusi\'on: ¡Cambia de puerta!



\subsection{Problema del cumpleaños}

Este problema se presenta a menudo como ejemplo en el c\'alculo de
probabilidades Ec. (\ref{eq:probabilidad}). Este problema puede ser enunciado de la siguiente
forma. 

\textit{En un grupo de $n$ personas escogidas al azar, ¿cuál es la probabilidad de que
al menos dos de ellas tengan el mismo d\'ia de cumpleaños?}

Se asume que los nacimientos se distribuyen de manera uniforme a lo largo de
todo el año, y, por simplicidad, también asumimos que no existe el 29 de
febrero.

Otra forma alternativa de presentar el problema es la siguiente:

\textit{¿Cuál es el número mínimo de personas necesario en un grupo para que sea más
probable encontrar al menos dos con el mismo d\'ia de cumpleaños?}

Resolveremos la primera versión del problema, ya que con esto podremos hallar
también la respuesta a la segunda versión.

Soluci\'on.

Comenzaremos por resolver el problema para un tamaño muestral $n = 1, 2$ y $3$,
y posteriormente extenderemos el razonamiento al caso general. Para $n = 1$ , es
decir, cuando s\'olo hay una persona, la probabilidad $P_1$ de coincidir con otra
persona es cero, dado que no existe otra persona. Para $n = 2$, esto es, hay dos
personas, la probabilidad $P_2$ de que coincidan en su d\'ia de cumpleaños de
acuerdo con la Ec. (\ref{eq:probabilidad}) es:


\begin{equation}
P_2 = \frac{365}{365^2} = \frac{1}{365}
\end{equation}

ya que hay 365 maneras diferentes de que su día de cumpleaños coincida
(una por cada día del año), y $365 \cdot 365 = 3652$ maneras diferentes de que
se produzcan sus cumpleaños.

Para $n = 3$ personas, podemos calcular la probabilidad $p_3$ de que al menos
dos de ellas coincidan en su día de cumpleaños, a partir de la Ec.
(\ref{eq:complementRule}), como 1 menos la probabilidad del suceso contrario, es
decir, menos la probabilidad de que todas ellas cumplan años en d\'ias
diferentes.

De esta forma, tenemos:

\begin{equation}
p_3 = 1- \frac{364}{365} = 1 - \frac {364 \cdot 363}{365^2}
\end{equation}

%ya que la segunda persona puede cumplir años en cualquiera de los 365 días del
%año, pero s\'olo en 364 de ellos \'este no coincidirá con el cumpleaños de la
%primera persona; de igual forma, el cumpleaños de la tercera persona no
%coincidir\'a con el cumpleaños de las otras dos en 363 de los 365 casos
%posibles. Aplicando el mismo razonamiento para cualquier $n$ entre 1 y 365,
%obtenemos:
%
%\begin{equation}
%	\begin{array}{l}
%		P_n = 1 - \frac{364}{365} \cdot \frac{364}{365} \cdot \frac{364}{365} ...
%	
%			= 1 - \frac{365 \cdot - n + 1}{365} \\
%	
%			= \frac {364 \cdot 362 \cdot 363 \cdot 362 ... ( 365-n+1)}
%				{365^{n-1}} \\
%	
%			= \frac {364 \cdot 362 \cdot 363 \cdot 362 ... (365-n+1)}
%				{365^{n}} \\
%	
%			= \frac{365!} {365^n \cdot (365-n)!}
%
%	\end{array}
%\end{equation}
%
%Con lo que damos soluci\'on al problema 1. Obviamente, $P_n$ es 1 para todo $n$
%mayor que 365 (en una reuni\'on de más de 365 personas, necesariamente ha de
%haber al menos dos personas con el mismo d\'ia de cumpleaños). En el cuadro 1
%mostramos los valores que toma la probabilidad $p_n$ para algunos valores de n,
%seg\'un la expresi\'on (\ref{e6}) que acabamos de calcular.
%




\subsection{Resumen de técnicas de conteo}

\begin{tcolorbox}[colback=gray!5!white,colframe=gray!60!black,title=Resumen: Técnicas de conteo]

	\begin{itemize}
		\item \textbf{Permutaciones} de $n$ objetos:
		\begin{center}
		$n!$
		\end{center}

		\item \textbf{\textit{k}-Permutaciones} de $n$ objetos:
		\begin{center}
		$\frac{n!}{(n-k)!}$
		\end{center}

		\item \textbf{Combinaciones} de $k$ de $n$ objetos:
		\begin{center}
		$\binom{n}{k} = \frac{n!}{k!(n-k)!}$
		\end{center}

		\item \textbf{Particiones} de $n$ objetos en $r$ grupos, con el \textit{i}-ésimo
		grupo con $n_i$ objetos:
		\begin{center}
			$\binom{n}{n_1,n_2,...,n_r} = \frac{n!}{n_1!,n_2!,...,n_r!}$
		\end{center}

	\end{itemize}
	
\end{tcolorbox}


\newpage
\subsection{Factorial}
En muchas ocasiones, para poder obtener un resultado principal, es necesario el
c\'alculo secundario del factorial de un n\'umero muy grande, resultando en una
limitante computacional. Un ejemplo de esto es el teorema de \textit{Chebichev}.

La f\'ormula Stirling, nombrada así en honor al matem\'atico escoc\'es del siglo
XVIII James Stirling, es una aproximaci\'on para el c\'alculo del factorial de
un n\'umero muy grande y dif\'icil de computar con la definici\'on est\'andar.

La aproximaci\'on de Stirling dice lo siguiente:
\begin{eqnarray}
	n!\approx \sqrt{2\pi n}(\frac{n}{e})^n\\
	n!\approx \sqrt{2\pi n} (n^n)(e^{-n})
	\label{stirling}
\end{eqnarray}

La Ec. \eqref{stirling} permite una forma r\'apida de calcular el factorial.
Aunque el error va disminuyendo de forma exponencial conforme crece el n\'umero,
existe una mejor aproximaci\'on que involucra un t\'ermino m\'as.


\begin{equation}
	n!\approx \sqrt{2\pi n}(\frac{n}{e})^n(1+\frac{1}{12n})
	\label{Stirling}
\end{equation}
\begin{figure}[h!]
	\centering
\includegraphics[scale=0.50]{figures/stirling.png}
\caption{Error de aproximaci\'on stirling respecto del factorial.}
\label{stirling_error}
\end{figure}

\pagebreak
\subsection{Código para realizar la aproximación de Stirling}
\lstinputlisting[language=Python]{../src/stirling.py}



\documentclass[12pt]{article}
\usepackage[spanish,es-tabla,es-nodecimaldot]{babel}
\usepackage{tikz}
\usepackage{amsmath, amssymb, amsbsy}    %Caracteres especiales de matematicas
\usepackage[hidelinks]{hyperref}    % Vinculos [que no estén subrayados]
\usepackage{listings}   %Introducir codigo
\usepackage{setspace}
\usepackage{color}
\usepackage{pdflscape}  %Voltear la página
\usepackage{graphicx}
\usepackage{caption}


\begin{document}
\title{Ejercicios de Técnicas de Conteo}
\author{Ana Maritza Bello Ya\~nez}
\maketitle
\setlength{\parindent}{0pt}
\setlength{\parskip}{1em}

\section*{Ejercicios de Combinatoria (Schaum)}
% Problemas 1.7, 1.8, 1.29, 1.39, 1.40, 1.77, 1.78, 1.87
% 5 ejemplos de muestras ordenadas

\subsection*{Problema 1.7}
Demostrar que un número palindrómico (decimal) de longitud par es divisible por
11.

La prueba inductiva explota el hecho de que cuando se quitan el primer y el
último carácter de un palíndromo, queda un palíndromo. Así, sea $N$ un número
palindrómico de longitud $2k$. Si $k = 1$, el teorema obviamente se cumple. Si
$k \geq 2$, tenemos

$ N = a_{2k-1} 10^{2k-1} + a_{2k-2} 10^{2k-2} + ... + a_{k} 10^{k} + ... +
a_{2k-2} 10^{1} + a_{2k-1} 10^{0} $

$ = a_{2k-1}(10^{2k}+10^0) + (a_{2k-2}10^{2k-2}+...+ a_{2k-2}10^{1}) $

$ \equiv a_{2k-1} P + Q$

Donde:

$P = \underbrace{100...001}_{\text{longitud } 2k} =
\underbrace{9090...9091}_{\text{longitud } 2k-2}$

y ya sea $Q=0$ (divisible por 11) o, para algún $1 \leq r \leq k-1$,

$Q = 10^r$ \{palindrome de longitud $ 2(k-r)\} = 10^r\{11R\} $

donde el último paso se sigue de la hipótesis de inducción. Por lo tanto, $N$ es
divisible por 11 y la demostración está completa.

\subsection*{Problema 1.8}

En un palíndromo binario, el primer dígito es 1 y cada dígito subsiguiente puede
ser 0 o 1. Cuente los palíndromos binarios de longitud $n$.

Tenemos $[(n + 1)/21- 1 ]= [(n - 1)/2]$ posiciones libres, por lo tanto el número
deseado es:

$2^{[(n-1)/2]}$


\subsection*{Problema 1.29}
Encuentre la probabilidad $p_n$ de que un grupo de $n$ personas reunidas al azar
incluya al menos 2 personas con el mismo cumpleaños (día del año).

\textbf{Solución}

Aquí no tratamos con una muestra de personas, sino con una muestra de
cumpleaños, es decir, números enteros del 1 al 365 inclusivo. Nuestra noción de
probabilidad es:

$\text{Probabilidad} = \frac{\text{número de muestras favorables}}{\text{Número total de muestras}}$

En este problema es simple considerar el evento complementario: todos los $n$
cumpleaños son distintos. Este evento es realizado en $P(365,n)$ muestras; y el
total de muestras es $365^n$. Por lo tanto, $1-p_n = P(365,n)/365^n$ o:

$p_n = 1 - \frac{P(365,n)}{365^n} = 1 -
\frac{(365)(365-1)(365-2)...[365-(n-1)]}{365^n} $

$= 1-(1 - \frac{1}{365})(1 - \frac{2}{365})...(1 - \frac{n-1}{365})$

Puede verificarse que $p_n \geq 1/2 $ cuando $n>25$.

\subsection*{Problema 1.39}

\subsection*{Problema 1.40}

\subsection*{Problema 1.77}

\subsection*{Problema 1.78}

\subsection*{Problema 1.87}

\section*{Ejercicios de Probabilidad (Schaum)}
% Problemas 2.29, 2.30, 2.68 al 2.74

\subsection*{Problema 2.29}


\subsection*{Problema 2.30}


\subsection*{Problema 2.68}

\subsection*{Problema 2.69}


\subsection*{Problema 2.70}


\subsection*{Problema 2.71}

\subsection*{Problema 2.72}

\subsection*{Problema 2.73}

\subsection*{Problema 2.74}

\end{document}

\section{Examen}

\textbf{Problema}
Una empresa de manufactura que emplea tres planos analíticos para el diseño y
desarrollo de un productos específico. Por razones de costos los tres se
utilizan en momentos diferentes. De hecho, los planos 1, 2 y 3 se utilizan para
30\%, 20\% y 50\% de los productos respectivamente. La tasa de defectos difiere
en los 3 procedimientos de la siguiente manera,\

\begin{center}
$P(D|P_1)=0.01 P(D|P_2)=0.03 P(D|P_3)=0.02 $ \
\end{center}

En donde $P(D|P_j)$ es la probabilidad de que un producto esté defectuoso, dado
el plano j.\

Si se observa un producto al azar y se descubre que está defectuoso, ¿cuál de
los planos tiene más probabilidades de haberse utilizado y, por lo tanto, de ser
el responsable?\\

\textbf{Solución} \

A partir del planteamiento del problema \

\begin{center}
$P(P_1)=0.30, P(P_2)=0.20 y P(P_3)=0.50$ debemos calcular $P(P_j|D)$ para $j =
1, 2, 3.$
\end{center}

La regla de Bayes muestra que\

\begin{center}
$P(P_1|D)=\frac{(P(P_1)P(D|P1))}{P(P_1)P(P_1|D)+P(P_2)P(D|P_2)+P(P_3)P(D|P_3)} = $
\end{center}

\begin{center}
$\frac{(0.30)(0.01)}{(0.3)(0.01)+(0.20)(0.03)+(0.50)(0.02)}=\frac{0.003}{0.019}=0.158$\ 
\end{center}

De igual manera,\

\begin{center}
$P(P_2|D) = \frac{(0.03)(0.20)}{0.019} = 0.316$ y
\end{center}

\begin{center}
$P(P_3|D)=\frac{(0.02)(0.50)}{0.019}=0.526.$
\end{center}

La probabilidad condicional de un defecto, dado el plano 3, es la mayor de las
tres; por consiguiente, un defecto en un producto elegido al azar tiene más
probabilidad de ser el resultado de haber usado el plano 3.

\textbf{Problema}
Un bit es 0 ó 1, un byte una secuencia de 8 bits. Encuentra (a) El número de
bytes (b) El número de bytes que comienzan con 11 y no terminan con 11 (c) El
número de bytes que comienzan con 11 o terminan con 11. (d) El número de bytes
que comienzan con 11 o terminan con 11

\textbf{Solución}\

(a) $2^8 = 256$.

(b) Si un byte tiene 8 bits, entonces las 4 posiciones de en medio pueden
llenarse de $2^4 = 16$ maneras. 

(c) $2^6 = 64$ bytes comienzan con 11; por lo tanto, menos los $2^4 = 16$
espacios que no terminan en 11, tenemos $64 - 16 = 48$.

(d) Si usamos (b), 64 bytes comienzan con 11; y de igual manera 64 bytes
terminan con 11. Sumándolos, tenemos 64+64 = 128, y cada byte que empieza y
termina con 11 se cuenta doble. Así, la respuesta es 128 - 16 = 112 bytes. 

\textbf{Problema}
Entre un grupo de programadores, 49 estudiaron Pascal. 37 estudiaron COBOL y 21
estudiaron FORTRAN. Si 9 de esos programadores estudiaron Pascal y COBOL, 5
estudiaron Pascal y FORTRAN, 4 estudiaron COBOL y FORTRAN, 3 estudiaron Pascal,
cobol y FORTRAN, ¿cuanto programadores hay en este grupo?

\textbf{Solución}

Pascal = C, Cobol = C, Fortran = F, el número de programadores en el grupo está
dado por: $|P \cup C \cup F|$.

El total de estudiantes está dado por $n_1 = |P|+|C|+|F| = 49+37+21=107$ $n_2
=|P \cap C| + |P \cap F| + |C \cap F| = 9+5+4 = 18$ $n_3 = |P \cap C \cap F| =
3$

Por lo tanto por el principio de inclusión-exclusión $|P \cup C \cup F| = n_1 - n_2 +
n_3 = 107-18+3 = 92$

\textbf{Problema}
¿Cuál es la probabilidad de que un número entero positivo seleccionado de un
conjunto de enteros positivos, que no excedan 100, sea divisible ya sea entre 2
o 5. 

\textbf{Solución}

Sea el $E_1$ el evento en el que el entero seleccionado es divisible entre 2, y
sea $E_2$ el evento en el que es divisible entre 5. Entonces $E_1 \cup E_2$ es
el evento en el que el número seleccionado es divisible entre 2 y entre .

Por otra parte $E_1 \cap E_2$ es el evento en el que el número seleccionado es
divisible entre 2 y entre 6, o equivalentemente, que sea divisible entre 10.

Tenemos que:
$|E_1|=50, |E_2|=20,|E_1 \cap E_2|=10,$

entonces

\begin{center}
$p(E_1 \cup E_2)=p(E_1)+p(E_2)-p(E_1 \cap E_2) = \frac{50}{100} + \frac{20}{100}
- \frac{10}{100} = \frac{3}{5}$
\end{center}


\textbf{Problema}

Los alces moran en cierto bosque. Hay $N$ alces, de los cuales se captura y
etiqueta una muestra aleatoria simple de tamaño $n$ ("muestra aleatoria simple"
significa que todos los $\binom{n}{k}$ conjuntos de $n$ alces son igualmente
probables). Los alces capturados se devuelven a la población y luego se extrae
una nueva muestra, esta vez de tamaño $m$. Este es un método importante que se
usa ampliamente en ecología, conocido como \textit{captura-recaptura}. ¿Cuál es
la probabilidad de que exactamente $k$ de los $m$ alces en la nueva muestra
hayan sido marcados previamente? Suponga que un alce que fue capturado antes no
tiene más o menos probabilidades de ser capturado nuevamente.

\textbf{Solución}

Asumimos que todas las muestras de tamaño $m$ son igualmente probables. Para
tener exactamente k sea elk etiquetado, necesitamos elegir $k$ de los $n$ alces
etiquetados, y luego $m-k$ de los $N-n$ elk no etiquetados. Entonces la
probabilidad es

\begin{center}
$\frac{\binom{n}{k}*\binom{N-n}{m-k}}{\binom{N}{m}}$
\end{center}

para $k$ tal que $0 \leq k \leq n$ y $0 \leq m-k \leq N-n$, y la probabilidad es
0 para todos los demas valores de $k$ (por ejemplo, si $k > n$ la
probabilidad es 0 ya que entonces ni siquiera hay $k$ alces etiquetados en toda
la población).

\textbf{Problema}

Hay 100 pasajeros en fila para abordar un avión con 100 asientos (con cada
asiento asignado a uno de los pasajeros). El primer pasajero en la fila decide
locamente sentarse en un asiento elegido al azar (con todos los asientos
igualmente probables). Cada pasajero subsiguiente toma su asiento asignado si
está disponible y, de lo contrario, se sienta en un asiento disponible al azar.

¿Cuál es la probabilidad de que el último pasajero de la fila se siente en su
asiento asignado?

\textbf{Solución}

Observemos la situación cuando entra el $k$-ésimo pasajero. Ninguno de los
pasajeros anteriores mostró preferencia por el $k$-ésimo asiento frente al
asiento del primer pasajero. Esto en particular es cierto cuando $k=n$. Pero el
$n$-ésimo pasajero solo puede ocupar su asiento o el del primer pasajero. Por lo
tanto la probabilidad es $\frac{1}{2}$.

\textbf{El problema de los dos mentirosos}

Los sujetos $A$ y $B$ dicen la verdad con la probabilidad de $1/3$ y mienten con
una probabilidad de $(2/3)$.

$A$ dice una declaración, y $B$ confirma que la declaración hecha por $A$ es
cierta.

¿Cuál es la probabilidad que $A$ estuviera diciendo la verdad?

\textbf{Solución}

Sea $E$ el evento en el que $A$ hace una declaración y $B$ la confirma. Esto
puede pasar de dos maneras:

\begin{itemize}
    \item $T$: $A$ y $B$, ambos dicen la verdad.
    \item $L$: $A$ y $B$, ambos están mintiendo.
\end{itemize}

\begin{center}
$P(E)=P(T)+P(L)=\frac{1}{3}*\frac{1}{3}+\frac{2}{3}*\frac{2}{3}-\frac{5}{9}$
\end{center}

Dado que observamos $E$, estamos interesados en encontrar la probabilidad de que
el proceso subyacente sea, de hecho, $T$, es decir, $P(T|E)$. De la fórmula de las
probabilidades condicionales, tenemos que:

\begin{center}
$P(T|E)=\frac{P(T \cap E)}{P(E)}$  y $P(T \cap E) = P(T)$
\end{center}

ya que $T \subset E$.

En conclusión:

\begin{center}
$P(T|E) = \frac{P(T)}{P(E)} = \frac{1/9}{5/9} = \frac{1}{5}$
\end{center}

\textbf{La ameba sobreviviente}

Una población comienza con una sola ameba. Para esta y para las generaciones
posteriores, existe una probabilidad de 3/4 de que una ameba individual se
divida para crear dos amebas, y una probabilidad de 1/4 de que se extinga sin
producir descendencia. ¿Cuál es la probabilidad de que el árbol genealógico de
la ameba original continúe para siempre?

\textbf{Solución}

Esta es una caminata aleatoria sesgada disfrazada con un estado de absorción en
0 que comienza en 1. Si esta caminata aleatoria está en $N$, considerar $N$
movimientos corresponde a considerar el destino de $N$ amebas simultáneamente.
Esto produce $1-\frac{1-p}{p}=1-\frac{1/4}{3/4}=\frac{2}{3}$ por los argumentos
habituales.

%\subsection{Duelo justo}
%
%Pepe y Juan se han retado a un duelo. Tomarán turnos disparando uno al otro
%hasta que uno sea herido. Pepe, quien puede atinarle a Juan solamente el 40\% de
%las veces, es el disparador más débil, por lo que le será permitido disparar
%primero. Ellos han determinado que en el duelo no se favorezca a nadie.
%
%¿Cuál es la probabilidad de herir a Pepe?





\section{Probabilidad y conteo}

\subsection{Definici\'on cl\'asica}

La definici\'on cl\'asica de la probabilidad \cite{faber2012statistics} de un
evento $A$, puede ser formulada de la siguiente manera:

\begin{theorem}{Probabilidad}{probabilidad}
    \begin{equation}
        P(A)= \frac{n_A}{n_{total}}
    \end{equation}
    \label{eq:probabilidad}
\end{theorem}

Donde $n_A$ es el n\'umero de formas igualmente probables por las cuales un
experimento puede conducir a A; $n_{total}$ n\'umero total de formas igualmente
probables en el experimento.

- En la definici\'on cl\'asica el experimento no necesariamente se lleva a cabo
ya que la respuesta es conocida de antemano.

- La teor\'ia cl\'asica no da soluci\'on a menos de que todas las formas
igualmente posibles puedan ser derivadas anal\'iticamente.

La definición clásica desprende una serie de propiedades:

\begin{tcolorbox}[colback=gray!5!white,colframe=gray!60!black,title=Axiomas de la probabilidad]

\begin{itemize}
\item $P(A) \geq 0$. La probabilida se he definido como el cociente del número
de casos favorables al suceso $A$ y el número de casos posibles, por lo que el
cociente no puede ser negativo, y su límite inferior 0 se alcanza cuando el
número de casos favorables sea nulo.

\item $P(A) \leq 1$. El número de casos favorables nunca puede ser mayor que el
número total de casos, a lo sumo igual.


Estas dos propiedade conducen  que la probabilidad de un suceso esté acotada, 
\begin{equation}
    0 \leq P(S) \leq 1
    \label{eq:acotadaProb}
\end{equation}

\item Si $A$ y $B$ son conjuntos disjuntos, entonces:

\begin{equation}
    P(A \cup B) = P(A) + P(B)
    \label{eq:probaAunionB}
\end{equation}

\item $P(0)=\emptyset$

\item \textbf{Regla del complemento}

Sea $S$ el espacio muestral de un experimento $E$. Si $A$ es un evento, es decir
$A \subseteq S$, entonces:

\begin{equation}
    P(\bar{A}) = 1 - P(A)
    \label{eq:complementRule}
\end{equation}

\end{itemize}
\end{tcolorbox}



\section{Probabilidad Condicional}

La probabilidad condicional nos provee una forma de razonar acerca de la salida
o resultado de un experimento, basado en información parcial. Algunos ejemplos
de estos experimentos podrían ser los siguientes:

\begin{itemize}
\item En un experimento en el cual tiramos dos dados sucesivamente, te dicen que
la suma de los dados es 9. ¿Qué tan probable es que el primer dado haya caído 6?

\item En un juego de adivinanzas de palabras, la primera letra de la palabra es
\textit{t}. ¿Cuál es la probabilidad de que la siguiente palabra sea \textit{h}?

\item ¿Qué  tan probable es que una persona tenga cierta enfermedad dado que su
examen médico dió negativo?
\end{itemize}

En términos más precisos, dado un experimento, un espacio de muestreo y una ley
de probabilidad, suponiendo que sabemos que la salida está dentro de un evento
dado $B$. Deseamos cuantificar la probabilidad de que la salida pertenece a
algún otro evento $A$. Así, podemos construir una nueva probabilidad que tome en
cuenta el conocimiento disponible: una ley de probabilidad para cuaquier evento
$A$. Especificamente, la probabilidad condicional de $A$ dado $B$, denotado por
$P(A|B)$.

Para calcular $P(A|B)$ consideramos aquellos resultados del evento $A$ que están en
el evento $B$. Esto nos da los resultados en el evento $A \cap B$ y nos lleva al
teorema de la probabilidad condicional.

\begin{theorem}{Probabilidad Condicional}{conditional_probability}
Sea $S$ el espacio de muestro para un experimento $E$ y $A$, $B \subseteq S$,
entonces la probabilidad condicional de $A$ dado $B$ está dada por:
    \begin{equation}
        P(A|B) = \frac{P(A \cap B)}{P(B)}
        \label{eq:conditionalProbability}
    \end{equation}

Siempre que $P(B)>0$.

Particularmente, todas las propiedades de las leyes de la probabilidad
permanecen válidas para las leyes de la probabilidad condicional.

\begin{itemize}
    \item La probabilidad condicional puede verse también como una ley de
    probabilidad en un nuevo universo $B$, ya que toda la probabilidad condicional
    está concentrada en $B$.

    \item Si todos los resultados son finitos e igualmente probables, entonces:
        \begin{equation}
            P(A|B)=\frac{\text{número de elementos de }A \cap B}{\text{número de elementos de }B}
        \end{equation}
\end{itemize}
\end{theorem}


\def\firstcircle{(0,0) circle (1.5cm)}
\def\secondcircle{(0:2cm) circle (1.5cm)}
\def\rectangle{(-2,-2) rectangle (4,2)}

\colorlet{circle edge}{black!50}
\colorlet{circle area}{gray!20}

\tikzset{filled/.style={fill=circle area, draw=circle edge, thick},
    outline/.style={draw=circle edge, thick}}

\setlength{\parskip}{5mm}

\begin{figure}[h]
    \centering
    \begin{tikzpicture}
        \draw \rectangle;
        \begin{scope}
            \clip \firstcircle;
            \fill[filled] \secondcircle;
        \end{scope}
        \draw[outline] \firstcircle node {$A$};
        \draw[outline] \secondcircle node {$B$};
        \node[anchor=south] at (current bounding box.south) {$A \cap B$};
        \node at (3.5,1.5){$\mathbb S$};
    \end{tikzpicture}
    \caption{Representación del concepto intuitivo de la probabilidad condicional.}
    \label{fig:coditionalProbability}
\end{figure}

\def\firsttcircle{(90:1.75cm) circle (1.5cm)}
\def\seconddcircle{(170:1.75cm) circle (1.5cm)}
\def\thirddcircle{(-25:2cm) circle (1.5cm)}
\begin{figure}
    \centering
\begin{tikzpicture}
      \begin{scope}
    \clip \seconddcircle;
    \fill[cyan] \thirddcircle;
      \end{scope}
      \begin{scope}
    \clip \firsttcircle;
    \fill[cyan] \thirddcircle;
      \end{scope}
      \draw \firsttcircle node[text=black,above] {$A$};
      \draw \seconddcircle node [text=black,below left] {$B$};
      \draw \thirddcircle node [text=black,below right] {$C$};
\end{tikzpicture}
\end{figure}

\textbf{Observaciones y consideraciones de la probabilidad condicional:}

\begin{itemize}
\item La probabilidad $P(A|B)$ es una actualización de $P(A)$, basada en el
conocimiento de que ocurrió el evento $B$.

\item De la Ec.(\ref{eq:conditionalProbability}), tanto $P(A \cap B)$ como
$P(A)$ se calculan a partir del espacio muestral original.

\item Las probabilidades tienen sutiles cambios dependiendo de la información
exacta de la condición implicada en el evento A.
\end{itemize}

\subsection{Propiedades de las leyes de la probabilidad}

Las leyes de la probabilidad tienen propiedades que pueden deducirse de los
axiomas. Algunas de ellas, son las que se muestran a continuación.

\begin{tcolorbox}[colback=blue!5!white,colframe=blue!60!black,title=Resumen: Propiedades de las leyes de la probabilidad]
    Sean $A$, $B$ y $C$ eventos:

    \begin{itemize}
        \item
        \begin{equation}
            \text{Si } A \subset B, \text{entonces } P(A) \leq P(B)
            \label{eq:propProb1}
        \end{equation}
        
        \item 
        \begin{equation}
        P(A \cup B) = P(A) + P(B) - P(A \cap B)
        \label{eq:propProb2}
        \end{equation}

        \item 
        \begin{equation}
            P(A \cup B) \leq P(A) + P(B)
            \label{eq:propProb3}
        \end{equation}

    \item
    $P(A \cup B \cup C)=$
    \begin{equation}
        P(A) + P(B) + P(C) - P(A \cap B) - P(A \cap C) - 
        P(B \cap C) + P(A \cap B \cap C)
        \label{eq:propProb4}
    \end{equation}
    \end{itemize}
\end{tcolorbox}

De la Ec.(\ref{eq:conditionalProbability}) podemos hacer que:

\begin{center}
$P(B \cap A) = P(A \cap B) = P(A) \  P(B|A)$, 
\end{center}

y cambiando los roles de $A$ y de $B$, tenemos que:

\begin{center}
    $P(A \cap B) = P(B \cap A) = P(B) \ P(A|B)$, 
\end{center}

esto resulta en:

\begin{center}
    $P(A)P(B|A) = P(A \cap B) = P(B) \ P(A|B)$,
\end{center}

que comunmente llamamos \textit{regla de la multiplicación}.

Para verificar algunas de las propiedades de las leyes de la probabilida,
podemos usar diagramas de Venn, Fig. (poner diagramas de Venn). Si $A \subset
B$, entonces $B$ es la unión de eventos los eventos disjuntos $A$ y $A_c \cap
B$, por lo que por el axioma de adición, tenemos:

\begin{center}
    $P(B) = P(A) + P(A^c \cap B) \geq P(A)$.
\end{center}

Como observamos en la Fig. (\ref{fig:difer}), podemos expresar los eventos $A
\cup B$ y $B$ como unión de conjuntos disjuntos:

\begin{center}
    $A \cup B = A \cup (A^c \cap B)$,
\end{center}

\begin{center}
    $B = (A \cap B) \cap (A^c \cap B)$
\end{center}

Usando el axioma de adición, tenemos:

\begin{center}
    $P(A \cup B) = P(A) + P(A^c \cap B)$,
\end{center}

\begin{center}
    $P(B) = P(A \cap B) + P(A^c \cap B)$
\end{center}

Si despejamos $P(A^c \cap B)$ de la seguda igualdad y la sustituimos en la
primera, tenemos:

\begin{center}
    $P(A \cup B) = P(A) + P(B) - P(A \cap B)$,
\end{center}

con lo que podemos verificar la Ec. \eqref{eq:propProb3}

\begin{figure}
% Set B but not A
\centering
\begin{tikzpicture}
    \draw \rectangle;
    \begin{scope}
        \clip \secondcircle;
        \draw[filled, even odd rule] \firstcircle
                                     \secondcircle node {$B$};
    \end{scope}
    \draw[outline] \firstcircle node {$A$}
                   \secondcircle;
    \node[anchor=south] at (current bounding box.north) {$A^c \cap B$};
    \node at (1,0){$A \cap B$};
\end{tikzpicture}
\caption{Representación gráfica de los eventos $A \cup B$, como unión de eventos disjuntos.}
\label{fig:difer}
\end{figure}

\subsection{Teorema de la probabilidad total}
\begin{theorem}{Teorema de la probabilidad total}{TotalProbabilityTheorem}
Sean $A_1, ..., A_n$ eventos disjuntos que forman una partición de un espacio
muestral, Fig(poner figura!!), y asumiendo $P(A_i)>0$, para todo $i$. Entonces,
para cualquier evento $B$, tenemos:

\begin{center}
    $P(B) = P(A_1 \cap B) + ... + P(A_n \cap B)$
\end{center}

\begin{equation}
    =P(A_1)\ P(B|A_1) + ... + P(A_n) \ P(B|A_n)
\end{equation}

\end{theorem}


\subsection{Teorema de Bayes}

\begin{theorem}{Teorema de Bayes}{BayesTheorem}
Sean $A$ y $B$ dos eventos cuyas probabilidades son diferentes de cero,
entonces:
    \begin{equation}
        P(B|A) = \frac{P(A|B) \ P(B)}{P(A)}
        \label{eq:BayesTheorem}
    \end{equation}

La implicación más importante de la Ec. (\ref{eq:BayesTheorem}) es que permite
encontrar probabilidades condicionadas $P(B|A)$ en términos de $P(A|B)$, cuando
esta última resulta más fácil de calcular directamente.
\end{theorem}



\section{Convenio de suma de Einstein}

Se denomina notación de Einstein o notación indexada a la convención utilizada
para abreviar la escritura de las sumatorias, donde se suprime el término de la
sumatoria $(\sum)$. Este convenio fue introducido por Albert Einstein en 1916.

Dada una expresión lineal en $\mathbb{R^n}$ en la que se escriban todos sus
términos en forma explícita:

\begin{equation}
    u = u_1 + u_2 + ... = u_n
\end{equation}

Se puede escribir de la forma:

\begin{equation}
    u = \sum_{i=1}^{n} u_ix_i
\end{equation}

La notación de Einstein obtiene una expresión aún más condensada eliminando el
signo de la sumatoria y entendiendo que la expresión resultante en un índice
indica la suma sobre todos los posibles valores del mismo.

\begin{equation}
    u = u_ix_i
\end{equation}

\subsection{Tensor Levi-Civita}
Está definido por:

\begin{equation}
    \epsilon_{ijk}=
    \left\lbrace\begin{array}{clllr} 
        +1 & ijk, & kji, & jki & \text{Permutación cíclica o par}\\ 
        -1 & ikj, & kji, & jik & \text{Permutación anticíclica o impar}\\
        0  & iii, & jjj, & kkk & \text{Si se repiten}
    \end{array}\right.
\end{equation}

\subsection{Producto escalar de dos vectores}

También conocido como producto punto, es una operación algebraíca que toma dos
vectores y retorna un escalar. esta definido de la siguiente manera:

\begin{equation}
    \bar{A} \cdot \bar{B}= \bar{\abs{A}} \bar{\abs{B}} \cos \theta
    = \sum_{i=1}^{n} a_i b_i
\end{equation}

Que de acuerdo al convenio de suma de Einstein podemos escribir como:

\begin{equation}
    \bar{A} \cdot \bar{B} = a_i b_i
\end{equation}

\subsection{Producto vectorial de dos vectores}

EL producto vectorial o producto cruz de dos vectores es una operación binaria
entre dos vectores en un espacio tridimensional. El resultado es un vector
perpendicular a los vectores que se multiplican, y por lo tanto normal al plano
que los contiene.

Está definido de la siguiente manera:

\begin{equation}
    \bar{A} \times \bar{B}= \bar{\abs{A}} \bar{\abs{B}} \sin \theta \hat{r}
\end{equation}

Donde $\hat{r}$ es el vector unitario y ortogonal a los vectores $\bar{A}$ y
$\bar{B}$, y $\theta$ el ángulo entre $\bar{A}$ y $\bar{B}$.

Que mediante determinantes tenemos:

\begin{equation}
    \bar{A} \times \bar{B}=
    \begin{bmatrix}
        \hat{i} & \hat{j} & \hat{k}\\ 
         a_1 & a_1 & a_1\\ 
         b_1 & b_1 & b_1
    \end{bmatrix}
\end{equation}

Desarrollamos:

\begin{equation}
    \bar{A} \times \bar{B}= \hat{i}(a_2 b_3 - a_3 b_2) - \hat{j}(a_1 b_3 - a_3 b_1) + \hat{k}(a_1 b_2 - a_2 b_1)
\end{equation}

Desde un punto de vista tensorial el producto generalizado de $n$ vectores vendrá
dado por:

\begin{equation}
    (\bar{A} \times \bar{B})_{i} = \epsilon_{ijk} a_{j} b_{k}
\end{equation}

Desarrollando:

\begin{equation}
    \epsilon_{ijk} a_{jbk} =
    \begin{array}{ccccc}
        \cancel{\epsilon_{111} a_1 b_1} & + & \cancel{\epsilon_{112} a_1 b_2}   & + & \cancel{\epsilon_{112} a_1 b_3}   \\
        \cancel{\epsilon_{121} a_2 b_1} & + & \cancel{\epsilon_{122} a_2 b_2}   & + & \epsilon_{123} a_2 b_3            \\
        \cancel{\epsilon_{131} a_3 b_1} & + & \epsilon_{132} a_3 b_2            & + & \cancel{\epsilon_{133} a_3 b_3}
    \end{array}
    = a_2 b_3 - a_3 b_2
\end{equation}


\subsection{Congruencia de Zeller}

La congruencia de Zeller es un algoritmo que permite obtener, a partir de una
fecha, el día de la semana que le corresponde.

Se atribuye su creación a Julius Christian Johannes Zeller, un sacerdote
protestante alemán que vivió en el siglo XIX.

Zeller observó que existía una dependencia entre las fechas del calendario
gregoriano y el día de la semana que les correspondía. A raíz de esa
observación, obtuvo (se dice que por tanteo), esta fórmula, en apariencia
mágica, que lleva su nombre.

La fórmula en sí es muy sencilla, y se basa en algunas operaciones de aritmética
modular (el resto, también llamado módulo, de las divisiones)

Es necesario tener en cuenta que la fórmula presentada a continuación es válida
sólo para el calendario gregoriano, promulgado por el papa Gregorio XIII en
1582, pero adoptado en distintas fechas en cada país.

Para el caledario gregoriano la congruencia de Zeller es:

\begin{equation}
    h = (q + [\frac{13(m+1)}{5}] + K [\frac{K}{4}] + [\frac{J}{4}] - 2J ) \text{ mod } 7
\end{equation}

Para el calendario juliano es:

\begin{equation}
    h = (q + [\frac{13(m+1)}{5}] + K [\frac{K}{4}] + 5 - J ) \text{ mod } 7
\end{equation}

Donde:

\begin{itemize}
    \item $h$ es el día de la semana.
    \item $q$ es el día del mes.
    \item $m$ es el mes.
    \item $K$ el año del siglo (año mod 100)
    \item $J$ es el siglo de base cero.
\end{itemize}

Veáse código en Sec. (\ref{sec:zeller})


\section{Matriz de rotaci\'on}

fig.(\ref{permutaciones_tree})\\
\begin{figure}[h]
		\begin{eqnarray*}
			x^{'m} = a^{m}\hfill_{\nu} x^{\nu}\\
			m = 0,1,2,3\\
			\nu = 0,1,2,3
		\end{eqnarray*}		
		\begin{eqnarray*}
			x^{'(0)} =
			\left( {\begin{array}{cc}
					a^{0} _0 x^{_0} + a^{0} _1 x^{_1} + a^{0} _2 x^{_2} + a^{0} _3 x^{_3}\\
			\end{array} } \right)\\
		x^{'(1)} =
		\left( {\begin{array}{cc}
				a^{1} _0 x^{_0} + a^{1} _1 x^{_1} + a^{1} _2 x^{_2} + a^{1} _3 x^{_3}\\
		\end{array} } \right)\\
		x^{'(2)} =
		\left( {\begin{array}{cc}
				a^{2} _0 x^{_0} + a^{2} _1 x^{_1} + a^{2} _2 x^{_2} + a^{2} _3 x^{_3}\\
		\end{array} } \right)\\
		x^{'(3)} =
		\left( {\begin{array}{cc}
				a^{3} _0 x^{_0} + a^{3} _1 x^{_1} + a^{3} _2 x^{_2} + a^{3} _3 x^{_3}\\
		\end{array} } \right)
		\end{eqnarray*}
		\begin{equation*}
			\left(
			\begin{array}{ccc}
				x^{'0}\\
				x^{'1}\\
				x^{'2}\\
				x^{'3}\\ 
			\end{array}
			\right)
			=
			\left(
			\begin{array}{c}
				a^{m}\nu
			\end{array}
			\right)
			{}
			\left(
			\begin{array}{ccc}
				x^{0}\\
				x^{1}\\
				x^{2}\\
				x^{3}\\ 
			\end{array}
			\right)
		\end{equation*}		
	\caption{Matriz de rotaci\'on}
	\label{permutaciones_tree}
\end{figure}


\chapter{Matrices de rotaci\'on}
	Una matriz de rotaci\'on es una matriz que representa una rotaci\'on en el espacio euclidiano.
	Las matrices de rotaci\'on pueden ser en dos o tres dimensiones.
	
	Por ejemplo, si la matriz de rotaci\'on es de dos dimensiones y se busca rotar en sentido horario, la matriz tendrá la siguiente forma:
	\begin{equation}
		R(\theta)=
		\begin{bmatrix}
			\cos \theta & -\sin \theta\\
			\sin \theta & \cos \theta
		\end{bmatrix}
	\end{equation}
	De esta manera, las nuevas coordenadas, una vez aplicando la rotación son de la siguiente manera:
	\begin{eqnarray}
		\left(
		\begin{matrix}
			x'\\
			y'
		\end{matrix}
		\right)
		=
		\begin{bmatrix}
			\cos \theta & -\sin \theta\\
			\sin \theta & \cos \theta
		\end{bmatrix}
		\left(
		\begin{matrix}
			x\\
			y
		\end{matrix}
		\right)\\
		x'=x\cos \theta -y\sin \theta\\
		y'=x\sin \theta + y \cos \theta
	\end{eqnarray}

En caso de querer que la rotación sea antihoraria, se utiliza la siguiente matriz:
\begin{equation}
	R(-\theta)=
	\begin{bmatrix}
		\cos \theta & \sin \theta\\
		-\sin \theta & \cos \theta
	\end{bmatrix}
\end{equation}

Si lo que se busca es realizar rotaciones en el espacio tridimensional, se deben modificar las matrices de la siguiente manera:
\begin{equation}
	R_x(\theta)=
	\begin{bmatrix}
		1 & 0 & 0\\
		0 & \cos \theta & \sin \theta\\
		0 & -\sin \theta & \cos \theta
	\end{bmatrix}
\end{equation}

\begin{equation}
	R_y(\theta)=
	\begin{bmatrix}
		\cos \theta & 0 & \sin \theta\\
		0 & 1 & 0\\
		-\sin \theta & 0 & \cos \theta
	\end{bmatrix}
\end{equation}

\begin{equation}
	R_z(\theta)=
	\begin{bmatrix}
		\cos \theta & -\sin \theta & 0\\
		\sin \theta & \cos \theta & 0\\
		0 & 0 & 1
	\end{bmatrix}
\end{equation}

\subsection*{Ángulos de Euler}

Los ángulos de Euler constituyen un conjunto de tres coordenadas angulares que
sirven para especificar la orientación de un sistema de referencia de ejes
ortogonales, normalmente móvil, respecto a otro sistema de referencia de ejes
ortogonales normalmente fijos.

Dados dos sistemas de coordenadas $xyz$ y $XYZ$ con origen común, es posible
especificar la posición de un sistema en términos del otro usando tres ángulos
$\alpha$, $\beta$, $\gamma$.

La definición matemática es estática y se basa en escoger dos planos, uno en el
sistema de referencia y otro en el triedro rotado. En el esquema adjunto serían
los planos $xy$ y $XY$. Escogiendo otros planos se obtendrían distintas convenciones
alternativas, las cuales se llaman de Tait-Bryan cuando los planos de referencia
son no-homogéneos (por ejemplo $xy$ y $XY$ son homogéneos, mientras $xy$ y $XZ$ no lo
son).

La intersección de los planos coordenados $xy$ y $XY$ escogidos se llama línea de
nodos, y se usa para definir los tres ángulos:

\begin{itemize}
    \item $\alpha$ es el ángulo entre el eje x y la línea de nodos.
    \item $\beta$  es el ángulo entre el eje z y el eje Z.
    \item $\gamma$  es el ángulo entre la línea de nodos y el eje X.
\end{itemize}

más adelante se establecerá que los tres ángulos de Euler descritos son los
valores de las tres rotaciones intrínsecas que describen el sistema.

Notar que también se considera la notación: $\alpha =\phi$, $\gamma =\psi$,
$\beta =\theta$

Las rotaciones de Euler son los movimientos resultantes de variar uno de los
ángulos de Euler dejando fijo los otros dos. Son los siguientes:

- Preseción
- Nutación
- Rotación intrínseca

Si escribimos la rotación de ángulos $\phi$
,$\theta$ ,$\psi$ como una composición de estas tres rotaciones:

\begin{equation}
    A(\phi ,\theta ,\psi )=R(\phi ,\theta ,\psi )N(\phi ,\theta )P(\phi)\
\end{equation}

entonces se cumple:

\begin{equation}
    \begin{array}{ll}
    A(\delta \phi +\phi ,\theta ,\psi ) & = P(\delta \phi )A(\phi ,\theta ,\psi) \\
    A(\phi ,\delta \theta +\theta ,\psi ) & = N(\phi ,\delta \theta )A(\phi ,\theta ,\psi ) \\
    A(\phi,\theta ,\delta \psi +\psi ) &=R(\phi ,\theta ,\delta \psi )A(\phi ,\theta ,\psi) \\
    \end{array}
\end{equation}

Como consecuencia de estas propiedades, estas rotaciones son conmutativas
entre ellas:

\begin{equation}
    P(\delta \phi )N(\delta \theta )A(\phi ,\theta ,\psi)=A(\delta \phi +\phi ,\delta \theta +\theta ,\psi )=N(\delta \theta )P(\delta \phi )A(\phi ,\theta ,\psi )
\end{equation}

lo que también podría verse intuitivamente usando la analogía entre los ángulos
de Euler y los de un soporte Cardán.

%\begin{figure}[h]
    \includegraphics[scale=0.7]{figures/euler_angles.png}
%    \caption{Unos gatos molones}
%    \label{fig:gatos}
%\end{figure}


\pagebreak
\section{Teorema de la incompletitud matemática de Gödel}

\begin{wrapfigure}{l}{0.4\textwidth}
    \includegraphics[width=5cm]{figures/Kurt_Godel.jpg}
    \caption{Gödel como estudiante a los 19 años de edad, 5 años antes de la 
    demostración de los teoremas.}
\end{wrapfigure}

Gödel, nacido el 28 de abril de 1906 en Brno (hoy parte de la República Checa),
demostró como parte de su tesis doctoral, en 1929, el primer teorema de
incompletitud, que afirma que todo sistema axiomático coherente que englobe las
propiedades aritméticas básicas de los números naturales padece una dicotomía
\footnote{División de un concepto o una materia teórica en dos aspectos,
especialmente cuando son opuestos o están muy diferenciados entre sí.}
fundamental: o bien no se puede implementar en un algoritmo, o bien los axiomas
no son capaces de determinar la veracidad de todos los enunciados posibles.
Gödel demostró que en todo “universo matemático” imaginable habrá propiedades
que no podemos demostrar.

Los enunciados matemáticos indecidibles son imposibles de evitar. Un ejemplo es
la hipótesis del continuo, que afirma que todo subconjunto infinito de los
números reales se puede identificar con los números naturales o con los reales.
Otro ejemplo es el axioma de elección, que afirma que dada una colección de
cajas (o conjuntos) no vacías, es posible escoger un elemento de cada caja.
Puede resultar sorprendente que este axioma sea problemático, sobre todo si sólo
pensamos en un número finito de cajas. Sin embargo, en un contexto infinito,
tiene consecuencias inesperadas: Stefan Banach y Alfred Tarski demostraron
partiendo de dicho axioma que se puede descomponer una bola de madera maciza en
un número finito de piezas, las cuales, recolocadas de cierta manera, dan como
resultado dos bolas del mismo volumen.

El teorema de Gödel o teorema de incompletitud limita las posibilidades de las
matemáticas de demostrar fórmulas a través de la deducción. Dibuja el límite a
lo que es posible conocer a través de la lógica formal tal y como se plantea en
física y otras disciplinas.  Muy resumidamente hablando, el teorema de Gödel
viene a decir que «no se puede demostrar cualquier fórmula matemática, aunque
sea verdadera».

La demostración del teorema de incompletitud se apoya en dos ideas claves: por
un lado, Gödel tuvo la destreza de codificar frases y enunciados a través de
números. Al hablar de números, ahora hablamos también de enunciados. Por otro
lado, utilizó un argumento diagonal semejante al que usó Georg Cantor para
demostrar que, pese a que hay tantos números racionales como naturales, hay
muchos más números reales que naturales.

A raíz de los trabajos de Gödel se consolidaron diversas disciplinas dentro de
la lógica matemática: por un lado, la teoría de conjuntos como paradigma de un
formalismo autosuficiente, la teoría de la recursión y de la demostración, con
un enfoque sintáctico y algorítmico, así como la teoría de modelos, en la que
trabajamos los autores de este artículo, que se concentra en las propiedades
semánticas de los objetos matemáticos.

David Hilbert (1862-1943) se preguntaba si las matemáticas eran completas,
consitentes y decidibles.

El teorema de la incompletitud de Gödel significa que la verdad y la
demostrabilidad no son lo mismo en absoluto. Hilbert estaba equivocado. Siempre
habrá afirmaciones verdaderas sobre la matemática que no podamos demostrar.

También demostro que la matemática no es consistente (libre de contradicciones),
ya que cualquier sistema formal consistente de la matemática es incapaz de
probar su propia consistencia (paradojas de autoreferencia). Así que tomando
ambos teoremas de la incompletitud de Gödel, estos dicen que lo mejor a lo que
podemos aspirar es a un sistema consistente pero incompleto de la matemática.
Pero un sistema coo ese no puede probar su propia consistencia, por lo que
algunas contradicciones podrían surgir en el futuro revelando que el sistema en
el que se trabaja ha sido inconsistente todo el tiempo.

% Eso nos deja la tercera y última pregunta de Hilbert; ¿es la matemática
% decidible?, ¿Hay algún algoritmo que pueda siempre determinar si una afirmación
% surge de los axiomas?.


% doxygen



\section{Axiomas de Peano}\label{sect: Axiomas de Peano}
Los axiomas de Peano son un sistema de postulados para la aritmética ideados por el matem\'atico Giuseppe Peano en el siglo XIX para definir los n\'umeros naturales.

Estos axiomas fueron publicados en 1889 en un art\'iculo denominado \textit{Aritmetices principia, nova methodo exposita}.

Los cinco axiomas son los siguientes:
\begin{enumerate}
	\item $N(1)$: El \textbf{1} es un n\'umero natural.
	\item $\forall x(N(x)\to N(x'))$:Todo n\'umero natural, tiene un sucesor \textbf{$n^*$}
	\item $\neg \exists x(N(x)\land 1=x')$: El \textbf{1} no es el sucesor de ning\'un n\'umero natural.
	\item $\forall x \forall y ((N(x)\land N(y)\land x'=y')\to x=y)$: Si hay dos n\'umeros naturales n y m con el mismo sucesor, entonces n y m son el mismo n\'umero natural.
	\item $\left(\phi(1)\land \forall x(\phi(x)\to \phi(x'))\right) \to \forall x \phi(x)$: Si el 1 pertenece a un conjunto de n\'umeros naturales, y dado un elemento cualquiera, el sucesor tambi\'en pertenece al conjunto. Entonces, todos los n\'umeros naturales pertencen a ese conjunto.
\end{enumerate}

Cabe mencionar que el axioma 5 es el principio de la inducci\'on matem\'atica.

Existe un debate para determinar si el $0$ pertenece o no al conjunto de los n\'umeros naturales. Sin embargo, se tomar\'a en cuenta dependiendo de la aplicaci\'on. Es por esa raz\'on que los 5 axiomas existen en la versi\'on donde el $0$ es el n\'umero natural. As\'i, los axiomas 1 al 5, donde se indique el n\'umero 1, se cambia por el 0.

Adem\'as de sus 5 axiomas, Peano define la suma y la multiplicaci\'on:
\begin{itemize}
	\item suma
	\begin{eqnarray*}
		\forall n(n+1=n')\\
		\forall n\forall m(n+m'=(n+m)')
	\end{eqnarray*}
	\item Multiplicaci\'on
	\begin{eqnarray*}
		\forall n(nx1=n)\\
		\forall n \forall m(nxm'=(nxm)+n)
	\end{eqnarray*}
\end{itemize}
De igual manera, cuando interviene el 0, \'este se colocar\'a en lugar del 1.

\newpage
\section{Ley de Benford}
\subsection*{Introducción}
En 1981, Simon Newcomb publicó un art\'iculo que comienza diciendo que la
frecuencia de aparición de los 10 dígitos en algún número no era equitativa y
que esto debía ser evidente al observar las tablas de logaritmos y notar que las
primeras p\'aginas estaban m\'as que las del final. Esto result\'o para Newcomb
en el siguiente enunciado:

\textit{La ley de probabilidad de ocurrencia de los n\'umeros es tal que la mantisa 
de su logaritmo es equiprobable.}

Despu\'es de estos resultados, en 1937, Frank Benford retoma los resultados de
Newcomb y publica su \textit{Ley de los n\'umeros an\'omalos}.

El m\'etodo de estudio consiste en seleccionar cualquier tabla de datos que no
esté restringida en rango numérico o condicionada de alguna manera y hacer el
conteo del n\'umero de veces que los n\'umeros naturales del 1 al 9 curren como
primer dígito. Si aparece un punto decimal o un cero antes del primer n\'umero
natural, estos deben ser ignorados.
\subsection*{Ley para n\'umeros grandes}
Para comrpobar su hip\'otesis, Benford recolectó informaci\'on de distintas
fuentes. Estos datos se encuentran naturalmente de forma aleatoria. Una vez que
realizó el conteo de aproximadamente 20,229 observaciones, obtuvo el promedio de
apariciones de cada dígito en la primera posici\'on de los n\'umeros
Fig.(\ref{T1Benford}). 

\begin{figure}[h]
	\centering
	\includegraphics[scale=0.4]{figures/benford_table.jpg}
	\caption{Tabla de porcentajes de aparici\'on de los n\'umeros naturales para diferentes experimentos.}
	\label{T1Benford}
\end{figure}

\begin{figure}[h]
	\centering
	\includegraphics[scale=0.4]{figures/benford_frequency.jpg}
\caption{Gr\'afica de comportamiento para la Ec\eqref{BenfordFrecuency}.}
	\label{G1Benford}
\end{figure}
De la Fig.(\ref{T1Benford}), si se toma el promedio final de cada columna y se
estudia, por ejemplo, el 0.306, en lugar de 30.6. Se debe notar que la
frecuencia de aparici\'on del 1 como primer dígito es equivalente a $log 2$;
mientras que la frecuencia de aparici\'on del 2 como primer d\'igito (promedio
0.185), es igual al $log 3-log 2$ Ec.\eqref{BenfordFrecuency}. Este
comportamiento persiste hasta el promedio de la columna 10, donde 0.047 es
equivalente al $log \frac{10}{9}$.

Así, la frecuencia con la que aprece cada uno de los 9 n\'umeros como primer
d\'igito est\'a dada por la Ec.\eqref{BenfordFrecuency}. Mientras que en la
Fig.(\ref{G1Benford}) se muestra la gr\'afica de comportamiento de la frecuencia
con la que aparece cada uno de los n\'umeros como primer d\'igito.
\begin{equation}
	F_a=log(\frac{a+1}{a})
	\label{BenfordFrecuency}
\end{equation}

\subsection*{Comportamiento para la q-\'esima posici\'on}
En el caso de la frecuencia de aparici\'on de cada d\'igito como segunda
posici\'on, el cero debe ser contemplado, así como el d\'igito $a$ de la primera
posici\'on. De acuerdo con un sistema posicional, sea $ab$ el n\'umero formado
por el d\'igito de la posici\'on 1 y 2, respectivamente. 

De acuerdo con la Ec.\eqref{BenfordFrecuency}, La frecuencia de aparici\'on del
n\'umero compuesto como primer n\'umero es
\begin{equation}
	F_ab=log (\frac{ab+1}{ab})
\end{equation}
Sin embargo, no se ha considerado la probabilidad de $a$ en la primer posici\'on
de $ab$. Por lo tanto, el problema se puede modelar de forma similar a una
probabilidad condicional: $P(ab|a)$. De esta forma, se tiene la siguiente
expresi\'on
\begin{equation}
	F_b=\frac{log (\frac{ab+1}{ab})}{log(\frac{a+1}{a})}
\end{equation}
De forma m\'as general...
\begin{equation}
	F_q=\frac{log(\frac{(abc...pq)+1)}{abc...pq})}{log(\frac{(abc...op)+1}{abc...p})}
\end{equation}

\pagebreak
\section{Explosi\'on del transbordador espacial Challenger: un an\'alisis estad\'istico del accidente del Challenger}


Una comprensi\'on profunda del desastre requiere un an\'alisis estad\'istico de
los datos disponibles de las juntas t\'oricas que contienen informaci\'on sobre
la temperatura y las tasas de \'exito/fallo. Analizar las fallas de las juntas
t\'oricas requiere una comprensi\'on de la estructura subyacente de Challenger.
Los propulsores de cohetes del transbordador ten\'ian cuatro segmentos. 

\begin{wrapfigure}{r}{0.5\textwidth} \vspace{-20pt} 
    \begin{center}
\includegraphics[width=0.48\textwidth]{figures/challenger.jpg} \end{center}
\vspace{-20pt} \caption{Challenger} \vspace{-10pt} \end{wrapfigure} 

En primer lugar, lleno de combustible. En segundo lugar, oxidante. Y la NASA los
ensambl\'o y sell\'o con juntas t\'oricas, un componente cr\'itico que evit\'o
la fuga de gas caliente durante el lanzamiento.Sin embargo, la capacidad de
respuesta de los sellos no se prob\'o en temperaturas fr\'ias. \\


El d\'ia del lanzamiento, la temperatura era de 31 grados Fahrenheit,
significativamente m\'as baja que las temperaturas en las que se probaron las
juntas t\'oricas. Despu\'es del despegue, una de las juntas t\'oricas se
rompi\'o, lo que provoc\'o que el gas caliente calentara el ox\'igeno l\'iquido
y el hidr\'ogeno dentro de los tanques. finalmente rompiendo los propulsores y
destrozando el transbordador 1 . Entonces, ¿por qu\'e se tom\'o la decisi\'on de
continuar con el lanzamiento, a pesar de las advertencias de los ingenieros? \\


Antes del lanzamiento del Challenger, nadie hab\'ia analizado la asociaci\'on
entre la temperatura y la capacidad de respuesta de las juntas t\'oricas. La
NASA decidi\'o arriesgarse al lanzar el transbordador, sin conocer la verdadera
probabilidad estad\'istica de fallas en las juntas t\'oricas a 31 grados. \\ 

Las preocupaciones de los ingenieros se basaban en sospechas y an\'alisis
incompletos. Los ingenieros solo observaron los datos de las juntas t\'oricas de
baja temperatura que inclu\'ian 4 fallas. Al ignorar los datos sobre vuelos a
temperaturas m\'as altas, la probabilidad de falla calculada fue menor de lo que
deber\'ia haber sido 2 . Este an\'alisis no solo es incorrecto sino tambi\'en
peligroso y condujo al desastre. \\

Se deben tener en cuenta los datos de todos los vuelos que se registraron. Ha
habido muchas fallas de juntas t\'oricas tanto a altas como a bajas
temperaturas, pero no se ha medido la fuerza de la asociaci\'on entre estas dos
variables. Los ingenieros no pueden simplemente ignorar los vuelos con
temperaturas m\'as altas.\\

Se requiere un an\'alisis estad\'istico profundo de la eficiencia de las juntas
t\'oricas a varias temperaturas para comprender realmente la falla del
Challenger. Las intuiciones y los an\'alisis sesgados, especialmente por parte
de la administraci\'on de la NASA, no son suficientes para determinar la
probabilidad de falla de las juntas t\'oricas.\\

A continuaci\'on se muestra una muestra de los datos de fallas de las juntas
t\'oricas:\\


\begin{landscape} 


\begin{table}[h]
\caption{Muestra de datos de O-Rings}
\begin{turn}{0}

\centering

\begin{tabular}{|l|l|l|l|} 
\hline
Vuelo & Fecha    & N\'umero de fallas de O-Ring primarias en juntas de campo & Temperatura de junta  \\ 
\hline
1     & 4/12/81  & 0                                                               & 66                    \\
2     & 11/12/81 & 1                                                               & 70                    \\
3     & 3/22/82  & 0                                                               & 69                    \\

5     & 11/11/82 & 0                                                               & 68                    \\
6     & 4/04/83  & 0                                                               & 67                    \\
7     & 6/18/83  & 0                                                               & 72                    \\
8     & 8/30/83  & 0                                                               & 73                    \\
9     & 11/28/83 & 0                                                               & 70                    \\
41-B  & 2/03/84  & 1                                                               & 57                    \\
41-C  & 4/06/84  & 1                                                               & 63                    \\
41-D  & 8/30/84  & 1                                                               & 70                    \\
41-G  & 10/05/84 & 0                                                               & 78                    \\
51-A  & 11/08/84 & 0                                                               & 67                    \\
51-C  & 1/24/85  & 2 (hubo otra, secundaria, falla en el O-Ring)                     & 53                    \\
51-D  & 4/12/85  & 0                                                               & 67                    \\
51-B  & 4/29/85  & 0                                                               & 75                    \\
51-G  & 6/17/85  & 0                                                               & 70                    \\
51-F  & 7/29/85  & 0                                                               & 81                    \\
51-I  & 8/27/85  & 0                                                               & 76                    \\
51-J  & 10/03/85 & 0                                                               & 79                    \\
61-A  & 10/30/85 & 2                                                               & 75                    \\
61-B  & 11/26/85 & 0                                                               & 76                    \\
61-C  & 1/12/86  & 1                                                               & 58                    \\
\hline
\end{tabular}
\end{turn}
\end{table}

\end{landscape} 

Se puede ejecutar una prueba de permutaci\'on para evaluar la importancia
estad\'istica de la diferencia entre la tasa de falla de la junta t\'orica a
bajas temperaturas y la tasa de falla a altas temperaturas. Esto puede
determinar si las juntas t\'oricas de temperatura m\'as baja tienden a
experimentar m\'as fallas que las juntas t\'oricas de temperatura alta, o si la
distribuci\'on de fallas es la misma entre las dos. Las hip\'otesis de esta
prueba estad\'istica son las siguientes:\\

\begin{equation} 
    H_o: Falla_{bajaTemperatura} - Falla_{altaTemperatura} = 0
\end{equation} 

La verdadera diferencia media entre la tasa de falla de la junta t\'orica a baja
y alta temperatura es 0. Las diferencias observadas se deben al azar.


\begin{equation}   
    H_o: Falla_{bajaTemperatura} - Falla_{altaTemperatura} > 0
\end{equation} 


La verdadera diferencia media entre la tasa de falla de la junta t\'orica a baja
y alta temperatura es mayor que 0. Las diferencias observadas no se deben al
azar, sino a una asociaci\'on entre la temperatura y la tasa de falla.\\

La permutaci\'on se ejecuta agrupando cada punto de datos en dos grupos. Alta y
baja temperatura. Mirando la distribuci\'on de temperatura. Las observaciones de
juntas t\'oricas se pueden dividir en menos de 65 grados (baja temperatura). Y
por encima de 65 grados (alta temperatura).\\

La diferencia media observada de fallos entre juntas t\'oricas de baja y alta
temperatura es de 1,3; la columna de temperatura se permuta (baraja). Y la
diferencia de falla media simulada la calculamos nuevamente entre los dos grupos
de juntas t\'oricas. La ejecuci\'on de 10 000 simulaciones de esta prueba de
permutaci\'on produce 10 000 estad\'isticas de prueba simuladas. A partir de
esto, podemos calcular el valor p para evaluar la hip\'otesis nula:

\begin{equation} \label{eq1}
    \begin{split}
\frac{sum(diferencias\;simuladas> diferencia\;observada)}{total\; intentos})=\\ \\
= \frac{sum(diferencias \;simuladas>1.3)}{1000}= \\ \\
=0.012
    \end{split}
\end{equation}


Solo el $1,2 \%$ de las diferencias medias de fallas simuladas son mayores que
la diferencia media de fallas observada. Un valor p de 0.012 es
estad\'isticamente significativo para rechazar la hip\'otesis nula que establece
que las tasas de falla de las juntas t\'oricas en temperaturas bajas y altas
provienen de la misma distribuci\'on.\\

La diferencia observada es mucho mayor que las diferencias en los datos de las
juntas t\'oricas barajadas. Esto sugiere que la tasa de fallas entre las juntas
t\'oricas de baja y alta temperatura no tiene la misma distribuci\'on. \\Los
datos muestran evidencia que respalda la hip\'otesis alternativa que establece
que las juntas t\'oricas de baja temperatura tienen una mayor tasa media de
fallas que las juntas t\'oricas de alta temperatura.\\

Definitivamente parece haber una diferencia entre las juntas t\'oricas probadas
a bajas temperaturas. Y las juntas t\'oricas se probaron a altas temperaturas,
pero es \'util examinar el modelo subyacente de los datos.\\

Es importante estimar la probabilidad de falla de la junta t\'orica a una
temperatura determinada. Esto se puede hacer ajustando un modelo de regresi\'on
log\'istica a los datos anteriores. Con temperatura como entrada y un valor
binario (1: al menos una falla, 0: ninguna falla) como salida. Ajustamos una
regresi\'on log\'istica con los siguientes par\'ametros:

\begin{equation} 
    P(Falla)= \frac{1}{1+exp^{-(\beta_0+\beta_1*temperatura)}}
\end{equation} 


Cuando $\beta_0$ es el coeficiente de temperatura que mide la asociaci\'on y  
$\beta_0$ es el sesgo.Adem\'as, probamos la fuerza de la asociaci\'on calculando
un valor z con respecto a las siguientes hip\'otesis:


\begin{equation} 
H_0 : \beta_1 = 0
\end{equation} 

El verdadero coeficiente de temperatura para el modelo es 0, lo que indica que
no hay relaci\'on entre la temperatura y la falla de la junta t\'orica.

\begin{equation} 
H_0 : \beta_1 \not {0}
\end{equation} 

El verdadero coeficiente de temperatura para el modelo no es 0, lo que indica
alguna relaci\'on entre la temperatura y la falla de la junta t\'orica.\\

Ajustar los datos a una regresi\'on log\'istica genera un modelo con los
siguientes pesos:

\begin{equation} 
P(Falla)= \frac{1}{1+exp^{-(10.88+0.17*temperatura)}}
\end{equation} 

Una gr\'afica a continuaci\'on visualiza la distribuci\'on de fallas y el ajuste
del modelo:

\begin{center} 
    \begin{figure}[h]
    \includegraphics[width=12cm, height=9cm]{figures/challenger_plots.jpg}
    \caption{Gr\'aficas de regresi\'on log\'istica de datos de juntas te\'oricas.}
    \end{figure}
\end{center} 

Como se observa en la gr\'afica, una temperatura de 31 grados parece
incre\'iblemente probable que tenga al menos una falla. El modelo genera un
valor de 0,996, lo que indica un $99,6 \%$ de probabilidad de falla dados los
datos observados de lanzamientos de transbordadores anteriores.

Esto est\'a lejos de ser una decisi\'on cerrada y ciertamente no es una apuesta
arriesgada cuando hay siete compañeros de tripulaci\'on a bordo.

Adem\'as, la fuerza de la asociaci\'on entre estas dos variables. Analizamos con
el estad\'istico z generado en el peso del modelo del coeficiente de
temperatura. El valor p calculado de la hip\'otesis nula:


\begin{equation} 
H_0 : \beta_1 = 0
\end{equation} 


resulta ser 0,04, que es estad\'isticamente significativo cuando se utiliza un
umbral de 0,05. Debido al bajo valor de p, rechazamos la hip\'otesis nula que
establece que el coeficiente de temperatura es 0. Existe una fuerte evidencia en
los datos que respalda la hip\'otesis alternativa:

\begin{equation}  
H_0 : \beta_1 > 0
\end{equation} 


lo que sugiere que el coeficiente es mayor que 0 y que, de hecho, existe una
asociaci\'on entre la temperatura y la falla de la junta t\'orica. Por lo tanto,
existe una fuerte evidencia de que las bajas temperaturas pueden provocar fallas
en las juntas t\'oricas. Adem\'as, los ingenieros deber\'ian haber presentado un
an\'alisis de regresi\'on log\'istica como evidencia para retrasar el
lanzamiento del Challenger.\\

El desastre del Challenger muestra por qu\'e los cient\'ificos e ingenieros
necesitan comprender las t\'ecnicas estad\'isticas. Los an\'alisis b\'asicos
descritos brindan una s\'olida evidencia cuantitativa de que el transbordador no
deber\'ia haberse lanzado debido a la probabilidad significativamente alta de
falla de la junta t\'orica, un componente cr\'itico de la nave espacial.\\



\section{Criterios de divisibilidad}

Los criterios de divisibilidad son herramientas matem\'aticas que nos permiten
identificar por simple inspecci\'on si un n\'umero es divisible\footnote{Se dice
que $a$ es divisible por $b$ si al realizar la operaci\'on, el resultado es un
n\'umero entero y el residuo es 0} por otro.

Cuando se desea conocer la divisibilidad de un n\'umero por otro en forma
aritm\'etica (``calculando a mano''), se tienen distintas pruebas, enunciadas en
el siguiente teorema:


	\begin{itemize}
\item \textbf{Divisibilidad por 2:} Un n\'umero es divisible por 2 cuando
termina en cero o cifra par.
\item \textbf{Divisibilidad por 5:} Un n\'umero es divisible entre 5 cuando
termina en cero o cinco.
\item \textbf{Divisibilidad por 4:} Un n\'umero es divisible entre 4 cuando sus
dos \'ultimas cifras de la derecha son ceros o forman un m\'ultiplo de 4.
\item  \textbf{Divisibilidad por 3:} Un n\'umero es divisible entre 3, si la
suma de los valores absolutos de sus cifras es un m\'ultiplo de 3.
\item \textbf{Divisibilidad por 11:} Un n\'umero es divisible entre 11 cuando la
diferencia entre la suma de los valores absolutos de sus cifras de lugar impar y
la suma de los valores absolutos de sus cifras con lugar par, de derecha a
izquierda, es 0 o m\'ultiplo de 11.
\item \textbf{Divisibilidad por 7:} Un n\'umero es divisible entre 7 cuando
separando la primera cifra de la derecha, multiplic\'andola por 2, restando el
producto del resto de cifras que quedaron a la izquierda y as\'i sucesivamente,
da cero o un m\'ultiplo de 7.		
\item \textbf{Divisibilidad por 13:} Un n\'umero es divisible por 13 si, al
tomar la primera cifra de la derecha, multiplicarla por 9 y rest\'andola del
resto de cifras de la izquierda y as\'i sucesivamente, resulta 0 o un m\'ultiplo
de 13.
\item \textbf{Divisibilidad por 17:} Un n\'umero es divisible por 17 cuando,
tomando la primera cifra de la derecha, multiplic\'andola por 5 y rest\'andola
del resto de sus cifras de la izquierda y as\'i sucesivamente, resulta 0 o un
m\'ultiplo de 17.
	\end{itemize}

\pagebreak

Un n\'umero en sistema decimal se representa en sistema binario mediante una
cadena de 1's y 0's, a los que podemos denotar como alfabeto. Las palabras
\textbf{cadena} y \textbf{alfabeto} nos hacen pensar que es posible analizar un
n\'umero mediante un \textbf{aut\'omata finito}. Adem\'as, que los criterios de
divisibilidad pueden ser llevados acabo por un aut\'omata finito y, como \'estos
tienen una representaci\'on en grafos, entonces existe un grafo para cada
criterio de divisibilidad.

Consideremos el caso de la divisibilidad por 2. Es claro que cualquier n\'umero
dividido por 2 tiene dos posibles residuos: 1 si es impar y 0 si es par. Si los
estados del grafo son los posibles residuos de dividir cualquier n\'umero por
dos, entocnes hay dos estados: 0 y 1.

Los n\'umeros a evaluar, son cadenas de 1's y 0's. Por lo tanto, las
transiciones a cada estado dependen de qu\'e caracter se va ``leyendo''.
\begin{itemize}
\item Grafo de divisibilidad por 2:
	\begin{figure}[h]
		\centering
		\begin{tikzpicture}[node distance=2cm]
			% \draw [help lines] (-5,-5) grid (5,5);
\node (q0)[state, initial, accepting, initial text={Inicio}] at (0,0) {$0$};
\node (q1)[state, right=of q0] {$1$};
			\path[-stealth,thick]
(q0) edge [loop above] node {0} () (q0) edge [bend right] node [below] {1} (q1)
(q1) edge [bend right] node [above] {0} (q0) (q1) edge [loop below] node {1} ();
		\end{tikzpicture}
	\end{figure}
\item Grafo de divisibilidad por 3:
	\begin{figure}[h]
		\centering
		\begin{tikzpicture}[node distance=2cm]
			% \draw [help lines] (-5,-5) grid (5,5);
\node (q0)[state, initial, accepting, initial text={Inicio}] at (0,0) {$0$};
\node (q1)[state, right=of q0] {$1$}; \node (q2)[state, right=of q1] {$2$};
			
			\path[-stealth,thick]
(q0) edge [loop above] node {0} () (q0) edge [bend right] node [below] {1} (q1)
(q1) edge [bend right] node [above] {1} (q0) (q1) edge [bend right] node [below]
{0} (q2) (q2) edge [bend right] node [above] {0} (q1) (q2) edge [loop below]
node {1} ();
		\end{tikzpicture}
	\end{figure}
\pagebreak
\item Grafo de divisibilidad por 5:
	\begin{figure}[h]
		\centering
		\begin{tikzpicture}[node distance=2cm]
			% \draw [help lines] (-5,-5) grid (5,5);
\node (q0)[state, initial, accepting, initial text={Inicio}] at (0,0) {$0$};
\node (q1)[state, right=of q0] {$1$}; \node (q2)[state, right=of q1] {$2$};
\node (q3)[state, below=of q1] {$3$}; \node (q4)[state, below=of q2] {$4$};
			
			\path[-stealth,thick]
(q0) edge [loop above] node {0} () (q0) edge node [below] {1} (q1) (q1) edge
node [above] {0} (q2) (q1) edge node [right] {1} (q3) (q2) edge [bend right]
node [above] {1} (q0) (q2) edge node [right] {0} (q4) (q3) edge [bend left] node
[left] {0} (q1) (q3) edge node [right] {1} (q2) (q4) edge node [below] {0} (q3)
(q4) edge [loop below] node {1} ();
		\end{tikzpicture}
	\end{figure}

\item Grafo de divisibilidad por 7:
	\begin{figure}[h]
		\centering
		\begin{tikzpicture}[node distance=2cm]
			% \draw [help lines] (-5,-5) grid (5,5);
\node (q0)[state, initial, accepting, initial text={Inicio}] at (0,0) {$0$};
\node (q1)[state, right=of q0] {$1$}; \node (q2)[state, below=of q1] {$2$};
\node (q3)[state, right=of q1] {$3$}; \node (q4)[state, below=of q3] {$4$};
\node (q6)[state, right=of q3] {$6$}; \node (q5)[state, below=of q6] {$5$};
			
			\path[-stealth,thick]
(q0) edge [loop above] node {0} () (q0) edge node [above] {1} (q1) (q1) edge
node [above] {1} (q3) (q1) edge node [left] {0} (q2) (q2) edge [bend right] node
[below] {1} (q5) (q2) edge node [below] {0} (q4) (q3) edge node [above] {0} (q6)
(q3) edge [bend right] node [right=1cm] {1} (q0) (q4) edge node [above] {0} (q1)
(q4) edge [bend right] node [above] {1} (q2) (q5) edge node [above] {0} (q3)
(q5) edge node [above] {1} (q4) (q6) edge [loop above] node [above] {1} () (q6)
edge node [right] {0} (q5);
		\end{tikzpicture}
	\end{figure}
\end{itemize}

\pagebreak
\section{Definiciones}

\subsection{Parámetro}

Variable y parámetro son dos términos muy utilizados en matemáticas y física.
Una variable es una entidad que cambia con respecto a otra entidad. Un parámetro
es una entidad que se utiliza para conectar variables. Los conceptos de variable
y parámetro son muy importantes en campos como las matemáticas, la física, la
estadística, el análisis y cualquier otro campo que tenga usos de las
matemáticas.

Un parámetro es una cantidad que influye en la salida o el comportamiento de un
objeto matemático, pero se considera que se mantiene constante. Los parámetros
están estrechamente relacionados con las variables y, a veces, la diferencia es
solo una cuestión de perspectiva. Se considera que las variables cambian,
mientras que los parámetros generalmente no cambian o cambian más lentamente. En
algunos contextos, uno puede imaginarse realizando múltiples experimentos, donde
las variables cambian a través de cada experimento, pero los parámetros se
mantienen fijos durante cada experimento y solo cambian entre experimentos.

Un lugar donde aparecen los parámetros es dentro de las funciones. Por ejemplo,
una función podría ser una función cuadrática genérica como $f(x)=ax2+bx+c$.
Aquí, la variable $x$ se considera como la entrada de la función. Los símbolos
$a$, $b$ y $c$ son parámetros que determinan el comportamiento de la función $f$.
Para cada valor de los parámetros, obtenemos una función diferente

\subsection{Población y muestra}

\begin{tabular}{l|p{6cm}|p{6cm}}
                & \textbf{Población}    &   \textbf{Muestra}    \\
\hline

Definición  & 	Universo de elementos que se van a estudiar.    &   Selección de
una parte de la población que se va a ser sujeto de estudio.   \\

\hline

Características &   

\begin{itemize} 
    \item Se puede clasificar según la cantidad de individuos que la conforman. 
    \item Posee variables estadísticas. 
\end{itemize}
& \begin{itemize}
    \item Forma parte de la población: debería comprender entre 5\% y 10\% para ser más
efectiva. 
    \item Los elementos deben ser aleatorios. 
    \item Debe ser representativa de la población.
\end{itemize} \\
\hline

    Objetivos & Analizar los datos recabados referentes a las características comunes que
comparten los elementos con diversos propósitos. & Estudiar el comportamiento,
características, gustos o propiedades de una parte representativa de la
población. \\

\hline

Ejemplos & \begin{itemize} \item Las personas que habitan un país.
\item La cantidad de carros en una ciudad.
\item Los estudiantes de un país.
\end{itemize} & Para el estudio del desempeño de los estudiantes de cinco
universidades de una ciudad en una materia específica, se toma como muestra a
500 estudiantes aleatoriamente (100 de cada institución) que estén cursando el
mismo nivel para que la muestra sea representativa.

\end{tabular}


\subsection{Tipos de poblaciones}

\begin{itemize} 
    \item Población finita: es aquella que se puede contar y se pueden
estudiar con mayor facilidad a sus integrantes. Por ejemplo, la cantidad de
personas inscritas en un gimnasio. 
    \item Población infinita: son inmensas poblaciones
donde se hace muy difícil contabilizar a sus integrantes, por lo que suele
tomarse en cuenta solo una porción de ella a la hora de realizar un estudio,
seleccionando así una muestra. Por ejemplo, la cantidad de granos de arena en
una playa. 
    \item Población real: son grupos de integrantes tangibles. Por ejemplo, la
cantidad de animales en un zoológico.
\end{itemize}

\subsection{Tipos de muestras}

\textbf{Muestreo aleatorio}. Es una técnica que ofrece la misma posibilidad a los
elementos de ser seleccionados, por ser tomados al azar. Los tipos de muestreo
aleatorio son:

\begin{itemize}

    \item Muestreo aleatorio simple: los elementos se eligen de una lista al azar.
Funciona más eficazmente cuando el universo es reducido y homogéneo. 

\item Muestreo sistemático: el primer elemento se elige al azar y luego se
escogen a intervalos constantes los elementos restantes. 

\item Muestreo estratificado: se realiza dividiendo a la población en partes o
estratos que respondan a características establecidas y luego se eligen
aleatoriamente los individuos que se van a estudiar. 

\item Muestreo por conglomerado: la población se divide en grupos heterogéneos y
éstos a su vez se subdividen en grupos homogéneos con características comunes
para ser estudiados de acuerdo a lo requerido por el investigador.
\end{itemize}

\textbf{Muestreo no aleatorio o por selección intencionada} Se elige con
base en el manejo de información de los elementos a estudiar, por lo que la
representatividad de la muestra puede ser subjetiva. En este caso, se corre el
riesgo de que los resultados sean sesgados.

\section{Variable aleatoria}

Una variable es una entidad que cambia en un sistema dado. Considere un ejemplo
simple de una partícula en movimiento a través del espacio. En tal caso,
entidades como el tiempo, la distancia recorrida por la partícula, la dirección
de viaje se denominan variables.

Hay dos tipos principales de variables en un experimento dado. Estas se conocen
como variables independientes y variables dependientes. Las variables
independientes son las variables que se cambian o que son naturalmente
inmutables. En un ejemplo simple, si se mide la tensión de una banda elástica
mientras se cambia la tensión de la banda, la tensión es la variable dependiente
y la tensión es la variable independiente. La dependencia se aplica cuando la
variable dependiente es dependiente de la variable independiente.

Las variables también se pueden categorizar como variables discretas y variables
continuas. Esta clasificación se utiliza principalmente en matemáticas y
estadística. Los problemas se pueden categorizar dependiendo del número de
variables. El número de variables es muy importante en campos como las
ecuaciones diferenciales y la optimización.

Una variable aleatoria $X$ es una función definida sobre el espacio muestral
$\omega$ (conjunto de los resultados de un experimento aleatorio) que toma
valores en el cuerpo de los números reales $\mathbb{R}$, es decir:


$ X : \omega \rightarrow \mathbb{R} $


Una variable aleatoria puede ser discreta o continua según sea el rango de esta
aplicación.

%\subsection{Variable aleatoria discreta}

Una \textbf{variable aleatoria es discreta} si toma un número de valores finito
o infinito numerable. Estas variables corresponden a experimentos en los que se
cuenta el número de veces que ha ocurrido un suceso.

%\subsection{Variable aleatoria continua}

Una \textbf{variable aleatoria es continua} cuando puede tomar cualquier valor
de un intervalo real de la forma $(a, b)$,$(a,\infty)$,$(-\infty, b)$,$(-\infty,
+\infty)$ o uniones de ellos. Por ejemplo, el peso de una persona, el tiempo de
duración de un suceso, etc.



\pagebreak

\section{Función de distribución}

Toda variable aleatoria tiene asociada una función llamada función de
distribución.

\begin{tcolorbox}[colback=gray!5!white,colframe=gray!60!black,title=Definición: Función de Distribución]

    La función de distribución de una variable aleatoria $X$ es la función 
    $F(x): \mathbb{R} \rightarrow [0,1]$, definida de la siguiente manera:

\begin{equation}
    F(x) = P(X \leq x)
\end{equation}

\end{tcolorbox}


Cuando sea necesario especificar la variable aleatoria en cuestión se escribe
$F_X (x)$, pero en general se omite el subíndice $X$ cuando no haya posibilidad
de confusión. El argumento de la función es la letra minúscula $x$ que puede
tomar cualquier valor real. Por razones obvias a esta función se le conoce
también con el nombre de función de acumulación de probabilidad, o función de
probabilidad acumulada. Observe que la función de distribución de una variable
aleatoria está definida sobre la totalidad del conjunto de números reales, y
siendo una probabilidad, toma valores en el intervalo $[0, 1]$.


La función de distribución es importante pues, contiene toda la información de
la variable aleatoria y la correspondiente medida de probabilidad.

\begin{tcolorbox}[colback=gray!5!white,colframe=gray!60!black,title=Propiedades: Función de Distribución]
    Sea $F(x)$ la función de distribución de una variable aleatoria. Entonces:

    \begin{enumerate}
        \item \begin{equation}
            \lim_{x \rightarrow +\infty} F(x) =1
        \end{equation}

        \item \begin{equation}
            \lim_{x \rightarrow -\infty} F(x) =1
        \end{equation}

        \item Si $x_1 \leq x_2$
        \begin{equation}
            F(x_1) \leq F(x_2)
        \end{equation}

        \item $F(x)$ es continua por la derecha, es decir,
        \begin{equation}
            F(x+) = F(x)
        \end{equation}
    \end{enumerate}

Donde $F(x+)$, es el límite por la derecha de la función $F$ en el punto $x$, y
$F(x=)$, es el límite por la izquierda de la función $F$ en el punto $x$.

\end{tcolorbox}

\subsection{Demostración de las propiedades de la función de distribución}

Una función $F(x) : \mathbb{R} \rightarrow [0,1]$ es llamada función de
distribución si cumple con las siguientes propiedades:

\begin{enumerate}
\item Sea $x_1, x_2, ...$ una sucesión cualquiera de números reales creciente a
infinito, y sean los eventos $A_n = (X \leq x_n)$. Entonces ${A_n : n \in
\mathbb{N}}$ es una sucesión de eventos creciente cuyo límite es $\omega$. Por
la propiedad de continuidad:

\begin{equation}
    \lim_{n \rightarrow \infty} F(x_n) = \lim_{n \rightarrow \infty} P(A_n) = P(\omega) = 1
\end{equation}

Dado que $\mathbb{R}$ es un espacio métrico, lo anterior implica que $F(x)$
converge a 1 cuando $x$ tiende a infinito.

\item Sea ahora $\{x_n : n \in \mathbb{N} \}$ una sucesión cualquiera de números
reales decreciente a menos infinito, y sean los eventos $A_n = (X \leq x_n )$.
Entonces $\{A_n : n \in \mathbb{N}\}$ es una sucesión de eventos decreciente al
conjunto vacío. Nuevamente por la propiedad de continuidad

\begin{equation}
    \lim_{n \rightarrow \infty} F(x_n) = \lim_{n \rightarrow \infty} P(A_n) = P(0) = 0
\end{equation}

Por lo tanto, $F(x)$ converge a cero cuando $x$ tiende a menos infinito.

\item Para $x_1 \leq x_2$:

\begin{equation}
    \begin{array}{ll}
        F(x)    & \leq F(x_1) + P(x_1 < X \leq x_2) \\
                & = P[(X \leq x_1) \cup (x_1 < X \leq x_2)] \\
                & = P(X \leq x_2) \\
                & = P(x_2) \\
    \end{array}
\end{equation}

\item Sea $x_1 , x_2 ,...$ una sucesión cualquiera de números reales no negativos
y decreciente a cero. Entonces:

\begin{equation}
    F (x + x_n ) = F (x) + P (x < X \leq x + x_n )
\end{equation}

en donde $A_n = (x < X \leq x + x_n )$ es una sucesión de eventos decreciente al
conjunto vacío. Por lo tanto $\lim_{n \rightarrow \infty} F(x + x_n ) = F (x)$.
Es decir $F (x+) = F (x)$.

\end{enumerate}

Por lo tanto basta dar una función de distribución específica para saber que
existe un cierto espacio de probabilidad y una variable aleatoria definida sobre
él y cuya función de distribución es la especificada. Este es el punto de vista
que a menudo se adopta en el estudio de las variables aleatorias, quedando un
espacio de probabilidad no especificado en el fondo como elemento base en todas
las consideraciones.


\subsection{Variable aleatoria discreta}

\begin{tcolorbox}[colback=gray!5!white,colframe=gray!60!black,title=Definición: Variable aleatoria discreta]
La variable aleatoria $X$ se llama discreta si su correspondiente función de
distribución $F(x)$ es una función constante por pedazos, Fig. (\ref{fig: discrete_function}). Sean $x_1, x_2 , ...$
los puntos de discontinuidad de $F(x)$. En cada uno de estos puntos el tamaño de
la discontinuidad es $P(X = x_i ) = F (x_i ) - F (x_i -) > 0$. A la función
$f(x)$ que indica estos incrementos se le llama función de probabilidad de $X$,
y se define como sigue:

\begin{equation}
    f(x) =
    \left\lbrace
    \begin{array}{ll} 
        P(X=x)  & \text{si } x = x_1, x_2, ... \\
        0       & \text{otro caso.}
    \end{array}\right.
\end{equation}

La función de distribución se reconstruye de la forma siguiente:

\begin{equation}
    F(x) = \sum_{u \leq x} f(u)
\end{equation}

\end{tcolorbox}

En este caso se dice también que la función de distribución es discreta, además
la función de probabilidad $f(x)$ siempre existe, y se le llama también función
de masa de probabilidad. También se acostumbra usar el término función de
densidad, como una analogía con el caso de variables aleatorias continuas
definidas más adelante. Observe que la función de probabilidad $f(x)$ es una
función no negativa que suma uno en el sentido $\sum_{i} f(x_i) = 1$.
Recíprocamente, toda función que cumpla las dos propiedades anteriores se le
llama función de probabilidad, sin que haya necesariamente una variable
aleatoria de por medio.

\begin{figure}[h!]
    \centering
\begin{tikzpicture}
    \begin{axis}[
        clip=false,
        jump mark left,
        ymin=0,ymax=1,
        xmin=0, xmax=5,
        discontinuous,
        table/create on use/cumulative distribution/.style={
            create col/expr={\pgfmathaccuma + \thisrow{f(x)}}
            }
        ]
        \addplot [red] table [y=cumulative distribution]{
            x f(x)
            0 1/15
            1 2/15
            2 1/5
            3 4/15
            4 1/3
            5 0
        };
    \end{axis}
\end{tikzpicture}
\label{fig: discrete_function}
\caption{Función de distribución discreta}
\end{figure}


\subsection{Variable aleatoria continua}

\begin{tcolorbox}[colback=gray!5!white,colframe=gray!60!black,title=Definición: Variable aleatoria continua]
La variable aleatoria continua $X$ con función de distribución $F(x)$ se llama
absolutamente continua, si existe una función no negativa e integrable $f$ tal que
para cualquier valor de $x$ se cumple que:

\begin{equation}
    F(x) = \int_{- \infty}^{x} f(x) dx.
    \label{eq:continous_distribution}
\end{equation}

En tal caso $f(x)$ se le llama función de densidad de $X$.

\end{tcolorbox}

Aún cuando exista una función no negativa e integrable $f$ que cumpla con lo
anterior, ésta puede no ser única, pues basta modificarla en un punto para que
sea ligeramente distinta pero aún así lo seguirá cumpliendo. A pesar de ello,
nos referiremos a la función de densidad como si ésta fuera única, y ello se
justifica por el hecho de que las probabilidades son las mismas, ya sea usando
una función de densidad o modificaciones de ella que cumplan. Es claro que la
función de densidad de una variable aleatoria absolutamente continua es no
negativa y su integral sobre toda la recta real es uno. Recíprocamente, toda
función $f(x)$ no negativa que integre uno en $\mathbb{R}$ se llama función de
densidad. Si $X$ es absolutamente continua con función de distribución $F(x)$ y
función de densidad continua $f(x)$, entonces el teorema fundamental del cálculo
establece que, a partir de la Ec. \eqref{eq:continous_distribution}, $F'(x) = f
(x)$. Además, la probabilidad de que $X$ tome un valor en el intervalo $(a, b)$
es el área bajo la función de densidad sobre dicho intervalo, esto se ilustra en
la Fig. \ref{fig:continous_distribution}.

% GAUSSIANs: basic properties

\begin{figure}[h!]
    \centering
    \begin{tikzpicture}
    \message{Cumulative probability^^J}
    
    \def\B{11};
    \def\Bs{3.0};
    \def\xmax{\B+3.2*\Bs};
    \def\ymin{{-0.1*gauss(\B,\B,\Bs)}};
    \def\h{0.07*gauss(\B,\B,\Bs)};
    \def\a{\B-0.8*\Bs};
    
    \begin{axis}[every axis plot post/.append style={
                 mark=none,domain={-0.05*(\xmax)}:{1.08*\xmax},samples=\N,smooth},
                 xmin={-0.1*(\xmax)}, xmax=\xmax,
                 ymin=\ymin, ymax={1.1*gauss(\B,\B,\Bs)},
                 axis lines=middle,
                 axis line style=thick,
                 enlargelimits=upper, % extend the axes a bit to the right and top
                 ticks=none,
                 xlabel=$x$,
                 every axis x label/.style={at={(current axis.right of origin)},anchor=north},
                 width=0.7*\textwidth, height=0.55*\textwidth,
                 y=700pt,
                 clip=false
                ]
      
      % PLOTS
      \addplot[red,thick,name path=B] {gauss(x,\B,\Bs)};
      
      % FILL
      \path[name path=xaxis]
        (0,0) -- (\pgfkeysvalueof{/pgfplots/xmax},0);
      \addplot[red!25] fill between[of=xaxis and B, soft clip={domain=-1:{\a}}];
      
      % LINES
      \addplot[mydarkred,dashed,thick]
        coordinates {({\a},{1.2*gauss(\a,\B,\Bs)}) ({\a},{-\h})}
        node[mydarkred,below=-2pt] {$a$};
      \node[mydarkred,above right] at ({\B+\Bs},{1.2*gauss(\B+\Bs,\B,\Bs)}) {$f(x)$};
      \node[red!60!black,above left] at ({0.85*(\a)},{1.0*gauss(0.85*(\a),\B,\Bs)}) {};
      
    \end{axis}
  \end{tikzpicture}
  \label{fig:continous_distribution}
  \caption{Función de distribución continua}
\end{figure}

\input{sections/distributions/momentum}

\pagebreak

\section{Caminata aleatoria}

El concepto de random walk comenzó en 1827 cuando el botánico Robert Brown
estudió el movimiento del polen. También es importante la aportación de Albert
Einstein de 1905 sobre el movimiento browniano de pequeñas partículas.

Una caminata aleatoria simple sobre el conjunto de números enteros $\mathbb{Z}$
es un proceso estocástico a tiempo discreto $X_n : n=0,1, ...$ que evoluciona de
la siguiente forma: iniciando en el estado 0, al tiempo 1 el proceso puede pasar
al estado 1 con probabilidad $p$, o al estado 1 con probabilidad $q$, en donde
$p+q=1$. Se usa la misma regla para los siguientes tiempos, es decir, estando en
el estado $k$, al siguiente instante el proceso pasa al estado $k+1$ con
probabilidad $p$, o al estado $k-1$ con probabilidad $q$. El valor de $X_n$ es
el estado donde se encuentra el proceso al tiempo n y puede escribirse como

\begin{equation}
    X_n = \xi_1 + \xi_2 + ... + \xi_n,
\end{equation}

en donde $\xi_1, \xi_2,...$  es una sucesión de variables aleatorias
independientes con idéntica distribución

\begin{equation}
    P(\xi=+1) = p
\end{equation}
\begin{equation}
    P(\xi=-1) = q = 1-p
\end{equation}

\begin{figure}[h!]
    \centering
    \includegraphics[scale=0.8]{figures/Random_walk.png}
    \caption{Caminata aleatoria en dos dimensiones.}
\end{figure}

La independencia de las variables aleatorias $\xi_1, \xi_2,...$ tiene como
consecuencia que la caminata aleatoria simple cumpla la propiedad de Markov y
que sea un proceso estocástico con incrementos independientes y estacionarios.
En términos generales el objetivo es estudiar el comportamiento de la v.a. $X_n$
al paso del tiempo. Uno de los elementos fundamentales para estudiar a las
caminatas aleatorias está dado por las probabilidades de transición. Para
cualesquiera enteros $i$ y $j$, las probabilidades de transición en un paso de
la caminata aleatoria simple son:

\begin{equation}
    P(X_{n+1}=j | X_n = i) = \left\lbrace \begin{array}{ll}
        p & \text{si} j = i+1 \\
        q & \text{si} j = i-1 \\
        0 & \text{en otro caso}
    \end{array}\right.
\end{equation}

Como estas probabilidades no dependen de n, se dice que son homogéneas en el
tiempo, es decir, son las mismas para cualquier valor de n. Pueden consi-
derarse también caminatas aleatorias que inician en cualquier valor entero o que
permiten saltar de un estado en sí mismo al siguiente instante (es decir, no
cambiar de estado), o bien, que los saltos no sean unitarios ni los tiempos en
los que efectúan los saltos sean necesariamente tiempos enteros, e incluso
pueden definirse caminatas en dimensiones mayores, por ejemplo, en Z 2 . Las
caminatas aleatorias son procesos que pueden representar versiones discretas de
procesos a tiempo continuo más complejos.

\section{Cadenas de Markov}

Este tipo de procesos es también de amplia aplicación y se cuenta con una teoría
matemática bastante desarrollada. \\

\begin{tcolorbox}[colback=gray!5!white,colframe=gray!60!black,title=Cadenas de Markov]
Una cadena de Markov es un proceso estocástico a tiempo discreto ${X_n : n=0,
1,...}$ con espacio de estados discreto y tal que 0 y satisface la propiedad de
Markov, esto es, para cualquier entero $n \geq 0$ cualesquiera estados $x_0 , x_1 , . . .
, x_{n+1}$ se cumple la identidad:

\begin{equation}
    \begin{array}{l}
    P(X_{n+1} =  x_{n+1} | X_n = x_n, ..., X_1 = x_1, X_0=x_0) \\
    P(X_{n+1} = x_{n+1} = x_{n+1} | X_n = x_n).
    \end{array}
\end{equation}

\end{tcolorbox} 


A las probabilidades condicionales mencionadas en la definición anterior se les
llama probabilidades de transición del tiempo $n$ al tiempo $n+1$, y se les
denota por $p_{x_n, x_{n+1}}(n, n +1)$. Adicionalmente, se dice que la cadena es
estacionaria u homogénea en el tiempo si estas probabilidades no dependen
explícitamente de los tiempos particulares $n$ y $n+1$, sino únicamente de los
estados involucrados. De esta forma, si de manera general se considera la
probabilidad de transición del estado $i$ al estado $j$ de una unidad de tiempo
cualquiera a la siguiente unidad de tiempo, la probabilidad de transición del
primer estado al segundo se escribe como $p_{ij}$ , o $p_{ij}(1)$ , o también
como $p_{ij}^{1}$, es decir, sin importar el valor del entero $n$:

\begin{equation}
    p_{ij} = P(X_n = j | X_{n-1} = i)
\end{equation}

A estas probabilidades se les llama probabilidades de transición en un paso. En
general se considera como espacio de estados para una cadena de Markov el
conjunto discreto $0, 1, ...$ o algún subconjunto finito de él. Haciendo variar
los valores de los índices $i$ y $j$ en este espacio de estados se forma la
matriz de probabilidades de transición en un paso:

\begin{equation}
    P = \begin{pmatrix}
        p_{00}  & p_{01}    & p_{02}    & ... \\
        p_{10}  & p_{11}    & p_{12}    & ... \\
        p_{10}  & p_{21}    & p_{22}    & ... \\
        .       & .         & .         &     \\
        .       & .         & .         &     \\
        .       & .         & .         &
    \end{pmatrix}
\end{equation}



\section{Ley de los grandes números}

Esta ley establece que, bajo ciertas condiciones, el promedio de variables
aleatorias converge a una constante cuando el número de
sumandos crece a infinito. Demostraremos dos versiones de esta afirmación, las
cuales se distinguen por el tipo de convergencia de la que se trate. La ley
débil establece la convergencia en probabilidad y la ley fuerte dice que la
convergencia es casi segura. La ley fuerte implica entonces la ley débil.\\


\begin{tcolorbox}[colback=gray!5!white,colframe=gray!60!black,title=Teorema de Bernoulli: Ley débil de los grandes números]
Sean $X_1 , X_2 , ...$ independientes e idénticamente distribuidas con media $\mu$.
Entonces:

\begin{equation}
    \frac{1}{n} \sum_{i=1}^{n} X_i \rightarrow^p \mu
    \label{eq:big_number_law}
\end{equation}

\end{tcolorbox}


Sea $S_n = (X_1 + · · · + X_n )/n$, y sea $\phi (t)$ la función característica
de cualquier elemento $X$ de la sucesión. Como $X$ tiene esperanza
finita $\mu$ y por la expansión \eqref{eq:big_number_law},

\begin{equation}
    \phi (t) = 1 + it(\mu + o(1)) \text{, cuando } t \rightarrow 0
\end{equation}

Por independencia la función característica de $S_n$ es entonces

\begin{equation}
    \phi S_n(t) = \phi^n (t/n)= (1 + i(t/n)(\mu + o(1)))^n \text{, cuando } t \rightarrow 0
\end{equation}

Haciendo $n \rightarrow \infty$ se obtiene $\phi_{S_n} (t) \rightarrow e^{i \mu
t}$, en donde $e^{i \mu t}$ es la función característica de la variable
aleatoria constante $\mu$. Esto implica que $S_n \rightarrow^d \mu$. El
resultado se obtiene al recordar que la convergencia en distribución a una
constante es equivalente a la convergencia en probabilidad.

Este mismo resultado puede demostrarse fácilmente a partir de la desigualdad de
Chebyshev bajo la hipótesis adicional de existencia de la varianza.

Sea nuevamente $S_n = (X_1 + ... + X_n )/n$. Entonces $E(S_n ) = \mu$ y
$\text{Var}(S_n) = \sigma^2/n$, suponiendo $\text{Var}(X) = \sigma^2 < \infty$.
La desigualdad de Chebyshev aplicada a la variable $S_n$ asegura que para
cualquier $\epsilon > 0$ se cumple $P (|S_n - \mu| \geq \epsilon) \leq \sigma^2
/n \epsilon^2$. Basta ahora tomar el límite cuando $n$ tiende a infinito para
obtener el resultado.

\section{Distribución Bernoulli}

Un ensayo Bernoulli se define como aquel experimento aleatorio con únicamente
dos posibles resultados, llamados genéricamente: éxito y fracaso. Supondremos
que las probabilidades de estos resultados son $p$ y $1 - p$, respectivamente.
Si se define la variable aleatoria $X$ como aquella función que lleva el
resultado éxito al número 1 y el resultado fracaso al número 0, enton- ces
decimos que $X$ tiene una distribución Bernoulli con parámetro $p \in (0, 1)$ y
escribimos $X \sim Ber(p)$. La función de probabilidad se puede escribir de la
siguiente forma.

\begin{equation}
    f(x) = \left\lbrace
        \begin{array}{ll} 
            1-p & \text{ si } x = 0, \\
            p   & \text{ si } x=1, \\
            0   & \text{ en otro caso.}
        \end{array}\right.
\end{equation}

O bien de manera compacta,

\begin{equation}
    f(x) = \left\lbrace
        \begin{array}{ll} 
            p^x (1-p)^{1-x} & \text{ si } x = 0,1, \\
            0   & \text{ en otro caso.}
        \end{array}\right.
\end{equation}

En la realización de todo experimento aleatorio siempre es posible pregun-
tarse por la ocurrencia o no ocurrencia de un evento cualquiera. Este es el
esquema general donde surge esta distribución de probabilidad. La distri-
bución Bernoulli es sencilla, pero de muy amplia aplicación.

\subsection{Ejemplo de evento tipo Bernoulli}

Un jugador de baloncesto tiene un 80\% de acierto en tiros libres. Si tira tres
lanzamientos seguidos, ¿Cuál es la probabilidad de que acierte los tres?

Para un tiro:

Probabilidad de éxito $p=0.8$

Probabilidad de fracaso: $q = 1 - 0.8 = 0.2$

Para tres tiros:

Probabilidad de éxito $P(A \cap A \cap A) = p \cdot p \cdot p = 0.8 \cdot 0.8
\cdot 0.8 = 0.512$

Probabilidad de fracaso $P(\bar{A} \cap \bar{A} \cap \bar{A}) = q \cdot q \cdot q = 0.2 \cdot 0.2
\cdot 0.2 = 0.008$


\documentclass{beamer}

\mode<presentation>{
  \usetheme{Warsaw}
}


\usepackage[spanish,es-tabla,es-nodecimaldot]{babel}
\usepackage{tikz}
\usepackage{pgf}  %Para realizar figures
\usepackage{xcolor} % Para los colores
\usepackage{enumerate}
\usepackage{graphicx}
\usepackage{array}
\usepackage{cancel}
\usepackage{amssymb}
\usepackage{hyperref}
\usepackage{tcolorbox}  %Cuadros de teoremas


\title[ \hspace{21mm} \insertframenumber \ de \inserttotalframenumber ]
{Distribución binomial}

\subtitle
{Probabilidad, procesos aleatorios e inferencia}

\author[]
{Ana Maritza Bello Yáñez}


\institute[Instituto Polit\'ecnico Nacional]
{
  \inst{1}
  Centro de Investigaci\'on en Computaci\'on
  }

\date[Short Occasion]
{\today}

\keywords{}

\begin{document}

% https://www.youtube.com/watch?v=Nl7BsFe4xmY

\begin{frame}
  \titlepage
\end{frame}

\begin{frame}{Distribución Binomial}
  \begin{block}{Definición}
    Función de probabilidad de una distribución binomial se define como la
    probabilidad de que ocurran exactamente $k$ eventos exitosos de $n$ experimentos
    independientes de probabilidad $p$.

    \begin{equation}
      P(k) = C(n,k) p^k q^{n-k}
    \end{equation}

    Esto a menudo es escrito como:
    \begin{equation}
      b(k; n,p) = C(n,k) p^k q^{n-k}
    \end{equation}
    Donde $0 \leq k \leq n$
  \end{block}
\end{frame}

\begin{frame}{Eventos tipo Bernoulli}
  \begin{block}{Definición}
    Eventos en los cuales solo existen dos posibles resultados.

    Por ejemplo:

    \begin{itemize}
      \item Tirar una moneda
      \item Lanzar un dardo
      \item Lanzar una tiro para encestar
    \end{itemize}
  \end{block}

  \begin{block}{}
    Probabilidad de "éxito": $p$

    Probabilidad de "fracaso": $q = 1 - p$
  \end{block}

\end{frame}

\begin{frame}{Eventos tipo Bernoulli}

  \begin{block}{Ejemplo}
    Un jugador de baloncesto tiene un 80\% de cierto en tiros libres. Si tira tres
    lanzamientos seguidos, ¿Cuál es la probabilidad de que acierte los tres?
  \end{block}

  \begin{block}{}
    Para un tiro:

    Probabilidad de éxito: $p=0.8$

    Probabilidad de fracaso: $q = 1 - 0.8 = 0.2$
  \end{block}

  \begin{block}{}
    Para tres tiros:

    Probabilidad de éxito: $P(A \cap A \cap A) = p \cdot p \cdot p = 0.8 \cdot 0.8
    \cdot 0.8 = 0.512$

  \end{block}

  \begin{block}{}

    Probabilidad de fracaso: $P(\bar{A} \cap \bar{A} \cap \bar{A}) = q \cdot q \cdot q = 0.2 \cdot 0.2
    \cdot 0.2 = 0.008$

  \end{block}

\end{frame}

\begin{frame}{Distribución Binomial}
  \begin{block}{Ejemplo}
    Un jugador de baloncesto tiene un 80\% de cierto en tiros libres. Si tira tres
    lanzamientos seguidos, ¿Cuál es la probabilidad de que acierte dos de los tres
    lanzamientos?
  \end{block}

  \begin{block}{}
    Sabiendo que:    $p = 0.8$ y  $q = 0.2$

    Entonces:

$P(\text{dos aciertos}) = P(A \cap A \cap \bar{A}) 
                          \cup P(A \cap \bar{A} \cap A)
                          \cup P(\bar{A} \cap A \cap A )$
  \end{block}

  \begin{block}{}  
    Por lo que:
    $P(A \cap A \cap \bar{A}) = p \cdot p \cdot q 
                              = 0.8 \cdot 0.8 \cdot 0.2 
                              = 0.128$
  \end{block}

  \begin{block}{}
    Así:
    
    $P(\text{dos aciertos}) = 0.128 + 0.128 + 0.128 = 3 \cdot 0.128$
  \end{block}
\end{frame}

\begin{frame}{Distribución binomial}
  \begin{block}{Ejemplo}
    Notemos que:

    $P(\text{dos aciertos}) = P(A \cap A \cap \bar{A}) 
                          \cup P(A \cap \bar{A} \cap A)
                          \cup P(\bar{A} \cap A \cap A )$
                          
    Estamos haciendo una combinatoria:

    $C(3,2)$

  \end{block}

  \begin{block}{}
    De esta manera podemos hacerlo para experimentos más grandes. Digamos, ¿cuál
    sería la probabilidad de acertar 3 canastas si hiciera 15 lanzamientos?

    $C(15,3) = 455$
  \end{block}
\end{frame}

\begin{frame}{Distribución binomial}
  \begin{block}{Aplicando la función de probabilidad...}
      Un jugador de baloncesto tiene un 80\% de acierto en tiros libres. Si tira 20
      lanzamientos seguidos, ¿cuál es la probabilidad de que acierte 13 de los
      lanzamientos?
      
    $P(13) = C(20,13) 0.8^{13} \cdot 0.2^{7} = 0.0545$
  \end{block}
\end{frame}

\begin{frame}{Distribución binomial}
  \begin{block}{}
Ahora podemos obtener la distribución de la probabilidad pasando por todos los
valores posibles desde 0 hasta $k$ en:

    $P(k) = C(n,k) p^k q^{n-k}$
  \end{block}

  \begin{center}
  \includegraphics[scale=0.5]{figures/binomial_distribution.png}
  \end{center}
\end{frame}

\begin{frame}
  \begin{block}{Nota}
Cuando el éxito y el fracaso son igualmente probables, la distribución binomial
es una distribución normal. Por lo que si cambiamos el valor de $p$ a 0.5,
obtenemos la siguiente gráfica de distribución normal.
  \end{block}

  \begin{center}
    \includegraphics[scale=0.5]{figures/binomial_distribution_normal.png}
  \end{center}

\end{frame}

%\begin{frame}{Definición de distribución}
%    \begin{block}{Definición}
%      Una distribución estadística, o distribución de probabilidad, describe cómo se
%      distribuyen los valores para un campo. En otras palabras, la distribución
%      estadística muestra qué valores son comunes y poco comunes.
%    \end{block}  
%    
%\end{frame}


\end{document}



\documentclass{beamer}

\mode<presentation>{
  \usetheme{Warsaw}
}


\usepackage[spanish,es-tabla,es-nodecimaldot]{babel}
\usepackage{tikz}
\usepackage{pgf}  %Para realizar figures
\usepackage{xcolor} % Para los colores
\usepackage{enumerate}
\usepackage{graphicx}
\usepackage{array}
\usepackage{cancel}
\usepackage{amssymb}
\usepackage{hyperref}
\usepackage{tcolorbox}  %Cuadros de teoremas


\title[ \hspace{21mm} \insertframenumber \ de \inserttotalframenumber ]
{Distribución de Poisson}

\subtitle
{Probabilidad, procesos aleatorios e inferencia}

\author[]
{Ana Maritza Bello Yáñez}


\institute[Instituto Polit\'ecnico Nacional]
{
  \inst{1}
  Centro de Investigaci\'on en Computaci\'on
  }

\date[Short Occasion]
{\today}

\keywords{}

\begin{document}



\begin{frame}
  \titlepage
\end{frame}

\begin{frame}{Distribución de Poisson}
  \begin{block}{}
    Esta distribución es útil cuando se quiere estudiar la ocurrencia de eventos por
    unidad de tiempo:\\
    \vfill
    errores/mes, quejas/semana, defectos/día. \\
    \vfill
    Para su aplicación, la probabilidad de ocurrencia del evento debe ser constante
    en tiempo o espacio y debe haber independencia de ocurrencia de eventos.
  \end{block}
\end{frame}

\begin{frame}{Distribución de Poisson}
  \begin{block}{Definición}
    También se puede usar como una aproximación de la distribución binomial, esto es
    cuando:

    $n \to \infty $ y $p \to 0 $

    De manera que el promedio $\lambda = np$ se hace constante.
  \end{block}

  \begin{block}{}
    La expresión binomial de la función de densidad de probabilidad para tales
    sucesos tiende a la siguiente forma simplificada:

    \begin{equation}
      P(x)=\frac{\lambda ^ x}{x!}e^{-\lambda}
    \end{equation}
  \end{block}
\end{frame}


\begin{frame}{Distribución binomial}
  \begin{block}{Retomando la distribución binomial...}
La variable aleatoria $X$ representa el número de éxitos con probabilidad $p$,
obtenidos en $n$ intentos.

La función de densidad de probabilidad está dada por:

    \begin{equation}
      P(X=x) = f(x) = \binom{n}{x} p^x q^{n-x}
    \end{equation}

    Donde $\binom{n}{x}$ es la combinatoria de $n$ en $x$.

    La función de distribución está dada por:

    \begin{equation}
      P(X \leq x) = \sum_{i}^{} f(x_i)
    \end{equation}

  \end{block}
\end{frame}


\begin{frame}{Distribución Binomial}
  Para una v.a. discreta binomial podemos definir tres parámetros importantes:

  \begin{enumerate}
    \item Valor esperado:
    \begin{equation}
      \mu = E(x) = \sum_{x}^{} x f(x) = \sum_{x}^{} x \binom{n}{x} p^x (1-p)^(n-x) = np
    \end{equation}

    \item Varianza:
    \begin{equation}
      \sigma^2 = \sum_{x} x^2 \binom{n}{x} p^x (1-p)^{n-x} - (\sum_{x} \binom{n}{x} p^x (1-p)^{n-x})^2 = np(1-p)
    \end{equation}

    \item Desviación estándar:
    \begin{equation}
      \sigma = \sqrt{\sigma^2}
    \end{equation}

  \end{enumerate}

\end{frame}


\begin{frame}{Aproximación de la dist. Binomial a Poisson}
  La distribución binomial depende de 3 parámetros:

  \begin{equation}
    f(x,n,p) = \binom{n}{x} p^x (1-p)^{n-x}
  \end{equation}

\end{frame}


\begin{frame}{}
\end{frame}



\end{document}



from scipy.stats import expon
import matplotlib.pyplot as plt

n=20
p=0.5
k_values = list(range(n+1))

mean, var = binom.stats(n, p)
print("mean: ",mean, "var: ",var)
#mean: número de tiros exitosos en promedio
#var: 

distribution = [binom.pmf(k, n, p) for k in k_values]
#Probability Mass Function (PMF): Probabilidad total de lograr r éxito y nr fracasos


plt.plot(k_values,distribution, color="black", linestyle="dashed")
plt.bar(k_values,distribution, color="gray")
plt.title("Distribución Binomial")
plt.ylabel("Probabilidad(k)")
plt.xlabel("Valores k (lanzamientos exitosos)")
filepath = "slides/figures/binomial_distribution_normal.png"
plt.savefig(filepath)
plt.clf()


\pagebreak
\bibliography{../references/references.bib} 
\bibliographystyle{unsrt}

\end{document}

%poner alef 1 y alef2
% Axiomas de peano
