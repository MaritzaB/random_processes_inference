\documentclass{article}
\usepackage[spanish,es-tabla,es-nodecimaldot]{babel}
\usepackage{tikz}


\begin{document}
\title{Probabilidad, procesos aleatorios e inferencia}
\author{Ana Maritza Bello Ya\~nez}
\maketitle
\setlength{\parindent}{-3pt}
\setlength{\parskip}{1em}

\tableofcontents

\section{Probabilidad y conteo}

\subsection{Definici\'on cl\'asica}

La definici\'on cl\'asica de la probabilidad \cite{faber2012statistics} de un evento $A$, puede ser formulada
de la siguiente manera:

\begin{equation}
    P(A)= \frac{n_A}{n_(total)}
\end{equation}

Donde $n_A$ es el n\'umero de formas igualmente probables por las cuales un
experimento puede conducir a A; $n_(total)$ n\'umero total de formas igualmente
probables en el experimento.

- En la definici\'on cl\'asica el experimento no necesariamente se lleva a cabo ya que
la respuesta es conocida de antemano.

- La teor\'ia cl\'asica no da soluci\'on a menos de que todas las formas igualmente
posibles puedan ser derivadas anal\'iticamente.

\section{Experimentos deterministas vs aleatorios}

\subsection{Definici\'on: Observaci\'on}

Cualquier registro de informaci\'on, ya sea num\'erico o categ\'orico. Por
consiguiente, los n\'umeros 2, 0, 1 y 2, que representan el n\'umero de accidentes
que ocurrieron cada mes, de enero a abril, durante el a\~no pasado en la
intersecci\'on de Driftwood Lane y Royal Oak Drive, constituyen un conjunto de
observaciones. Lo mismo ocurre con los datos categ\'oricos N, D, N, N y D, que
representan los art\'iculos defectuosos o no defectuosos cuando se inspeccionan
cinco art\'iculos y se registran como observaciones \cite{walpole2012probabilidad}.

\subsection{Definici\'on: Experimento}

Cualquier proceso que genere un conjunto de datos. Un ejemplo simple de
experimento estad\'istico es el lanzamiento de una moneda al aire. En tal
experimento s\'olo hay dos resultados posibles: cara o cruz
\cite{walpole2012probabilidad}.

\subsection{Experimento determinista}
Un experimento determinista es aquel que produce el mismo resultado cuando se le
repite bajo las mismas condiciones.

\subsection{Experimento aleatorio}
Un experimento aleatorio es aquel que, cuando se le repite bajo las mismas
condiciones, el resultado que se observa no siempre es el mismo y tampoco es
predecible.

%Fig.(\reference{gatito})
%Tab.()
%Ec.()
\section{Tri\'angulo de Sierpinski}

El matem\'atico polaco Waclav Sierpinski (1882-1969), construy\'o este fractal en
1919 del modo siguiente: tom\'o un tri\'angulo equil\'atero, uni\'o los puntos medios de
los lados y form\'o cuatro tri\'angulos interiores: tres tri\'angulos equil\'ateros
sombreados y un hueco que es otro tri\'angulo equil\'atero. Repiti\'o el proceso en
cada uno de los tri\'angulos sombreados, y sigui\'o hasta el infinito el proceso en
los tres tri\'angulos restantes como el primero.
\begin{figure}[htp]
    \centering
    \usetikzlibrary{lindenmayersystems}
    \def\trianglewidth{3cm}
    \pgfdeclarelindenmayersystem{Sierpinski triangle}{
        \symbol{X}{\pgflsystemdrawforward}
        \symbol{Y}{\pgflsystemdrawforward}
        \rule{X -> X-Y+X+Y-X}
        \rule{Y -> YY}
    }
    \foreach \level in {1,3,5}{
        \tikzset{
            l-system={step=\trianglewidth/(2^\level), order=\level, angle=-120}
        }

        \begin{tikzpicture}
            \fill [black] (0,0) -- ++(0:\trianglewidth) -- ++(120:\trianglewidth) -- cycle;
            \draw [draw=none] (0,0) l-system
            [l-system={Sierpinski triangle, axiom=X},fill=white];
        \end{tikzpicture}
    }
    \caption{Tri\'angulo de Sierpinski para los niveles 1,3 y 5.}
    \label{fig:sierpinski}
\end{figure}

Como se muestra en la Fig.(\ref{fig:sierpinski})

\section{La belleza y utilidad de las matem\'aticas}

Las matem\'aticas han impregnado todos los campos de la actividad cient\'ifica y
desempe\~nan un papel inestimable en la biolog\'ia, la f\'isica, la qu\'imica, la
econom\'ia, la sociolog\'ia y la ingenier\'ia. Las matem\'aticas se pueden utilizar para
ayudar a explicar los colores de una puesta de sol o la arquitectura de nuestro
cerebro. Las matem\'aticas nos ayudan a construir aviones supers\'onicos y monta\~nas
rusas, simular el flujo de los recursos naturales de la Tierra, explorar
realidades cu\'anticas subat\'omicas y obtener im\'agenes de galaxias lejanas.

Las matem\'aticas han cambiado la forma en que vemos el cosmos.

La utilidad de las matem\'aticas nos permite construir naves espaciales e
investigar la geometr\'ia de nuestro universo. Los n\'umeros pueden ser nuestro
primer medio de comunicaci\'on con razas alien\'igenas inteligentes. Algunos f\'isicos
incluso han especulado que una comprensi\'on de las dimensiones superiores y de la
topolog\'ia, el estudio de las formas y sus interrelaciones, podr\'ia alg\'un d\'ia
permitirnos escapar de nuestro universo, cuando termine en un gran calor o fr\'io,
y entonces podr\'iamos llamar a todo el espacio. -tiempo nuestro hogar.

El descubrimiento simult\'aneo ha ocurrido a menudo en la historia de las
matem\'aticas.

En 1858 el matem\'atico alem\'an August Möbius (1790–1868) descubri\'o de forma simult\'anea e independiente
la cinta de Möbius (un maravilloso objeto retorcido de un solo lado) junto con
un erudito contempor\'aneo, el matem\'atico alem\'an Johann Benedict Listado
(1808–1882). Este descubrimiento simult\'aneo de la banda de Möbius por parte de
Möbius y Listing, al igual que el del c\'alculo del erudito ingl\'es Isaac Newton
(1643-1727) y el matem\'atico alem\'an Gottfried Wilhelm Leibniz (1646-1716), nos
hacen preguntarnos por qu\'e se hicieron tantos descubrimientos cient\'ificos en al
mismo tiempo por personas que trabajan de forma independiente. Para dar otro
ejemplo, los naturalistas brit\'anicos Charles Darwin (1809–1882) y Alfred Wallace
(1823–1913) desarrollaron la teor\'ia de la evoluci\'on de manera independiente y
simult\'anea. De manera similar, el matem\'atico h\'ungaro J\'anos Bolyai (1802–1860) y
el matem\'atico ruso Nikolai Lobachevsky (1793–1856) parec\'ian haber desarrollado
la geometr\'ia hiperb\'olica de forma independiente y al mismo tiempo. \\

Lo m\'as probable es que tales descubrimientos simult\'aneos hayan ocurrido porque
el momento era propicio para tales descubrimientos, dado el conocimiento
acumulado por la humanidad en el momento en que se realizaron los
descubrimientos. A veces, dos cient\'ificos se sienten estimulados al leer la
misma investigaci\'on preliminar de uno de sus contempor\'aneos. Por otro lado, los
m\'isticos han sugerido que existe un significado m\'as profundo para tales
coincidencias. El bi\'ologo austriaco Paul Kammerer (1880-1926) escribi\'o:
\textit{“Llegamos as\'i a la imagen de un mosaico mundial o caleidoscopio c\'osmico que, a
pesar de las constantes reorganizaciones y reorganizaciones, tambi\'en se ocupa de
unir lo similar”}. Compar\'o los eventos en nuestro mundo con la parte superior de
las olas del oc\'eano que parecen aisladas y sin relaci\'on. De acuerdo con su
controvertida teor\'ia, notamos la parte superior de las olas, pero debajo de la
superficie puede existir alg\'un tipo de mecanismo sincr\'onico que misteriosamente
conecta eventos en nuestro mundo y hace que se agrupen.

Georges Ifrah en The Universal History of Numbers analiza la simultaneidad al
escribir sobre las matem\'aticas mayas:

\textit{Por lo tanto, vemos una vez m\'as c\'omo las personas que han estado muy separadas
en el tiempo o en el espacio han... llegado a resultados muy similares, si no
id\'enticos... En algunos casos, la explicaci\'on de esto se puede encontrar en los
contactos e influencias entre diferentes grupos de personas... La verdadera
explicaci\'on est\'a en lo que antes hemos llamado la unidad profunda de la cultura:
la inteligencia del Homo sapiens es universal y su potencial es notablemente
uniforme en todas partes del mundo.}

Los pueblos antiguos, como los griegos, ten\'ian una profunda fascinaci\'on por los
n\'umeros. ¿Podr\'ia ser que en tiempos dif\'iciles los n\'umeros fueran lo \'unico
constante en un mundo en constante cambio? Para los pitag\'oricos, una antigua
secta griega, los n\'umeros eran tangibles, inmutables, c\'omodos, eternos, m\'as
confiables que los amigos, menos amenazantes que Apolo y Zeus.

Muchos aspectos del universo se pueden expresar con n\'umeros enteros. Los
patrones num\'ericos describen la disposici\'on de los p\'etalos en una margarita, la
reproducci\'on de los conejos, la \'orbita de los planetas, las armon\'ias de la
m\'usica y las relaciones entre los elementos de la tabla peri\'odica.

Desde la \'epoca de Pit\'agoras, el papel de las proporciones de n\'umeros enteros en
las escalas musicales ha sido ampliamente apreciado. M\'as importante a\'un, los
n\'umeros enteros han sido cruciales en la evoluci\'on de la comprensi\'on cient\'ifica
de la humanidad. Por ejemplo, el qu\'imico franc\'es Antoine Lavoisier (1743–1794)
descubri\'o que los compuestos qu\'imicos se componen de proporciones fijas de
elementos que corresponden a las proporciones de n\'umeros enteros peque\~nos. Esta
fue una evidencia muy fuerte de la existencia de los \'atomos. En 1925, ciertas
relaciones enteras entre las longitudes de onda de las l\'ineas espectrales
emitidas por los \'atomos excitados dieron las primeras pistas sobre la estructura
de los \'atomos. Las proporciones casi enteras de los pesos at\'omicos fueron
evidencia de que el n\'ucleo at\'omico est\'a formado por un n\'umero entero de
nucleones similares (protones y neutrones).

Las desviaciones de las proporciones de enteros condujeron al descubrimiento de
is\'otopos elementales (variantes con un comportamiento qu\'imico casi id\'entico pero
con diferente n\'umero de neutrones). Peque\~nas divergencias en las masas at\'omicas
de los is\'otopos puros de los n\'umeros enteros exactos confirmaron la famosa
ecuaci\'on de Einstein $E = mc^2$ y tambi\'en la posibilidad de bombas at\'omicas. Los
n\'umeros enteros est\'an en todas partes en la f\'isica at\'omica. Las relaciones
enteras son hebras fundamentales en el entramado matem\'atico, o como dijo el
matem\'atico alem\'an Carl Friedrich Gauss (1777-1855): "Las matem\'aticas son la
reina de las ciencias, y la teor\'ia de n\'umeros es la reina de las matem\'aticas".

Nuestra descripci\'on matem\'atica del universo crece para siempre, pero nuestros
cerebros y habilidades lingü\'isticas permanecen arraigados. Se est\'an descubriendo
o creando nuevos tipos de matem\'aticas todo el tiempo, pero necesitamos nuevas
formas de pensar y comprender. 

Cuando Werner Heisenberg se preocup\'o de que los seres humanos nunca llegaran a
entender verdaderamente los \'atomos, Niels Bohr fue un poco m\'as optimista. \'el
respondi\'o a principios de la d\'ecada de 1920: \textit{"Creo que a\'un podemos
hacerlo, pero en el proceso es posible que tengamos que aprender qu\'e significa
realmente la palabra comprensi\'on"}. Hoy, usamos computadoras para ayudarnos a
razonar m\'as all\'a de las limitaciones de nuestra propia intuici\'on. De hecho, los
experimentos con computadoras est\'an llevando a los matem\'aticos a descubrimientos
e ideas nunca antes so\~nados antes de la ubicuidad de estos dispositivos. Las
computadoras y los gr\'aficos por computadora permiten a los matem\'aticos descubrir
resultados mucho antes de que puedan probarlos formalmente y abren campos
matem\'aticos completamente nuevos. 

En ocasiones, se han utilizado teor\'ias matem\'aticas para predecir fen\'omenos que
no se confirmaron hasta a\~nos despu\'es. Por ejemplo, las ecuaciones de Maxwell,
llamadas as\'i por el f\'isico James Clerk Maxwell, predijeron las ondas de radio.
Las ecuaciones de campo de Einstein sugirieron que la gravedad doblar\'ia la luz y
que el universo se est\'a expandiendo. El f\'isico Paul Dirac se\~nal\'o una vez que las
matem\'aticas abstractas que estudiamos ahora nos dan una idea de la f\'isica en el
futuro. De hecho, sus ecuaciones predijeron la existencia de antimateria, que
posteriormente fue descubierta. De manera similar, el matem\'atico Nikolai
Lobachevsky dijo que “no hay rama de las matem\'aticas, por abstracta que sea, que
alg\'un d\'ia no pueda aplicarse a los fen\'omenos del mundo real”.

%Buscar 10 problemas de las matem\'aticas
% Problemas en los que usamos el residuo de una divisi\'on
% Investigar que es congruencia zeller

\pagebreak
\bibliography{../references/references.bib} 
\bibliographystyle{apalike}

\end{document}
