\documentclass{article}
\usepackage[spanish,es-tabla,es-nodecimaldot]{babel}
\usepackage{tikz}
\usepackage{amsmath}
\usepackage{hyperref}
\usepackage{listings} 
\usepackage{color} 
\usepackage{algorithm}
\usepackage{algorithmic}
\usepackage{amssymb}
\usepackage[makeroom]{cancel}

\providecommand{\abs}[1]{\lvert#1\rvert}

\input{specifications}

\begin{document}
\title{Probabilidad, procesos aleatorios e inferencia}
\author{Ana Maritza Bello Ya\~nez}
\maketitle
\setlength{\parindent}{-3pt}
\setlength{\parskip}{1em}

\tableofcontents

\section{Probabilidad y conteo}

\subsection{Definici\'on cl\'asica}

La definici\'on cl\'asica de la probabilidad \cite{faber2012statistics} de un
evento $A$, puede ser formulada de la siguiente manera:

\begin{equation}
    P(A)= \frac{n_A}{n_(total)}
\end{equation}

Donde $n_A$ es el n\'umero de formas igualmente probables por las cuales un
experimento puede conducir a A; $n_(total)$ n\'umero total de formas igualmente
probables en el experimento.

- En la definici\'on cl\'asica el experimento no necesariamente se lleva a cabo
ya que la respuesta es conocida de antemano.

- La teor\'ia cl\'asica no da soluci\'on a menos de que todas las formas
igualmente posibles puedan ser derivadas anal\'iticamente.

\section{Experimentos deterministas vs aleatorios}

\subsection{Definici\'on: Observaci\'on}

Cualquier registro de informaci\'on, ya sea num\'erico o categ\'orico. Por
consiguiente, los n\'umeros 2, 0, 1 y 2, que representan el n\'umero de
accidentes que ocurrieron cada mes, de enero a abril, durante el a\~no pasado en
la intersecci\'on de Driftwood Lane y Royal Oak Drive, constituyen un conjunto
de observaciones. Lo mismo ocurre con los datos categ\'oricos N, D, N, N y D,
que representan los art\'iculos defectuosos o no defectuosos cuando se
inspeccionan cinco art\'iculos y se registran como observaciones
\cite{walpole2012probabilidad}.

\subsection{Definici\'on: Experimento}

Cualquier proceso que genere un conjunto de datos. Un ejemplo simple de
experimento estad\'istico es el lanzamiento de una moneda al aire. En tal
experimento s\'olo hay dos resultados posibles: cara o cruz
\cite{walpole2012probabilidad}.

\subsection{Experimento determinista}
Un experimento determinista es aquel que produce el mismo resultado cuando se le
repite bajo las mismas condiciones.

\subsection{Experimento aleatorio}
Un experimento aleatorio es aquel que, cuando se le repite bajo las mismas
condiciones, el resultado que se observa no siempre es el mismo y tampoco es
predecible.

\section{Binomio de Newton}

La fórmula matemática del binomio de Newton es la siguiente:

\begin{equation}
    (a+b)^n = \sum_{k=0}^{n} \binom{n}{k} a^{n-k} b^k
\end{equation}

\subsection{Desarrollo del binomio de Newton de la potencia 1 a la 5}



\section{Convenio de suma de Einstein}

Se denomina notación de Einstein o notación indexada a la convención utilizada
para abreviar la escritura de las sumatorias, donde se suprime el término de la
sumatoria $(\sum)$. Este convenio fue introducido por Albert Einstein en 1916.

Dada una expresión lineal en $\mathbb{R^n}$ en la que se escriban todos sus
términos en forma explícita:

\begin{equation}
    u = u_1 + u_2 + ... = u_n
\end{equation}

Se puede escribir de la forma:

\begin{equation}
    u = \sum_{i=1}^{n} u_ix_i
\end{equation}

La notación de Einstein obtiene una expresión aún más condensada eliminando el
signo de la sumatoria y entendiendo que la expresión resultante en un índice
indica la suma sobre todos los posibles valores del mismo.

\begin{equation}
    u = u_ix_i
\end{equation}

\subsection{Tensor Levi-Civita}
Está definido por:

\begin{equation}
    \epsilon_{ijk}=
    \left\lbrace\begin{array}{clllr} 
        +1 & ijk, & kji, & jki & \text{Permutación cíclica o par}\\ 
        -1 & ikj, & kji, & jik & \text{Permutación anticíclica o impar}\\
        0  & iii, & jjj, & kkk & \text{Si se repiten}
    \end{array}\right.
\end{equation}

\subsection{Producto escalar de dos vectores}

También conocido como producto punto, es una operación algebraíca que toma dos
vectores y retorna un escalar. esta definido de la siguiente manera:

\begin{equation}
    \bar{A} \cdot \bar{B}= \bar{\abs{A}} \bar{\abs{B}} \cos \theta
    = \sum_{i=1}^{n} a_i b_i
\end{equation}

Que de acuerdo al convenio de suma de Einstein podemos escribir como:

\begin{equation}
    \bar{A} \cdot \bar{B} = a_i b_i
\end{equation}

\subsection{Producto vectorial de dos vectores}

EL producto vectorial o producto cruz de dos vectores es una operación binaria
entre dos vectores en un espacio tridimensional. El resultado es un vectore
perpendicular a los vectores que se multiplican, y por lo tanto normal al plano
que los contiene.

Está definido de la siguiente manera:

\begin{equation}
    \bar{A} \times \bar{B}= \bar{\abs{A}} \bar{\abs{B}} \sin \theta \hat{r}
\end{equation}

Donde $\hat{r}$ es el vector unitario y ortogonal a los vectores $\bar{A}$ y
$\bar{B}$, y $\theta$ el ángulo entre $\bar{A}$ y $\bar{B}$.

Qué mediante determinantes tenemos:

\begin{equation}
    \bar{A} \times \bar{B}=
    \begin{bmatrix}
        \hat{i} & \hat{j} & \hat{k}\\ 
         a_1 & a_1 & a_1\\ 
         b_1 & b_1 & b_1
    \end{bmatrix}
\end{equation}

Desarrollamos:

\begin{equation}
    \bar{A} \times \bar{B}= \hat{i}(a_2 b_3 - a_3 b_2) - \hat{j}(a_1 b_3 - a_3 b_1) + \hat{k}(a_1 b_2 - a_2 b_1)
\end{equation}

Desde un punto de vista tensorial el producto generalizado de n vectores vendrá
dado por:

\begin{equation}
    (\bar{A} \times \bar{B}) = \epsilon_{ijk} a_{jbk}
\end{equation}

Desarrollando:

\begin{equation}
    \epsilon_{ijk} a_{jbk} =
    \begin{array}{ccccc}
        \cancel{\epsilon_{111} a_1 b_1} & + & \cancel{\epsilon_{112} a_1 b_2}   & + & \cancel{\epsilon_{112} a_1 b_3}   \\
        \cancel{\epsilon_{121} a_2 b_1} & + & \cancel{\epsilon_{122} a_2 b_2}   & + & \epsilon_{123} a_2 b_3            \\
        \cancel{\epsilon_{131} a_3 b_1} & + & \epsilon_{132} a_3 b_2            & + & \cancel{\epsilon_{133} a_3 b_3}
    \end{array}
    = a_2 b_3 - a_3 b_2
\end{equation}

\section{Tri\'angulo de Sierpinski}

El matem\'atico polaco Waclav Sierpinski (1882-1969), construy\'o este fractal
en 1919 del modo siguiente: tom\'o un tri\'angulo equil\'atero, uni\'o los
puntos medios de los lados y form\'o cuatro tri\'angulos interiores: tres
tri\'angulos equil\'ateros sombreados y un hueco que es otro tri\'angulo
equil\'atero. Repiti\'o el proceso en cada uno de los tri\'angulos sombreados, y
sigui\'o hasta el infinito el proceso en los tres tri\'angulos restantes como el
primero.


\begin{figure}[h]
    \centering
    \usetikzlibrary{lindenmayersystems}
\def\trianglewidth{3cm} \pgfdeclarelindenmayersystem{Sierpinski triangle}{
        \symbol{X}{\pgflsystemdrawforward}
        \symbol{Y}{\pgflsystemdrawforward}
        \rule{X -> X-Y+X+Y-X}
        \rule{Y -> YY}
} \foreach \level in {1,3,5}{ \tikzset{
l-system={step=\trianglewidth/(2^\level), order=\level, angle=-120} }

        \begin{tikzpicture}
\fill [black] (0,0) -- ++(0:\trianglewidth) -- ++(120:\trianglewidth) -- cycle;
\draw [draw=none] (0,0) l-system [l-system={Sierpinski triangle,
axiom=X},fill=white];
        \end{tikzpicture}
}
    \caption{Tri\'angulo de Sierpinski para los niveles 1,3 y 5.}
    \label{fig:sierpinski}
\end{figure}

Como se muestra en la Fig.(\ref{fig:sierpinski})

\section{La belleza y utilidad de las matem\'aticas}

Las matem\'aticas han impregnado todos los campos de la actividad cient\'ifica y
desempe\~nan un papel inestimable en la biolog\'ia, la f\'isica, la qu\'imica,
la econom\'ia, la sociolog\'ia y la ingenier\'ia.


El descubrimiento simult\'aneo ha ocurrido a menudo en la historia de las
matem\'aticas.

Por ejemplo el del c\'alculo del erudito ingl\'es Isaac Newton (1643-1727) y el
matem\'atico alem\'an Gottfried Wilhelm Leibniz (1646-1716). Esto nos hacen
preguntarnos por qu\'e se hicieron estos descubrimientos cient\'ificos en al
mismo tiempo por personas que trabajan de forma independiente. Para dar otro
ejemplo, los naturalistas brit\'anicos Charles Darwin (1809–1882) y Alfred
Wallace (1823–1913) desarrollaron la teor\'ia de la evoluci\'on de manera
independiente y simult\'anea. De manera similar, el matem\'atico h\'ungaro
J\'anos Bolyai (1802–1860) y el matem\'atico ruso Nikolai Lobachevsky
(1793–1856) parec\'ian haber desarrollado la geometr\'ia hiperb\'olica de forma
independiente y al mismo tiempo.

Lo m\'as probable es que tales descubrimientos simult\'aneos hayan ocurrido
porque el momento era propicio para tales descubrimientos, dado el conocimiento
acumulado por la humanidad en el momento en que se realizaron los
descubrimientos. A veces, dos cient\'ificos se sienten estimulados al leer la
misma investigaci\'on preliminar de uno de sus contempor\'aneos. Por otro lado,
los m\'isticos han sugerido que existe un significado m\'as profundo para tales
coincidencias.

En ocasiones, se han utilizado teor\'ias matem\'aticas para predecir fen\'omenos
que no se confirmaron hasta a\~nos despu\'es. Por ejemplo, las ecuaciones de
Maxwell, llamadas as\'i por el f\'isico James Clerk Maxwell, predijeron las
ondas de radio. Las ecuaciones de campo de Einstein sugirieron que la gravedad
doblar\'ia la luz y que el universo se est\'a expandiendo. El f\'isico Paul
Dirac se\~nal\'o una vez que las matem\'aticas abstractas que estudiamos ahora
nos dan una idea de la f\'isica en el futuro. De hecho, sus ecuaciones
predijeron la existencia de antimateria, que posteriormente fue descubierta. De
manera similar, el matem\'atico Nikolai Lobachevsky dijo que “no hay rama de las
matem\'aticas, por abstracta que sea, que alg\'un d\'ia no pueda aplicarse a los
fen\'omenos del mundo real”.

% Buscar 10 problemas de las matem\'aticas Problemas en los que usamos el
% residuo de una divisi\'on Investigar que es congruencia zeller Buscar
% propiedades del triángulo de Pascal

\subsection{Congruencia de Zeller}

% \begin{lstlisting}
% \lstinputlisting[language=Python]{../src/zeller_congruency.py}
% \end{lstlisting}





\pagebreak
\bibliography{../references/references.bib} 
\bibliographystyle{unsrt}

\end{document}
