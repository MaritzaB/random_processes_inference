\documentclass[12pt]{article}
\usepackage[spanish,es-tabla,es-nodecimaldot]{babel}
\usepackage{tikz}
\usepackage{amsmath}    %Caracteres especiales de matematicas
\usepackage[hidelinks]{hyperref}    % Vinculos [que no estén subrayados]
\usepackage{listings}   %Introducir codigo
\usepackage{setspace}
\usepackage{color}
\usepackage{amssymb}
\usepackage[makeroom]{cancel}
\usepackage{pdflscape}  %Voltear la página
\usepackage{tcolorbox}  %Cuadros de teoremas
\usepackage{fancyhdr, lastpage}
\usepgflibrary{fpu}
\usepackage{ifthen}
\usepackage{graphicx}
\usepackage{caption}
\usepackage{subcaption}
\usepackage{multirow}
\usepackage{tikz-3dplot}
\usepackage{pgfplots}
\usepgfplotslibrary{external}

\tcbuselibrary{listings,theorems}

\pagestyle{fancy}   %Encabezado en las páginas
\chead{Probabilidad, procesos aleatorios e inferencia}  %Encabezado central
\lhead{}    %Encabezado izquiero vacio
\rhead{}    %Encabezado derecho vacio
%% Cambiar letra a tamaño 12

\usepackage[Glenn]{fncychap}
% Sonny, Lenny, Glenn, Conny, Rejne, Bjarne, Bjornstrup

\providecommand{\abs}[1]{\lvert#1\rvert}

\input{custom/specifications}
\input{custom/tikz-3dplot_documentation_figures.tex}

\begin{document}
\title{Probabilidad, procesos aleatorios e inferencia}
\author{Ana Maritza Bello Ya\~nez}
\maketitle


%%%%%%%%%%%%%%%%%%%%%%%%%%%%%%%%%%%%%%%%%%%%%%%%%%%%%%%%%%%%%%%%%%%%%
%                       Caja del teorema                            %
%                                                                   %
\newtcbtheorem{theorem}{Teorema}             %
{colback=gray!5,colframe=gray!35!black,fonttitle=\bfseries}{th}   %
%%%%%%%%%%%%%%%%%%%%%%%%%%%%%%%%%%%%%%%%%%%%%%%%%%%%%%%%%%%%%%%%%%%%%

\tableofcontents

\setlength{\parindent}{0pt}
\setlength{\parskip}{0em}
\pagebreak
\section{Resumen: La belleza y utilidad de las matem\'aticas \ref{hestenes2012clifford}}

Las matem\'aticas han impregnado todos los campos de la actividad cient\'ifica y
desempe\~nan un papel inestimable en la biolog\'ia, la f\'isica, la qu\'imica,
la econom\'ia, la sociolog\'ia y la ingenier\'ia.


El descubrimiento simult\'aneo ha ocurrido a menudo en la historia de las
matem\'aticas.

Por ejemplo el del c\'alculo del erudito ingl\'es Isaac Newton (1643-1727) y el
matem\'atico alem\'an Gottfried Wilhelm Leibniz (1646-1716). Esto nos hacen
preguntarnos por qu\'e se hicieron estos descubrimientos cient\'ificos en al
mismo tiempo por personas que trabajan de forma independiente. Para dar otro
ejemplo, los naturalistas brit\'anicos Charles Darwin (1809-1882) y Alfred
Wallace (1823-1913) desarrollaron la teor\'ia de la evoluci\'on de manera
independiente y simult\'anea. De manera similar, el matem\'atico h\'ungaro
J\'anos Bolyai (1802-1860) y el matem\'atico ruso Nikolai Lobachevsky
(1793-1856) parec\'ian haber desarrollado la geometr\'ia hiperb\'olica de forma
independiente y al mismo tiempo.

Lo m\'as probable es que tales descubrimientos simult\'aneos hayan ocurrido
porque el momento era propicio para tales descubrimientos, dado el conocimiento
acumulado por la humanidad en el momento en que se realizaron los
descubrimientos. A veces, dos cient\'ificos se sienten estimulados al leer la
misma investigaci\'on preliminar de uno de sus contempor\'aneos. Por otro lado,
los m\'isticos han sugerido que existe un significado m\'as profundo para tales
coincidencias.

En ocasiones, se han utilizado teor\'ias matem\'aticas para predecir fen\'omenos
que no se confirmaron hasta a\~nos despu\'es. Por ejemplo, las ecuaciones de
Maxwell, llamadas as\'i por el f\'isico James Clerk Maxwell, predijeron las
ondas de radio. Las ecuaciones de campo de Einstein sugirieron que la gravedad
doblar\'ia la luz y que el universo se est\'a expandiendo. El f\'isico Paul
Dirac se\~nal\'o una vez que las matem\'aticas abstractas que estudiamos ahora
nos dan una idea de la f\'isica en el futuro. De hecho, sus ecuaciones
predijeron la existencia de antimateria, que posteriormente fue descubierta. De
manera similar, el matem\'atico Nikolai Lobachevsky dijo que “no hay rama de las
matem\'aticas, por abstracta que sea, que alg\'un d\'ia no pueda aplicarse a los
fen\'omenos del mundo real”.



\section*{Experimentos deterministas vs aleatorios}


\begin{tcolorbox}[colback=blue!5!white,colframe=blue!60!black,title=Definición: Observación]
    Cualquier registro de informaci\'on, ya sea num\'erico o categ\'orico.
\end{tcolorbox}

Ejemplos: 

Los n\'umeros 2, 0, 1 y 2, que representan el n\'umero de accidentes que
ocurrieron cada mes, de enero a abril, durante el a\~no pasado, constituyen un
conjunto de observaciones. 

Lo mismo ocurre con los datos categ\'oricos N, D, N, N y D, que representan los
art\'iculos defectuosos o no defectuosos cuando se inspeccionan cinco
art\'iculos y se registran como observaciones \cite{walpole2012probabilidad}.

\begin{tcolorbox}[colback=blue!5!white,colframe=blue!60!black,title=Definición: Experimento]
    Cualquier proceso que genere un conjunto de datos.    
\end{tcolorbox}

Un ejemplo simple de experimento estad\'istico es el lanzamiento de una moneda
al aire. En tal experimento s\'olo hay dos resultados posibles: águila o sol.
\cite{walpole2012probabilidad}.

\begin{tcolorbox}[colback=blue!5!white,colframe=blue!60!black,title=Definición: Tipos de experimentos]

    \textbf{Experimento determinista.}
    Un experimento determinista es aquel que produce el mismo resultado cuando se le
    repite bajo las mismas condiciones.

    \tcblower

    \textbf{Experimento aleatorio.}
    Un experimento aleatorio es aquel que, cuando se le repite bajo las mismas
    condiciones, el resultado que se observa no siempre es el mismo y tampoco es
    predecible.
\end{tcolorbox}



\section{Teoría de conjuntos}

\subsection{Introducción}

El concepto de conjunto, es fundamental en matemáticas. Su estudio, se basa en
el hecho de que éstos pueden ser combinados, mediante ciertas operaciones, para
formar otros conjuntos. 

El estudio de las operaciones con los conjuntos, constituye el álgebra de
conjuntos; que tiene semejanzas formales (aunque también presenta diferencias)
con el álgebra de los números. El álgebra de conjuntos, resulta valiosa en la
reducción de los conceptos matemáticos a sus fundamentos lógicos. 

La disciplina matemática que estudia las propiedades generales de los conjuntos
es la teoría de conjuntos. Esta disciplina se comenzó a desarrollar,
rigurosamente, a finales del siglo XIX y principios del XX. El fundador de dicha
teoría es el matemático alemán de origen ruso, George Ferdinand Ludwing
Phillipp Cantor (1845, 1918). Las ideas y conceptos de la teoría de conjuntos
han irrumpido literalmente en todas las ramas de las matemáticas y cambiaron su
faz por completo.

\subsection{Definición básica de conjunto}

Un conjunto es cualquier colección de objetos bien definidos por medio de alguna
o algunas propiedades en común, de dichos objetos. Por objeto entenderemos no
sólo cosas físicas, como discos, computadores, etc., si no también abstractos,
como son números, letras, etc. A los objetos se les llama elementos del
conjunto.

Representamos a los conjuntos por medio de letras mayúsculas, así A, B, C, etc.
nos representan conjuntos.

\subsection{Elemento}

Se llaman elementos o miembros a los objetos que componen un conjunto; y se
denotan con letras minúsculas, como: $a, b, x, y$, etc. Para indicar que un
objeto $x$ pertenece o es miembro de un conjunto $A$, escribimos $x \in A$, que
se lee \textit{“x elemento de A”} y, si no pertenece al conjunto $A$, escribimos
$x \notin A$, que se lee \textit{“x no es elemento de A”}.

\subsection{Conjunto universo}
Llamamos conjunto universo y lo denotamos por U, al conjunto del cual se
seleccionan los elementos para formar conjuntos.

En general identificamos al conjunto universo como un todo, pero no
representamos de una manera única a este “todo”, así habrá ocasiones que un
conjunto sea considerado conjunto universo y en otras no, por ejemplo:
considerando el conjunto P de los número pares, se tiene que $\mathbb Z$
(conjunto enteros) es un conjunto universo para él; pero si consideramos el
mismo conjunto $\mathbb Z$, tenemos que $\mathbb Q$ (racionales) es un conjunto
universo de él. Así Z es un conjunto universo en ocasiones y en otras no. Además
puede haber varios conjuntos universos para un solo conjunto, por ejemplo el
mismo conjunto P de números pares tiene por conjunto universo a $\mathbb Z$
(conjunto de los enteros) o a Q (conjunto se los racionales) o a $\mathbb R$
(conjunto de los reales) o incluso a $\mathbb C$ (conjunto de los complejos), en
general podemos decir que para números el conjunto universo más grande es
$\mathbb C$ (números complejos).

\subsection{Conjunto vacío}

El conjunto vacío o nulo es un conjunto que no tiene elementos, el conjunto
vacío se representa por: $\emptyset$, o bien por {}.

No confundir el conjunto $A = \emptyset $ con el conjunto $ A = {\emptyset} $,
ya que el primer conjunto indica que no tiene ningún elemento y el segundo
conjunto indica que tiene un elemento y ese elemento es el conjunto vacío.

\subsection{Igualdad de conjuntos}

Decimos que dos conjuntos $A$ y $B$ son iguales y denotamos $A=B$, si $A$ y $B$
constan de los mismos elementos, i.e. si cada elemento de $A$ pertenece a $B$ y
si cada elemento de $B$ pertenece a $A$.

\subsection{Subconjunto}

Si todos los elementos de un conjunto A son también elementos de un conjunto
$B$, esto es, si cuando $x \in A$ entonces $x \in B$ (simbólicamente $x \in A
\rightarrow x \in B$), decimos que A es un subconjunto de B o que A está
contenido en B y se escribe:

\begin{equation}
    \begin{array}{l}
        A \subseteq B \\
        B \subseteq A
    \end{array}
\end{equation}

Si $A$ no es subconjunto de $B$ se escribe $A \not \subset B$ .

Si además existe un elemento de $B$ que no este en $A$, decimos que $A$ es un
subconjunto propio de $B$ y se denota $A \subset B$ o $B \subset A$.

\subsection{Propiedades de los conjuntos}

\begin{enumerate}
    \item Si A es cualquier conjunto diferente del vacío, entonces $A \subseteq A$ . Esto es, cualquier conjunto
    diferente del vacío es subconjunto de si mismo.
    
\item El conjunto vacío es subconjunto de cualquier conjunto distinto del vacío,
esto es $\emptyset \subset A$

\item Si $n>0$ es el número de elementos de $A$ entonces el número de elementos de $P(A)$ es $2^n$
\end{enumerate}

\begin{theorem}{Conjuntos}{conjuntos}
Sean $A$ y $B$ dos conjuntos. Decimos que $A=B$ si y solo si $A \subseteq B$ y
$B \subseteq A$
\end{theorem}

\begin{theorem}{Conjuntos}{conjuntos}
    Si $ A \subset B $ y $ B \subset C $, entonces $ A \subset C $
\end{theorem}

\subsection{Cardinalidad}

Sea $A$ un conjunto, la cardinalidad de $A$ es el número de elementos diferentes del conjunto
$A$ y se representa por $|A|$.

\subsection{Conjunto potencia}
Sea $A$ un conjunto finito, llamaremos conjunto potencia al conjunto formado por todos los
subconjuntos de $A$. El conjunto potencia se denota como $P(A)$ o bien $2^A$.

\subsection{Operaciones entre conjuntos}

% Definition of circles
\def\firstcircle{(0,0) circle (1.5cm)}
\def\secondcircle{(0:2cm) circle (1.5cm)}

\colorlet{circle edge}{blue!50}
\colorlet{circle area}{blue!20}

\tikzset{filled/.style={fill=circle area, draw=circle edge, thick},
    outline/.style={draw=circle edge, thick}}

\setlength{\parskip}{5mm}

\subsubsection{Unión}

La unión La unión (o reunión) de dos conjuntos $A$ y $B$, denotada por $ A \cup
B $ (que se lee “A unión B”) es un nuevo conjunto formado por los elementos que
pertenecen a A o a B o a ambos conjuntos.

\begin{equation}
    A \cup B = {x | x \in A \vee x \in B}
\end{equation}

Si $x \in A \cup B $ entonces $x \in A $ ó $x \in B$ o $x$ pertenece a ambos
conjuntos.

% Set A or B
\begin{figure}[h]
    \centering
    \begin{tikzpicture}
        \draw[filled] \firstcircle node {$A$}
                      \secondcircle node {$B$};
        \node[anchor=south] at (current bounding box.north) {$A \cup B$};
    \end{tikzpicture}
\end{figure}

\subsubsection{Intersección}

La intersección de dos conjuntos $A$ y $B$, denotada por $A \cap B$ (que se lee “ A intersección B”),
es un nuevo conjunto formado por los elementos que pertenecen a A y a B al mismo tiempo, es
decir, por los elementos comunes a ambos conjuntos.

\begin{equation}
    A \cap B={x | x \in A \wedge x \in B}
\end{equation}

% Set A and B
\begin{figure}[h]
    \centering
    \begin{tikzpicture}
        \begin{scope}
            \clip \firstcircle;
            \fill[filled] \secondcircle;
        \end{scope}
        \draw[outline] \firstcircle node {$A$};
        \draw[outline] \secondcircle node {$B$};
        \node[anchor=south] at (current bounding box.north) {$A \cap B$};
    \end{tikzpicture}
\end{figure}

\subsubsection{Complemento}

Sea $A \subseteq U$ un conjunto. El complemento de A denotado por $A'$, $A^C$ ó $U-A$, se define
como el conjunto de todos los elementos que están en $U$ pero no están en $A$.
Simbólicamente:

\begin{equation}
    \bar{A}= {x | x \in U \wedge x \notin A}
\end{equation}

Es decir, el complemento de $A$ es el conjunto de los todos los elementos que no
están en el conjunto $A$. Simbólicamente: $ \bar{A}={x | x \notin A}$.

\subsubsection{Diferencia}

% Set A but not B
\begin{figure}[h]
    \centering
    \begin{subfigure}[b]{0.45\textwidth}
        \centering
        \begin{tikzpicture}
            \begin{scope}
                \clip \firstcircle;
                \draw[filled, even odd rule] \firstcircle node {$A$}
                                             \secondcircle;
            \end{scope}
            \draw[outline] \firstcircle
                           \secondcircle node {$B$};
            \node[anchor=south] at (current bounding box.north) {$A - B$};
        \end{tikzpicture}
    \end{subfigure}
\hfill
    \begin{subfigure}[b]{0.45\textwidth}
        \centering
        % Set B but not A
        \begin{tikzpicture}
            \begin{scope}
                \clip \secondcircle;
                \draw[filled, even odd rule] \firstcircle
                                             \secondcircle node {$B$};
            \end{scope}
            \draw[outline] \firstcircle node {$A$}
                           \secondcircle;
            \node[anchor=south] at (current bounding box.north) {$B - A$};
        \end{tikzpicture}
\end{subfigure}
\end{figure}

\subsubsection{Diferencia simétrica}



%Set A or B but not (A and B) also known a A xor B
\begin{figure}[h]
    \centering
    \begin{tikzpicture}
        \draw[filled, even odd rule] \firstcircle node {$A$}
                                     \secondcircle node{$B$};
        \node[anchor=south] at (current bounding box.north) {$\overline{A \cap B}$};
    \end{tikzpicture}
\end{figure}






\section{Binomio de Newton}

El binomio de Newton consiste en una fórmula que permite obtener los
coeficientes de un término enésimo de un binomio elevado a un exponente
determinado.

La fórmula matemática del binomio de Newton es la siguiente:

\begin{equation}
    (a+b)^n = \sum_{k=0}^{n} \binom{n}{k} a^{n-k} b^k
\end{equation}

\subsection{Desarrollo del binomio de Newton de la potencia 1 a la 10}

\begin{landscape}
	\begin{equation}
		\begin{split}
			(a+b)^0 &   = 1 \\
			(a+b)^1 &   = a+b \\
			(a+b)^2 &   = a^2+2ab+b^2 \\
			(a+b)^3 &   = a^3+3a^2b+3ab^2+b^3 \\
			(a+b)^4 &   = a^4+4a^3b+6a^2b^2+4ab^3+b^4 \\
			(a+b)^5 &   = a^5+5a^4b+10a^3b^2+10a^2b^3+5ab^4+b^5\\
			(a+b)^6 &   = a^6+6a^5b+15a^4b^2+20a^3b^3+15a^2b^4+6ab^5+b^6\\
			(a+b)^7 &   = a^7+7a^6b+21a^5b^2+35a^4b^3+35a^3b^4+21ba^2b^5+21ab^6+b^7 \\
			(a+b)^8 &   = a^8+8a^7b+28a^6b^2+56a^5b^3+70a^4b^4+56a^3b^5+28a^2b^6+8ab^7+b^8 \\
			(a+b)^9 &   = a^9+9a^8b+36a^7b^2+84a^6b^3+126a^5b^4+126a^4b^5+84a^3b^6+36a^2b^7+9ab^8+b \\
			(a+b)^{10} &    = a^{10}+10a^9b+45a^8b^2+120a^7b^3+210a^6b^4+252a^5b^5+210a^4b^6+120a^3b^7+45a^2b^8+10ab^9+b^{10} \\
		\end{split}
	\end{equation}
\end{landscape}


\section*{Triángulo de Pascal}

En las matemáticas, el triángulo de Pascal es una representación de los
coeficientes binomiales ordenados en forma de triángulo. Es llamado así en honor
al filósofo y matemático francés Blaise Pascal, quien introdujo esta notación en
1654, en su Tratado del triángulo aritmético.

\newcommand{\pasc}[2]{
	\pgfkeys{/pgf/fpu}
	\pgfmathparse{round(#1!/((#1-#2)!*#2!))}
	\pgfmathfloattoint{\pgfmathresult}
	\pgfmathresult
}

\begin{figure}[h]
	\centering
	\begin{tikzpicture}
	\pgfmathsetmacro{\N}{10};
	\foreach \i in {0,...,\N}{
		\foreach \j in {0,...,\i}{
			%\pgfmathtruncatemacro{\var}{\pasc{\i}{\j}}
			\node at ({-0.5*\i+\j-0.2},-\i){\pasc{\i}{\j}};
			
			%\ifthenelse{\pasc{\i}{\j}=}{then clause}{else clause}
			% \draw ({-0.5*\i+\j-0.5},-\i+0.5) rectangle
			% ({-0.5*\i+\j-0.5},-\i-0.5);
			\draw ({-0.5*\i+\j},-\i) circle (0.5) ;
		}
	}
	\end{tikzpicture}
	\caption{Triángulo de Pascal para n=10}
	\label{fig:pascalTriangle}
\end{figure}

Este triángulo fue ideado para desarrollar las potencias de binomios.

La fórmula del binomio de Newton desarrolla los coeficientes de cada fila en el
triángulo de Pascal. Es por esto que existe una estrecha relación entre el
triángulo de Pascal y el binomios de Newton.

\subsection{Propiedades del triángulo de Pascal}

Algunas de las propiedades más representativas del triángulo de Pascal son las
siguientes:

\begin{itemize}
	\item Cada número es la suma de los dos números encima de él.
	\item Todos los números exteriores son iguales a 1.
	\item El triángulo de Pascal es simétrico.
	\item La primera diagonal muestra los números de conteo.
	\item Las sumas de las filas dan las potencias del 2.
	\item Cada fila da los dígitos de las potencias del 11.
	\item Cada elemento representa a la combinación , en donde, m es la fila del elemento y n es la posición del elemento en la fila.
	\item Cada fila representa a los coeficientes binomiales.
	\item Los números Fibonacci están a lo largo de las diagonales.
\end{itemize}

\subsection{Patrones del triángulo de Pascal}

\subsubsection{Suma de las filas}

Una de las propiedades interesantes del triángulo es que la suma de los números
en una fila es igual a $2^n$, en donde, n corresponde al número de la fila. Por
ejemplo, tenemos:

$ 1 = 1 = 2^0 $ \\
$ 1 + 1 = 1 = 2 = 2^1 $ \\
$ 1 + 2 = 1 = 4 = 2^2 $\\

como se puede observar en la Fig. (\ref{fig:pascalTriangle})

\subsubsection{Números primos en el triángulo}

Otro patrón visible en el triángulo se relaciona a los números primos. Si es que
una fila empieza con un número primo o es una fila con número primo, todos los
números que están en esa fila, sin incluir al 1, son divisibles para ese número
primo.

Si es que miramos a la fila 5 (1, 5, 10, 10, 5, 1) podemos ver que el 5 y el 10
son divisibles por 5. Sin embargo, para una fila compuesta como la fila 8 (1, 8,
28, 56, 70, 56, 28, 8, 1), 28 y 70 no son divisibles por 8.

\subsubsection{Sucesión de Fibonacci en el triángulo}
En el triángulo de Pascal se puede apreciar una relación entre un modo de sumar
las diagonales y la sucesión de Fibonacci. Los primeros términos de esta
sucesión son: $1,1,2,3,5,8,13,21,34,55$ como se puede apreciar en la Fig.
(\ref{fig:pascalTriangle}).

\subsubsection{Expansión binomial con el triángulo de Pascal}

El triángulo de Pascal define a los coeficientes que aparecen en las expresiones
binomiales. Eso significa que la fila n del triángulo de Pascal contiene a los
coeficientes de la expresión expandida del binomio $(x+y)^n$.


\section{T\'ecnicas de conteo}

Muchos problemas de combinatoria implican conteo. Ya que el número de objetos a
contar podría ser muy grande, es importante ser capaz de conta el conjunto de
objetos sin tener que enlistarlos a todos.
% ARREGLAR SALTOS DE PÁGINA
% EIGENVALORES
% VALORES PROPIOS
\subsection{Principio de multiplicaci\'on}

El principio de multiplicación lo utilizamos para contar el número de formas en
las que pueden ocurrir dos eventos simultáneos. El principio de la multiplicación
establece que si un evento puede ocurrir de $n_1$ formas diferentes, y para cada
una de estas ppuede ocurrir un segundo evento simultáneo en $n_2$ formas
diferentes, entonces los dos eventos pueden ocurrir de $n_1 * n_2$ formas
diferentes.

Así, para una serie de $k$ eventos, tenemos:
\begin{theorem}{Principio de multiplicación}{multiplicación}
	\begin{equation}
		n_1*n_2*n_3*...*n_k
		\label{eq:reglaMultiplicacion}
	\end{equation}
\end{theorem}


\subsection{Diagrama de Árbol}

Los diagramas de árbol muestran todos los resultados posibles de un evento. Cada
rama en un diagrama de árbol representa un posible resultado.
Los diagramas de \'arbol pueden usarse para encontrar el n\'umero de resultados
posibles y calcular la probabilidad de los posibles resultados.

%Por ejemplo,

\begin{figure}[h]
	\begin{center}		
	\begin{tikzpicture}[level distance=1cm, level 1/.style={sibling distance=3cm},
		level 2/.style={sibling distance=2cm}, every node/.style={circle, draw,
		align=center} ]
				\centering
				\node[circle,draw]{$P_n$}
		child{node{1} child{node{2} child{node{3}} } child{node{3} child{node{2}}} }
		child{node{2} child{node{1} child{node{3}} } child{node{3} child{node{1}}} }
		child{node{3} child{node{1} child{node{2}} } child{node{2} child{node{1}}} };
			\end{tikzpicture}
		\end{center}
		\caption{Permutaciones para 3 elementos}
		\label{permutaciones_tree}
\end{figure}


\subsection{Principio de adición}

Supongamos que existen $k$ conjuntos de elementos con $n_1$ elementos en el
primer conjunto, $n_2$ elementos en el segundo conjuntos, etc. Si todos los
elementos son distintos, es decir, si si todos los pares dek conjunto $k$ son
disjuntos, entonces el número de elementos de la unión de los conjuntos es $n_1
+ n_2 + ... + n_k $.

Usando la nomenclatura de conjuntos, el principio de adición está definido de la
siguiente manera:

\begin{theorem}{Principio de adición}{adicion}
Sean $A$ y $B$ dos eventos disjuntos, entonces la probabilidad de la unión de
los conjuntos está dada por:

\begin{equation}
	P(A \cup B) = P(A) + P(B)
\end{equation}

De manera general, si el espacio de muestreo tiene un número infinito de
elementos y $A_1, A_2, ... $ son una secuencia de eventos disjuntos, entonces la
probabilidad de la unión es:

\begin{equation}
	P(A_1 \cup A_2 \cup ...) = P(A_1) + P(A_2) + ...
\end{equation}

\end{theorem}


\subsection{Principio del palomar o Principio de Dirichlet}

El principio del palomar, también llamado principio de Dirichlet o principio de
las cajas, establece que si $n$ palomas se distribuyen en $m$ palomares, y si $n
> m$, entonces al menos habrá un palomar con más de una paloma. Otra forma de
decirlo es que $m$ huecos pueden albergar como mucho $m$ objetos si cada uno de
los objetos está en un hueco distinto, así que el hecho de añadir otro objeto
fuerza a volver a utilizar alguno de los huecos, como se observa en la Fig.
(\ref{palomar}). A manera de ejemplo: si se toman trece personas, al menos dos
habrán nacido el mismo mes.

\begin{figure}[h]
	\centering
	\begin{tikzpicture}
		% draw the sets
		\filldraw[fill=gray!20, draw=gray!60] (-1.5,0) circle (2cm);
		\filldraw[fill=green!20, draw=green!60] (3,0) circle (2cm);
	
	
		% the texts
		\node at (-1.5,3) {Palomas};
		\node at (3,3) {Palomares};
	
		% the circles and the arrows
		\node (x1) at (-1.5,0.8) {$p_1$};
		\node (x2) at (-1.5,0.3) {$p_2$};
		\node (x3) at (-1.5,-0.3) {$p_3$};
		\node (x4) at (-1.5,-0.8) {$p_4$};
		
		\node (y1) at (3,0.7) {$1$};
		\node (y2) at (3,0) {$2$};
		\node (y3) at (3,-0.7) {$3$};
	
		% draw the arrows
		\draw[-latex] (x1) -- (y1);
		\draw[-latex] (x2) -- (y2);
		\draw[-latex] (x3) -- (y3);
		\draw[-latex] (x4) -- (y3);
	
	\end{tikzpicture}
\caption{Ilustración del principio del palomar. El conjunto de las palomas es
mayor al del palomar, por lo tanto en al menos un palomar deben de haber dos
palomas.}
\label{palomar}
	\end{figure}




\subsection{Permutaciones}

\begin{tcolorbox}[colback=gray!5!white,colframe=gray!60!black,title=Definición: Permutación]
	Una \textbf{permutaci\'on} de un conjunto es un arreglo u ordenamiento de sus
	elementos sin repeticiones ni omisiones.
	
	Si el conjunto está ordenado, el proceso de reordenar sus elementos, es una permutación.
	\label{Permutaciones_definition}
\end{tcolorbox}

Para saber el n\'umero de permutaciones que existen de un conjunto se debe
conocer su cardinalidad. Tomemos como ejemplo, el conjunto de n\'umeros enteros
$A=\{1, 2, 3\}$. Para cualquier permutaci\'on, en la primera posici\'on pueden
colocarse cualquiera de los tres elementos; en la segunda posición se pueden
colocar dos posibles elementos; mientras que al final, se puede colocar una
posibilidad.

En general, para cualquier conjunto de elementos $X$ de cardinalidad $|n|$, en
la primera posici\'on se pueden colocar $n$ elementos, para la siguiente
posici\'on $n-1$. Así sucesivamente hasta las \'ultimas posiciones, donde se
pueden colocar $3, 2$ y $1$ elementos. De esta manera, se dice que el n\'umero
de permutaciones es:

\begin{theorem}{Permutación}{permutacion}
	\begin{equation}
		P_n=n*(n-1)*(n-2)*...*3*2*1=n!
		\label{permutaciones_totales}
	\end{equation}
	
	Remarcando el hecho de que $0! = 1$.

\end{theorem}

\subsubsection{Permutaciones de $k$-elementos}

Sea un conjunto de elementos $X$ de cardinalidad $|n|$, como en el caso
anterior. Supongamos que deseamos tomar $k$ elementos del conjunto y contar las
diferentes formas en las que podríamos agarrarlos.

De primera instancia, podemos tomar cualquiera de los $n$ objetos como el primer
elemento. Posteriormente, solo hay $n-1$ elementos para la segunda opción y asi
sucesivamente hasta que llegamos al $k$-ésimo elemento. Lo que nos deja
$n-(k-1)$ opciones para la última opción.

Podemos partir de la Ec. (\ref{permutaciones_totales}) para hacer la permutación
de los $n$ elementos, sin embargo, tendríamos que ajustar la fórmula para que
escojamos solo $k$ de los $n$ elementos ($n-k$).
\begin{equation}
n(n-1)(n-2)...(n-(k-1))\overbrace{(n-k)(n-(k+1))...1}^{(n-k)!}
\end{equation}

Para quitar los elementos repetidos, dividimos la ecuación anterior entre el
número de elementos repetidos, de la siguiente manera:

\begin{equation}
 = \frac{ n (n-1) (n-2) ... (n-k+1) \overbrace{(n-k)(n-(k+1))...1}^{(n-k)!}}{\underbrace{(n-k)(n-k-1) ... 1}_{(n-k)!}}
\end{equation}

Lo que nos lleva a la Ec. (\ref{k-permutaciones}).

\begin{theorem}{Permutación}{permutacion}
El total de permutaciones de $n$ objetos distintos tomados de $k$ formas a la
vez, está dado por:

	\begin{equation}
		P(n,r) = \frac{n!}{(n-k)!}
		\label{k-permutaciones}
	\end{equation}
\end{theorem}


\subsubsection{Paridad de una permutación}

En una permutación, ocurre una \textbf{inversi\'on} cuando un elemento mayor
precede a un elemento menor. Para conocer el n\'umero de inversiones se siguen
los siguientes pasos:

\begin{itemize}
\item Tomar el primer elemento de la permutaci\'on
\item Contar los enteros menores a la derecha del elemento en cuesti\'on
\item Realizar los dos pasos anteriores para cada elemento de la permutaci\'on
\item Sumar el total de inversiones contadas para cada elemento
\end{itemize}

\textbf{Ejemplo}:

Se toma la permutación \textbf{$A=\{6,1,3,4,5,2\}$}\\

Primer elemento: $6$; menores a la derecha: $1,3,4,5,2$; n\'umero de
inversiones: \textbf{5}
Primer elemento: $1$; menores a la derecha: $\emptyset$; n\'umero de
inversiones: \textbf{0}
Primer elemento: $3$; menores a la derecha: $2$; n\'umero de inversiones:
\textbf{1}\\
Primer elemento: $4$; menores a la derecha: $2$; n\'umero de inversiones:
\textbf{1}\\
Primer elemento: $5$; menores a la derecha: $2$; n\'umero de inversiones:
\textbf{1}\\

Total de inversiones: \textbf{8}
\begin{tcolorbox}[colback=gray!5!white,colframe=gray!60!black,title=Definición: Tipos de permutaciones]
	\textbf{Permutaci\'on par:} aquella en la que el total de inversiones es un
	entero par.
	
	\tcblower

	\textbf{Permutaci\'on impar}: aquella en la que el total de inversiones es un
	entero impar.
	
\end{tcolorbox}


\subsection{Combinación}

Existe una estrecha relaci\'on entre el n\'umero de combinaciones y el
coeficiente binomial ya que el n\'umero de combinaciones es igual al coeficiente
binomial $\binom{n}{k}$.

Para contar el n\'umero de combinaciones, tenga en cuenta que al seleccionar una
$k$-permutación es lo mismo que seleccionar primero una combinación de $k$ elementos
y luego ordenarlos. Ya que hay $k!$ formas de ordenar los $k$ elementos
seleccionados, vemos que el número de $k$-permutaciones es igual al n\'umero de
combinaciones por $k!$. Por lo tanto, la n\'umero de combinaciones posibles,
viene dado por:

\begin{equation}
	\binom{n}{k} = \frac{n!}{k!(n-k)!}
\end{equation}

Ejemplo: El n\'umero de combinaciones de dos de las cuatro letras A, B, C, y D
se encuentra haciendo que $n = 4$ y $ k = 2$, está dado por:

\begin{center}
	$\binom{4}{2} = \frac{4!}{2!(4-2)!}$
\end{center}


\subsection{El problema de Monty Hall}

El problema de \textit{Monty Hall} o paradoja del presentador es un problema de
probailidad condicional basado en el concurso de televisi\'on estadounidense
\textit{Let's Make a Deal}. El problema fue bautizado as\'i por el nombre del
presentador de dicho programa.

En 1975, el matem\'atico Steve Selvin resuelve este problema y es publicado en
la revista \textit{American Statistician}. Posteriormente fue popularizado por
Marilyn vos Savant en la revista \textit{Parade Magazine}, en 1990. El problema
dice as\'i:

\textbf{Planteamiento del problema:}

Hay tres puertas. Detr\'as de una de ellas, hay un premio. Detr\'as de cada una
de las dos puertas restantes hay una cabra.

Cada puerta tiene la misma probabilidad de contener el premio.

Se te da a seleccionar una de las tres puertas como una oportunidad de ganar el
premio. Sin embargo, una vez que se ha seleccionado la puerta, el presentador
destapa una de ellas y te muestra que detrás hab\'ia una cabra.Una vez dada esa
informaci\'on, el presentador te pregunta si deseas cambiar de puerta o
conservar tu elecci\'on.

La pregunta crucial es:

Dado que el presentador ha destapado una puerta y mostrado que no estaba el
premio, ¿es m\'as conveniente cambiar mi opci\'on o conservar mi elecci\'on de
la puerta inicial?

%\begin{figure}[h]
%	\centering
%	\includegraphics[scale=0.5]{Images/Monty Hall}
%	\caption{Imagen ilustratuiva del problema de Monty Hall}
%\end{figure}

\subsection{Soluci\'on:}
Este problema es un ejercicios relativamente simple de probabilidad condicional.
Sin embargo, se debe tener cuidado, pues es un problema particularmente confuso.
Se abrodará desde dos enfoques.

Dos observaciones son cruciales para entender el problema:

\begin{itemize}
\item El presentador conoce la puerta donde se encuentra el premio y obviamente
no revelará esa puerta. 
\item Monty revelar\'a cualquiera de las otras dos puertas de forma aleatoria y
con con igual probabilidad de que esto ocurra.
\end{itemize}
Supongase que selecciona la puerta 1. Entonces pueden ocurrir tres casos:
\begin{enumerate}
\item El premio está en la puerta 1. Monty abre la puerta 2 o la 3. Si se cambia
la elecci\'on, se pierde. Si se conserva la elecci\'on, ganas.
\item El premio está en la puerta 2. Monty abre la puerta 3. Si se cambia de
elecci\'on, se gana. Si se conserva la elecci\'on, pierdes.
\item El premio está en la pierta 3. Monty abre la puerta 2. Si se cambia de
elecci\'on, se gana. Si se conserva la elecci\'on, pierdes.
\end{enumerate}
Se debe notar que al cambiar de elecci\'on se gana en 2 de los 3 casos; mientras
que al conservar la elecci\'on, se gana s\'olo en 1 de los 3 casos. Por lo
tanto, la probabilidad de ganar si se cambia de puerta, es de $\frac{2}{3}$. Por
otro lado, la probabilidad de ganar con la elecci\'on inicial es de
$\frac{1}{3}$.

Conclusi\'on: ¡Cambia de puerta!



\subsection{Problema del cumpleaños}

Este problema se presenta a menudo como ejemplo en el c\'alculo de
probabilidades Ec. (\ref{eq:probabilidad}). Este problema puede ser enunciado de la siguiente
forma. 

\textit{En un grupo de $n$ personas escogidas al azar, ¿cuál es la probabilidad de que
al menos dos de ellas tengan el mismo d\'ia de cumpleaños?}

Se asume que los nacimientos se distribuyen de manera uniforme a lo largo de
todo el año, y, por simplicidad, también asumimos que no existe el 29 de
febrero.

Otra forma alternativa de presentar el problema es la siguiente:

\textit{¿Cuál es el número mínimo de personas necesario en un grupo para que sea más
probable encontrar al menos dos con el mismo d\'ia de cumpleaños?}

Resolveremos la primera versión del problema, ya que con esto podremos hallar
también la respuesta a la segunda versión.

Soluci\'on.

Comenzaremos por resolver el problema para un tamaño muestral $n = 1, 2$ y $3$,
y posteriormente extenderemos el razonamiento al caso general. Para $n = 1$ , es
decir, cuando s\'olo hay una persona, la probabilidad $P_1$ de coincidir con otra
persona es cero, dado que no existe otra persona. Para $n = 2$, esto es, hay dos
personas, la probabilidad $P_2$ de que coincidan en su d\'ia de cumpleaños de
acuerdo con la Ec. (\ref{eq:probabilidad}) es:


\begin{equation}
P_2 = \frac{365}{365^2} = \frac{1}{365}
\end{equation}

ya que hay 365 maneras diferentes de que su día de cumpleaños coincida
(una por cada día del año), y $365 \cdot 365 = 3652$ maneras diferentes de que
se produzcan sus cumpleaños.

Para $n = 3$ personas, podemos calcular la probabilidad $p_3$ de que al menos
dos de ellas coincidan en su día de cumpleaños, a partir de la Ec.
(\ref{eq:complementRule}), como 1 menos la probabilidad del suceso contrario, es
decir, menos la probabilidad de que todas ellas cumplan años en d\'ias
diferentes.

De esta forma, tenemos:

\begin{equation}
p_3 = 1- \frac{364}{365} = 1 - \frac {364 \cdot 363}{365^2}
\end{equation}

%ya que la segunda persona puede cumplir años en cualquiera de los 365 días del
%año, pero s\'olo en 364 de ellos \'este no coincidirá con el cumpleaños de la
%primera persona; de igual forma, el cumpleaños de la tercera persona no
%coincidir\'a con el cumpleaños de las otras dos en 363 de los 365 casos
%posibles. Aplicando el mismo razonamiento para cualquier $n$ entre 1 y 365,
%obtenemos:
%
%\begin{equation}
%	\begin{array}{l}
%		P_n = 1 - \frac{364}{365} \cdot \frac{364}{365} \cdot \frac{364}{365} ...
%	
%			= 1 - \frac{365 \cdot - n + 1}{365} \\
%	
%			= \frac {364 \cdot 362 \cdot 363 \cdot 362 ... ( 365-n+1)}
%				{365^{n-1}} \\
%	
%			= \frac {364 \cdot 362 \cdot 363 \cdot 362 ... (365-n+1)}
%				{365^{n}} \\
%	
%			= \frac{365!} {365^n \cdot (365-n)!}
%
%	\end{array}
%\end{equation}
%
%Con lo que damos soluci\'on al problema 1. Obviamente, $P_n$ es 1 para todo $n$
%mayor que 365 (en una reuni\'on de más de 365 personas, necesariamente ha de
%haber al menos dos personas con el mismo d\'ia de cumpleaños). En el cuadro 1
%mostramos los valores que toma la probabilidad $p_n$ para algunos valores de n,
%seg\'un la expresi\'on (\ref{e6}) que acabamos de calcular.
%




\subsection{Resumen de técnicas de conteo}

\begin{tcolorbox}[colback=gray!5!white,colframe=gray!60!black,title=Resumen: Técnicas de conteo]

	\begin{itemize}
		\item \textbf{Permutaciones} de $n$ objetos:
		\begin{center}
		$n!$
		\end{center}

		\item \textbf{\textit{k}-Permutaciones} de $n$ objetos:
		\begin{center}
		$\frac{n!}{(n-k)!}$
		\end{center}

		\item \textbf{Combinaciones} de $k$ de $n$ objetos:
		\begin{center}
		$\binom{n}{k} = \frac{n!}{k!(n-k)!}$
		\end{center}

		\item \textbf{Particiones} de $n$ objetos en $r$ grupos, con el \textit{i}-ésimo
		grupo con $n_i$ objetos:
		\begin{center}
			$\binom{n}{n_1,n_2,...,n_r} = \frac{n!}{n_1!,n_2!,...,n_r!}$
		\end{center}

	\end{itemize}
	
\end{tcolorbox}

\documentclass[12pt]{article}
\usepackage[spanish,es-tabla,es-nodecimaldot]{babel}
\usepackage{tikz}
\usepackage{amsmath, amssymb, amsbsy}    %Caracteres especiales de matematicas
\usepackage[hidelinks]{hyperref}    % Vinculos [que no estén subrayados]
\usepackage{listings}   %Introducir codigo
\usepackage{setspace}
\usepackage{color}
\usepackage{pdflscape}  %Voltear la página
\usepackage{graphicx}
\usepackage{caption}


\begin{document}
\title{Ejercicios de Técnicas de Conteo}
\author{Ana Maritza Bello Ya\~nez}
\maketitle
\setlength{\parindent}{0pt}
\setlength{\parskip}{1em}

\section*{Ejercicios de Combinatoria (Schaum)}
% Problemas 1.7, 1.8, 1.29, 1.39, 1.40, 1.77, 1.78, 1.87
% 5 ejemplos de muestras ordenadas

\subsection*{Problema 1.7}
Demostrar que un número palindrómico (decimal) de longitud par es divisible por
11.

La prueba inductiva explota el hecho de que cuando se quitan el primer y el
último carácter de un palíndromo, queda un palíndromo. Así, sea $N$ un número
palindrómico de longitud $2k$. Si $k = 1$, el teorema obviamente se cumple. Si
$k \geq 2$, tenemos

$ N = a_{2k-1} 10^{2k-1} + a_{2k-2} 10^{2k-2} + ... + a_{k} 10^{k} + ... +
a_{2k-2} 10^{1} + a_{2k-1} 10^{0} $

$ = a_{2k-1}(10^{2k}+10^0) + (a_{2k-2}10^{2k-2}+...+ a_{2k-2}10^{1}) $

$ \equiv a_{2k-1} P + Q$

Donde:

$P = \underbrace{100...001}_{\text{longitud } 2k} =
\underbrace{9090...9091}_{\text{longitud } 2k-2}$

y ya sea $Q=0$ (divisible por 11) o, para algún $1 \leq r \leq k-1$,

$Q = 10^r$ \{palindrome de longitud $ 2(k-r)\} = 10^r\{11R\} $

donde el último paso se sigue de la hipótesis de inducción. Por lo tanto, $N$ es
divisible por 11 y la demostración está completa.

\subsection*{Problema 1.8}

En un palíndromo binario, el primer dígito es 1 y cada dígito subsiguiente puede
ser 0 o 1. Cuente los palíndromos binarios de longitud $n$.

Tenemos $[(n + 1)/21- 1 ]= [(n - 1)/2]$ posiciones libres, por lo tanto el número
deseado es:

$2^{[(n-1)/2]}$


\subsection*{Problema 1.29}
Encuentre la probabilidad $p_n$ de que un grupo de $n$ personas reunidas al azar
incluya al menos 2 personas con el mismo cumpleaños (día del año).

\textbf{Solución}

Aquí no tratamos con una muestra de personas, sino con una muestra de
cumpleaños, es decir, números enteros del 1 al 365 inclusivo. Nuestra noción de
probabilidad es:

$\text{Probabilidad} = \frac{\text{número de muestras favorables}}{\text{Número total de muestras}}$

En este problema es simple considerar el evento complementario: todos los $n$
cumpleaños son distintos. Este evento es realizado en $P(365,n)$ muestras; y el
total de muestras es $365^n$. Por lo tanto, $1-p_n = P(365,n)/365^n$ o:

$p_n = 1 - \frac{P(365,n)}{365^n} = 1 -
\frac{(365)(365-1)(365-2)...[365-(n-1)]}{365^n} $

$= 1-(1 - \frac{1}{365})(1 - \frac{2}{365})...(1 - \frac{n-1}{365})$

Puede verificarse que $p_n \geq 1/2 $ cuando $n>25$.

\subsection*{Problema 1.39}

\subsection*{Problema 1.40}

\subsection*{Problema 1.77}

\subsection*{Problema 1.78}

\subsection*{Problema 1.87}

\section*{Ejercicios de Probabilidad (Schaum)}
% Problemas 2.29, 2.30, 2.68 al 2.74

\subsection*{Problema 2.29}


\subsection*{Problema 2.30}


\subsection*{Problema 2.68}

\subsection*{Problema 2.69}


\subsection*{Problema 2.70}


\subsection*{Problema 2.71}

\subsection*{Problema 2.72}

\subsection*{Problema 2.73}

\subsection*{Problema 2.74}

\end{document}


\section{Probabilidad y conteo}

\subsection{Definici\'on cl\'asica}

La definici\'on cl\'asica de la probabilidad \cite{faber2012statistics} de un
evento $A$, puede ser formulada de la siguiente manera:

\begin{theorem}{Probabilidad}{probabilidad}
    \begin{equation}
        P(A)= \frac{n_A}{n_{total}}
    \end{equation}
    \label{eq:probabilidad}
\end{theorem}

Donde $n_A$ es el n\'umero de formas igualmente probables por las cuales un
experimento puede conducir a A; $n_{total}$ n\'umero total de formas igualmente
probables en el experimento.

- En la definici\'on cl\'asica el experimento no necesariamente se lleva a cabo
ya que la respuesta es conocida de antemano.

- La teor\'ia cl\'asica no da soluci\'on a menos de que todas las formas
igualmente posibles puedan ser derivadas anal\'iticamente.

La definición clásica desprende una serie de propiedades:

\begin{tcolorbox}[colback=gray!5!white,colframe=gray!60!black,title=Axiomas de la probabilidad]

\begin{itemize}
\item $P(A) \geq 0$. La probabilida se he definido como el cociente del número
de casos favorables al suceso $A$ y el número de casos posibles, por lo que el
cociente no puede ser negativo, y su límite inferior 0 se alcanza cuando el
número de casos favorables sea nulo.

\item $P(A) \leq 1$. El número de casos favorables nunca puede ser mayor que el
número total de casos, a lo sumo igual.


Estas dos propiedade conducen  que la probabilidad de un suceso esté acotada, 
\begin{equation}
    0 \leq P(S) \leq 1
    \label{eq:acotadaProb}
\end{equation}

\item Si $A$ y $B$ son conjuntos disjuntos, entonces:

\begin{equation}
    P(A \cup B) = P(A) + P(B)
    \label{eq:probaAunionB}
\end{equation}

\item $P(0)=\emptyset$

\item \textbf{Regla del complemento}

Sea $S$ el espacio muestral de un experimento $E$. Si $A$ es un evento, es decir
$A \subseteq S$, entonces:

\begin{equation}
    P(\bar{A}) = 1 - P(A)
    \label{eq:complementRule}
\end{equation}

\end{itemize}
\end{tcolorbox}



\section{Probabilidad Condicional}

La probabilidad condicional nos provee una forma de razonar acerca de la salida
o resultado de un experimento, basado en información parcial. Algunos ejemplos
de estos experimentos podrían ser los siguientes:

\begin{itemize}
\item En un experimento en el cual tiramos dos dados sucesivamente, te dicen que
la suma de los dados es 9. ¿Qué tan probable es que el primer dado haya caído 6?

\item En un juego de adivinanzas de palabras, la primera letra de la palabra es
\textit{t}. ¿Cuál es la probabilidad de que la siguiente palabra sea \textit{h}?

\item ¿Qué  tan probable es que una persona tenga cierta enfermedad dado que su
examen médico dió negativo?
\end{itemize}

En términos más precisos, dado un experimento, un espacio de muestreo y una ley
de probabilidad, suponiendo que sabemos que la salida está dentro de un evento
dado $B$. Deseamos cuantificar la probabilidad de que la salida pertenece a
algún otro evento $A$. Así, podemos construir una nueva probabilidad que tome en
cuenta el conocimiento disponible: una ley de probabilidad para cuaquier evento
$A$. Especificamente, la probabilidad condicional de $A$ dado $B$, denotado por
$P(A|B)$.

Para calcular $P(A|B)$ consideramos aquellos resultados del evento $A$ que están en
el evento $B$. Esto nos da los resultados en el evento $A \cap B$ y nos lleva al
teorema de la probabilidad condicional.

\begin{theorem}{Probabilidad Condicional}{conditional_probability}
Sea $S$ el espacio de muestro para un experimento $E$ y $A$, $B \subseteq S$,
entonces la probabilidad condicional de $A$ dado $B$ está dada por:
    \begin{equation}
        P(A|B) = \frac{P(A \cap B)}{P(B)}
        \label{eq:conditionalProbability}
    \end{equation}

Siempre que $P(B)>0$.

Particularmente, todas las propiedades de las leyes de la probabilidad
permanecen válidas para las leyes de la probabilidad condicional.

\begin{itemize}
    \item La probabilidad condicional puede verse también como una ley de
    probabilidad en un nuevo universo $B$, ya que toda la probabilidad condicional
    está concentrada en $B$.

    \item Si todos los resultados son finitos e igualmente probables, entonces:
        \begin{equation}
            P(A|B)=\frac{\text{número de elementos de }A \cap B}{\text{número de elementos de }B}
        \end{equation}
\end{itemize}
\end{theorem}


\def\firstcircle{(0,0) circle (1.5cm)}
\def\secondcircle{(0:2cm) circle (1.5cm)}
\def\rectangle{(-2,-2) rectangle (4,2)}

\colorlet{circle edge}{black!50}
\colorlet{circle area}{gray!20}

\tikzset{filled/.style={fill=circle area, draw=circle edge, thick},
    outline/.style={draw=circle edge, thick}}

\setlength{\parskip}{5mm}

\begin{figure}[h]
    \centering
    \begin{tikzpicture}
        \draw \rectangle;
        \begin{scope}
            \clip \firstcircle;
            \fill[filled] \secondcircle;
        \end{scope}
        \draw[outline] \firstcircle node {$A$};
        \draw[outline] \secondcircle node {$B$};
        \node[anchor=south] at (current bounding box.south) {$A \cap B$};
        \node at (3.5,1.5){$\mathbb S$};
    \end{tikzpicture}
    \caption{Representación del concepto intuitivo de la probabilidad condicional.}
    \label{fig:coditionalProbability}
\end{figure}

\def\firsttcircle{(90:1.75cm) circle (1.5cm)}
\def\seconddcircle{(170:1.75cm) circle (1.5cm)}
\def\thirddcircle{(-25:2cm) circle (1.5cm)}
\begin{figure}
    \centering
\begin{tikzpicture}
      \begin{scope}
    \clip \seconddcircle;
    \fill[cyan] \thirddcircle;
      \end{scope}
      \begin{scope}
    \clip \firsttcircle;
    \fill[cyan] \thirddcircle;
      \end{scope}
      \draw \firsttcircle node[text=black,above] {$A$};
      \draw \seconddcircle node [text=black,below left] {$B$};
      \draw \thirddcircle node [text=black,below right] {$C$};
\end{tikzpicture}
\end{figure}

\textbf{Observaciones y consideraciones de la probabilidad condicional:}

\begin{itemize}
\item La probabilidad $P(A|B)$ es una actualización de $P(A)$, basada en el
conocimiento de que ocurrió el evento $B$.

\item De la Ec.(\ref{eq:conditionalProbability}), tanto $P(A \cap B)$ como
$P(A)$ se calculan a partir del espacio muestral original.

\item Las probabilidades tienen sutiles cambios dependiendo de la información
exacta de la condición implicada en el evento A.
\end{itemize}

\subsection{Propiedades de las leyes de la probabilidad}

Las leyes de la probabilidad tienen propiedades que pueden deducirse de los
axiomas. Algunas de ellas, son las que se muestran a continuación.

\begin{tcolorbox}[colback=blue!5!white,colframe=blue!60!black,title=Resumen: Propiedades de las leyes de la probabilidad]
    Sean $A$, $B$ y $C$ eventos:

    \begin{itemize}
        \item
        \begin{equation}
            \text{Si } A \subset B, \text{entonces } P(A) \leq P(B)
            \label{eq:propProb1}
        \end{equation}
        
        \item 
        \begin{equation}
        P(A \cup B) = P(A) + P(B) - P(A \cap B)
        \label{eq:propProb2}
        \end{equation}

        \item 
        \begin{equation}
            P(A \cup B) \leq P(A) + P(B)
            \label{eq:propProb3}
        \end{equation}

    \item
    $P(A \cup B \cup C)=$
    \begin{equation}
        P(A) + P(B) + P(C) - P(A \cap B) - P(A \cap C) - 
        P(B \cap C) + P(A \cap B \cap C)
        \label{eq:propProb4}
    \end{equation}
    \end{itemize}
\end{tcolorbox}

De la Ec.(\ref{eq:conditionalProbability}) podemos hacer que:

\begin{center}
$P(B \cap A) = P(A \cap B) = P(A) \  P(B|A)$, 
\end{center}

y cambiando los roles de $A$ y de $B$, tenemos que:

\begin{center}
    $P(A \cap B) = P(B \cap A) = P(B) \ P(A|B)$, 
\end{center}

esto resulta en:

\begin{center}
    $P(A)P(B|A) = P(A \cap B) = P(B) \ P(A|B)$,
\end{center}

que comunmente llamamos \textit{regla de la multiplicación}.

Para verificar algunas de las propiedades de las leyes de la probabilida,
podemos usar diagramas de Venn, Fig. (poner diagramas de Venn). Si $A \subset
B$, entonces $B$ es la unión de eventos los eventos disjuntos $A$ y $A_c \cap
B$, por lo que por el axioma de adición, tenemos:

\begin{center}
    $P(B) = P(A) + P(A^c \cap B) \geq P(A)$.
\end{center}

Como observamos en la Fig. (\ref{fig:difer}), podemos expresar los eventos $A
\cup B$ y $B$ como unión de conjuntos disjuntos:

\begin{center}
    $A \cup B = A \cup (A^c \cap B)$,
\end{center}

\begin{center}
    $B = (A \cap B) \cap (A^c \cap B)$
\end{center}

Usando el axioma de adición, tenemos:

\begin{center}
    $P(A \cup B) = P(A) + P(A^c \cap B)$,
\end{center}

\begin{center}
    $P(B) = P(A \cap B) + P(A^c \cap B)$
\end{center}

Si despejamos $P(A^c \cap B)$ de la seguda igualdad y la sustituimos en la
primera, tenemos:

\begin{center}
    $P(A \cup B) = P(A) + P(B) - P(A \cap B)$,
\end{center}

con lo que podemos verificar la Ec. \eqref{eq:propProb3}

\begin{figure}
% Set B but not A
\centering
\begin{tikzpicture}
    \draw \rectangle;
    \begin{scope}
        \clip \secondcircle;
        \draw[filled, even odd rule] \firstcircle
                                     \secondcircle node {$B$};
    \end{scope}
    \draw[outline] \firstcircle node {$A$}
                   \secondcircle;
    \node[anchor=south] at (current bounding box.north) {$A^c \cap B$};
    \node at (1,0){$A \cap B$};
\end{tikzpicture}
\caption{Representación gráfica de los eventos $A \cup B$, como unión de eventos disjuntos.}
\label{fig:difer}
\end{figure}

\subsection{Teorema de la probabilidad total}
\begin{theorem}{Teorema de la probabilidad total}{TotalProbabilityTheorem}
Sean $A_1, ..., A_n$ eventos disjuntos que forman una partición de un espacio
muestral, Fig(poner figura!!), y asumiendo $P(A_i)>0$, para todo $i$. Entonces,
para cualquier evento $B$, tenemos:

\begin{center}
    $P(B) = P(A_1 \cap B) + ... + P(A_n \cap B)$
\end{center}

\begin{equation}
    =P(A_1)\ P(B|A_1) + ... + P(A_n) \ P(B|A_n)
\end{equation}

\end{theorem}


\subsection{Teorema de Bayes}

\begin{theorem}{Teorema de Bayes}{BayesTheorem}
Sean $A$ y $B$ dos eventos cuyas probabilidades son diferentes de cero,
entonces:
    \begin{equation}
        P(B|A) = \frac{P(A|B) \ P(B)}{P(A)}
        \label{eq:BayesTheorem}
    \end{equation}

La implicación más importante de la Ec. (\ref{eq:BayesTheorem}) es que permite
encontrar probabilidades condicionadas $P(B|A)$ en términos de $P(A|B)$, cuando
esta última resulta más fácil de calcular directamente.
\end{theorem}



\section{Convenio de suma de Einstein}

Se denomina notación de Einstein o notación indexada a la convención utilizada
para abreviar la escritura de las sumatorias, donde se suprime el término de la
sumatoria $(\sum)$. Este convenio fue introducido por Albert Einstein en 1916.

Dada una expresión lineal en $\mathbb{R^n}$ en la que se escriban todos sus
términos en forma explícita:

\begin{equation}
    u = u_1 + u_2 + ... = u_n
\end{equation}

Se puede escribir de la forma:

\begin{equation}
    u = \sum_{i=1}^{n} u_ix_i
\end{equation}

La notación de Einstein obtiene una expresión aún más condensada eliminando el
signo de la sumatoria y entendiendo que la expresión resultante en un índice
indica la suma sobre todos los posibles valores del mismo.

\begin{equation}
    u = u_ix_i
\end{equation}

\subsection{Tensor Levi-Civita}
Está definido por:

\begin{equation}
    \epsilon_{ijk}=
    \left\lbrace\begin{array}{clllr} 
        +1 & ijk, & kji, & jki & \text{Permutación cíclica o par}\\ 
        -1 & ikj, & kji, & jik & \text{Permutación anticíclica o impar}\\
        0  & iii, & jjj, & kkk & \text{Si se repiten}
    \end{array}\right.
\end{equation}

\subsection{Producto escalar de dos vectores}

También conocido como producto punto, es una operación algebraíca que toma dos
vectores y retorna un escalar. esta definido de la siguiente manera:

\begin{equation}
    \bar{A} \cdot \bar{B}= \bar{\abs{A}} \bar{\abs{B}} \cos \theta
    = \sum_{i=1}^{n} a_i b_i
\end{equation}

Que de acuerdo al convenio de suma de Einstein podemos escribir como:

\begin{equation}
    \bar{A} \cdot \bar{B} = a_i b_i
\end{equation}

\subsection{Producto vectorial de dos vectores}

EL producto vectorial o producto cruz de dos vectores es una operación binaria
entre dos vectores en un espacio tridimensional. El resultado es un vector
perpendicular a los vectores que se multiplican, y por lo tanto normal al plano
que los contiene.

Está definido de la siguiente manera:

\begin{equation}
    \bar{A} \times \bar{B}= \bar{\abs{A}} \bar{\abs{B}} \sin \theta \hat{r}
\end{equation}

Donde $\hat{r}$ es el vector unitario y ortogonal a los vectores $\bar{A}$ y
$\bar{B}$, y $\theta$ el ángulo entre $\bar{A}$ y $\bar{B}$.

Que mediante determinantes tenemos:

\begin{equation}
    \bar{A} \times \bar{B}=
    \begin{bmatrix}
        \hat{i} & \hat{j} & \hat{k}\\ 
         a_1 & a_1 & a_1\\ 
         b_1 & b_1 & b_1
    \end{bmatrix}
\end{equation}

Desarrollamos:

\begin{equation}
    \bar{A} \times \bar{B}= \hat{i}(a_2 b_3 - a_3 b_2) - \hat{j}(a_1 b_3 - a_3 b_1) + \hat{k}(a_1 b_2 - a_2 b_1)
\end{equation}

Desde un punto de vista tensorial el producto generalizado de $n$ vectores vendrá
dado por:

\begin{equation}
    (\bar{A} \times \bar{B})_{i} = \epsilon_{ijk} a_{j} b_{k}
\end{equation}

Desarrollando:

\begin{equation}
    \epsilon_{ijk} a_{jbk} =
    \begin{array}{ccccc}
        \cancel{\epsilon_{111} a_1 b_1} & + & \cancel{\epsilon_{112} a_1 b_2}   & + & \cancel{\epsilon_{112} a_1 b_3}   \\
        \cancel{\epsilon_{121} a_2 b_1} & + & \cancel{\epsilon_{122} a_2 b_2}   & + & \epsilon_{123} a_2 b_3            \\
        \cancel{\epsilon_{131} a_3 b_1} & + & \epsilon_{132} a_3 b_2            & + & \cancel{\epsilon_{133} a_3 b_3}
    \end{array}
    = a_2 b_3 - a_3 b_2
\end{equation}


\subsection{Congruencia de Zeller}

La congruencia de Zeller es un algoritmo que permite obtener, a partir de una
fecha, el día de la semana que le corresponde.

Se atribuye su creación a Julius Christian Johannes Zeller, un sacerdote
protestante alemán que vivió en el siglo XIX.

Zeller observó que existía una dependencia entre las fechas del calendario
gregoriano y el día de la semana que les correspondía. A raíz de esa
observación, obtuvo (se dice que por tanteo), esta fórmula, en apariencia
mágica, que lleva su nombre.

La fórmula en sí es muy sencilla, y se basa en algunas operaciones de aritmética
modular (el resto, también llamado módulo, de las divisiones)

Es necesario tener en cuenta que la fórmula presentada a continuación es válida
sólo para el calendario gregoriano, promulgado por el papa Gregorio XIII en
1582, pero adoptado en distintas fechas en cada país.

Para el caledario gregoriano la congruencia de Zeller es:

\begin{equation}
    h = (q + [\frac{13(m+1)}{5}] + K [\frac{K}{4}] + [\frac{J}{4}] - 2J ) \text{ mod } 7
\end{equation}

Para el calendario juliano es:

\begin{equation}
    h = (q + [\frac{13(m+1)}{5}] + K [\frac{K}{4}] + 5 - J ) \text{ mod } 7
\end{equation}

Donde:

\begin{itemize}
    \item $h$ es el día de la semana.
    \item $q$ es el día del mes.
    \item $m$ es el mes.
    \item $K$ el año del siglo (año mod 100)
    \item $J$ es el siglo de base cero.
\end{itemize}

Veáse código en Sec. (\ref{sec:zeller})

%
\section*{Tri\'angulo de Sierpinski}

El matem\'atico polaco Waclav Sierpinski (1882-1969), construy\'o este fractal
en 1919 del modo siguiente: tom\'o un tri\'angulo equil\'atero, uni\'o los
puntos medios de los lados y form\'o cuatro tri\'angulos interiores: tres
tri\'angulos equil\'ateros sombreados y un hueco que es otro tri\'angulo
equil\'atero. Repiti\'o el proceso en cada uno de los tri\'angulos sombreados, y
sigui\'o hasta el infinito el proceso en los tres tri\'angulos restantes como el
primero.


\begin{figure}[h]
    \centering
    \usetikzlibrary{lindenmayersystems}
\def\trianglewidth{3cm} \pgfdeclarelindenmayersystem{Sierpinski triangle}{
        \symbol{X}{\pgflsystemdrawforward}
        \symbol{Y}{\pgflsystemdrawforward}
        \rule{X -> X-Y+X+Y-X}
        \rule{Y -> YY}
} \foreach \level in {1,3,5}{ \tikzset{
l-system={step=\trianglewidth/(2^\level), order=\level, angle=-120} }

        \begin{tikzpicture}
\fill [black] (0,0) -- ++(0:\trianglewidth) -- ++(120:\trianglewidth) -- cycle;
\draw [draw=none] (0,0) l-system [l-system={Sierpinski triangle,
axiom=X},fill=white];
        \end{tikzpicture}
}
    \caption{Tri\'angulo de Sierpinski para los niveles 1,3 y 5.}
    \label{fig:sierpinski}
\end{figure}

Como se muestra en la Fig.(\ref{fig:sierpinski})




\section{Matriz de rotaci\'on}

fig.(\ref{permutaciones_tree})\\
\begin{figure}[h]
		\begin{eqnarray*}
			x^{'m} = a^{m}\hfill_{\nu} x^{\nu}\\
			m = 0,1,2,3\\
			\nu = 0,1,2,3
		\end{eqnarray*}		
		\begin{eqnarray*}
			x^{'(0)} =
			\left( {\begin{array}{cc}
					a^{0} _0 x^{_0} + a^{0} _1 x^{_1} + a^{0} _2 x^{_2} + a^{0} _3 x^{_3}\\
			\end{array} } \right)\\
		x^{'(1)} =
		\left( {\begin{array}{cc}
				a^{1} _0 x^{_0} + a^{1} _1 x^{_1} + a^{1} _2 x^{_2} + a^{1} _3 x^{_3}\\
		\end{array} } \right)\\
		x^{'(2)} =
		\left( {\begin{array}{cc}
				a^{2} _0 x^{_0} + a^{2} _1 x^{_1} + a^{2} _2 x^{_2} + a^{2} _3 x^{_3}\\
		\end{array} } \right)\\
		x^{'(3)} =
		\left( {\begin{array}{cc}
				a^{3} _0 x^{_0} + a^{3} _1 x^{_1} + a^{3} _2 x^{_2} + a^{3} _3 x^{_3}\\
		\end{array} } \right)
		\end{eqnarray*}
		\begin{equation*}
			\left(
			\begin{array}{ccc}
				x^{'0}\\
				x^{'1}\\
				x^{'2}\\
				x^{'3}\\ 
			\end{array}
			\right)
			=
			\left(
			\begin{array}{c}
				a^{m}\nu
			\end{array}
			\right)
			{}
			\left(
			\begin{array}{ccc}
				x^{0}\\
				x^{1}\\
				x^{2}\\
				x^{3}\\ 
			\end{array}
			\right)
		\end{equation*}		
	\caption{Matriz de rotaci\'on}
	\label{permutaciones_tree}
\end{figure}

\chapter{Matrices de rotaci\'on}
	Una matriz de rotaci\'on es una matriz que representa una rotaci\'on en el espacio euclidiano.
	Las matrices de rotaci\'on pueden ser en dos o tres dimensiones.
	
	Por ejemplo, si la matriz de rotaci\'on es de dos dimensiones y se busca rotar en sentido horario, la matriz tendrá la siguiente forma:
	\begin{equation}
		R(\theta)=
		\begin{bmatrix}
			\cos \theta & -\sin \theta\\
			\sin \theta & \cos \theta
		\end{bmatrix}
	\end{equation}
	De esta manera, las nuevas coordenadas, una vez aplicando la rotación son de la siguiente manera:
	\begin{eqnarray}
		\left(
		\begin{matrix}
			x'\\
			y'
		\end{matrix}
		\right)
		=
		\begin{bmatrix}
			\cos \theta & -\sin \theta\\
			\sin \theta & \cos \theta
		\end{bmatrix}
		\left(
		\begin{matrix}
			x\\
			y
		\end{matrix}
		\right)\\
		x'=x\cos \theta -y\sin \theta\\
		y'=x\sin \theta + y \cos \theta
	\end{eqnarray}

En caso de querer que la rotación sea antihoraria, se utiliza la siguiente matriz:
\begin{equation}
	R(-\theta)=
	\begin{bmatrix}
		\cos \theta & \sin \theta\\
		-\sin \theta & \cos \theta
	\end{bmatrix}
\end{equation}

Si lo que se busca es realizar rotaciones en el espacio tridimensional, se deben modificar las matrices de la siguiente manera:
\begin{equation}
	R_x(\theta)=
	\begin{bmatrix}
		1 & 0 & 0\\
		0 & \cos \theta & \sin \theta\\
		0 & -\sin \theta & \cos \theta
	\end{bmatrix}
\end{equation}

\begin{equation}
	R_y(\theta)=
	\begin{bmatrix}
		\cos \theta & 0 & \sin \theta\\
		0 & 1 & 0\\
		-\sin \theta & 0 & \cos \theta
	\end{bmatrix}
\end{equation}

\begin{equation}
	R_z(\theta)=
	\begin{bmatrix}
		\cos \theta & -\sin \theta & 0\\
		\sin \theta & \cos \theta & 0\\
		0 & 0 & 1
	\end{bmatrix}
\end{equation}
\subsection*{Ángulos de Euler}

Los ángulos de Euler constituyen un conjunto de tres coordenadas angulares que
sirven para especificar la orientación de un sistema de referencia de ejes
ortogonales, normalmente móvil, respecto a otro sistema de referencia de ejes
ortogonales normalmente fijos.

Dados dos sistemas de coordenadas $xyz$ y $XYZ$ con origen común, es posible
especificar la posición de un sistema en términos del otro usando tres ángulos
$\alpha$, $\beta$, $\gamma$.

La definición matemática es estática y se basa en escoger dos planos, uno en el
sistema de referencia y otro en el triedro rotado. En el esquema adjunto serían
los planos $xy$ y $XY$. Escogiendo otros planos se obtendrían distintas convenciones
alternativas, las cuales se llaman de Tait-Bryan cuando los planos de referencia
son no-homogéneos (por ejemplo $xy$ y $XY$ son homogéneos, mientras $xy$ y $XZ$ no lo
son).

La intersección de los planos coordenados $xy$ y $XY$ escogidos se llama línea de
nodos, y se usa para definir los tres ángulos:

\begin{itemize}
    \item $\alpha$ es el ángulo entre el eje x y la línea de nodos.
    \item $\beta$  es el ángulo entre el eje z y el eje Z.
    \item $\gamma$  es el ángulo entre la línea de nodos y el eje X.
\end{itemize}

más adelante se establecerá que los tres ángulos de Euler descritos son los
valores de las tres rotaciones intrínsecas que describen el sistema.

Notar que también se considera la notación: $\alpha =\phi$, $\gamma =\psi$,
$\beta =\theta$

Las rotaciones de Euler son los movimientos resultantes de variar uno de los
ángulos de Euler dejando fijo los otros dos. Son los siguientes:

- Preseción
- Nutación
- Rotación intrínseca

Si escribimos la rotación de ángulos $\phi$
,$\theta$ ,$\psi$ como una composición de estas tres rotaciones:

\begin{equation}
    A(\phi ,\theta ,\psi )=R(\phi ,\theta ,\psi )N(\phi ,\theta )P(\phi)\
\end{equation}

entonces se cumple:

\begin{equation}
    \begin{array}{ll}
    A(\delta \phi +\phi ,\theta ,\psi ) & = P(\delta \phi )A(\phi ,\theta ,\psi) \\
    A(\phi ,\delta \theta +\theta ,\psi ) & = N(\phi ,\delta \theta )A(\phi ,\theta ,\psi ) \\
    A(\phi,\theta ,\delta \psi +\psi ) &=R(\phi ,\theta ,\delta \psi )A(\phi ,\theta ,\psi) \\
    \end{array}
\end{equation}

Como consecuencia de estas propiedades, estas rotaciones son conmutativas
entre ellas:

\begin{equation}
    P(\delta \phi )N(\delta \theta )A(\phi ,\theta ,\psi)=A(\delta \phi +\phi ,\delta \theta +\theta ,\psi )=N(\delta \theta )P(\delta \phi )A(\phi ,\theta ,\psi )
\end{equation}

lo que también podría verse intuitivamente usando la analogía entre los ángulos
de Euler y los de un soporte Cardán.

%\begin{figure}[h]
    \includegraphics[scale=0.7]{figures/euler_angles.png}
%    \caption{Unos gatos molones}
%    \label{fig:gatos}
%\end{figure}




\pagebreak
\bibliography{../references/references.bib} 
\bibliographystyle{unsrt}

\end{document}

%poner alef 1 y alef2
% Axiomas de peano
