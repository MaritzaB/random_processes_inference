\documentclass{article}
\usepackage[a4paper,top=1cm,bottom=2cm,left=3cm,right=3cm,marginparwidth=1.75cm]{geometry}  
\usepackage[spanish,es-tabla,es-nodecimaldot]{babel}
%\usepackage{booktabs}
%\usepackage{caption}
%\usepackage{color}
%\usepackage{float}
%\usepackage{hyperref}

\begin{document}
\title{Probabilidad, procesos aleatorios e inferencia}
\author{Ana Maritza Bello Yañez}
\maketitle

\tableofcontents

\section{Probabilidad y conteo}

\subsection{Definición clásica \cite{faber2012statistics}}

La definición clásica de la probabilidad de un evento $A$, puede ser formulada
de la siguiente manera:

\begin{equation}
    P(A)= \frac{n_A}{n_(total)}
\end{equation}

Donde $n_A$ es el número de formas igualmente probables por las cuales un
experimento puede conducir a A; $n_(total)$ número total de formas igualmente
probables en el experimento.

- En la definición clásica el experimento no necesariamente se lleva a cabo ya que
la respuesta es conocida de antemano.

- La teoría clásica no da solución a menos de que todas las formas igualmente
posibles puedan ser derivadas analíticamente.

\section{Experimentos deterministas vs aleatorios}

\subsection{Definicion: Observación}

Cualquier registro de información, ya sea numérico o categórico. Por
consiguiente, los números 2, 0, 1 y 2, que representan el número de accidentes
que ocurrieron cada mes, de enero a abril, durante el año pasado en la
intersección de Driftwood Lane y Royal Oak Drive, constitu- yen un conjunto de
observaciones. Lo mismo ocurre con los datos categóricos N, D, N, N y D, que
representan los artículos defectuosos o no defectuosos cuando se inspeccionan
cinco artículos y se registran como observaciones \cite{walpole2012probabilidad}.

\subsection{Definicion: Experimento}

Cualquier proceso que genere un conjunto de datos. Un ejemplo simple de
experimento estadístico es el lanzamiento de una moneda al aire. En tal
experimento sólo hay dos resultados posibles: cara o cruz
\cite{walpole2012probabilidad}.

\subsection{Experimento determinista}
Un experimento determinista es aquel que produce el mismo resultado cuando se le
repite bajo las mismas condiciones.

\subsection{Experimento aleatorio}
Un experimento aleatorio es aquel que, cuando se le repite bajo las mismas
condiciones, el resultado que se observa no siempre es el mismo y tampoco es
predecible.

\pagebreak
\bibliography{../references/references.bib} 
\bibliographystyle{apalike}

\end{document}
