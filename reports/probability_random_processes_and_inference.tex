\documentclass[12pt]{article}
\usepackage[spanish,es-tabla,es-nodecimaldot]{babel}
\usepackage{tikz}
\usepackage{amsmath}    %Caracteres especiales de matematicas
\usepackage[hidelinks]{hyperref}    % Vinculos [que no estén subrayados]
\usepackage{listings}   %Introducir codigo
\usepackage{setspace}
\usepackage{color}
\usepackage{amssymb}
\usepackage[makeroom]{cancel}
\usepackage{pdflscape}  %Voltear la página
\usepackage{tcolorbox}  %Cuadros de teoremas
\usepackage{fancyhdr, lastpage}
\usepgflibrary{fpu}
\usepackage{ifthen}
\usepackage{graphicx}
\usepackage{caption}
\usepackage{subcaption}
\usepackage{multirow}
\usepackage{tikz-3dplot}
\usepackage{pgfplots}
\usepgfplotslibrary{external}

\tcbuselibrary{listings,theorems}

\pagestyle{fancy}   %Encabezado en las páginas
\chead{Probabilidad, procesos aleatorios e inferencia}  %Encabezado central
\lhead{}    %Encabezado izquiero vacio
\rhead{}    %Encabezado derecho vacio
%% Cambiar letra a tamaño 12

\usepackage[Glenn]{fncychap}
% Sonny, Lenny, Glenn, Conny, Rejne, Bjarne, Bjornstrup

\providecommand{\abs}[1]{\lvert#1\rvert}

\definecolor{Code}{rgb}{0,0,0} 
\definecolor{Decorators}{rgb}{0.5,0.5,0.5} 
\definecolor{Numbers}{rgb}{0.5,0,0} 
\definecolor{MatchingBrackets}{rgb}{0.25,0.5,0.5} 
\definecolor{Keywords}{rgb}{0,0,1} 
\definecolor{self}{rgb}{0,0,0} 
\definecolor{Strings}{rgb}{0,0.63,0} 
\definecolor{Comments}{rgb}{0,0.63,1} 
\definecolor{Backquotes}{rgb}{0,0,0} 
\definecolor{Classname}{rgb}{0,0,0} 
\definecolor{FunctionName}{rgb}{0,0,0} 
\definecolor{Operators}{rgb}{0,0,0} 
\definecolor{Background}{rgb}{0.98,0.98,0.98} 


\lstset{frame=tb,
numbers=left, 
numberstyle=\footnotesize, 
numbersep=1em, 
xleftmargin=1em, 
framextopmargin=2em, 
framexbottommargin=2em, 
showspaces=false, 
showtabs=false, 
showstringspaces=false, 
frame=l, 
tabsize=4, 
% Basic 
basicstyle=\ttfamily\small\setstretch{1}, 
backgroundcolor=\color{Background}, 
% Comments 
commentstyle=\color{Comments}\slshape, 
% Strings 
stringstyle=\color{Strings}, 
morecomment=[s][\color{Strings}]{"""}{"""}, 
morecomment=[s][\color{Strings}]{'''}{'''}, 
% keywords 
morekeywords={import,from,class,def,for,while,if,is,in,elif,else,not,and,or,print,break,continue,return,True,False,None,access,as,,del,except,exec,finally,global,import,lambda,pass,print,raise,try,assert}, 
keywordstyle={\color{Keywords}\bfseries}, 
% additional keywords 
morekeywords={[2]@invariant,pylab,numpy,np,scipy}, 
keywordstyle={[2]\color{Decorators}\slshape}, 
emph={self}, 
emphstyle={\color{self}\slshape}, 
}


\newcommand{\tdplotmainfig}{%
%Angle Definitions
%-----------------
%
%set the plot display orientation
%synatax: \tdplotsetdisplay{\theta_d}{\phi_d}
\tdplotsetmaincoords{60}{110}
%
%define polar coordinates for some vector
%TODO: look into using 3d spherical coordinate system
\pgfmathsetmacro{\rvec}{.8}
\pgfmathsetmacro{\thetavec}{30}
\pgfmathsetmacro{\phivec}{60}
%
%start tikz picture, and use the tdplot_main_coords style to implement the display coordinate transformation provided by 3dplot
\begin{tikzpicture}[scale=3,tdplot_main_coords]

	%set up some coordinates 
	%-----------------------
	\coordinate (O) at (0,0,0);

	%determine a coordinate (P) using (r,\theta,\phi) coordinates.  This command also determines (Pxy), (Pxz), and (Pyz): the xy-, xz-, and yz-projections of the point (P).
	%synatax: \tdplotsetcoord{Coordinate name without parentheses}{r}{\theta}{\phi}
	\tdplotsetcoord{P}{\rvec}{\thetavec}{\phivec}

	%draw figure contents
	%--------------------

	%draw the main coordinate system axes
	\draw[thick,->] (0,0,0) -- (1,0,0) node[anchor=north east]{$x$};
	\draw[thick,->] (0,0,0) -- (0,1,0) node[anchor=north west]{$y$};
	\draw[thick,->] (0,0,0) -- (0,0,1) node[anchor=south]{$z$};

	%draw a vector from origin to point (P) 
	\draw[-stealth,color=red] (O) -- (P);

	%draw projection on xy plane, and a connecting line
	\draw[dashed, color=red] (O) -- (Pxy);
	\draw[dashed, color=red] (P) -- (Pxy);

	%draw the angle \phi, and label it
	%syntax: \tdplotdrawarc[coordinate frame, draw options]{center point}{r}{angle}{label options}{label}
	\tdplotdrawarc{(O)}{0.2}{0}{\phivec}{anchor=north}{$\phi$}


	%set the rotated coordinate system so the x'-y' plane lies within the "theta plane" of the main coordinate system
	%syntax: \tdplotsetthetaplanecoords{\phi}
	\tdplotsetthetaplanecoords{\phivec}

	%draw theta arc and label, using rotated coordinate system
	\tdplotdrawarc[tdplot_rotated_coords]{(0,0,0)}{0.5}{0}{\thetavec}{anchor=south west}{$\theta$}

	%draw some dashed arcs, demonstrating direct arc drawing
	\draw[dashed,tdplot_rotated_coords] (\rvec,0,0) arc (0:90:\rvec);
	\draw[dashed] (\rvec,0,0) arc (0:90:\rvec);

	%set the rotated coordinate definition within display using a translation coordinate and Euler angles in the "z(\alpha)y(\beta)z(\gamma)" euler rotation convention
	%syntax: \tdplotsetrotatedcoords{\alpha}{\beta}{\gamma}
	\tdplotsetrotatedcoords{\phivec}{\thetavec}{0}

	%translate the rotated coordinate system
	%syntax: \tdplotsetrotatedcoordsorigin{point}
	\tdplotsetrotatedcoordsorigin{(P)}

	%use the tdplot_rotated_coords style to work in the rotated, translated coordinate frame
	\draw[thick,tdplot_rotated_coords,->] (0,0,0) -- (.5,0,0) node[anchor=north west]{$x'$};
	\draw[thick,tdplot_rotated_coords,->] (0,0,0) -- (0,.5,0) node[anchor=west]{$y'$};
	\draw[thick,tdplot_rotated_coords,->] (0,0,0) -- (0,0,.5) node[anchor=south]{$z'$};

	%WARNING:  coordinates defined by the \coordinate command (eg. (O), (P), etc.) cannot be used in rotated coordinate frames.  Use only literal coordinates.  

	%draw some vector, and its projection, in the rotated coordinate frame
	\draw[-stealth,color=blue,tdplot_rotated_coords] (0,0,0) -- (.2,.2,.2);
	\draw[dashed,color=blue,tdplot_rotated_coords] (0,0,0) -- (.2,.2,0);
	\draw[dashed,color=blue,tdplot_rotated_coords] (.2,.2,0) -- (.2,.2,.2);

	%show its phi arc and label
	\tdplotdrawarc[tdplot_rotated_coords,color=blue]{(0,0,0)}{0.2}{0}{45}{anchor=north west,color=black}{$\phi'$}

	%change the rotated coordinate frame so that it lies in its theta plane.  Note that this overwrites the original rotated coordinate frame
	%syntax: \tdplotsetrotatedthetaplanecoords{\phi'}
	\tdplotsetrotatedthetaplanecoords{45}

	%draw theta arc and label
	\tdplotdrawarc[tdplot_rotated_coords,color=blue]{(0,0,0)}{0.2}{0}{55}{anchor=south west,color=black}{$\theta'$}

\end{tikzpicture}
}

\newcommand{\threedcoord}[2]{%
\tdplotsetmaincoords{#1}{#2}
\begin{tikzpicture}[tdplot_main_coords]
	\draw[thick,->] (0,0,0) -- (1,0,0) node[anchor=north east]{$x$};
	\draw[thick,->] (0,0,0) -- (0,1,0) node[anchor=north west]{$y$};
	\draw[thick,->] (0,0,0) -- (0,0,1) node[anchor=south]{$z$};
\end{tikzpicture}
}

\newcommand{\threedrotcoordsystem}{%
\tdplotsetmaincoords{50}{140}
\begin{tikzpicture}[scale=5,tdplot_main_coords]
	\draw[thick,->] (0,0,0) -- (1,0,0) node[anchor=north east]{$x$};
	\draw[thick,->] (0,0,0) -- (0,1,0) node[anchor=north west]{$y$};
	\draw[thick,->] (0,0,0) -- (0,0,1) node[anchor=south]{$z$};

	\tdplotsetrotatedcoords{34}{26}{12}

	\draw[thick,color=blue,tdplot_rotated_coords,->] (0,0,0) -- (.5,0,0) node[anchor=north east]{$x'$};
	\draw[thick,color=blue,tdplot_rotated_coords,->] (0,0,0) -- (0,.5,0) node[anchor=north west]{$y'$};
	\draw[thick,color=blue,tdplot_rotated_coords,->] (0,0,0) -- (0,0,.5) node[anchor=south]{$z'$};
	
\end{tikzpicture}
}

\newcommand{\threedconventions}{%
\tdplotsetmaincoords{50}{110}
\begin{tikzpicture}[scale=2,tdplot_main_coords]
	\draw[thick,->] (0,0,0) -- (1,0,0) node[anchor=north east]{$x$};
	\draw[thick,->] (0,0,0) -- (0,1,0) node[anchor=north west]{$y$};
	\draw[thick,->] (0,0,0) -- (0,0,1) node[anchor=south]{$z$};

	\draw[color=blue] (0,0,0) -- (.5,.5,0);

	\draw[color=blue,->] (.5,0,0) arc (0:45:.5);	
\end{tikzpicture}
}


\newcommand{\threedalphabetagamma}{%
\beginpgfgraphicnamed{Figures/alphabetagamma}
\tdplotsetmaincoords{50}{140}
%
\begin{tikzpicture}[scale=2,tdplot_main_coords]
	\draw[thick,->] (0,0,0) -- (1,0,0) node[anchor=north east]{$x$};
	\draw[thick,->] (0,0,0) -- (0,1,0) node[anchor=north west]{$y$};
	\draw[thick,->] (0,0,0) -- (0,0,1) node[anchor=south]{$z$};

	\tdplotsetrotatedcoords{0}{0}{30}

	\draw[thick,color=red,tdplot_rotated_coords,->] (0,0,0) -- (.7,0,0) node[anchor=north]{$x'$};
	\draw[thick,color=green!50!black,tdplot_rotated_coords,->] (0,0,0) -- (0,.7,0) node[anchor=west]{$y'$};
	\draw[thick,color=blue,tdplot_rotated_coords,->] (0,0,0) -- (0,0,.7) node[anchor=west]{$z'$};
	
	\tdplotdrawarc[color=orange!50!black]{(0,0,0)}{.4}{0}{30}{anchor=north east}{$\gamma$}
\end{tikzpicture}
%
\begin{tikzpicture}[scale=2,tdplot_main_coords]
	\draw[thick,->] (0,0,0) -- (1,0,0) node[anchor=north east]{$x$};
	\draw[thick,->] (0,0,0) -- (0,1,0) node[anchor=north west]{$y$};
	\draw[thick,->] (0,0,0) -- (0,0,1) node[anchor=south]{$z$};

	\tdplotsetrotatedcoords{0}{0}{30}

	\draw[dashed,color=red,tdplot_rotated_coords] (0,0,0) -- (.5,0,0);
	\draw[dashed,color=green!50!black,tdplot_rotated_coords] (0,0,0) -- (0,.5,0);
	\draw[dashed,color=blue,tdplot_rotated_coords] (0,0,0) -- (0,0,.5);

	\tdplotsetrotatedcoords{0}{40}{30}

	\draw[thick,color=red,tdplot_rotated_coords,->] (0,0,0) -- (.7,0,0) node[anchor=north]{$x'$};
	\draw[thick,color=green!50!black,tdplot_rotated_coords,->] (0,0,0) -- (0,.7,0) node[anchor=west]{$y'$};
	\draw[thick,color=blue,tdplot_rotated_coords,->] (0,0,0) -- (0,0,.7) node[anchor=south]{$z'$};
	
	\tdplotsetthetaplanecoords{0}
	\tdplotdrawarc[tdplot_rotated_coords,color=orange!50!black]{(0,0,0)}{.4}{0}{40}{anchor=south}{$\beta$}

\end{tikzpicture}
%
\begin{tikzpicture}[scale=2,tdplot_main_coords]
	\draw[thick,->] (0,0,0) -- (1,0,0) node[anchor=north east]{$x$};
	\draw[thick,->] (0,0,0) -- (0,1,0) node[anchor=north west]{$y$};
	\draw[thick,->] (0,0,0) -- (0,0,1) node[anchor=south]{$z$};

	\tdplotsetrotatedcoords{0}{40}{30}

	\draw[dashed,color=red,tdplot_rotated_coords] (0,0,0) -- (.5,0,0);
	\draw[dashed,color=green!50!black,tdplot_rotated_coords] (0,0,0) -- (0,.5,0);
	\draw[dashed,color=blue,tdplot_rotated_coords] (0,0,0) -- (0,0,.5);

	\tdplotsetrotatedcoords{60}{0}{0}
	\draw[dotted,color=blue,tdplot_rotated_coords] (0,0,0) -- (.4,0,0);
	\tdplotsetrotatedcoords{60}{40}{30}

	\draw[thick,color=red,tdplot_rotated_coords,->] (0,0,0) -- (.7,0,0) node[anchor=north]{$x'$};
	\draw[thick,color=green!50!black,tdplot_rotated_coords,->] (0,0,0) -- (0,.7,0) node[anchor=west]{$y'$};
	\draw[thick,color=blue,tdplot_rotated_coords,->] (0,0,0) -- (0,0,.7) node[anchor=south]{$z'$};


	\tdplotdrawarc[color=orange!50!black]{(0,0,0)}{.2}{0}{60}{anchor=north east}{$\alpha$}
\end{tikzpicture}
\endpgfgraphicnamed
}


\begin{document}
\title{Probabilidad, procesos aleatorios e inferencia}
\author{Ana Maritza Bello Ya\~nez}
\maketitle


%%%%%%%%%%%%%%%%%%%%%%%%%%%%%%%%%%%%%%%%%%%%%%%%%%%%%%%%%%%%%%%%%%%%%
%                       Caja del teorema                            %
%                                                                   %
\newtcbtheorem{theorem}{Teorema}             %
{colback=gray!5,colframe=gray!35!black,fonttitle=\bfseries}{th}   %
%%%%%%%%%%%%%%%%%%%%%%%%%%%%%%%%%%%%%%%%%%%%%%%%%%%%%%%%%%%%%%%%%%%%%

\tableofcontents

\setlength{\parindent}{0pt}
\setlength{\parskip}{0em}
\pagebreak
\section*{Resumen: La belleza y utilidad de las matem\'aticas}

Las matem\'aticas han impregnado todos los campos de la actividad cient\'ifica y
desempe\~nan un papel inestimable en la biolog\'ia, la f\'isica, la qu\'imica,
la econom\'ia, la sociolog\'ia y la ingenier\'ia.


El descubrimiento simult\'aneo ha ocurrido a menudo en la historia de las
matem\'aticas.

Por ejemplo el del c\'alculo del erudito ingl\'es Isaac Newton (1643-1727) y el
matem\'atico alem\'an Gottfried Wilhelm Leibniz (1646-1716). Esto nos hacen
preguntarnos por qu\'e se hicieron estos descubrimientos cient\'ificos en al
mismo tiempo por personas que trabajan de forma independiente. Para dar otro
ejemplo, los naturalistas brit\'anicos Charles Darwin (1809–1882) y Alfred
Wallace (1823–1913) desarrollaron la teor\'ia de la evoluci\'on de manera
independiente y simult\'anea. De manera similar, el matem\'atico h\'ungaro
J\'anos Bolyai (1802–1860) y el matem\'atico ruso Nikolai Lobachevsky
(1793–1856) parec\'ian haber desarrollado la geometr\'ia hiperb\'olica de forma
independiente y al mismo tiempo.

Lo m\'as probable es que tales descubrimientos simult\'aneos hayan ocurrido
porque el momento era propicio para tales descubrimientos, dado el conocimiento
acumulado por la humanidad en el momento en que se realizaron los
descubrimientos. A veces, dos cient\'ificos se sienten estimulados al leer la
misma investigaci\'on preliminar de uno de sus contempor\'aneos. Por otro lado,
los m\'isticos han sugerido que existe un significado m\'as profundo para tales
coincidencias.

En ocasiones, se han utilizado teor\'ias matem\'aticas para predecir fen\'omenos
que no se confirmaron hasta a\~nos despu\'es. Por ejemplo, las ecuaciones de
Maxwell, llamadas as\'i por el f\'isico James Clerk Maxwell, predijeron las
ondas de radio. Las ecuaciones de campo de Einstein sugirieron que la gravedad
doblar\'ia la luz y que el universo se est\'a expandiendo. El f\'isico Paul
Dirac se\~nal\'o una vez que las matem\'aticas abstractas que estudiamos ahora
nos dan una idea de la f\'isica en el futuro. De hecho, sus ecuaciones
predijeron la existencia de antimateria, que posteriormente fue descubierta. De
manera similar, el matem\'atico Nikolai Lobachevsky dijo que “no hay rama de las
matem\'aticas, por abstracta que sea, que alg\'un d\'ia no pueda aplicarse a los
fen\'omenos del mundo real”.



\section{Experimentos deterministas vs aleatorios}


\begin{tcolorbox}[colback=gray!5!white,colframe=gray!60!black,title=Definición: Observación]
    Cualquier registro de informaci\'on, ya sea num\'erico o categ\'orico.
\end{tcolorbox}

Ejemplos: 

Los n\'umeros 2, 0, 1 y 2, que representan el n\'umero de accidentes que
ocurrieron cada mes, de enero a abril, durante el a\~no pasado, constituyen un
conjunto de observaciones. 

Lo mismo ocurre con los datos categ\'oricos N, D, N, N y D, que representan los
art\'iculos defectuosos o no defectuosos cuando se inspeccionan cinco
art\'iculos y se registran como observaciones \cite{walpole2012probabilidad}.

\begin{tcolorbox}[colback=gray!5!white,colframe=gray!60!black,title=Definición: Experimento]
    Cualquier proceso que genere un conjunto de datos.    
\end{tcolorbox}

Un ejemplo simple de experimento estad\'istico es el lanzamiento de una moneda
al aire. En tal experimento s\'olo hay dos resultados posibles: águila o sol.
\cite{walpole2012probabilidad}.

\begin{tcolorbox}[colback=gray!5!white,colframe=gray!60!black,title=Definición: Tipos de experimentos]

    \textbf{Experimento determinista.}
    Un experimento determinista es aquel que produce el mismo resultado cuando se le
    repite bajo las mismas condiciones.

    \tcblower

    \textbf{Experimento aleatorio.}
    Un experimento aleatorio es aquel que, cuando se le repite bajo las mismas
    condiciones, el resultado que se observa no siempre es el mismo y tampoco es
    predecible.
\end{tcolorbox}



\section*{Teoría de conjuntos}

\subsection*{Introducción}

El concepto de conjunto, es fundamental en matemáticas. Su estudio, se basa en
el hecho de que éstos pueden ser combinados, mediante ciertas operaciones, para
formar otros conjuntos. 

El estudio de las operaciones con los conjuntos, constituye el álgebra de
conjuntos; que tiene semejanzas formales (aunque también presenta diferencias)
con el álgebra de los números. El álgebra de conjuntos, resulta valiosa en la
reducción de los conceptos matemáticos a sus fundamentos lógicos. 

La disciplina matemática que estudia las propiedades generales de los conjuntos
es la teoría de conjuntos. Esta disciplina se comenzó a desarrollar,
rigurosamente, a finales del siglo XIX y principios del XX. El fundador de dicha
teoría es el matemático alemán de origen ruso, George Ferdinand Ludwing
Phillipp Cantor (1845, 1918). Las ideas y conceptos de la teoría de conjuntos
han irrumpido literalmente en todas las ramas de las matemáticas y cambiaron su
faz por completo.

\subsection*{Definición básica de conjunto}

Un conjunto es cualquier colección de objetos bien definidos por medio de alguna
o algunas propiedades en común, de dichos objetos. Por objeto entenderemos no
sólo cosas físicas, como discos, computadores, etc., si no también abstractos,
como son números, letras, etc. A los objetos se les llama elementos del
conjunto.

Representamos a los conjuntos por medio de letras mayúsculas, así $A$, $B$, $C$,
etc., nos representan conjuntos.

\subsection*{Elemento}

Se llaman elementos o miembros a los objetos que componen un conjunto; y se
denotan con letras minúsculas, como: $a, b, x, y$, etc. Para indicar que un
objeto $x$ pertenece o es miembro de un conjunto $A$, escribimos $x \in A$, que
se lee \textit{“x elemento de A”} y, si no pertenece al conjunto $A$, escribimos
$x \notin A$, que se lee \textit{“x no es elemento de A”}.

\subsection*{Conjunto universo}
Llamamos conjunto universo y lo denotamos por $\mathbb U$, al conjunto del cual se
seleccionan los elementos para formar conjuntos.

En general identificamos al conjunto universo como un todo, pero no
representamos de una manera única a este “todo”, así habrá ocasiones que un
conjunto sea considerado conjunto universo y en otras no, por ejemplo:
considerando el conjunto P de los número pares, se tiene que

\begin{tabular}{ |l|l|l| }
    
\multirow{2}{4em}{$\mathbb C$ conjunto de los complejos} & \multirow{2}{4em}{$\mathbb R$ números reales} \\
    & & $\mathbb I$ números irracionales \\
    & & $\mathbb Q$ números racionales
    %& & {$\mathbb Z$ números enteros}
\end{tabular}



\subsection*{Conjunto vacío}

El conjunto vacío o nulo es un conjunto que no tiene elementos, el conjunto
vacío se representa por: $\emptyset$, o bien por \{\}.

No confundir el conjunto $A = \emptyset $ con el conjunto $ A = \{\emptyset\} $,
ya que el primer conjunto indica que no tiene ningún elemento y el segundo
conjunto indica que tiene un elemento y ese elemento es el conjunto vacío.

\subsection*{Igualdad de conjuntos}

Decimos que dos conjuntos $A$ y $B$ son iguales y denotamos $A=B$, si y solo si
$A$ es un subconjunto de $B$, y $B$ es un subconjunto de $A$.

\begin{equation}
    A=B \iff A \subseteq B \land B \subseteq A
\end{equation}

\subsection*{Subconjunto}

Si todos los elementos de un conjunto $A$ son también elementos de un conjunto
$B$, esto es, si cuando $x \in A$ entonces $x \in B$ (simbólicamente $x \in A
\rightarrow x \in B$), decimos que $A$ es un subconjunto de $B$ o que $A$ está
contenido en $B$ y se escribe:

\begin{equation}
    A \subseteq B \iff (\forall x)|[x \in A \rightarrow x \in B]
\end{equation}

Si $A$ no es subconjunto de $B$ se escribe $A \not \subseteq B$ .

Si además existe un elemento de $B$ que no este en $A$, decimos que $A$ es un
subconjunto propio de $B$ y se denota $A \subset B$ o $B \subset A$ y es
equivalente a la siguiente preoposición:

\begin{equation}
    (\exists \ x \in \mathbb U)[x \in A \land x \not\in B]
\end{equation}

\subsection*{Propiedades de los conjuntos}

\begin{enumerate}
    \item Si $A$ es cualquier conjunto diferente del vacío, entonces $A \subseteq A$ .
    Esto es, cualquier conjunto diferente del vacío es subconjunto de si mismo.

    \item El conjunto vacío es subconjunto de cualquier conjunto distinto del vacío,
    esto es $\emptyset \subset A$

    \item Si $n>0$ es el número de elementos de $A$ entonces el número de elementos
de $P(A)$ es $2^n$
\end{enumerate}

\begin{theorem}{Conjuntos}{conjuntos}
Sean $A$ y $B$ dos conjuntos. Decimos que $A=B$ si y solo si $A \subseteq B$ y
$B \subseteq A$
\end{theorem}

\begin{theorem}{Conjuntos}{conjuntos}
    Si $ A \subset B $ y $ B \subset C $, entonces $ A \subset C $
\end{theorem}

\subsection*{Cardinalidad}

Sea $A$ un conjunto, la cardinalidad de $A$ es el número de elementos diferentes
del conjunto $A$ y se representa por $|A|$.

\subsection*{Conjunto potencia}
Sea $A$ un conjunto finito, llamaremos conjunto potencia al conjunto formado por
todos los subconjuntos de $A$. El conjunto potencia se denota como $P(A)$ o bien
$2^A$.

\subsection*{Operaciones entre conjuntos}

% Definition of circles
\def\firstcircle{(0,0) circle (1.5cm)}
\def\secondcircle{(0:2cm) circle (1.5cm)}
\def\rectangle{(-2,-2) rectangle (4,2)}

\colorlet{circle edge}{blue!50}
\colorlet{circle area}{blue!20}

\tikzset{filled/.style={fill=circle area, draw=circle edge, thick},
    outline/.style={draw=circle edge, thick}}

\setlength{\parskip}{5mm}

\subsubsection*{Unión}

La unión de dos conjuntos $A$ y $B$, denotada por $ A \cup B $ (que se lee “A
unión B”) es un nuevo conjunto formado por los elementos que pertenecen a $A$ o
a $B$ o a ambos conjuntos.

\begin{equation}
    A \cup B = \{x | x \in A \vee x \in B\}
\end{equation}

Si $x \in A \cup B $ entonces $x \in A $ ó $x \in B$ o $x$ pertenece a ambos
conjuntos.

% Set A or B
\begin{figure}[h]
    \centering
    \begin{tikzpicture}
        \draw \rectangle;
        \draw[filled] \firstcircle node {$A$}
                      \secondcircle node {$B$};
        \node[anchor=south] at (current bounding box.north) {$A \cup B$};
    \end{tikzpicture}
    \caption{Representación gráfica de la unión de conjuntos $A \cup B$}
    \label{fig:unionConjuntos}
\end{figure}

\subsubsection*{Intersección}

La intersección de dos conjuntos $A$ y $B$, denotada por $A \cap B$ (que se lee
“ A intersección B”), es un nuevo conjunto formado por los elementos que
pertenecen a A y a B al mismo tiempo, es decir, por los elementos comunes a
ambos conjuntos.

\begin{equation}
    A \cap B=\{x | x \in A \wedge x \in B\}
\end{equation}

% Set A and B
\begin{figure}[h]
    \centering
    \begin{tikzpicture}
        \draw \rectangle;
        \begin{scope}
            \clip \firstcircle;
            \fill[filled] \secondcircle;
        \end{scope}
        \draw[outline] \firstcircle node {$A$};
        \draw[outline] \secondcircle node {$B$};
        \node[anchor=south] at (current bounding box.north) {$A \cap B$};
    \end{tikzpicture}
    \caption{Representación gráfica de la intersección de los conjuntos de $A \cap B$.}
    \label{fig:interseccionConjuntos}
\end{figure}

\subsubsection*{Complemento}

Sea $A \subseteq U$ un conjunto. El complemento de A denotado por $\bar{A}$,
$A'$, $A^C$ ó $U-A$, se define como el conjunto de todos los elementos que están
en $U$ pero no están en $A$. Simbólicamente:

\begin{equation}
    \bar{A}= \{x | x \in \mathbb U \wedge x \notin A\}
\end{equation}

Es decir, el complemento de $A$ es el conjunto de los todos los elementos que no
están en el conjunto $A$. Simbólicamente: $ \bar{A}={x | x \notin A}$.

% Set A^c
\begin{figure}[h]
    \centering
    \begin{tikzpicture}
        \draw[filled] \rectangle node [anchor=north west] at (current bounding box.north) {$\bar{A}$};
        \begin{scope}
            \clip \firstcircle;
            \fill[white!80] \firstcircle;
            \draw[draw=circle edge,thick] \firstcircle node {$A$};
        \end{scope}
        \node[anchor=south] at (current bounding box.north) {$\bar{A}$};
    \end{tikzpicture}
    \caption{Representación gráfica del complemento de $\bar{A}$}
    \label{fig:unionConjuntos}
\end{figure}

\subsubsection*{Diferencia}

Sean $A$ y $B$ dos conjuntos, definimos la diferencia de $A$ y $B$, que se
denota por $A-B$, como el conjunto de todos los elementos que pertenecen a $A$
pero no pertenecen a $B$. Simbólicamente:

\begin{equation}
    A-B = \{x | x \in A \land x \not\in B\}
\end{equation}

% Set A but not B
\begin{figure}[h]
    \centering
    \begin{subfigure}[b]{0.45\textwidth}
        \centering
        \begin{tikzpicture}
            \draw \rectangle;
            \begin{scope}
                \clip \firstcircle;
                \draw[filled, even odd rule] \firstcircle node {$A$}
                                             \secondcircle;
            \end{scope}
            \draw[outline] \firstcircle
                           \secondcircle node {$B$};
            \node[anchor=south] at (current bounding box.north) {$A - B$};
        \end{tikzpicture}
    \end{subfigure}
\hfill
    \begin{subfigure}[b]{0.45\textwidth}
        \centering
        % Set B but not A
        \begin{tikzpicture}
            \draw \rectangle;
            \begin{scope}
                \clip \secondcircle;
                \draw[filled, even odd rule] \firstcircle
                                             \secondcircle node {$B$};
            \end{scope}
            \draw[outline] \firstcircle node {$A$}
                           \secondcircle;
            \node[anchor=south] at (current bounding box.north) {$B - A$};
        \end{tikzpicture}
\end{subfigure}
\caption{Representación gráfica de la diferencia de los conjuntos de $A-B$ y de $B-A$.}
\label{fig:diferenciaConjuntos}
\end{figure}

\subsubsection*{Diferencia simétrica}
La diferencia simétrica de dos conjuntos $A$ y $B$, se denota como $A \triangle
B$, se define como la unión de $A-B$ y $B-A$. Simbólicamente:

\begin{equation}
    A \triangle B = (A-B) \cup (B-A)
\end{equation}


%Set A or B but not (A and B) also known a A xor B
\begin{figure}[h]
    \centering
    \begin{tikzpicture}
        \draw \rectangle;
        \draw[filled, even odd rule] \firstcircle node {$A$}
                                     \secondcircle node{$B$};
        \node[anchor=south] at (current bounding box.north) {$A \triangle B$};
    \end{tikzpicture}
    \caption{Representación gráfica de la diferencia simétrica de los conjuntos de $A \triangle B$.}
    \label{fig:diferenciaSimetricaConjuntos}
\end{figure}

\subsection*{Propiedades del álgebra de conjuntos}

\begin{itemize}

\item \textbf{Idempotencia}
\begin{equation}
    \begin{array}{ll}
        A \cup A & = A \\
        A \cap A & = A
    \end{array}
\end{equation}

\item \textbf{Conmutatividad}
\begin{equation}
    \begin{array}{ll}
        A \cup B & = B \cup A \\
        A \cap B & = A \cap B
    \end{array}
\end{equation}

\item \textbf{Asociatividad}
\begin{equation}
    \begin{array}{ll}
        (A \cup B) \cup C & = A \cup (B \cup C) \\
        (A \cap B) \cap C & = A \cap (B \cap C)
    \end{array}
\end{equation}

\item \textbf{Distributividad}
\begin{equation}
    \begin{array}{ll}
        A \cup (B \cap C) & = (A \cup B) \cap (A \cup B) \\
        A \cap (B \cup C) & = (A \cap B) \cup (A \cap B)
    \end{array}
\end{equation}

\item \textbf{Identidad}
\begin{equation}
    \begin{array}{ll}
        A \cup \emptyset & = A \\
        A \cap \mathbb{U} & = A
    \end{array}
\end{equation}

\item \textbf{Dominación}
\begin{equation}
    \begin{array}{ll}
        A \cap \emptyset & = A \\
        A \cup \mathbb{U} & = A
    \end{array}
\end{equation}

\item \textbf{Inversa}
\begin{equation}
    \begin{array}{ll}
        A \cup \bar{A} & = \mathbb U \\
        A \cap \bar{A} & = \emptyset
    \end{array}
\end{equation}

\item \textbf{Complemento}
\begin{equation}
    \begin{array}{ll}
        \bar{\bar{A}} & = A \\
        \bar{\mathbb U} & = \emptyset \\
        \bar{\emptyset} & = \mathbb U \\
    \end{array}
\end{equation}

\item \textbf{DeMorgan}
\begin{equation}
    \begin{array}{ll}
        \bar{A \cup B} & = \bar{A} \cap \bar{B} \\
        \bar{A \cap B} & = \bar{A} \cup \bar{B}
    \end{array}
\end{equation}

\item \textbf{Absorción}
\begin{equation}
    \begin{array}{ll}
        A \cup (A \cap B) & = A \\
        A \cap (A \cup B) & = A
    \end{array}
\end{equation}

\end{itemize}



\section{Binomio de Newton}

El binomio de Newton consiste en una fórmula que permite obtener los
coeficientes de un término enésimo de un binomio elevado a un exponente
determinado.

La fórmula matemática del binomio de Newton es la siguiente:

\begin{equation}
    (a+b)^n = \sum_{k=0}^{n} \binom{n}{k} a^{n-k} b^k
\end{equation}

\subsection{Desarrollo del binomio de Newton de la potencia 1 a la 10}

\begin{landscape}
	\begin{equation}
		\begin{split}
			(a+b)^0 &   = 1 \\
			(a+b)^1 &   = a+b \\
			(a+b)^2 &   = a^2+2ab+b^2 \\
			(a+b)^3 &   = a^3+3a^2b+3ab^2+b^3 \\
			(a+b)^4 &   = a^4+4a^3b+6a^2b^2+4ab^3+b^4 \\
			(a+b)^5 &   = a^5+5a^4b+10a^3b^2+10a^2b^3+5ab^4+b^5\\
			(a+b)^6 &   = a^6+6a^5b+15a^4b^2+20a^3b^3+15a^2b^4+6ab^5+b^6\\
			(a+b)^7 &   = a^7+7a^6b+21a^5b^2+35a^4b^3+35a^3b^4+21ba^2b^5+21ab^6+b^7 \\
			(a+b)^8 &   = a^8+8a^7b+28a^6b^2+56a^5b^3+70a^4b^4+56a^3b^5+28a^2b^6+8ab^7+b^8 \\
			(a+b)^9 &   = a^9+9a^8b+36a^7b^2+84a^6b^3+126a^5b^4+126a^4b^5+84a^3b^6+36a^2b^7+9ab^8+b \\
			(a+b)^{10} &    = a^{10}+10a^9b+45a^8b^2+120a^7b^3+210a^6b^4+252a^5b^5+210a^4b^6+120a^3b^7+45a^2b^8+10ab^9+b^{10} \\
		\end{split}
	\end{equation}
\end{landscape}


\section{Triángulo de Pascal}

En las matemáticas, el triángulo de Pascal es una representación de los
coeficientes binomiales ordenados en forma de triángulo. Es llamado así en honor
al filósofo y matemático francés Blaise Pascal, quien introdujo esta notación en
1654, en su tratado del triángulo aritmético.

\newcommand{\pasc}[2]{
	\pgfkeys{/pgf/fpu}
	\pgfmathparse{round(#1!/((#1-#2)!*#2!))}
	\pgfmathfloattoint{\pgfmathresult}
	\pgfmathresult
}

\begin{figure}[h]
	\centering
	\begin{tikzpicture}
	\pgfmathsetmacro{\N}{10};
	\foreach \i in {0,...,\N}{
		\foreach \j in {0,...,\i}{
			\node at ({-0.5*\i+\j-0.2},-\i){\pasc{\i}{\j}};
			\draw ({-0.5*\i+\j},-\i) circle (0.5) ;
		}
	}
	\end{tikzpicture}
	\caption{Triángulo de Pascal para n=10}
	\label{fig:pascalTriangle}
\end{figure}

Este triángulo fue ideado para desarrollar las potencias de binomios.

La fórmula del binomio de Newton desarrolla los coeficientes de cada fila en el
triángulo de Pascal. Es por esto que existe una estrecha relación entre el
triángulo de Pascal y el binomios de Newton.

\subsection{Propiedades del triángulo de Pascal}

Algunas de las propiedades más representativas del triángulo de Pascal son las
siguientes:

\begin{itemize}
	\item Cada número es la suma de los dos números encima de él.
	\item Todos los números exteriores son iguales a 1.
	\item El triángulo de Pascal es simétrico.
	\item La primera diagonal muestra los números de conteo.
	\item Las sumas de las filas dan las potencias del 2.
	\item Cada fila da los dígitos de las potencias del 11.
	\item Cada elemento representa a la combinación , en donde, m es la fila del elemento y n es la posición del elemento en la fila.
	\item Cada fila representa a los coeficientes binomiales.
	\item Los números Fibonacci están a lo largo de las diagonales.
\end{itemize}

\subsection{Patrones del triángulo de Pascal}

\subsubsection{Suma de las filas}

Una de las propiedades interesantes del triángulo es que la suma de los números
en una fila es igual a $2^n$, en donde, n corresponde al número de la fila. Por
ejemplo, tenemos:

$ 1 = 1 = 2^0 $ \\
$ 1 + 1 = 1 = 2 = 2^1 $ \\
$ 1 + 2 = 1 = 4 = 2^2 $\\

como se puede observar en la Fig. (\ref{fig:pascalTriangle})

\subsubsection{Números primos en el triángulo}

Otro patrón visible en el triángulo se relaciona a los números primos. Si es que
una fila empieza con un número primo o es una fila con número primo, todos los
números que están en esa fila, sin incluir al 1, son divisibles para ese número
primo.

Si es que miramos a la fila 5 (1, 5, 10, 10, 5, 1) podemos ver que el 5 y el 10
son divisibles por 5. Sin embargo, para una fila compuesta como la fila 8 (1, 8,
28, 56, 70, 56, 28, 8, 1), 28 y 70 no son divisibles por 8.

\subsubsection{Sucesión de Fibonacci en el triángulo}
En el triángulo de Pascal se puede apreciar una relación entre un modo de sumar
las diagonales y la sucesión de Fibonacci. Los primeros términos de esta
sucesión son: $1,1,2,3,5,8,13,21,34,55$ como se puede apreciar en la Fig.
(\ref{fig:pascalTriangle}).

\subsubsection{Expansión binomial con el triángulo de Pascal}

El triángulo de Pascal define a los coeficientes que aparecen en las expresiones
binomiales. Eso significa que la fila n del triángulo de Pascal contiene a los
coeficientes de la expresión expandida del binomio $(x+y)^n$.

\section{T\'ecnicas de conteo}
% ARREGLAR SALTOS DE PÁGINA
% EIGENVALORES
% VALORES PROPIOS
\subsection{Principio de multiplicaci\'on}

El principio de multiplicación lo utilizamos para contar el número de formas en
las quepueden ocurrir dos eventos simultáneos. El principio de la multiplicación
establece que si un evento puede ocurrir de $n_1$ formas diferentes, y para cada
una de estas ppuede ocurrir un segundo evento simultáneo en $n_2$ formas
diferentes, entonces los dos eventos pueden ocurrir de $n_1 * n_2$ formas
diferentes.

Así, para una serie de $k$ eventos, tenemos:
\begin{theorem}{Principio de multiplicación}{multiplicación}
	\begin{equation}
		n_1*n_2*n_3*...*n_k
		\label{eq:reglaMultiplicacion}
	\end{equation}
\end{theorem}


\subsection{Diagrama de Árbol}

Los diagramas de árbol muestran todos los resultados posibles de un evento. Cada
rama en un diagrama de árbol representa un posible resultado.
Los diagramas de \'arbol pueden usarse para encontrar el n\'umero de resultados
posibles y calcular la probabilidad de los posibles resultados.

%Por ejemplo,

\begin{figure}[h]
	\begin{center}		
	\begin{tikzpicture}[level distance=1cm, level 1/.style={sibling distance=3cm},
		level 2/.style={sibling distance=2cm}, every node/.style={circle, draw,
		align=center} ]
				\centering
				\node[circle,draw]{$P_n$}
		child{node{1} child{node{2} child{node{3}} } child{node{3} child{node{2}}} }
		child{node{2} child{node{1} child{node{3}} } child{node{3} child{node{1}}} }
		child{node{3} child{node{1} child{node{2}} } child{node{2} child{node{1}}} };
			\end{tikzpicture}
		\end{center}
		\caption{Permutaciones para 3 elementos}
		\label{permutaciones_tree}
\end{figure}


\subsection{Principio de adición}

Supongamos que existen $k$ conjuntos de elementos con $n_1$ elementos en el
primer conjunto, $n_2$ elementos en el segundo conjuntos, etc. Si todos los
elementos son distintos, es decir, si si todos los pares dek conjunto $k$ son
disjuntos, entonces el número de elementos de la unión de los conjuntos es $n_1
+ n_2 + ... + n_k $.

Usando la nomenclatura de conjuntos, el principio de adición está definido de la
siguiente manera:

\begin{theorem}{Principio de adición}{adicion}
Sean $A$ y $B$ dos eventos disjuntos, entonces la probabilidad de la unión de
los conjuntos está dada por:

\begin{equation}
	P(A \cup B) = P(A) + P(B)
\end{equation}

De manera general, si el espacio de muestreo tiene un número infinito de
elementos y $A_1, A_2, ... $ son una secuencia de eventos disjuntos, entonces la
probabilidad de la unión es:

\begin{equation}
	P(A_1 \cup A_2 \cup ...) = P(A_1) + P(A_2) + ...
\end{equation}

\end{theorem}

\subsection{Permutación}


\begin{tcolorbox}[colback=blue!5!white,colframe=blue!60!black,title=Definición: Permutación]
	Una \textbf{permutaci\'on} de un conjunto es un arreglo de dichos elementos en
	alg\'un orden sin repeticiones ni omisiones.
		\label{Permutaciones_definition}
\end{tcolorbox}

Para saber el n\'umero de permutaciones que existen de un conjunto se debe
conocer su cardinalidad. Tomemos como ejemplo, el conjunto de n\'umeros enteros
$A=\{1, 2, 3\}$. Para cualquier permutaci\'on, en la primera posici\'on pueden
colocarse cualquiera de los tres elementos; en la segunda posición se pueden
colocar dos posibles elementos; mientras que al final, se puede colocar una
posibilidad.

En general, para cualquier conjunto de elementos $X$ de cardinalidad $|n|$, en
la primera posici\'on se pueden colocar $n$ elementos, para la siguiente
posici\'on $n-1$. Así sucesivamente hasta las \'ultimas posiciones, donde se
pueden colocar $3, 2$ y $1$ elementos. De esta manera, se dice que el n\'umero
de permutaciones es:

\begin{theorem}{Permutación}{permutacion}
	\begin{equation}
		P_n=n*(n-1)*(n-2)*...*3*2*1=n!
		\label{permutaciones_totales}
	\end{equation}
\end{theorem}

\subsubsection{Paridad de una permutación}

En una permutación, ocurre una \textbf{inversi\'on} cuando un elemento mayor
precede a un elemento menor. Para conocer el n\'umero de inversiones se siguen
los siguientes pasos:

\begin{itemize}
\item Tomar el primer elemento de la permutaci\'on
\item Contar los enteros menores a la derecha del elemento en cuesti\'on
\item Realizar los dos pasos anteriores para cada elemento de la permutaci\'on
\item Sumar el total de inversiones contadas para cada elemento
\end{itemize}

\textbf{Ejemplo}:

Se toma la permutación \textbf{$A=\{6,1,3,4,5,2\}$}\\

Primer elemento: $6$; menores a la derecha: $1,3,4,5,2$; n\'umero de
inversiones: \textbf{5}
Primer elemento: $1$; menores a la derecha: $\emptyset$; n\'umero de
inversiones: \textbf{0}
Primer elemento: $3$; menores a la derecha: $2$; n\'umero de inversiones:
\textbf{1}\\
Primer elemento: $4$; menores a la derecha: $2$; n\'umero de inversiones:
\textbf{1}\\
Primer elemento: $5$; menores a la derecha: $2$; n\'umero de inversiones:
\textbf{1}\\

Total de inversiones: \textbf{8}
\begin{tcolorbox}[colback=blue!5!white,colframe=blue!60!black,title=Definición: Tipos de permutaciones]
	\textbf{Permutaci\'on par:} aquella en la que el total de inversiones es un
	entero par.
	
	\tcblower

	\textbf{Permutaci\'on impar}: aquella en la que el total de inversiones es un
	entero impar.
	
\end{tcolorbox}

\subsection{Combinación}

\subsection{Problema de Monty Hall}

\subsection{Problema del cumpleaños}


\subsection{Principio del palomar o Principio de Dirichlet}

El principio del palomar, también llamado principio de Dirichlet o principio de
las cajas, establece que si $n$ palomas se distribuyen en $m$ palomares, y si $n
> m$, entonces al menos habrá un palomar con más de una paloma. Otra forma de
decirlo es que $m$ huecos pueden albergar como mucho $m$ objetos si cada uno de
los objetos está en un hueco distinto, así que el hecho de añadir otro objeto
fuerza a volver a utilizar alguno de los huecos, como se observa en la Fig.
(\ref{palomar}). A manera de ejemplo: si se toman trece personas, al menos dos
habrán nacido el mismo mes.

\begin{figure}[h]
	\centering
	\begin{tikzpicture}
		% draw the sets
		\filldraw[fill=blue!20, draw=blue!60] (-1.5,0) circle (2cm);
		\filldraw[fill=green!20, draw=green!60] (3,0) circle (2cm);
	
	
		% the texts
		\node at (-1.5,3) {Palomas};
		\node at (3,3) {Palomares};
	
		% the circles and the arrows
		\node (x1) at (-1.5,0.8) {$p_1$};
		\node (x2) at (-1.5,0.3) {$p_2$};
		\node (x3) at (-1.5,-0.3) {$p_3$};
		\node (x4) at (-1.5,-0.8) {$p_4$};
		
		\node (y1) at (3,0.7) {$1$};
		\node (y2) at (3,0) {$2$};
		\node (y3) at (3,-0.7) {$3$};
	
		% draw the arrows
		\draw[-latex] (x1) -- (y1);
		\draw[-latex] (x2) -- (y2);
		\draw[-latex] (x3) -- (y3);
		\draw[-latex] (x4) -- (y3);
	
	\end{tikzpicture}
\caption{Ilustración del principio del palomar. El conjunto de las palomas es
mayor al del palomar, por lo tanto en al menos un palomar deben de haber dos
palomas.}
\label{palomar}
	\end{figure}



\documentclass[12pt]{article}
\usepackage[spanish,es-tabla,es-nodecimaldot]{babel}
\usepackage{tikz}
\usepackage{amsmath, amssymb, amsbsy}    %Caracteres especiales de matematicas
\usepackage[hidelinks]{hyperref}    % Vinculos [que no estén subrayados]
\usepackage{listings}   %Introducir codigo
\usepackage{setspace}
\usepackage{color}
\usepackage{pdflscape}  %Voltear la página
\usepackage{graphicx}
\usepackage{caption}


\begin{document}
\title{Ejercicios de Técnicas de Conteo}
\author{Ana Maritza Bello Ya\~nez}
\maketitle
\setlength{\parindent}{0pt}
\setlength{\parskip}{1em}

\section*{Ejercicios de Combinatoria (Schaum)}
% Problemas 1.7, 1.8, 1.29, 1.39, 1.40, 1.77, 1.78, 1.87
% 5 ejemplos de muestras ordenadas

\subsection*{Problema 1.7}
Demostrar que un número palindrómico (decimal) de longitud par es divisible por
11.

\textbf{Solución}

La prueba inductiva explota el hecho de que cuando se quitan el primer y el
último carácter de un palíndromo, queda un palíndromo. Así, sea $N$ un número
palindrómico de longitud $2k$. Si $k = 1$, el teorema obviamente se cumple. Si
$k \geq 2$, tenemos

$ N = a_{2k-1} 10^{2k-1} + a_{2k-2} 10^{2k-2} + ... + a_{k} 10^{k} + ... +
a_{2k-2} 10^{1} + a_{2k-1} 10^{0} $

$ = a_{2k-1}(10^{2k}+10^0) + (a_{2k-2}10^{2k-2}+...+ a_{2k-2}10^{1}) $

$ \equiv a_{2k-1} P + Q$

Donde:

$P = \underbrace{100...001}_{\text{longitud } 2k} =
\underbrace{9090...9091}_{\text{longitud } 2k-2}$

y ya sea $Q=0$ (divisible por 11) o, para algún $1 \leq r \leq k-1$,

$Q = 10^r$ \{palindrome de longitud $ 2(k-r)\} = 10^r\{11R\} $

donde el último paso se sigue de la hipótesis de inducción. Por lo tanto, $N$ es
divisible por 11 y la demostración está completa.

\subsection*{Problema 1.8}

En un palíndromo binario, el primer dígito es 1 y cada dígito subsiguiente puede
ser 0 o 1. Cuente los palíndromos binarios de longitud $n$.

\textbf{Solución}

Tenemos $[(n + 1)/2] - 1 = [(n - 1)/2]$ posiciones libres, por lo tanto el número
deseado es:

$2^{[(n-1)/2]}$


\subsection*{Problema 1.29}
Encuentre la probabilidad $p_n$ de que un grupo de $n$ personas reunidas al azar
incluya al menos 2 personas con el mismo cumpleaños (día del año).

\textbf{Solución}

Aquí no tratamos con una muestra de personas, sino con una muestra de
cumpleaños, es decir, números enteros del 1 al 365 inclusivo. Nuestra noción de
probabilidad es:

$\text{Probabilidad} = \frac{\text{número de muestras favorables}}{\text{Número total de muestras}}$

En este problema es simple considerar el evento complementario: todos los $n$
cumpleaños son distintos. Este evento es realizado en $P(365,n)$ muestras; y el
total de muestras es $365^n$. Por lo tanto, $1-p_n = P(365,n)/365^n$ o:

$p_n = 1 - \frac{P(365,n)}{365^n} = 1 -
\frac{(365)(365-1)(365-2)...[365-(n-1)]}{365^n} $

$= 1-(1 - \frac{1}{365})(1 - \frac{2}{365})...(1 - \frac{n-1}{365})$

Puede verificarse que $p_n \geq 1/2 $ cuando $n>25$.

\subsection*{Ejercicio 1.39}
Pruebe que 
\begin{enumerate}
	\item 
    \begin{equation*}
		\Sigma_{r=0}^nC(n,r)=2^n
    \end{equation*}

\textbf{Solución}

\begin{eqnarray*}
	\Sigma_{r=0}^{n} \binom{n}{r}=\Sigma_{r=0}^{n} \binom{n}{r}(1^{n-r})(1^r)=(1+1)^n=2^n
\end{eqnarray*}
	
    \item 
    \begin{equation*}
		\Sigma_{r=0}^n(-1)^rC(n,r)=0
	\end{equation*}

\textbf{Solución}

\begin{eqnarray*}
	\Sigma_{r=0}^{n} (-1)^r \binom{n}{r}=\Sigma_{r=0}^{n} \binom{n}{r}(1^{n-r})(-1^r)=(1-1)^n=0
\end{eqnarray*}

\item
\begin{equation*}
	\Sigma_{r=even}^nC(n,r)=\Sigma_{r=odd}^nC(n,r)=2^{n-1}
\end{equation*}

\textbf{Solución}

\begin{eqnarray*}
	\Sigma_{r=even}^nC(n,r)=\frac{\Sigma_{r=0}^{n} \binom{n}{r}}{2}=\frac{2^n}{2}=2^{n-1}\\
	\Sigma_{r=odd}^nC(n,r)=\frac{\Sigma_{r=0}^{n} \binom{n}{r}}{2}=\frac{2^n}{2}=2^{n-1}
\end{eqnarray*}
	
\end{enumerate}

\subsection*{Ejercicio 1.39}
Obtenga nuevamente los resultados del ejercicio 1.39 con argumentos combinatorios.

\textbf{Solución}

\begin{enumerate}
	
\item Si se considera un conjunto de $n$ elementos, se pueden formar $2^n$
subconjuntos (conjunto potencia). El conjunto potencia incluye los subconjuntos
que se pueden formar con r elementos. Así:

\begin{equation*}
	|\mathcal{P}|=2^n=\binom{n}{0}+\binom{n}{1}+...+\binom{n}{n-1}+\binom{n}{n}=\Sigma_{r=0}^{n} \binom{n}{r}
\end{equation*} 

\item Se establece que X tiene tantos subconjuntos con $r$ elementos pares como
$r$ elementos impares.

Caso 1: Esto se cumple si n es impar.
	
Caso 2: Cuando $n$ es par.

Se descompone $X=X'\cup \{j\}$. Donde $j$ es un elemento de $X$\\
	
Ahora $X'$ está compuesto por un n\'umero impar de elementos, por lo que tiene
la misma cantidad de subconjuntos formados con $r$ par ($X_p$) o impar ($X_i$),
pero como se había eliminado un elemento $|X_p|=|X_i|=2^{n-1}$

\end{enumerate}

\subsection*{Ejercicio 1.77}

Muestre que en un grupo de n personas, al menos 2 personas conocen al mismo
n\'umero de personas.

\textbf{Solución}

Sea $k$ el n\'umero de personas que no conoce a nadie del grupo

\begin{itemize}
\item Caso $k>1$: Hay m\'as de una persona que no conoce a nadie y entonces se
cumple el enunciado.
	
\item Caso $k=0$: Como no hay persona que no conozca a nadie. Todas las personas
pueden conocer al menos a 1 persona o hasta $n-1$ personas. 
	
Con el principio del palomar (principio de Dirichlet), supongamos que las cajas
(pichoneras) están de acuerdo al n\'umero de personas conocidas. Es decir, que
puede haber hasta $n-1$ pichoneras y que le han preguntado a las $n-1$ personas
a cuántos de los otros conocen. En el caso extremo, uno a uno se ir\'an
colocando en las cajas correspondientes a sus n\'umeros.
	
Cuando le pregunten a la persona $n$, forsozamente tendr\'a que elegir un
n\'umero que alguien m\'as haya elegido y entonces el enunciado se cumple. 
	
\item Caso $k=1$: Se puede despreciar a esa persona que no conoce a nadie y
entonces se hace el an\'alisis para el caso k=0 con un grupo de $n-1$ personas.

\end{itemize}

\subsection*{Problema 1.78}

Considera un torneo in el cual cada uno de los $n$ jugadores juega contra cada
uno de los otros jugadores y cada jugador gana al menos 1 juego. Prueba que hay
al menos dos jugadores que tienen el mismo n\'umero de victorias.

El n\'umero de victorias de cada jugador es al menos una y por mucho $n-1$
victorias. Estos $n-1$ n\'umeros corresponde a las pichoneras para acomodar $n$
palomas.

\subsection*{Problema 1.87}
Hay 12 computadoras y 8 impresoras l\'aser en una oficina. Encuentra el n\'umero
m\'inimo de conexiones que se tienen que hacer para garantizar que si 8 o menos
computadoras quieren imprimir al mismo tiempo, cada una de ellas ser\'a capaz de
usar una impresora diferente.\\

Suponemos que las impresoras se denotan por $P_{j} (j=1,2,...,8)$ y las
computadoras por $C_{i} (i=1,2,..,12)$.\\

Conectar la primera impresora a las primeras cinco computadoras. Despu\'es
conectar la segunda impresora a las 5 impresoras consecutivas empezando por
$C_{2}$. Despu\'es conectar la tercera impresora con las 5 computadoras
consecutivas empezando por $C_{3}$ y as\'i consecutivamente para generar la
matriz de conexiones.

\begin{table}[h!]
	\begin{tabular}{l|lllllllll}
		       &$P_{1}$&$P_{2}$&$P_{3}$ &$P_{4}$&$P_{5}$&$P_{6}$&$P_{7}$&$P_{8}$&\\\hline
		$C_{1}$&  1&  0&  0&   0& 	 0&   0&   0&  0&\\ 
		$C_{2}$&  1&  1&  0&   0& 	 0&   0&   0&  0&\\
		$C_{3}$&  1&  1&  1&   0& 	 0&   0&   0&  0&\\
		$C_{4}$&  1&  1&  1&   1& 	 0&   0&   0&  0&\\
		$C_{5}$&  1&  1&  1&   1& 	 1&   0&   0&  0&\\
		$C_{6}$&  0&  1&  1&   1& 	 1&   1&   0&  0&\\
		$C_{7}$&  0&  0&  1&   1& 	 1&   1&   1&  0&\\
		$C_{8}$&  0&  0&  0&   1& 	 1&   1&   1&  1&\\
		$C_{9}$&  0&  0&  0&   0& 	 1&   1&   1&  1&\\
		$C_{10}$&  0&  0&  0&   0& 	 0&   1&   1&  1&\\
		$C_{11}$&  0&  0&  0&   0& 	 0&   0&   1&  1&\\
		$C_{12}$&  0&  0&  0&   0& 	 0&   0&   0&  1&\\
	\end{tabular}
\end{table}

Sean las 8 computadoras que requieren una impresora $C_{i_{1}}, C_{i_{2}},...,
C_{i_{8}}$ donde $i_{1} <i_{2}<...<i_{8} < $. Si 8 computadoras pueden ser
acomodadas, entonces se puede acomodar cualquier n\'umero m\'as peque\~no. La
observaci\'on crucial es:

\begin{equation}
	s \leq i_{s} \leq s+4   (s=1,2,...,8)
\end{equation}

En efecto, si $i_{s} < s$ existen $s$ enteros positivos menores que $s$ y si
$i_{s} > s+5$, al menos $12-(s+6)+1=7-s$ valores estar\'ian disponibles para los
$8-s$ indices restantes.\\


\section*{Ejercicios de Probabilidad (Schaum)}
% Problemas 2.29, 2.30, 2.68 al 2.74

\subsection*{Problema 2.29}
Construya el diagrama de árbol para las permutaciones de $\{ a,b,c \}$.

\begin{figure}[h]
	\begin{center}		
	\begin{tikzpicture}[level distance=1cm, level 1/.style={sibling distance=3cm},
		level 2/.style={sibling distance=2cm}, every node/.style={circle, draw,
		align=center} ]
				\centering
				\node[circle,draw]{$P_n$}
		child{node{a} child{node{b} child{node{c}} } child{node{c} child{node{b}}} }
		child{node{b} child{node{a} child{node{c}} } child{node{c} child{node{a}}} }
		child{node{c} child{node{a} child{node{b}} } child{node{b} child{node{a}}} };
			\end{tikzpicture}
		\end{center}
		\caption{Permutaciones para 3 elementos}
		\label{permutaciones_tree}
\end{figure}

\subsection*{Problema 2.30}

Una persona tiene tiempo para jugar a la ruleta cinco veces como máximo. En cada
jugada gana o pierde un dólar.

La persona comienza con un dólar y dejará de jugar antes de las cinco veces si
pierde todo su dinero o si gana tres dólares, es decir, si tiene cuatro dólares.
Encuentre el número de formas en que pueden ocurrir las apuestas.


El diagrama de árbol a la derecha describe la forma en que se pueden realizar
las apuestas. Cada número en el diagrama denota la cantidad de dólares que tiene
el hombre en ese momento. Observe que las apuestas pueden ocurrir de 11 maneras
diferentes. Tenga en cuenta que dejará de apostar antes de que se cumplan los
cinco tiempos en solo tres de los casos.

% Set the overall layout of the tree
\tikzstyle{level 1}=[level distance=1.5cm, sibling distance=3.5cm]
\tikzstyle{level 2}=[level distance=1.5cm, sibling distance=3.5cm]
\tikzstyle{level 3}=[level distance=1.5cm, sibling distance=3cm]
\tikzstyle{level 4}=[level distance=1.0cm, sibling distance=2.5cm]
\tikzstyle{level 5}=[level distance=1.0cm, sibling distance=0.5cm]

% Define styles for bags and leafs
\tikzstyle{bag} = [text width=1em, text centered]
\tikzstyle{end} = [circle, minimum width=0.5pt,fill, inner sep=0pt]

\begin{tikzpicture}[grow=right, sloped]
    \node[bag] {1}
    child {
        node[bag] {0}
        }
    child {
        node[bag] {2}% This is the first of three "Bag 2"
        child {
                node[bag] {3}
                child{
                        node[bag] {2}
                                                    child {
                             node[bag] {3}
                                                         child {
                             node[bag] {4}
                }
                            child {
                             node[bag] {2}
                }
                }
                            child {
                             node[bag] {1}
                                                         child {
                             node[bag] {2}
                }
                            child {
                             node[bag] {0}
                }
                }
                }
                child{
                        node[bag] {4}
                }
            }
            child {
                node[bag] {1}
                child {
        node[bag] {2}
            child {% Here are three children, hence three end branches
                node[bag] {3}
                            child {
                             node[bag] {2}
                }
                            child {
                             node[bag] {0}
                }
            }
            child {
                node[bag] {1}
                            child {
                             node[bag] {2}
                }
                            child {
                             node[bag] {0}
                }
            }
    }
            }
    }
    ;
    \end{tikzpicture}

\begin{landscape}

\subsection*{Problema 2.68}
Construye el diagrama de árbol para el número de permutaciones de $\{ a,b,c,d
\}$

\begin{tikzpicture}[
    auto,
    level 1/.style={sibling distance=58mm},
    level 2/.style={sibling distance=18mm},
    level 3/.style={sibling distance=8mm},
    ]
    \node [circle,draw] (z){}
        child {
            node[circle,draw] (b) {a}
            child { 
                node[circle,draw] (c) {b}
                child {
                    node[circle,draw] (d) {c}
                    child {
                        node[circle,draw] (e) {d}
                    }
                }
                child {
                    node[circle,draw] (d) {d}
                    child {
                        node[circle,draw] (e) {c}
                    }
                }               
            }
             child { 
                node[circle,draw] (c) {c}
                child {
                    node[circle,draw] (d) {b}
                    child {
                        node[circle,draw] (e) {d}
                    }
                }
                child {
                    node[circle,draw] (d) {d}
                    child {
                        node[circle,draw] (e) {b}
                    }
                }               
            } 
            child { 
                node[circle,draw] (c) {d}
                child {
                    node[circle,draw] (d) {b}
                    child {
                        node[circle,draw] (e) {c}
                    }
                }
                child {
                    node[circle,draw] (d) {c}
                    child {
                        node[circle,draw] (e) {b}
                    }
                }               
            }  
        }
        child {
            node[circle,draw] (b) {b}
                        child { 
                node[circle,draw] (c) {a}
                child {
                    node[circle,draw] (d) {c}
                    child {
                        node[circle,draw] (e) {d}
                    }
                }
                child {
                    node[circle,draw] (d) {d}
                    child {
                        node[circle,draw] (e) {c}
                    }
                }               
            }
                       child { 
                node[circle,draw] (c) {c}
                child {
                    node[circle,draw] (d) {a}
                    child {
                        node[circle,draw] (e) {d}
                    }
                }
                child {
                    node[circle,draw] (d) {d}
                    child {
                        node[circle,draw] (e) {a}
                    }
                }               
            }
                       child { 
                node[circle,draw] (c) {d}
                child {
                    node[circle,draw] (d) {a}
                    child {
                        node[circle,draw] (e) {c}
                    }
                }
                child {
                    node[circle,draw] (d) {c}
                    child {
                        node[circle,draw] (e) {a}
                    }
                }               
            }
        }
        child {
            node[circle,draw] (b) {c}
                        child { 
                node[circle,draw] (c) {a}
                child {
                    node[circle,draw] (d) {b}
                    child {
                        node[circle,draw] (e) {d}
                    }
                }
                child {
                    node[circle,draw] (d) {d}
                    child {
                        node[circle,draw] (e) {b}
                    }
                }               
            }
                        child { 
                node[circle,draw] (c) {b}
                child {
                    node[circle,draw] (d) {a}
                    child {
                        node[circle,draw] (e) {d}
                    }
                }
                child {
                    node[circle,draw] (d) {d}
                    child {
                        node[circle,draw] (e) {a}
                    }
                }               
            }
                        child { 
                node[circle,draw] (c) {d}
                child {
                    node[circle,draw] (d) {a}
                    child {
                        node[circle,draw] (e) {b}
                    }
                }
                child {
                    node[circle,draw] (d) {b}
                    child {
                        node[circle,draw] (e) {a}
                    }
                }               
            }
        }
        child {
            node[circle,draw] (b) {d}
                       child { 
                node[circle,draw] (c) {a}
                child {
                    node[circle,draw] (d) {b}
                    child {
                        node[circle,draw] (e) {c}
                    }
                }
                child {
                    node[circle,draw] (d) {c}
                    child {
                        node[circle,draw] (e) {b}
                    }
                }               
            }
                       child { 
                node[circle,draw] (c) {b}
                child {
                    node[circle,draw] (d) {a}
                    child {
                        node[circle,draw] (e) {c}
                    }
                }
                child {
                    node[circle,draw] (d) {c}
                    child {
                        node[circle,draw] (e) {a}
                    }
                }               
            }
                        child { 
                node[circle,draw] (c) {c}
                child {
                    node[circle,draw] (d) {a}
                    child {
                        node[circle,draw] (e) {b}
                    }
                }
                child {
                    node[circle,draw] (d) {b}
                    child {
                        node[circle,draw] (e) {a}
                    }
                }               
            }
        }
        
        ;
\end{tikzpicture}


\end{landscape}

\subsection*{Problema 2.69}


\subsection*{Problema 2.70}


\subsection*{Problema 2.71}

\subsection*{Problema 2.72}

\subsection*{Problema 2.73}

\subsection*{Problema 2.74}

\end{document}


\section{Probabilidad y conteo}

\subsection{Definici\'on cl\'asica}

La definici\'on cl\'asica de la probabilidad \cite{faber2012statistics} de un
evento $A$, puede ser formulada de la siguiente manera:

\begin{equation}
    P(A)= \frac{n_A}{n_(total)}
\end{equation}

Donde $n_A$ es el n\'umero de formas igualmente probables por las cuales un
experimento puede conducir a A; $n_(total)$ n\'umero total de formas igualmente
probables en el experimento.

- En la definici\'on cl\'asica el experimento no necesariamente se lleva a cabo
ya que la respuesta es conocida de antemano.

- La teor\'ia cl\'asica no da soluci\'on a menos de que todas las formas
igualmente posibles puedan ser derivadas anal\'iticamente.



\section{Probabilidad Condicional}

La probabilidad condicional nos provee una forma de razonar acerca de la salida
o resultado de un experimento, basado en información parcial. Algunos ejemplos
de estos experimentos podrían ser los siguientes:

\begin{itemize}
\item En un experimento en el cual tiramos dos dados sucesivamente, te dicen que
la suma de los dados es 9. ¿Qué tan probable es que el primer dado haya caído 6?

\item En un juego de adivinanzas de palabras, la primera letra de la palabra es
\textit{t}. ¿Cuál es la probabilidad de que la siguiente palabra sea \textit{h}?

\item ¿Qué  tan probable es que una persona tenga cierta enfermedad dado que su
examen médico dió negativo?
\end{itemize}

En términos más precisos, dado un experimento, un espacio de muestreo y una ley
de probabilidad, suponiendo que sabemos que la salida está dentro de un evento
dado $B$. Deseamos cuantificar la probabilidad de que la salida pertenece a
algún otro evento $A$. Así, podemos construir una nueva probabilidad que tome en
cuenta el conocimiento disponible: una ley de probabilidad para cuaquier evento
$A$. Especificamente, la probabilidad condicional de $A$ dado $B$, denotado por
$P(A|B)$.

Para calcular $P(A|B)$ consideramos aquellos resultados del evento $A$ que están en
el evento $B$. Esto nos da los resultados en el evento $A \cap B$ y nos lleva al
teorema de la probabilidad condicional.

\begin{theorem}{Probabilidad Condicional}{conditional_probability}
Sea $S$ el espacio de muestro para un experimento $E$ y $A$, $B \subseteq S$,
entonces la probabilidad condicional de $A$ dado $B$ está dada por:
    \begin{equation}
        P(A|B) = \frac{P(A \cap B)}{P(B)}
        \label{eq:conditionalProbability}
    \end{equation}

Siempre que $P(B)>0$.

Particularmente, todas las propiedades de las leyes de la probabilidad
permanecen válidas para las leyes de la probabilidad condicional.

\begin{itemize}
    \item La probabilidad condicional puede verse también como una ley de
    probabilidad en un nuevo universo $B$, ya que toda la probabilidad condicional
    está concentrada en $B$.

    \item Si todos los resultados son finitos e igualmente probables, entonces:
        \begin{equation}
            P(A|B)=\frac{\text{número de elementos de }A \cap B}{\text{número de elementos de }B}
        \end{equation}
\end{itemize}
\end{theorem}


\def\firstcircle{(0,0) circle (1.5cm)}
\def\secondcircle{(0:2cm) circle (1.5cm)}
\def\rectangle{(-2,-2) rectangle (4,2)}

\colorlet{circle edge}{black!50}
\colorlet{circle area}{gray!20}

\tikzset{filled/.style={fill=circle area, draw=circle edge, thick},
    outline/.style={draw=circle edge, thick}}

\setlength{\parskip}{5mm}

\begin{figure}[h]
    \centering
    \begin{tikzpicture}
        \draw \rectangle;
        \begin{scope}
            \clip \firstcircle;
            \fill[filled] \secondcircle;
        \end{scope}
        \draw[outline] \firstcircle node {$A$};
        \draw[outline] \secondcircle node {$B$};
        \node[anchor=south] at (current bounding box.south) {$A \cap B$};
        \node at (3.5,1.5){$\mathbb S$};
    \end{tikzpicture}
    \caption{Representación del concepto intuitivo de la probabilidad condicional.}
    \label{fig:coditionalProbability}
\end{figure}

\def\firsttcircle{(90:1.75cm) circle (1.5cm)}
\def\seconddcircle{(170:1.75cm) circle (1.5cm)}
\def\thirddcircle{(-25:2cm) circle (1.5cm)}
\begin{figure}
    \centering
\begin{tikzpicture}
      \begin{scope}
    \clip \seconddcircle;
    \fill[cyan] \thirddcircle;
      \end{scope}
      \begin{scope}
    \clip \firsttcircle;
    \fill[cyan] \thirddcircle;
      \end{scope}
      \draw \firsttcircle node[text=black,above] {$A$};
      \draw \seconddcircle node [text=black,below left] {$B$};
      \draw \thirddcircle node [text=black,below right] {$C$};
\end{tikzpicture}
\end{figure}

\textbf{Observaciones y consideraciones de la probabilidad condicional:}

\begin{itemize}
\item La probabilidad $P(A|B)$ es una actualización de $P(A)$, basada en el
conocimiento de que ocurrió el evento $B$.

\item De la Ec.(\ref{eq:conditionalProbability}), tanto $P(A \cap B)$ como
$P(A)$ se calculan a partir del espacio muestral original.

\item Las probabilidades tienen sutiles cambios dependiendo de la información
exacta de la condición implicada en el evento A.
\end{itemize}

\subsection{Propiedades de las leyes de la probabilidad}

Las leyes de la probabilidad tienen propiedades que pueden deducirse de los
axiomas. Algunas de ellas, son las que se muestran a continuación.

\begin{tcolorbox}[colback=blue!5!white,colframe=blue!60!black,title=Resumen: Propiedades de las leyes de la probabilidad]
    Sean $A$, $B$ y $C$ eventos:

    \begin{itemize}
        \item Si $A \subset B$, entonces $P(A) \leq P(B)$.
        
        \item $P(A \cup B) = P(A) + P(B) - P(A \cap B)$.

        \item $P(A \cup B) \leq P(A) + P(B)$.

    \item $P(A \cup B \cup C) = P(A) + P(B) + P(C) - P(A \cap B) - P(A \cap C) - P(B
        \cap C) + P(A \cap B \cap C)$
    \end{itemize}
\end{tcolorbox}

De la Ec.(\ref{eq:conditionalProbability}) podemos hacer que:

\begin{center}
$P(B \cap A) = P(A \cap B) = P(A)P(B|A)$, 
\end{center}

y cambiando los roles de $A$ y de $B$, tenemos que:

\begin{center}
    $P(A \cap B) = P(B \cap A) = P(B)P(A|B)$, 
\end{center}

esto resulta en:

\begin{center}
    $P(A)P(B|A) = P(A \cap B) = P(B)P(A|B)$,
\end{center}

Que comunmente llamamos \textit{regla de la multiplicación}.

\begin{figure}
% Set B but not A
\centering
\begin{tikzpicture}
    \draw \rectangle;
    \begin{scope}
        \clip \secondcircle;
        \draw[filled, even odd rule] \firstcircle
                                     \secondcircle node {$B$};
    \end{scope}
    \draw[outline] \firstcircle node {$A$}
                   \secondcircle;
    \node[anchor=south] at (current bounding box.north) {$A^c \cap B$};
    \node at (1,0){$A \cap B$};
\end{tikzpicture}
\caption{Representación gráfica de la diferencia de los conjuntos de $A-B$ y de $B-A$.}
\label{fig:difer}
\end{figure}

\subsection{Teorema de Bayes}

\begin{theorem}{Teorema de Bayes}{BayesTheorem}
Sean $A$ y $B$ dos eventos cuyas probabilidades son diferentes de cero,
entonces:
    \begin{equation}
        P(B|A) = \frac{P(A|B) P(B)}{P(A)}
        \label{eq:BayesTheorem}
    \end{equation}

La implicación más importante de la Ec. (\ref{eq:BayesTheorem}) es que permite
encontrar probabilidades condicionadas $P(B|A)$ en términos de $P(A|B)$, cuando
esta última resulta más fácil de calcular directamente.
\end{theorem}



\section*{Convenio de suma de Einstein}

Se denomina notación de Einstein o notación indexada a la convención utilizada
para abreviar la escritura de las sumatorias, donde se suprime el término de la
sumatoria $(\sum)$. Este convenio fue introducido por Albert Einstein en 1916.

Dada una expresión lineal en $\mathbb{R^{n}}$ en la que se escriban todos sus
términos en forma explícita:

\begin{equation}
    u = u_1 x_1 + u_2 x_2 + ...u_n x_n
\end{equation}

Se puede escribir de la forma:

\begin{equation}
    u = \sum_{i=1}^{n} u_ix_i
\end{equation}

La notación de Einstein obtiene una expresión aún más condensada eliminando el
signo de la sumatoria y entendiendo que la expresión resultante en un índice
indica la suma sobre todos los posibles valores del mismo.

\begin{theorem}{Convenio de la suma de Einstein}{convenioSuma}
    \begin{equation}
        u = u_ix_i
    \end{equation}
\end{theorem}

\subsection*{Tensor Levi-Civita}

Está definido por:

\begin{theorem}{Tensor de Levi-Civita}{leviCivita}
\begin{equation}
    \epsilon_{ijk}=
    \left\lbrace\begin{array}{ccccl} 
        +1 & ijk, & kji, & jki & \text{Permutación cíclica o par}\\ 
        -1 & ikj, & kji, & jik & \text{Permutación anticíclica o impar}\\
        0  & iii, & jjj, & kkk & \text{Si se repiten}
    \end{array}\right.
\end{equation}
\end{theorem}

\subsection*{Producto escalar de dos vectores}

También conocido como producto punto, es una operación algebraíca que toma dos
vectores y retorna un escalar. esta definido de la siguiente manera:

\begin{theorem}{Producto escalar de dos vectores}{productoEscalar}
    \begin{equation}
        \bar{A} \cdot \bar{B}= \bar{\abs{A}} \bar{\abs{B}} \cos \theta
        = \sum_{i=1}^{n} a_i b_i
    \end{equation}
\end{theorem}

Que de acuerdo al convenio de suma de Einstein podemos escribir como:

\begin{equation}
    \bar{A} \cdot \bar{B} = a_i b_i
\end{equation}

\subsection*{Producto vectorial de dos vectores}

EL producto vectorial o producto cruz de dos vectores es una operación binaria
entre dos vectores en un espacio tridimensional. El resultado es un vector
perpendicular a los vectores que se multiplican, y por lo tanto normal al plano
que los contiene.

Está definido de la siguiente manera:

\begin{theorem}{Producto Vectorial de dos vectores}{productoVectorial}
\begin{equation}
    \bar{A} \times \bar{B}= \bar{\abs{A}} \bar{\abs{B}} \sin \theta \  \hat{r}
\end{equation}
\end{theorem}

Donde $\hat{r}$ es el vector unitario y ortogonal a los vectores $\bar{A}$ y
$\bar{B}$, y $\theta$ el ángulo entre $\bar{A}$ y $\bar{B}$.

Que mediante determinantes tenemos:

\begin{equation}
    \bar{A} \times \bar{B}=
    \begin{bmatrix}
        \hat{i} & \hat{j} & \hat{k}\\ 
         a_1 & a_1 & a_1\\ 
         b_1 & b_1 & b_1
    \end{bmatrix}
\end{equation}

Desarrollamos:

\begin{equation}
    \bar{A} \times \bar{B}= \hat{i}(a_2 b_3 - a_3 b_2) - \hat{j}(a_1 b_3 - a_3 b_1) + \hat{k}(a_1 b_2 - a_2 b_1)
\end{equation}

Desde un punto de vista tensorial el producto generalizado de $n$ vectores vendrá
dado por:

\begin{theorem}{Generalización del producto vectorial desde el punto de vista tensorial}{productoTensorial}
    \begin{equation}
        (\bar{A} \times \bar{B})_{i} = \epsilon_{ijk} a_{j} b_{k}
    \end{equation}
\end{theorem}

Desarrollando:

\begin{equation}
    \epsilon_{ijk} a_{jbk} =
    \begin{array}{ccccc}
        \cancel{\epsilon_{111} a_1 b_1} & + & \cancel{\epsilon_{112} a_1 b_2}   & + & \cancel{\epsilon_{112} a_1 b_3}   \\
        \cancel{\epsilon_{121} a_2 b_1} & + & \cancel{\epsilon_{122} a_2 b_2}   & + & \epsilon_{123} a_2 b_3            \\
        \cancel{\epsilon_{131} a_3 b_1} & + & \epsilon_{132} a_3 b_2            & + & \cancel{\epsilon_{133} a_3 b_3}
    \end{array}
    = a_2 b_3 - a_3 b_2
\end{equation}



\subsection{Congruencia de Zeller}

\lstinputlisting[language=Python]{../src/zeller_congruency.py}



%
\section{Tri\'angulo de Sierpinski}

El matem\'atico polaco Waclav Sierpinski (1882-1969), construy\'o este fractal
en 1919 del modo siguiente: tom\'o un tri\'angulo equil\'atero, uni\'o los
puntos medios de los lados y form\'o cuatro tri\'angulos interiores: tres
tri\'angulos equil\'ateros sombreados y un hueco que es otro tri\'angulo
equil\'atero. Repiti\'o el proceso en cada uno de los tri\'angulos sombreados, y
sigui\'o hasta el infinito el proceso en los tres tri\'angulos restantes como el
primero.


\begin{figure}[h]
    \centering
    \usetikzlibrary{lindenmayersystems}
\def\trianglewidth{3cm} \pgfdeclarelindenmayersystem{Sierpinski triangle}{
        \symbol{X}{\pgflsystemdrawforward}
        \symbol{Y}{\pgflsystemdrawforward}
        \rule{X -> X-Y+X+Y-X}
        \rule{Y -> YY}
} \foreach \level in {1,3,5}{ \tikzset{
l-system={step=\trianglewidth/(2^\level), order=\level, angle=-120} }

        \begin{tikzpicture}
\fill [black] (0,0) -- ++(0:\trianglewidth) -- ++(120:\trianglewidth) -- cycle;
\draw [draw=none] (0,0) l-system [l-system={Sierpinski triangle,
axiom=X},fill=white];
        \end{tikzpicture}
}
    \caption{Tri\'angulo de Sierpinski para los niveles 1,3 y 5.}
    \label{fig:sierpinski}
\end{figure}

Como se muestra en la Fig.(\ref{fig:sierpinski})




\section{Matriz de rotaci\'on}
La matriz de rotación es la siguiente:
%CORREGIR


	\begin{eqnarray*}
		x^{'m} = a^{m}\hfill_{\nu} x^{\nu}\\
		m = 0,1,2,3\\
		\nu = 0,1,2,3
	\end{eqnarray*}		
	\begin{eqnarray*}
		x^{'(0)} =
		\left( {\begin{array}{cc}
				a^{0} _0 x^{_0} + a^{0} _1 x^{_1} + a^{0} _2 x^{_2} + a^{0} _3 x^{_3}\\
		\end{array} } \right)\\
	x^{'(1)} =
	\left( {\begin{array}{cc}
			a^{1} _0 x^{_0} + a^{1} _1 x^{_1} + a^{1} _2 x^{_2} + a^{1} _3 x^{_3}\\
	\end{array} } \right)\\
	x^{'(2)} =
	\left( {\begin{array}{cc}
			a^{2} _0 x^{_0} + a^{2} _1 x^{_1} + a^{2} _2 x^{_2} + a^{2} _3 x^{_3}\\
	\end{array} } \right)\\
	x^{'(3)} =
	\left( {\begin{array}{cc}
			a^{3} _0 x^{_0} + a^{3} _1 x^{_1} + a^{3} _2 x^{_2} + a^{3} _3 x^{_3}\\
	\end{array} } \right)
	\end{eqnarray*}
\begin{equation*}
	\left(
	\begin{array}{ccc}
		x^{'0}\\
		x^{'1}\\
		x^{'2}\\
		x^{'3}\\ 
	\end{array}
	\right)
	=
	\left(
	\begin{array}{c}
		a^{m}\nu
	\end{array}
	\right)
	{}
	\left(
	\begin{array}{ccc}
		x^{0}\\
		x^{1}\\
		x^{2}\\
		x^{3}\\ 
	\end{array}
	\right)
\end{equation*}		
%\caption{Matriz de rotaci\'on}
%\label{permutaciones_tree}

\chapter{Matrices de rotaci\'on}
	Una matriz de rotaci\'on es una matriz que representa una rotaci\'on en el espacio euclidiano.
	Las matrices de rotaci\'on pueden ser en dos o tres dimensiones.
	
	Por ejemplo, si la matriz de rotaci\'on es de dos dimensiones y se busca rotar en sentido horario, la matriz tendrá la siguiente forma:
	\begin{equation}
		R(\theta)=
		\begin{bmatrix}
			\cos \theta & -\sin \theta\\
			\sin \theta & \cos \theta
		\end{bmatrix}
	\end{equation}
	De esta manera, las nuevas coordenadas, una vez aplicando la rotación son de la siguiente manera:
	\begin{eqnarray}
		\left(
		\begin{matrix}
			x'\\
			y'
		\end{matrix}
		\right)
		=
		\begin{bmatrix}
			\cos \theta & -\sin \theta\\
			\sin \theta & \cos \theta
		\end{bmatrix}
		\left(
		\begin{matrix}
			x\\
			y
		\end{matrix}
		\right)\\
		x'=x\cos \theta -y\sin \theta\\
		y'=x\sin \theta + y \cos \theta
	\end{eqnarray}

En caso de querer que la rotación sea antihoraria, se utiliza la siguiente matriz:
\begin{equation}
	R(-\theta)=
	\begin{bmatrix}
		\cos \theta & \sin \theta\\
		-\sin \theta & \cos \theta
	\end{bmatrix}
\end{equation}

Si lo que se busca es realizar rotaciones en el espacio tridimensional, se deben modificar las matrices de la siguiente manera:
\begin{equation}
	R_x(\theta)=
	\begin{bmatrix}
		1 & 0 & 0\\
		0 & \cos \theta & \sin \theta\\
		0 & -\sin \theta & \cos \theta
	\end{bmatrix}
\end{equation}

\begin{equation}
	R_y(\theta)=
	\begin{bmatrix}
		\cos \theta & 0 & \sin \theta\\
		0 & 1 & 0\\
		-\sin \theta & 0 & \cos \theta
	\end{bmatrix}
\end{equation}

\begin{equation}
	R_z(\theta)=
	\begin{bmatrix}
		\cos \theta & -\sin \theta & 0\\
		\sin \theta & \cos \theta & 0\\
		0 & 0 & 1
	\end{bmatrix}
\end{equation}
\subsection{Ángulos de Euler}


Los ángulos de Euler constituyen un conjunto de tres coordenadas angulares que
sirven para especificar la orientación de un sistema de referencia de ejes
ortogonales, normalmente móvil, respecto a otro sistema de referencia de ejes
ortogonales normalmente fijos.

Dados dos sistemas de coordenadas $xyz$ y $XYZ$ con origen común, es posible
especificar la posición de un sistema en términos del otro usando tres ángulos
$\alpha$, $\beta$, $\gamma$.

La definición matemática es estática y se basa en escoger dos planos, uno en el
sistema de referencia y otro en el triedro rotado. En el esquema adjunto serían
los planos $xy$ y $XY$. Escogiendo otros planos se obtendrían distintas convenciones
alternativas, las cuales se llaman de Tait-Bryan cuando los planos de referencia
son no-homogéneos (por ejemplo $xy$ y $XY$ son homogéneos, mientras $xy$ y $XZ$ no lo
son).

La intersección de los planos coordenados $xy$ y $XY$ escogidos se llama línea de
nodos, y se usa para definir los tres ángulos, Fig. (\ref{fig:euler_angles}) :

\begin{itemize}
    \item $\alpha$ es el ángulo entre el eje $x$ y la línea de nodos.
    \item $\beta$  es el ángulo entre el eje $z$ y el eje $Z$.
    \item $\gamma$  es el ángulo entre la línea de nodos y el eje X.
\end{itemize}

Más adelante se establecerá que los tres ángulos de Euler descritos son los
valores de las tres rotaciones intrínsecas que describen el sistema.

Notar que también se considera la notación: $\alpha =\phi$, $\gamma =\psi$,
$\beta =\theta$

Las rotaciones de Euler son los movimientos resultantes de variar uno de los
ángulos de Euler dejando fijo los otros dos. Son los siguientes:

\begin{itemize}
    \item Preseción
    \item Nutación
    \item Rotación intrínseca
\end{itemize}

Si escribimos la rotación de ángulos $\phi$
,$\theta$ ,$\psi$ como una composición de estas tres rotaciones:

\begin{equation}
    A(\phi ,\theta ,\psi )=R(\phi ,\theta ,\psi )N(\phi ,\theta )P(\phi)\
\end{equation}

entonces se cumple:

\begin{equation}
    \begin{array}{ll}
    A(\delta \phi +\phi ,\theta ,\psi ) & = P(\delta \phi )A(\phi ,\theta ,\psi) \\
    A(\phi ,\delta \theta +\theta ,\psi ) & = N(\phi ,\delta \theta )A(\phi ,\theta ,\psi ) \\
    A(\phi,\theta ,\delta \psi +\psi ) &=R(\phi ,\theta ,\delta \psi )A(\phi ,\theta ,\psi) \\
    \end{array}
\end{equation}

Como consecuencia de estas propiedades, estas rotaciones son conmutativas
entre ellas:

\begin{equation}
    P(\delta \phi )N(\delta \theta )A(\phi ,\theta ,\psi)=A(\delta \phi +\phi ,\delta \theta +\theta ,\psi )=N(\delta \theta )P(\delta \phi )A(\phi ,\theta ,\psi )
\end{equation}

lo que también podría verse intuitivamente usando la analogía entre los ángulos
de Euler y los de un soporte Cardán.

\begin{figure}[h!]
	\begin{center}
		\threedalphabetagamma
	\end{center}
	\caption{Posicionamiento del marco de coordenadas girado $(x', y', z')$ 
	usando ángulos de Euler $(\alpha, \beta, \gamma)$.}
	\label{fig:euler_angles}
\end{figure}






\pagebreak
\bibliography{../references/references.bib} 
\bibliographystyle{unsrt}

\end{document}

%poner alef 1 y alef2
% Axiomas de peano
